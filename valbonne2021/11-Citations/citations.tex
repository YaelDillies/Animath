\setlength{\parindent}{0cm}

%\item \textit{AUTEUR} : \og CITATION \fg


\begin{itemize}

\item \textit{Tristan} : \og Je fais des études de maths, ça fait des années que j'ai pas vu de chiffres. \fg

\item \textit{Rémi} : \og Non tu ne peux pas être à deux endroits en meme temps. Les physiciens y arrivent, mais toi non. \fg \\
\textit{Tristan} : \og C'est plutôt positif, ça veut dire que t'es pas physicien. \fg

\item \textit{Aurélien} : \og Si tu choisis des amis fachistes ça va se voir. \fg

\item \textit{Pierre-Marie} : \og Moi je dis pas de bêtises, demandes-en à Aurélien. \fg

\item \textit{Zinedine} : \og Surtout Paul, c'est un humain. \fg

\item \textit{Théo, parlant de graphes} : \og Cet arbre est moche, on veut le couper. \fg

\item \textit{Théo} : \og Donc là y a une chose dont Tristan se rend compte. Enfin normalement il s'en est déjà rendu compte. Mais bon ! C'est Tristan. \fg

\item \textit{Pierre-Marie} : \og On a une jolie petite équation toute mignonne qui se balade en forêt. \fg

\item \textit{Pierre-Marie} : \og Que tous ceux qui utilisent Zsigmondy sautent par la fenêtre. \fg \\
\textit{Plus tard dans la preuve,} Bon bah là on va devoir utiliser Zsigmondy.

\item \textit{Rémi} : \og Les gens qui ont eu 7/7, ça se compte sur les doigts d'un manchot. \fg

\item \textit{Anatole, 13 ans, à des élèves de première} : \og Bonne nuit les petits ! \fg

\item \textit{Gaspard} : \og Tic tac toc sans pif paf pof c'est beaucoup mieux que tic tac. \fg

\item \og Je dois écrire une lettre à quelqu'un. \fg \\
\textit{Martin} : \og Est-ce qu'elle s'appelle Élise, au cas où ? \fg

\item \textit{Rémi} : \og Je suis quelqu'un d'assez primitif. \fg

 \item \og Quand est-ce qu'on aura les T-shirts ? \fg \\
\textit{Rémi} : \og Oui. \fg \\
\textit{Les T-shirts n'arrivèrent que le dernier jour…}

\item \textit{Georges} : \og Les anims sont tous jeunes et dynamiques, même François Lo Jacomo est jeune et dynamique ! \fg

\item \textit{Martin} : \og Il y a fa, il y a do. Il y a quoi d'autre comme note déjà ? \fg

\item \textit{Alexander} : \og Tout est valorisé. \fg \\
\textit{Pierre-Marie} : \og Rien n'est grave. \fg

\item \textit{Victor} : \og Un carré $4\times 3$ \fg

\item \textit{Victor} : \og Il y a des chauves avec zéro cheveux. \fg

\item \textit{Théo} : \og Alors oui, j'écris très mal… Si vous n'arrivez pas à me déchiffrer, c'est de votre faute. \fg

\item \textit{À un départ d'animatheurs} : \\
\og Au revoir ! \fg \\
\og Je ne pense pas qu'ils nous entendent. \fg \\
\og Je peux crier ! \fg \\
\og Même si tu cries, je ne compte pas sur toi pour les faire revenir. \fg

\item \textit{Emile} : \og Pour le cours, je fais une pause à 16h30. \fg

\item \textit{Eva} : \og Ce serait trop bien d'avoir le compas dans l'œil. \fg

\item \textit{Théo} : \og Tu confonds la division euclidienne avec la division euclidienne \fg

\item \textit{Emma} : \og Je n'ai jamais rien complété de ma vie. \fg

\item \textit{Matthieu Vogel} : \og Vous allez souvent au stage de Bodo ? \fg \\
\textit{Amélie} : \og Oui, à chaque fois. \fg \\
\textit{Matthieu V.} : \og C'est pour ça que je suis nul en maths. \fg

\item \textit{Théo, durant la réunion des anims} : \og Ce matin, j'ai fait cours de dormir au groupe moi. \fg

\item \textit{Stéphane} : \og Stanislas, c'est moderne, mais à l'ancienne. \fg

\item \textit{Emma} : \og Je vois le monde en couleur, et ça fait mal à la tête. \fg

\item \textit{Théo} : \og Mais oui c'est possible avec la carte kiwi. \fg

\item \textit{Raphaël} : \og Tiens, il y a un tournage en bas. \fg \\
Tout le monde regarde vers la fenêtre. \\
\textit{Raphaël, d'un air dépité} : \og Okay… vous pouvez aller voir deux minutes. \fg \\
Tout le monde se précipite vers la fenêtre.



\end{itemize}