Dans ce TD, nous avons évoqué la notion de degré d'un polynôme, de division euclidienne et de racines. Voir \url{https://maths-olympiques.fr/wp-content/uploads/2019/02/stage_ete_2018.pdf} à partir de la page $176$, ou \url{http://maths-olympiques.fr/wp-content/uploads/2017/09/polynomes.pdf}.


\subsubsection{Degré}


\begin{exo}
Déterminer tous les polynômes réels $P$ vérifiant $P(2X) = 2P(X)$.
\end{exo}


\begin{exo}
Déterminer tous les polynômes $P$ réels tels que $P(X^2) = (X^2 + 1)P(X)$
\end{exo}


\begin{exo}
Soit $P_1,\dots, P_{2021}$ des polynômes réels non constants tels que $P_1\circ P_2 = P_2\circ P_3 = \dots = \circ P_{2021}P_1$. Montrer que $\deg P_1 = \deg P_2 = \dots = \deg P_{2021}$
\end{exo}


\begin{exo}
Déterminer tous les polynômes réels $P$ vérifiant, pour tous réels $a, b, c$, $P(a + b-2c) + P(b + c - 2a) + P(a + c - 2b) = 3P(a - b) + 3P(b - c) + 3P(c - a)$.
\end{exo}


\subsubsection{Division euclidienne}


\begin{exo}
Soit $P$ un polynôme réel, $a$ un réel. Quel est le reste de la division euclidienne de $P$ par $(X - a)$ ?
\end{exo}


\begin{exo}
Soit $P$ un polynôme réel tel que $P(2) = 5$ et $P(1) = 3$. Déterminer le reste de la division euclidienne de $P$ par $(X - 2)(X - 1)$.
\end{exo}


\begin{exo}
Soit $n\ge 2$, déterminer le reste de la division euclidienne de $X^n$ par $X^2 - 4X + 3$.
\end{exo}


\begin{exo}
Soit $P$ et $Q$ deux polynômes à coefficients entiers, avec $P$ non constant et unitaire. On suppose que $P(n)$ divise $Q(n)$ pour une infinité d'entiers $n$, montrer que $P$ divise $Q$ dans $\Z[X]$.
\end{exo}


\subsubsection{Racines}


\begin{exo}
Montrer que $|x|$ n'est pas un polynôme.
\end{exo}


\begin{exo}
Montrer qu'il n'existe pas de polynôme réel $P$ tel que $P(n) = \sqrt[3]{n^2 + 1}$ pour tout entier $n$ positif.
\end{exo}


\begin{exo}
Soit $P$ un polynôme réel de degré au plus $n - 1$, vérifiant $P(k) = \frac{1}{k}$ pour tout entier $k$ vérifiant $1\le k \le n + 1$. Calculer $P(0)$.
\end{exo}


\begin{exo}
Déterminer tous les polynômes réels $P$ tels que $P(0) = 0$ et $P(X^2 + 1) = P(X)^2 + 1$.
\end{exo}


\begin{exo}
Déterminer tous les entiers positifs $n$ pour lesquels il existe $k_1,\dots, k_n$ $n$ entiers deux à deux distincts, et un polynôme $P$ de degré au plus $n$ tel que $P(k_i) = n$ pour tout entier $i$ vérifiant $1\le i \le n$, et admettant une racine dans $\Z$.
\end{exo}


\subsubsection{Solutions}


\begin{sol}
Supposons $P\neq 0$ solution de l'équation, notons $c\neq 0$ son coefficient dominant. Le coefficient dominant de $2P(X)$ est $2c$, celui de $P(2X)$ est $2^nc$. Ainsi $2^nc = 2c$, donc $2^n = 2$, donc $n = 1$. $P$ est alors de la forme $aX + b$ pour $a$ et $b$ réels.

En évaluant en $0$, on obtient que $P(0) = 2P(0)$ donc $P(0) = 0$, donc $b = 0$. Ainsi, dans tous les cas, $P$ est de la forme $aX$ pour $a\in \R$.

Réciproquement tout polynôme de la forme $aX$ pour $a\in \R$ vérifie $P(2X) = a\times 2X = 2aX = 2P(X)$ est bien solution. Donc les polynômes solutions sont ceux de la forme $aX$ pour $a\in \R$.
\end{sol}


\begin{sol}
Supposons $P$ non nul et solution de l'équation, notons $n$ son degré. Le degré de $P(X^2)$ est $2n$, et celui de $(X^2 + 1)P(X)$ est $2 + n$, donc $2n = n + 2$, ce qui donne $n + 2$. On pose alors $P = aX^2 + bX + c$ avec $a, b, c$ réels.

L'équation se réécrit $aX^4 + bX^2 + c = (aX^2 + bX + c)(X^2 + 1) = aX^4 + bX^3 + (a + c)X^2 + bX + c$, par identification l'équation implique $5$ équations suivante : $a = a$, $b = 0$, $a + c = b$, $b = 0$, $c = c$, donc équivalent à $b = 0$ et $c = -a$, ainsi dans tous les cas $P$ est de la forme $a(X^2 - 1)$ avec $a\in \R$.

Vérifions : si $P$ est de la forme $a(X^2 - 1)$, $P(X^2) = a(X^4 - 1) = a(X^2 + 1)(X^2 - 1) = a(X^2 + 1)P(X)$. Ainsi les polynômes solutions sont les polynômes de la forme $a(X^2 - 1)$ avec $a\in \R$.
\end{sol}


\begin{sol}
Comme le degré de $P\circ Q$ vaut $\deg P \deg Q$, on obtient que $\deg(P_1)\deg(P_2) = \deg(P_2)\deg(P_3) = \dots = \deg(P_{2020})\deg(P_{2021}) = \deg(P_{2021})\deg(P_1) = \deg(P_1)\deg(P_2)$.

On en déduit que $\deg(P_1) = \deg(P_3) = \deg(P_5) = \dots = \deg(P_{2021}) = \deg(P_2) = \deg(P_4) = \dots = \deg(P_{2020})$ d'où le résultat voulu.
\end{sol}


\begin{sol}
Comme dans une équation fonctionnelle, on essaie des valeurs particulières de $a, b, c$. Pour $a = b = c = 0$, on obtient $3P(0) = 9P(0)$ donc $P(0) = 0$.

Pour $b = c = 0$, on obtient $P(a) + P(-2a) + P(a) = 3P(a) + 3P(-a)$ pour tout $a$ réel, donc par rigidité $P(X) + P(-2X) + P(X) = 3P(X) + 3P(-X)$. Supposons $P$ différent de $0$, soit $c$ son coefficient dominant et $n$ son degré. Le terme en $X^N$ de $P(X) + P(-2X) + P(X)$ vaut $c\times (2 + (-2)^n)$ celui de $3P(X) + 3P(-X)$ vaut $3c(1 + (-1)^n)$. On obtient alors que $2 + (-2)^n = 1 + (-1)^n$, donc $(-2)^n = -1 + 3(-1)^n$, donc $(-2)^n$ vaut $4$ ou $-2$. Ainsi $n = 1$ ou $2$.

Dans tous les cas, $P$ est de la forme $aX^2 + bX$ avec $a,b$ des réels.\smallskip

Pour la vérification, notons que l'ensemble des solutions de l'équation est stable par addition et multiplication par un scalaire. En particulier, si on montre que $X^2$ et $X$ vérifient l'équation, alors tout polynôme de la forme $aX^2 + bX$ seront aussi solution. La vérification pour $X$ donne $0 = 0$ qui est vrai. Pour $X^2$, en développant à gauche on obtient $6(a^2 + b^2 + c^2) - 6(ab + bc + ca)$, et en développant à droite on obtient la même chose, ainsi $X$ et $X^2$ vérifient l'équation.

L'ensemble des solutions est l'ensemble des polynômes de la forme $aX^2 + bX$ pour $a,b$ des réels.
\end{sol}


\begin{sol}
Le reste de la division euclidienne de $P$ par $(X - a)$ est de degré au plus $0$, c'est donc une constante notée $C$. Il existe ainsi $Q$ un polynôme à coefficient réel tel que $P(X) = Q(X)(X - a) + C$. En évaluant en $a$, on obtient $C = P(a)$. Ainsi le reste de la division euclidienne est $P(a)$.
\end{sol}


\begin{sol}
Soit $R$ le reste de la division euclidienne de $P$ par $(X - 2)(X - 1)$. Il existe un polynôme $Q$ tel que $P(X) = Q(X)(X - 1)(X - 2) + R(X)$. De plus le degré de $R$ est strictement inférieur au degré de $(X - 2)(X - 1)$ donc à $2$. On peut ainsi écrire $R$ de la forme $aX + b$.

En évaluant $P(X) = Q(X)(X - 1)(X - 2) + R(X)$ en $1$ et en $2$, on obtient $5 = R(2) = 2a + b$ et $3 = R(1) = a + b$. En soustrayant, on obtient $a = 2$, puis de $a + b = 3$, on déduit $b = 1$. Ainsi le reste vaut $2X + 1$.
\end{sol}


\begin{sol}
Notons que $X^2 - 4X + 3 = (X - 1)(X - 3)$. Notons $Q_n$ et $R_n$ le quotient et le reste de $X^n$ par $(X - 1)(X - 3)$, on a donc $X^n = Q_n(X - 1)(X - 3) + R_n$. En évaluant en $1$ et $3$, on a $R_n(1) = 1^n = 1$ et $R_n(3) = 3^n$.

De plus le degré de $R_n$ est strictement inférieur au degré de $(X - 3)(X - 1)$ donc à $2$. On peut ainsi écrire $R_n$ de la forme $a_nX + b_n$. On a donc $3a_n + b_n = 3^n$ et $a_n + b_n = 1$, en faisant la différence on obtient que $a_n = \frac{3^n - 1}{2}$, puis $b_n = 1 - a_n = \frac{3 - 3^n}{2}$, donc $R_n = \frac{3^n - 1}{2}X + \frac{3 - 3^n}{2}$.
\end{sol}


\begin{sol}
Comme $P$ est unitaire, on peut effectuer la division euclidienne de $Q$ par $P$ dans $\Z[X]$, on a $Q = AP + B$ pour $A$ et $B$ deux polynômes à coefficients entiers, avec $\deg B<\deg A$.

Or pour une infinité d'entiers $n$, $P(n)$ divise $Q(n)$, donc $P(n)$ divise $R(n) = Q(n) - A(n)P(n)$. Or comme $\deg P > \deg Q$ et $P$ est non constant, cela signifie que si $n$ est assez grand en valeur absolue, $|P(n)|>|R(n)|$. En particulier, il y a un nombre fini de $n$ tels que $|P(n)|\le |R(n)|$. Comme il y a une infinité d'entiers $n$ tels que $P(n)$ divise $R(n)$, il y en a une infinité qui vérifient également $|P(n)|>|R(n)|$, donc $R(n) = 0$. $R$ a ainsi une infinité de racines, donc $R = 0$, donc $Q = AP$ : $P$ divise bien $Q$ dans $\Z[X]$.
\end{sol}


\begin{sol}
Supposons qu'il existe un polynôme $P$ tel que $P(x) = |x|$ pour tout $x$ réel. Pour tout $x$ réel positif, $P(x) = x$, donc $P(x) - x = 0$. En particulier le polynôme $P(X) - X$ a une infinité de racines, donc $P(X) - X = 0$, i.e. $P(X) = X$. En évaluant en $-1$, $-1 = P(-1) = |-1| = 1$, contradiction. Ainsi $x\mapsto |x|$ n'est pas un polynôme.
\end{sol}


\begin{sol}
Supposons qu'il existe un polynôme réel $P$ tel que $P(n) = \sqrt[3]{n^2 + 1}$ pour tout entier $n$ positif. On en déduit que pour tout entier $n$ positif, $P(n)^3 = n^2°1$. En particulier, le polynôme $P(X)^3 - X^2 - 1$ a une infinité de racines, donc $P(X)^3 = X^2 + 1$. En prenant le degré, $3\deg P = 2$, ce qui est impossible. Il n'existe pas de polynôme $P$ vérifiant l'énoncé.
\end{sol}


\begin{sol}
Notons pour la culture qu'un tel polynôme existe par interpolation de Lagrange.

L'hypothèse nous donne que $kP(k) - 1 = 0$ pour tout entier $k$ entre $1$ et $n$. Ainsi le polynôme $XP(X) - 1$ a $1,2,\dots, n$ comme racine il s'écrit donc de la forme $XP(X) - 1 = (X - 1)\dots (X - n)Q(X)$ avec $Q$ un polynôme. Le degré de $XP(X) - 1$ étant au plus $n$, celui de $Q$ est au plus $0$, donc $Q$ est constant.

En évaluant en $0$, $(-1)\times \dots\times (-n)Q(0) = -1$.

En particulier comme $Q$ est constant $(-1)P(-1) - 1 = (-2)\times\dots \times (-(n + 1))Q(0) = -(n + 1)$

Ainsi, $P(-1) = n$.
\end{sol}


\begin{sol}
Soit $P$ vérifiant l'équation. En évaluant en $0$, $P(1) = 1$. En évaluant en $1$, $P(2) = 2$. En évaluant en $2$, $P(5) = 5$. Il semble qu'en itérant, on obtienne pleins d'entiers $n$ tels que $P(n) = n$.

Définissons la suite suivant par récurrence : on pose $u_0 = 0$ et $u_{n + 1} = u_n^2 + 1$. On montre facilement avec l'équation par récurrence sur $n$ que $P(u_n) = u_n$, que $u_n\in \N$ pour tout entier positif $n$. De plus $u_{n + 1}>u_n^2\ge u_n$ car $u_n(u_n - 1)\ge 0$ comme $u_n$ est un entier positif.

Ainsi la suite $(u_n)$ est strictement croissante, donc contient une infinité de termes qui sont racines de $P(X) - X$. Ainsi $P(X) - X$ est le polynôme nul, donc $P(X) = X$.

Réciproquement si $P(X) = X$, on a bien $P(0) = 0$ et $P(X^2 + 1) = X^2 + 1 = P(X)^2 + 1$, c'est donc l'unique solution de cette équation.
\end{sol}


\begin{sol}
Soit $Q(X) = P(X) - n$. Puisque pour tout indice $i$ compris entre $1$ et $n$, $Q(k_i) = P(k_i) - n = n - n = 0$ donc $Q$ a $n$ racine et est de degré $n$. Ainsi il existe $c$ tel que
$$Q(X) = c(X - k_1)(X - k_2)\cdots (X - k_n)$$ avec $c$ entier car $P$ est à coefficients entiers.
Or d'une part il existe $z$ entier tel que $P(z) = 0$, donc $Q(z) = P(z) - n =  - n$ et d'autre part $Q(0) = c(z - k_1)\cdot(z - k_2)\cdots (z - k_n)$
Comme les $z - k_i$ sont deux à deux distincts et non nuls, au plus deux des $z - k_i$ sont de valeur absolue égale à $1$, donc au moins $n - 2$ des $k_i$ sont de valeur absolue supérieure ou égale à $2$. Ainsi $\mid c(z - k_1)\cdot(z - k_2)\cdots (z - k_n)\mid \ge 1\cdot 1\cdot 2\cdots 2 = 2^{n - 2}$. Ainsi $n\ge 2^{n - 2}$.

Montrons par récurrence que si $m\ge 5$ est un entier, $2^{m - 2}>m + 2$. Si $m = 5$, on a bien $5<2^3 = 8$. On suppose que $2^{m - 2}>m$ et on souhaite montrer que $2^{m + 1 - 2}>m + 1$.
Or $2^{m - 1} = 2\cdot 2^{m - 2}>2m>m + 1$ en utilisant l'hypothèse de récurrence. Ceci achève la récurrence.

De cette discussion on déduit que $n\le 4$.
Or si $n = 1$, le polynôme $X$ satisfait la condition (avec $k_1 = 1$). Si $n = 2$, le polynôme $-(X - 1)(X - 2) + 2$ satisfait la condition (avec $k_1 = 1$ et $k_2 = 2$). Si $n = 3$, le polynôme $-(X - 1)(X + 1)(X - 3) + 3$ satisfait la condition (avec $k_1 = 1, k_2 = -1, k_3 = 3$). Enfin, si $n = 4$, le polynôme $-(X - 1)(X + 1)(X + 2)(X - 2) + 4$ satisfait la condition (avec $k_1 = 1, k_2 = -1,k_3 = -2,k_4 = 2$).

Les solutions sont donc les entiers $1$, $2$, $3$ et $4$.
\end{sol}
