%Matthieu Bouyer, Auguste de Lambilly

\subsubsection{Premiers exercices}


\begin{exo}
Trouver tous les entiers $n$ strictement positifs tels que $1!+2!+\ldots+n!$ soit un carré parfait.
\end{exo}

\begin{exo}
Trouver tous les $(x,y)$ entiers positifs tels que $x^2=y^2+7y+6$.
\end{exo}

\begin{exo}
Trouver tous les couples d'entiers $(x,y)$ tels que $x^2=y^5+7$.

(Indication : Regarder modulo un nombre premier $p$ à conjecturer avec Fermat.)
\end{exo}

\begin{exo}
Montrer que si $n\ge2$ divise $2^n+1,$ alors $n$ est un multiple de $3$.
\end{exo}

\begin{exo}
Trouver tous les entiers $n$ tels que $n^3-3n^2+n+2$ soit une puissance de $5$.
\end{exo}

\subsubsection{Encore plus d'exercices}


\begin{exo}
Trouver tous les triplets d’entiers naturels $(x,y,z)$ tels que :
$x^2+y^2=3\cdot2016^z+77$.
\end{exo}

\begin{exo}
Trouver tous les couples d'entiers positifs $(a,b)$ tels que $|3^a-2^b|=1$
\end{exo}


\begin{exo}
Trouver tous les couples de nombres premiers $(p,q)$ tels que $p\mid5^q+1$ et $q\mid5^p+1$.
\end{exo}


\begin{exo}
Trouver tous les entiers $x$ tels que $x^4+x^3+x^2+x+1$ est un carré parfait.
\end{exo}


\begin{exo}
Trouver tous les entiers positifs $a,b,c$ tels que $2^a3^b+9=c^2$
\end{exo}


\begin{exo}
Soient $n$ et $m$ deux entiers strictement positifs. Montrer que $5^m + 5^n$ s’écrit comme une
somme de deux carres si et seulement si $n$ et $m$ ont même parité.
\end{exo}



\begin{exo} Si $p$ est un nombre premier, montrer que $7p+3^p-4$ n'est pas un carré.
\end{exo}


\subsubsection{Plus exotique…}


\begin{exo} Soient $a_1,\dots,a_n \in \{-1,1\}$ tels que : $a_1a_2+a_2a_3+\dots+a_{n-1}a_n+a_na_1=0$.
Montrer que $4\mid n$.
\end{exo}


\begin{exo}
Soit $n\ge1$ un entier. Montrer qu'il existe un multiple de $n$ dont la somme des chiffres vaut $n$.
\end{exo}


\subsubsection{Solutions}


\begin{sol} Si $n\ge5$, alors $ 1!+2!+\ldots+n!\equiv 1+2+6+24\equiv 3 [5]$.
Or, modulo $5$, on a :
\begin{center}
$\begin{array}{|r|c|c|c|c|l|}
\hline
x & 0 & 1 & 2 & 3 & 4 \\
x^2 & 0 & 1 & 4 & 4 & 1 \\
\hline
\end{array}$
\end{center}
Ainsi, $n\le5$. En testant manuellement, on trouve que seuls $n=1$ et $n=3$ conviennent.
\end{sol}


\begin{sol}
Soit $(x,y)$ une éventuelle solution.
Pour $y>3,$ on a $(y+3)^2=y^2+6y+9<y^2+7y+6=x^2<y^2+8y+16=(y+4)^2$, absurde.
Donc $y\in`\{0;1;2.3\}$. En testant les trois cas, seul $y=3$ donne un carré parfait ($x=6$).

La seule solution est donc $(6,3)$.
\end{sol}


\begin{sol}
On travaille modulo $p=2\cdot5+1=11$ qui est premier et on a :
\begin{center}
$\begin{array}{|r|c|c|c|c|c|c|c|c|c|c|l|}
\hline
x & 0 & 1 & 2 & 3 & 4 & 5 & 6 & 7 & 8 & 9 & 10 \\
x^2 & 0 & 1 & 4 & 9 & 5 & 3 & 3 & 5 & 9 & 4 & 1 \\
x^5 & 0 & 1 & 10 & 1 & 1 & 1 & 10 & 10 & 10 & 1 & 10 \\
\hline
\end{array}$
\end{center}

Ainsi, il n'y a aucune solution possible.

\medskip

\textit{Explication :} Après s'être convaincu qu'on ne trouverait pas de solutuion, on choisit $p$ premier de manière à limiter le nombre de valeurs que prennent les restes des $x^2$ et des $x^5$ modulo $p$.

On remarque plus généralement que si $q$ premier ne divise pas $p-1$, alors : $$\forall x,y\in\mathbb Z,~x^q\equiv y^q[p]\Longrightarrow x\equiv y[p]$$ donc $x\mapsto x^q$ est injectif, donc bijectif dans $\mathbb Z/p\mathbb Z$ ! C'est pourquoi on choisit $p$ tel que $p-1$ est divisible par $2$ et $5$, le plus petit possible pour limiter les cas à tester.
\end{sol}


\begin{sol}
Comme $n\ge2$, considérons un de ses diviseurs premiers $p$. Alors $2^n\equiv-1[p]$ donc $2^{2n}\equiv1[p]$.
On en déduit que $p$ divise $2^{2n}-1\wedge2^{p-1}-1$ par petit Fermat.
Un lemme classique (quon va redémontrer plus tard) garantit que $p$ divise $2^{(2n)\wedge (p-1)}-1$.
Supposons alors que $p$ est le plus petit diviseur premier de $n$, de sorte que $p-1\wedge n=1$. Alors $p\mid2^2-1=3$ d'où $p=3$.
\end{sol}



\begin{sol} On a alors : $(n-2)(n^2-n-1)=5^a$. Ainsi, on a : $n-2=5^x$ et $n^2-n-1=5^y$. Il suit $5^y-5^{2x}-3\cdot5^x=1$. Donc, si $x,y\ge1$ alors $0\equiv1[5]$, ce qui est absurde. Donc $x=0$ ou $y=0$.
Si $x=0$, alors $y=1$, ce qui donne $n=3$.
Si $y=0$, alors $5^{2x}+3\cdot5^x=0$, ce qui est absurde.
La seule solution est ainsi $n=3$.
\end{sol}



\begin{sol}
On suppose $z\ge1$, comme $7\mid 2016$, $x^2+y^2\equiv0[7]$.
Or, modulo $7$, on a :
\begin{center}
$\begin{array}{|r|c|c|c|c|c|c|l|}
\hline
x & 0 & 1 & 2 & 3 & 4 & 5 & 6 \\
x^2 & 0 & 1 & 4 & 2 & 2 & 4 & 1 \\
\hline
\end{array}$
\end{center}

Ainsi, le seul cas possible est $x\equiv y \equiv 0[7]$. On a alors $x=7a,y=7b$. Il suit : $7(a^2+b^2)=11+3\cdot288\cdot 2016^{z-1}$. Si, $z\ge2$, alors la partie de droite ne serait pas divisible par $7$. Donc, $z=1$. On a alors : $7(a^2+b^2)=11+3\cdot288=875$, donc $a^2+b^2=125$, donc nécessairement, $a,b<12$.
On trouve manuellement : $\{a,b\}\in\{\{11,2\},\{10,5\} \}$.
Ce qui donne $(x,y,z)\in\{(77,14,1),(14,77,1),(70,35,1),(35,70,1)\}$.
Sinon, on a $z=0$, et alors $x^2+y^2=80$, donc $x,y<9$. On teste manuellement et on obtient comme autres solutions : $(x,y,z)\in\{(8,4,0),(4,8,0)\}$.
\end{sol}


\begin{sol}
On fait une disjonction de cas.

Premier cas :
$3^a-2^b=1$. On a donc : $-(-1)^b\equiv 1 [3]$, d'où $b\equiv 1 [2]$.
Ainsi, avec $b=2c+1$, $3^a-2\cdot4^c=1$. Si $c\ge1$, alors $3^a\equiv 1 [8]$, donc $a=2d$. Il suit : $3^{2d}-2^{2b+1}=1$.
Ainsi : $(3^d-1)(3^d+1)=2^{2b+1}$.
Il suit qu'il existe $x,y$ tels que $x+y=2b+1$ et $3^d-1=2^x$ et $3^d+1=2^y$. Donc $2^x+2=2^y$. Si $x\ge2$, alors $2^y\equiv2[4]$ mais $2^y\ge4$, ce qui est absurde. D'où $x\in\{0,1\}$.
Un rapide calcul mène à $x=1$ et $y=2$, il suit $b=1$ et $d=1$ soit : $(a,b)=(2,3)$.
Sinon, $c=0$, et alors, $3^a=3$, d'où $a=1$. Donc $(a,b)=(1,1)$ est une autre solution.

Deuxième cas :
$2^b-3^a=1$. On a donc, si $a\ge1$, $(-1)^b\equiv 1 [3]$, donc $b=2c$, ainsi, $2^{2c}-3^a=1$. Donc $(2^c+1)(2^c-1)=3^a$. Donc $2^c-1=3^x$ et $2^c+1=3^y$ avec $x+y=a$. Il suit, $3^x+2=3^y$. Si $y\ge1$, alors $x=0$, donc $y=1$. Sinon, $y=0$, ce qui est impossible.
Donc, on a $(a,b)=(1,2)$.
Dans le dernier cas, $a=0$, et ainsi, $b=1$ : on a comme autre solution $(a,b)=(0,1)$
\end{sol}



\begin{sol}
On se donne une solution $(p,q)$ et on suppose $q\le p$ sans perte de généralité. Alors $5^p\equiv-1[q]$ donc $5^{2p}\equiv1[q]$ : l'ordre $\omega$ de $5$ modulo $q$ divise donc $2p$.

$\bullet$ Si $\omega=1$, alors $q=2$, donc $p\mid5^2+1=26$ : $p\in\{2;13\}.$

$\bullet$ Si $\omega=2$, alors $q\mid5^2-1$ mais pas $5-1$ donc $q=3$, puis $p\mid126$ donc $p\in\{3;7\}$.

$\bullet$ Si $\omega=p$ ou $\omega=2p$, par le petit théorème de Fermat, $p\mid q-1$ (ou même $2p\mid q-1$) donc $p\le q-1<q$, absurde.

On vérifie aisément que toutes ces solutions conviennent ainsi que leur permutation.
\end{sol}


\begin{sol}
On se donne une éventuelle solution $x$. Encadrons alors $x^4+x^3+x^2+x+1$ par deux carrés consécutifs comme à l'exercice $2$.
On remarque que $A=4(x^4+x^3+x^2+x+1)$ est encore un carré parfait et :
$$(2x^2+x)^2=4x^4+4x^3+x^2<A<4x^4+4x^3+5x^2+2x+1=(2x^2+x+1)^2$$
dès que $3x^2+4x+4=2x^2+(x+2)^2>0$ (toujours vrai) et $x^2-2x-3=(x-3)(x+1)>0$.

Nécessairement on a donc $x\in\{-1;0;1;2;3\}$.

En testant les $5$ cas à la main, on obtient comme seules solutions $-1,0,3$.
\end{sol}


\begin{sol}
On a donc $2^a3^b=(c+3)(c-3)$.
D'où, il existe $x,y,u,v$ tels que $x+u=a, y+v=b, c-3=2^x3^y$ et $c+3=2^u3^v$. Ainsi, $2^x3^y+6=2^u3^v$.
Si $y\ge2$, alors $2^u3^v\equiv6[9]$, donc $v=1$.
Si $x\ge2$, alors $2^u3^v\equiv2[4]$, donc $u=1$.
Donc, si $x,y\ge2$, alors $u=v=1$, d'où $2^x3^y=0$ : absurde.
Ainsi, on a nécessairement $y\le1$ ou $x\le1$.
En testant les derniers cas restants (et en utilisant l'exercice $7$), on trouve comme solutions : $(a,b,c)\in \{(4,0,5),(4,5,51),(3,3,15),(4,3,21),(3,2,9)\}$
\end{sol}


\begin{sol}
Cf envoi d'arithmétique 2012-2013, exercice 5, page 6 :

\url{https://maths-olympiques.fr/wp-content/uploads/2017/10/ofm-2012-2013-envoi3-corrige.pdf}
\end{sol}


\begin{sol}
Ce n'est déjà pas le cas pour $p=2$. Supposons dorénavant $p$ impair.

On raisonne par l'absurde en supposant qu'il existe $a\in\Z$ tel que $7p+3^p-4=a^2$.
Alors $a^2\equiv0$ ou $1$ modulo $4$ et $7p+3^p-4\equiv-p+(-1)^p\equiv-p-1$.
Comme $p$ est impair, $p\equiv-1[4]$.

Mais alors en regardant modulo $p$ par petit Fermat : $a^2\equiv0+3-4\equiv-1[p]$.
Donc avec le petit théorème de Fermat, comme clairement $p$ ne divise pas $a$ : $$1\equiv a^{p-1}\equiv\big(a^2\big)^{\frac{p-1}2}\equiv(-1)^{\frac{p-1}2}\equiv-1[p]$$
car $p-1$ est divisible par $2$ mais pas par $4$. C'est absurde.
\end{sol}


\begin{sol}
On pose $a_{n+1}=a_1$.
En regardant modulo $2$, on a $$n\equiv a_1a_2+a_2a_3+\dots+a_{n-1}a_n+a_na_1\equiv 0[2]$$
On écrit $n=2m$. Alors exactement $m$ des $a_ia_{i+1}$ valent $1$ et exactement $m$ valent $-1$.
Comme leur produit vaut le produit des $a_i$ au carré (chacun étant présent deux fois) donc $1$, on en déduit que $1=1^m\cdot(-1)^m$ donc $m$ est pair donc $4\mid n$.
\end{sol}


\begin{sol}
On considère la suite des $(10^k)_{k\ge0}$. Par le principe des tiroirs, l'un des résidus $r$ modulo $n$ est pris par une infinité de termes de la suite : notons-les $(10^{a_k})_{k\ge0}$. On a alors que : $$m=10^{a_0}+10^{a_1}+\dots+10^{a_{n-1}}\equiv n\cdot r\equiv0[n]$$
Donc $n\mid m$ et $m$ s'écrit avec $n$ chiffres $1$ et que des $0$.
\end{sol}
