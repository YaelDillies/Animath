\soustitre{Sujet}


\begin{exo}
Soit $ABC$ un triangle non isocèle en $A$, $\Gamma_1$ un cercle passant par les points $B$ et $C$ et dont le centre $O$ se trouve sur la bissectrice de l'angle $\widehat{BAC}$. Soit $\Gamma_2$ un cercle passant par les points $O$ et $A$. On note $P$ et $Q$ les points d'intersection des cercles $\Gamma_1$ et $\Gamma_2$. Soit $X$ le point d'intersection des droites $(PQ)$ et $(AO)$. Montrer que le point $X$ appartient au segment $[BC]$.
\end{exo}


\begin{exo}
Soit $P$ un polynôme à coefficients entiers. On suppose qu'il existe quatre entiers $a, b, c$ et $d$ distincts tels que
$$P(a) = P(b) = P(c) = P(d) = 4$$
\noindent Montrer qu'il n'existe pas d'entier $e$ tel que $P(e) = 7$. 
\end{exo}


\begin{exo}
Soit $ABC$ un triangle et soient $P$ et $Q$ des points respectivement sur les segments $[AB]$ et $[AC]$ tels que $BP=CQ$. Les droites $(BQ)$ et $(CP)$ se coupent au point $R$. Les cercles circonscrits aux triangles $BPR$ et $CQR$ se recoupent au point $S$. Montrer que le point $S$ appartient à la bissectrice de l'angle $\widehat{BAC}$.
\end{exo}

\begin{exo}
Déterminer toutes les fonctions $f : \R \rightarrow \R$ qui vérifient l'équation
$$f\left(x^2+xy+f(y^2)\right)=xf(y)+x^2+f(y^2)$$
pour tous réels $x$ et $y$.
\end{exo}


\soustitre{Corrigé}


\begin{sol}
\begin{center}
\begin{tikzpicture}[scale=1]
\tkzInit[ymin=-5,ymax=7,xmin=-4,xmax=4]
\tkzClip 

\tkzDefPoint(2,6){A}
\tkzDefPoint(3,0){B}
\tkzDefPoint(-3,0){C}

\tkzDefMidPoint(B,C) \tkzGetPoint{M}
\tkzDefCircle[in](A,B,C) \tkzGetPoint{I}
\tkzDefLine[perpendicular=through M](B,C) \tkzGetPoint{o}
\tkzInterLL(A,I)(M,o) \tkzGetPoint{O}
\tkzDefMidPoint(O,A) \tkzGetPoint{L}
\tkzDefLine[perpendicular=through L](O,A) \tkzGetPoint{x}
\tkzDefLine[parallel=through A](B,C) \tkzGetPoint{y}
\tkzInterLL(L,x)(A,y) \tkzGetPoint{z}
\tkzDefMidPoint(z,L) \tkzGetPoint{o'}
\tkzInterCC(o',A)(O,B) \tkzGetPoints{P}{Q}
\tkzInterLL(A,O)(P,Q) \tkzGetPoint{X}
\tkzDefCircle[circum](A,B,C) \tkzGetPoint{U}

\tkzDrawSegment(A,B)
\tkzDrawSegment(B,C)
\tkzDrawSegment(C,A)
\tkzDrawSegment(A,O)
\tkzDrawCircle(o',A)
\tkzDrawCircle(O,B)
\tkzDrawCircle(U,B)
\tkzDrawSegment(P,Q)
\tkzDrawPoints[fill=white,color=black](A,B,C,O,X,P,Q)

\tkzLabelPoint(A){$A$}
\tkzLabelPoint(B){$B$}
\tkzLabelPoint[below left](C){$C$}
\tkzLabelPoint(O){$O$}
\tkzLabelPoint(X){$X$}
\tkzLabelPoint[right](P){$P$}
\tkzLabelPoint[below](Q){$Q$}
\end{tikzpicture}
\end{center}

La définition du point $O$ interpelle, il s'agit du pôle sud du point $A$ dans le triangle $ABC$, il appartient donc à son cercle circonscrit.

On dispose désormais de trois cercles, dont les axes radicaux sont les droites $(PQ), (BC)$ et $(AO)$, elles sont donc concourantes au point $X$ qui appartient donc au segment $[BC]$.
\end{sol}


\begin{sol}
Tout d'abord, l'hypothèse se traduit par le fait que les entiers $a,b,c$ et $d$ sont racines distinctes du polynôme $P(X)-4$. On peut donc factoriser $P$ : on dispose d'un polynôme $Q$ à coefficients entiers tel que 

\[P(X)-4= (X-a)(X-b)(X-c)(X-d)Q(X)\]

Supposons ensuite par l'absurde qu'il existe un entier $e$ tel que $P(e)=7$. On peut alors écrire
$$7=P(e)=(e-a)(e-b)(e-c)(e-d)Q(e) +4$$
et donc
$$3=(e-a)(e-b)(e-c)(e-d)Q(e)$$

Comme tous les termes du membre de droite de l'égalité sont des entiers, les nombres $e-a,e-b,e-c$ et $e-d$ sont des diviseurs de $3$. Comme les entiers $a,b,c$ et $d$ sont distincts, les entiers $e-a,e-b,e-c$ et $e-d$ sont également des diviseurs distincts de $3$. Les seuls diviseurs entiers de $3$ étant $-3,-1,1$ et $3$, on a forcément $\{e-a,e-b,e-c,e-d\}=\{-3,-1,1,3\}$. Mais comme $(e-a)(e-b)(e-c)(e-d)$ divise $3$ et vaut $9$, et on obtient bien une contradiction.
\end{sol}


\begin{sol}
\begin{center}
\begin{tikzpicture}
[scale=1]
\tkzInit[ymin=-2,ymax=7,xmin=-5,xmax=5]
\tkzClip

\tkzDefPoint(2,5){A}
\tkzDefPoint(3,0){B}
\tkzDefPoint(-3,0){C}
\tkzDefMidPoint(B,C) \tkzGetPoint{M}
\tkzDefLine[perpendicular=through M](B,C) \tkzGetPoint{n}
\tkzDefCircle[in](A,B,C) \tkzGetPoint{I}
\tkzDefLine[perpendicular=through A](A,I) \tkzGetPoint{i}
\tkzInterLL(M,n)(A,i) \tkzGetPoint{N}
\tkzDefPoint(2.7,1.5){P}
\tkzDefCircle[circum](N,A,P) \tkzGetPoint{O'}
\tkzDefCircle[circum](A,B,C) \tkzGetPoint{O}
\tkzInterLC(A,C)(O',A) \tkzGetPoints{Q}{A}
\tkzInterLL(B,Q)(C,P) \tkzGetPoint{R}
\tkzDefCircle[circum](B,P,R) \tkzGetPoint{O1}
\tkzDefCircle[circum](C,Q,R) \tkzGetPoint{O2}
\tkzInterCC(O1,P)(O2,Q) \tkzGetPoints{S}{R}

\tkzMarkSegment[color=blue,mark=s||](Q,C)
\tkzMarkSegment[color=blue,mark=s||](P,B)
\tkzDrawSegment(A,B)
\tkzDrawSegment(B,C)
\tkzDrawSegment(C,A)
\tkzDrawSegment(B,Q)
\tkzDrawSegment(C,P)
\tkzDrawLine[dashed](A,S)
\tkzDrawCircle(O,A)
\tkzDrawCircle(O1,B)
\tkzDrawCircle(O2,C)
%\tkzDrawCircle[dashed](O',A)
\tkzDrawPoints[fill=white,color=black](A,B,C,P,Q,R,S)

\tkzLabelPoint[above](A){$A$}
\tkzLabelPoint(B){$B$}
\tkzLabelPoint[below left](C){$C$}
\tkzLabelPoint(P){$P$}
\tkzLabelPoint[below right](S){$S$}
\tkzLabelPoint[above left](R){$R$}
\tkzLabelPoint[below](Q){$Q$}
\end{tikzpicture}
\end{center}

Le point $S$ est le centre de la similitude envoyant les points $B$ et $P$ sur les points $Q$ et $C$. Les triangles $BSP$ et $QSC$ sont donc semblables. Puisque $PB=QC$, les triangles sont en fait isométriques et $SB=SQ$.

On a de plus
$$\widehat{QSB}=\widehat{QSR}+\widehat{RSB}=\widehat{RCQ}+\widehat{RPA}=180^\circ-\widehat{QAB}$$
donc les points $A,B, S$ et $Q$ sont cocycliques. Le point $S$ est alors le pôle Sud du sommet $A$ dans le triangle $AQB$, le point $S$ appartient donc à la bissectrice de l'angle $\widehat{BAC}$. 
\end{sol}


\begin{sol}
En remplaçant $x$ par $0$, on trouve $f(f(y^2))=f(y^2)$ pour tout $y$ réel. Puis, en remplaçant $x$ par $-y$, on trouve 

\[f(f(y^2))= -yf(y)+y^2+f(y^2)\]

On déduit que $-yf(y)=y^2$, et donc que $f(y)=y$ pour tout $y\neq 0$. 

\medskip

Il reste à déterminer $f(0)$. Pour cela, on remplace $y$ par $0$ pour avoir 

\[f(x^2+f(0))=xf(0)+x^2+f(0)\]

Puis, en remplaçant $x$ par $-x$ on a 

\[f(x^2+f(0))=-xf(0)+x^2+f(0)\]

En faisant la différences de ces deux dernières équations, on trouve $2xf(0)=0$ pour tout réel $x$. Ceci force $f(0)=0$. 

\medskip

On déduit que $f(x)=x$ pour tout réel $x$. Réciproquement, la fonction identité est bien solution de l'équation, et c'est donc la seule.
\end{sol}

%\begin{sol}
%En posant $y=0$, on trouve $f(x^2+f(0))= xf(0)+f(x^2)$. En remplaçant $x$ par $-x$ dans cette équation, on trouve $f(x^2+f(0))=-xf(0)+f(x^2)$, et en combinant on obtient donc $2xf(0)=0$, soit $f(0)=0$. 
%
%\medskip
%
%En posant $x=0$, on trouve $f(f(y^2))=y^2+f(0)=y^2$. En posant $y=-x$, on trouve cette fois-ci que $f(f(y^2))=-yf(y)+y^2+f(y^2)$. On en déduit que 
%
%\[f(y^2)=yf(y)\]
%
%D'autre part, en posant $x=-f(y^2)/y$, on trouve 
%
%\[-\frac{f(y)f(y^2)}{y}+y^2 =0\]
%
%On déduit que si $y$ est non nul, alors $y^3=f(y)f(y^2)=yf(y)^2$, ou encore $f(y)^2=y^2$. Puisque si $y=0$, alors $f(0)=0$, on a donc pour tout $y$ réel que $f(y)=\pm y$.
%
%\medskip
%
%Dans la suite, on suppose qu'il existe $a$ et $b$ tels que $f(a)=a$ et $f(b)=-b$. Notons qu'alors $f(a^2)=a^2$ et $f(b^2)=-b^2$, puisque $f(x^2)=xf(x)$.
%
%En remplaçant $x=a$ et $y=b$ dans l'équation, on trouve 
%
%\[\pm (a^2+ab-b^2)=-ab+b^2+a^2\]
%
%%Cette équation est vraie pour tout couple de réels $a$ et $b$ tels que $f(a)=a$ et $f(b)=-b$. Par principe des tiroirs infinis, on sait qu'il existe une infinité de réels $a$ tels que $f(a)=a$ ou une infinité de réels $b$ tels que $f(b)=-b$. 
%%
%%Dans le premier cas, on fixe $b$ et on fait bouger $a$. Il existe soit une infinité de réels $a$ tels que le signe dans le membre de gauche soit un $+$, soit il en existe une infinité telle que le signe dans le membre de gauche soit un $-$. Dans les deux cas, les deux côtés de l'équation sont des polynômes en $a$, égaux pour une infinité de réels, et donc ils sont égaux. On vérifie que cela implique alors que $b=0$. 
%%
%%Le cas où il existe une infinité de réels $b$ tels que $f(b)=-b$ donne de la même façon que $a=0$. 
%
%Le cas où le signe du $LHS$ est un $+$ donne que $b(a-b)=0$. Si $b\neq 0$, alors $a=b$ ce qui donne $a=f(a)=f(b)=-b=-a$ donc $a=b=0$. Dans tous les cas, $a$ ou $b$ est nul. 
%
%Le cas où le signe du $LHS$ est un $-$ donne que $a^2=0$ donc $a=0$. 
%
%On déduit que les seules solutions sont les fonction $f\equiv x$ et $f\equiv -x$.      
%\end{sol}
%
%%\begin{sol}
%En remplaçant $x$ en $-x$ et $y$ en $-y$. Le membre de gauche est inchangé, ainsi $xf(y)+y^2+f(x^2)=-xf(-y)+y^2+f(x^2)$, ainsi avec $x=1$, $f$ est impaire. En particulier, $f(0)=0$.\\
%On considère $x=0$, on  a : $f(f(y^2))=y^2$, donc $f$ est involutive sur les positifs, par imparité, elle l'est aussi sur les négatifs. Donc $f$ est involutive (et bijective).\\
%On prend $y=-x$, on a $$x^2=f(f(x^2))=-xf(x)+x^2+f(x^2)$$
%donc $f(x^2)=xf(x)$, on change $x$ en $f(x)$ pour avoir $f(f(x)^2)=f(x)f(f(x))=xf(x)$, donc $f(f(x)^2)=f(x^2)$. Ainsi, par injectivité, $f(x)^2=x^2$ et $f(x)=\pm x$.\\
%On suppose qu'il existe $x$ et $y$ non nuls tels que $f(x)=x$ et $f(y)=-y$. Dans l'équation de départ en utilisant $f(x^2)=xf(x)$, on a $$f(x^2+xy-y^2)=-xy+y^2+x^2$$
%Dès lors, $x^2+xy-y^2=-xy+y^2+x^2$ ou $x^2+xy-y^2=xy-y^2-x^2$. Dans le premier cas on a $y=0$ ou $x=y$ et dans le deuxième, $x=0$, ce qui est exclu. Par conséquent $f:x\longmapsto x$ ou $f:x\longmapsto -x$.
%\\
%Ces fonctions conviennent c'est donc les solutions.
%\end{sol}