\begin{rappel}
Une équation fonctionnelle est une équation où l'inconnue à déterminer est une fonction. La plupart du temps on cherche toutes les fonctions qui vérifie une propriété. On utilise alors le raisonnement d'analyse synthèse : (analyse) on suppose que $f$ est une fonction qui réalise l'énoncé on trouve qu'elle nécessairement d'une certaine forme ensuite (synthèse) on vérifie que ces formes conviennent.
\end{rappel}
\begin{rappel}
Pour $f:X\to Y$ on dit que :
\begin{itemize}
    \item $f$ est injective si pour tout $x,y\in X$ : $f(x)=f(y)\implies x=y$
    \item $f$ est surjective si pour tout $y\in Y$, il existe $x\in X$ tel que $f(x)=y$
    \item $f$ est bijective si $f$ est injective et surjective.
    \item $f$ est paire si pour $x\in X$, on a $-x\in X$ et $f(-x)=f(x)$
    \item $f$ est impaire si pour $x\in X$, on a $-x\in X$ et $f(-x)=-f(x)$
\end{itemize}
\end{rappel}
\subsubsection{Petits conseils}
\begin{itemize}
    \item Chercher des valeurs particulières comme $f(0)$, $f(1)$ ... Si on ne les trouve pas c'est peut être des variables qu'on peut alors nommer.
    \item Chercher quelles solutions pourraient fonctionner pour orienter ses recherches.
    \item Utiliser les symétries pour obtenir d'autres équations.
    \item Utiliser les transformations qui changent peu ou pas un des membres de l'équation.
    \item Chercher les points fixes ($f(x)=x$) et les racines ($f(x)=0$).
    \item Il peut être utile de changer de fonction d'étude pour avoir une équation plus simple.
    \item Chercher s'il y a de l'parité, de la surjectivité, de l'injectivité ou de la bijectivité.
    \item Ne rien effacer sur ses brouillons et penser à mettre en valeur ses trouvailles importantes.
\end{itemize}

\begin{exo}
Déterminer dans chaque cas si $f:\R\to\R$ est injective, surjective et bijective :
\begin{itemize}
    \item $f(f(x)-1)=x+1$
    \item $f(y+f(x))=(y-1)f(x^2)+3x$
    \item $f(x+f(y))=f(x)+y^5$
    \item $f(f(x))=\sin x$
    \item $f(x+y^2)=f(x)f(y)+xf(y)-y^3f(x)$
\end{itemize}

\end{exo}
%\begin{exo}
%Trouver toutes les fonctions $f:\R\to\R$ telles que pour tous réels $x$ et $y$ : 
%$$f(x)f(y)+f(x+y)=xy$$
%\end{exo}
\begin{exo}
Trouver toutes les fonctions $f:\R\to\R$ telles que pour tous réels $x$ et $y$ : 
$$f(2f(x)+f(y))=2x+f(y)$$
\end{exo}
\begin{exo}
Trouver toutes les fonctions $f:\R\to\R$ telles que pour tous réels $x$ et $y$ : 
$$(y+1)f(x+y)-f(x)f(x+y^2)=yf(x)$$
\end{exo}
\begin{exo}
Trouver toutes les fonctions $f:\R\to\R$ telles que pour tous réels $x$ et $y$ : 
$$f(x)f(y)=f(x-y)$$
\end{exo}

\begin{exo}
Trouver toutes les fonctions $f:\R\to\R$ telles que pour tous réels $x$ et $y$ : 
$$f(x^{2022}+y)=f(x^{1747}+2y)+f(x^{42})$$
\end{exo}
\begin{exo}
Trouver toutes les fonctions $f:\R\to\R$ telles que pour tous réels $x$ et $y$ : 
$$f(f(x)^2+f(y))=xf(x)+y$$
\end{exo}
\begin{exo}
Trouver toutes les fonctions $f:\R\to\R$ telles que pour tous réels $x$ et $y$ : 
$$(x-y)f(x+y)-(x+y)f(x-y)=4xy(x^2-y^2)$$
\end{exo}
\begin{exo}
Trouver toutes les fonctions $f:\N\to\N$ telles que pour tous naturels $n$ et $m$ : 
$$f(3n+2m)=f(n)f(m)$$
\end{exo}
\begin{exo}
Trouver toutes les fonctions $f:\N\to\N$ telles que pour tous naturels $a$ et $b$ : 
$$f(f(a)+f(b))=a+b$$
\end{exo}
\begin{exo}
Trouver toutes les fonctions $f:\R\to\R$ telles que pour tous réels $x$ et $y$ : 
$$f(x^2-y^2)=xf(x)-yf(y)$$
\end{exo}
\begin{exo}
Trouver toutes les fonctions $f:\R\to\R$ telles que pour tous réels $x$ différent de 1 : 
$$f(x)+f\left(\frac1{1-x}\right)=x$$
\end{exo}
\begin{exo}
Trouver toutes les fonctions $f:\R\to\R$ telles que pour tous réels $x$ et $y$ :
$$f(x^2+f(y))=y+f(x)^2$$
\end{exo}
\subsubsection{Corrections}
\begin{sol}[][1]
\begin{itemize}
    \item Soient $a$ et $b$ deux réels tels que $f(a)=f(b)$, on a $f(f(a)-1)=f(f(b)-1)$ donc $a+1=b+1$, dès lors $f$ est injective.\\
    Soit $y$ un réel on a $f(f(y-1)-1)=y$ donc $f$ est surjective, donc $f$ est bijective.
    \item On considère $y=1$, on a $f(1+f(x))=3x$.\\ 
    Soient $a$ et $b$ deux réels tels que $f(a)=f(b)$, on a $f(1+f(a))=f(1+f(b))$ donc $a=b$, dès lors $f$ est injective.\\
    Soit $y$ un réel on a $f(1+f(\frac y3))=y$ donc $f$ est surjective, donc $f$ est bijective.
    \item Pour $x=0$ : $f(f(y))=y^5+f(0)$ or $y\to y^5+f(0)$ est bijective donc $f$ est bijective.
    \item Pour $x=0$ et $x=\pi$ : $f(f(0))=f(f(\pi))$, si $f(0)\ne f(\pi)$ ou $f(0)=f(\pi)$, dans les deux cas $f$ n'est pas injective.\\
    $\sin$ n'est pas surjective or si $f$ est surjective $f\circ f$ est surjective donc $f$ n'est ni surjective ni injective.
    \item En prenant $x=y=0$, $f(0)=f(0)^2$ donc $f(0)=0$ ou $f(0)=1$.\\
    Si $f(0)=1$, on considère $y=0$, on a alors $0=x$, c'est impossible. Ainsi, $f(0)=0$ :\\
    On considère $y=0$, on a $f(x)=0$ donc $f$ n'est ni surjective ni injective.
\end{itemize}
\end{sol}
%\begin{sol}[][2]
%On considère $y=0$ : on a $f(x)(f(0)+1)=0$.\\
%Ainsi soit $f=0$ soit $f(0)=-1$. Si $f=0$, on a pour tous $x$ et $y$ : $0=xy$ donc c'est impossible.\\
%Donc $f(0)=-1$\\
%On prend $x=-y=1$, on a $f(1)f(-1)=0$, donc $f(1)=0$ ou $f(-1)=0$.\\
%Si $f(1)=0$, avec $y=1$ : $f(x+1)=x$ donc $f:x\longmapsto x-1$, qui convient.\\
%Sinon, $f(-1)=0$, avec $y=-1$ : $f(x-1)=-x$ donc $f :\longmapsto -x-1$, qui convient aussi.

%\end{sol}
\begin{sol}[][2]
On prend $y=0$, on a $f(2f(x)+f(0))=2x+f(0)$. Soit $a$ et $b$ deux réels tels que $f(a)=f(b)$, on a alors $f(2f(a)+f(0))=f(2f(b)+f(0))$ d'où $2a+f(0)=2b+f(0)$ et $a=b$, donc $f$ est injective.\\
On considère $x=0$, on a alors $f(2f(0)+f(y))=f(y)$, par injectivité, $f(y)=y-2f(0)$, en particulier $f(0)=-f(0)$ donc $f(0)=0$ et $f$ est l'indentité. Elle convient c'est donc l'unique solution.
\end{sol}
\begin{sol}[][3]
On considère $y=0$, on a alors $f(x)(1-f(x))=0$.\\
Ainsi $f(x)=0$ ou $f(x)=1$, attention cela ne veut pas dire que $f$ est constante à $1$ ou à $0$, elle ne prend que ces valeurs mais peut changer.\\
On considère qu'il existe $a$ tel que $f(a)=0$, en considérant $x=a$, on a $(y+1)f(a+y)=0$, donc $f(a+y)=0$ pour $y\ne-1$ on considère $x=a-2$ par exemple et $y=1$, on a alors $f=0$. Cette fonction convient. \\
Sinon c'est que $f=1$ qui convient également.
\end{sol}
\begin{sol}[][4]
On considère $x=y=0$, on a $f(0)^2=f(0)$, donc $f(0)=0$ ou $1$.\\
On suppose $f(0)=0$, on considère $y=0$, on a $f(x)=0$, donc, $f=0$, qui convient.\\
Sinon $f(0)=1$, on prend $x=y$, on a $f(x)^2=1$, d'où $f(x)=\pm1$. Attention cela ne veut pas dire que la fonction est constante à $1$ ou $-1$. On considère $x=0$, on a $f(x)=f(-x)$ donc $f$ est paire. On prend ensuite $y=-x$, pour avoir $$1=f(x)^2=f(x)f(-x)=f(2x)$$
Pour un réel $y$, on peut considérer $x=\dfrac y2$, et on a $f(y)=1$, donc $f=1$ qui convient aussi. 
\end{sol}
\begin{sol}[][5]
On a des exposants compliqués on va donc essayer de les simplifier. Pour un $x$ donné, on a $x^{2022}+y=x^{1747}+2y$ si et seulement si $y=x^{2022}-x^{1747}$ on prend donc ce $y$. On a alors $f(2x^{2022}-x^{1747})=f(2x^{2022}-x^{1747})+f(x^{42})$ d'où : $f(x^{42})=0$, ainsi $f$ est nul sur les positifs. On considère $y=0$, on a alors pour tous les $x$ négatifs $f(x^{1747})=0$, d'où $f=0$. Cette fonction convient ($0+0=0$).
\end{sol}
\begin{sol}[][6]
On considère $x=0$, on a alors $f(f(0)^2+f(y))=y$ donc $f$ est bijective.\\
Soit $a$ l'antécédant de $0$, on considère $x=a$, on a $f(f(y))=y$, $f$ est involutive !\\
En évaluant en $f(x)$ au lieu de $x$ l'équation, le membre de droite ne change pas mais le membre de gauche si. Ainsi : $$f(f(x)^2+f(y))=f(x^2+f(y))$$ par injectivité : $x^2=f(x)^2$. Cela signifie que pour tout $x$, on a $f(x)=x$ ou $f(x)=-x$. (Attention cela ne vaut pas dire que $f:x\longmapsto x$ ou $f:x\longmapsto -x$)\\
On suppose qu'il existe $b$ et $c$ deux réels tels que $f(b)=b$ et $f(c)=-c$, on a alors en prenant $x=b$ et $y=c$ : $$f(b^2-c)=b^2+c$$ on a alors $b^2-c=b^2+c$ ou $b^2-c=-b^2-c$, dans les deux cas, $b$ ou $c$ est nul. Ainsi on a $f:x\longmapsto x$ ou $f:x\longmapsto -x$. Ces fonctions conviennent à l'énoncé, ce sont donc les solutions.

\end{sol}

\begin{sol}[][7]
L'expression semble au premier abord compliquée. On peut cependant remarquer la forte présence de $x+y$ et $x-y$. On pose alors $a=x+y$ et $b=x-y$, on a alors $x=\dfrac{a+b}2$ et $y=\dfrac{a-b}2$. L'équation initiale est alors équivalente à : $$bf(a)-af(b)=ab(a^2-b^2)$$ On prend $b=1$, on a alors $f(a)=a^3+a(f(1)-1)$. Ainsi, il existe $c$ un réel tel que $f(x)=x^3+xc$ pour tous les réels $x$, on vérifie que ces fonctions respectent l'équation initiale.
\end{sol}
\begin{sol}[][8]
Pour $n=m=0$ on a $f(0)=f(0)^2$ donc $f(0)=0$ ou $f(0)=1$\\
On suppose $f(0)=0$, avec $n=0$, $f(2m)=0$ et avec $m=0$, $f(3n)=0$. Puis avec $n=3$, $f(2m+9)=0$, ainsi $f(n)=0$ pour $n$ différent de $1$,$3$,$5$ et $7$. On prend ensuite $n=m$ égal à $3$, $5$ et $7$ pour avoir $f(n)^2=f(5n)=0$ d'où $f(n)=0$ pour $n\in\{3,5,7\}$. De même, $f(1)^2=f(5)=0$ et $f=0$. Cette fonction convient.\\
On suppose ensuite $f(0)=1$. On a $f(3 \times 4) = f(4)f(0) = f(2 \times 2) = f(2)f(0) = f(1)f(0) = f(1)$, et $f(3 \times2 +
2 \times 3) = f(2)f(3) = f(1)^2$ Donc $f(1) = f(1)^2$. Donc $f(1) = 1$ ou $f(1) = 0$.\\
\begin{itemize}
    \item Si $f(1)=0$. On a $f(2)=0$ donc $f(2n+3)=f(n)f(1)=0$ et $f(2n+6)=f(n)f(2)=0$. Ainsi $f(n)=0$ pour $n$ différent de $4$ (et de $0$). Enfin $f(20)=f(4)^2=0$ donc $f(n)=0$ pour $n\ne 0$. Cette fonction convient également.
    \item Si $f(1)=1$, $f(2n)=f(n)$ et $f(2n+3)=f(n)$. Ainsi pour $n>1$, il existe $m<n$ tel que $f(m)=f(n)$ ainsi par récurrence forte $f(n)=f(1)=1$. Cette fonction vérifie aussi l'énoncé.
\end{itemize}
On a donc trois solutions possibles :
\begin{itemize}
    \item $f(n)=0$ pour $n\in\N$
    \item $f(n)=1$ pour $n\in\N$
    \item $f(n)=0$ pour $n\ne 0$ et $f(0)=1$
\end{itemize}
\end{sol}
\begin{sol}[][9]
Pour $b=0$, $f(f(a)+f(0))=a$, on a alors $f$ injective (car si $f(n)=f(m)$, $f(f(n)+f(0))=f(f(m)+f(0))$) et surjective.\\
Soit $n$ un entier naturel, on prend $a=n$ et $b=1$ puis $a=n+1$ et $b=0$, on a alors 
$$f(f(n)+f(1))=n+1=f(f(n+1)+f(0))$$
Ainsi par injectivité : $f(n+1)=f(n)+f(1)-f(0)$, on note $f(0)=a$ et $f(1)=b$, on a alors $f(n+1)=f(n)+b-a$.\\
Par récurrence facile $f(n)=a+n(b-a)$, on peut alors conclure de deux façons, soit on réinjecte dans l'équation initiale pour avoir $a=0$ et $b=1$, sinon, la surjectivité dit que tout naturel est atteint donc $b-a=1$ et $a=0$. Dans tous les cas $f(n)=n$ pour tout $n$ et cette fonction convient bien.
\end{sol}
\begin{sol}[][10]
Pour $y=0$, on a $f(x^2)=xf(x)$, cela donne en particulier le fait que $f$ est impaire.\\
Par ailleurs cela donne $f(x^2-y^2)=f(x^2)+f(-y^2)$, ainsi $f(a+b)=f(a)+f(b)$ pour $a\ge0$ et $b\le0$. L'imparité permet d'avoir $f(a+b)=f(a)+f(b)$ dans les autres cas.\\
On prend $x=t+1$ et $y=t$, pour avoir $$2f(t)+f(1)=f(2t+1)=f((t+1)^2-t^2)=(t+1)f(t+1)-tf(t)=f(t)+(t+1)f(1)$$ Ainsi $f(t)=tf(1)$ et $f$ est une fonction linéaire, on vérifie qu'elles fonctionnent.
\end{sol}
\begin{sol}[][11]
Soit $x$ différent de $0$ et de $1$ \\
On applique l'équation à $\dfrac1{1-x}$ au lieu de $x$, on a alors : $$f\left(\dfrac1{1-x}\right)+f\left(1-\dfrac1x\right)=\dfrac1{1-x}$$
On réapplique $\dfrac1{1-x}$ au lieu de $x$ dans cette dernière équation : $$f\left(1-\dfrac1x\right)+f(x)=1-\dfrac1x$$
On a trois équations à trois inconnues, on peut donc résoudre le système. On prend l'équation de départ plus la dernière auquelle on soustrait la deuxième pour obtenir : $$2f(x)=x+1-\dfrac1x-\dfrac1{1-x}$$
D'où : $f(x)=\dfrac{x^3-2x}{2x(x-1)}$ pour $x\in\R-\{0,1\}$.\\
Pour $x=0$, l'équation de départ implique $f(0)+f(1)=0$, c'est la seule contrainte sur $f(0)$ et $f(1)$. \\
Ainsi les fonctions qui sont solutions (parce qu'elles conviennent) sont les $f:\R\longmapsto\R$ telles qu'il existe $c\in\R$ tel que $f(x)=\dfrac{x^3-2x}{2x(x-1)}$ pour $x$ différent de $0$ et de $1$, $f(0)=c$ et $f(1)=-c$
\end{sol}

\begin{sol}[][12]
On prend $x=0$, on a alors $f(f(y))=y+f(0)^2$, cela donne en particulier $f$ bijective.
On applique $f$ à l'équation initiale pour obtenir :
$$f(f(x^2+f(y)))=x^2+f(y)+f(0)^2=f(y+f(x)^2)$$
On prend maintenant $a$ l'antécédant de $0$ par $f$ et on prend $x=a$ dans l'équation précédente :\\
$a^2+f(y)+f(0)^2=f(y)$, ainsi $a^2+f(0)^2=0$. Un carré (d'un réel) est positif donc $a=f(0)=0$.\\
D'ailleurs $f(f(y))=y$.
On prend $y=0$ dans l'équation initiale, on a : $f(x^2)=f(x)^2$.\\
L'équation de départ donne donc $f(x^2+f(y))=y+f(x^2)$. On prend alors $y=f(-x^2)$ pour avoir :
$$f(x^2+f(f(-x^2)))=f(0)=0=f(-x^2)+f(x^2)$$
Ainsi $f$ est impaire.\\
En prenant $f(y)$ à la place de $y$ dans l'équation de départ on a :
$$f(x^2+f(f(y)))=f(x^2+y)=f(y)+f(x^2)$$
Ainsi $f(a+b)=f(a)+f(b)$ pour $a$ positif, en fait pour $a$ et $b$ réels quelconques en utilisant l'imparité. $f$ résoud l'équation de Cauchy donc $f$ est linéaire sur $\Q$, de plus on a $f(x^2)=f(x)^2$ donc $f(x)\ge0$ pour $x\ge0$, ainsi $f$ est croissante et en encadrant chaque réel par deux suites adjacentes de rationnels, on a $f $ linéaire sur $\R$. Comme $f$ est croissante et $f(x^2)=f(x)^2$, on a $f$ est l'identité qui convient.
\end{sol}