
Pour la 5ième fois, le Centre International de Valbonne (CIV) nous a accueilli du lundi 16 août vers X h au jeudi 26 août vers XX h, avec un effectif final de XX stagiaires et XX animatheurs. 


Parmi les presque XX candidats à la Coupe Animath, un peu moins de 500 ont franchi le cap des éliminatoires en ligne. Sur la base des résultats de la Coupe, nous devions accueillir XX stagiaires, dont environ XX de fin de première, 20 de seconde, 10 de troisième et 10 de quatrième. En prévision des EGMO, Olympiades Européennes Féminines de Mathématiques,
et de la JBMO, Olympiades Balkaniques Junior de Mathématiques, 
des bonifications ont été ajoutées pour favoriser les filles et les plus jeunes.




Le stage était structuré comme ceux des années précédentes : deux périodes de quatre jours (18 - 21 août et 22 - 26 août), trois de cours / exercices, un entraînement %(test) 
de type olympique le matin du quatrième jour (de 9h à 12h, ou, pour le groupe D, de 8h à 12h) et une après-midi récréative. Les élèves étaient répartis en 4 groupes A, B, C, et D en fonction de leur expérience en mathématiques olympiques.
Le programme est construit suivant ce qui est demandé lors des compétitions internationales : Arithmétique, Algèbre, Combinatoire et Géométrie.



En plus des cours étaient prévues, le soir, des conférences à vocation culturelle, permettant de découvrir de nouveaux pans des mathématiques. Merci à Pooran Memari pour son exposé sur les triangulations et leur utilisation dans la vie courante ; Colin Davalo pour sa présentation des jeux combinatoires (et avoir appris aux élèves à gagner à tous les coups au jeu de Nim !) ; Phong Nguyen pour son colloque sur l'utilisation de la théorie des nombres dans la vie courante et à  Victor Vermès pour sa (très actuelle) conférence sur la propagation d'une épidémie.



L'après-midi suivant le premier entraînement fut organisée un grand jeu par une petite équipe chapeautée par X. % METTRE UN PETIT COMMENTAIRE, et rajouter ce qui s'est passé lors du second entraînement.


Il est possible de retrouver les comptes rendus du stage au jour le jour sur le site de la POFM : \url{https://maths-olympiques.fr/?p=5193}

\vfill
\pagebreak
