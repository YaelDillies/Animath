\begin{exo}
Considérons $4n$ points dans le plan trois à trois non alignés. Montrer que l'on peut former $n$ quadrilatères non croisés disjoints dont les sommets sont ces points
\end{exo}

\begin{exo}
Soit $n\ge 1$ : on place dans le plan $2n$ points, trois quelconques non alignés. On en colorie $n$ en bleu et $n$ en rouge. Montrer qu’il est possible de tracer $n$ segments qui ne se croisent pas, chaque segment reliant un point bleu à un point rouge, de telle manière que chaque point soit utilisé une seule fois.
\end{exo}

\begin{exo}
On considère $2021$ droites du plan, deux à deux non parallèles et trois à trois non concourantes. On appelle E l’ensemble de leurs points d’intersection. On veut attribuer une couleur à chacun des points de E de sorte que deux quelconques de ces points qui appartiennent à une même droite et dont le segment qui les relient ne contient aucun autre point de E, soient de couleurs différentes.

Combien faut-il au minimum de couleurs pour pouvoir réaliser une telle coloration ?
\end{exo}

\begin{exo}
(Théorème de Helly en dimension $2$) On considère quatre parties convexes du plan telles que l’intersection de trois d’entre elles est toujours non vide.

$a)$ Montrer que l’intersection des quatre convexes est non vide.

$b)$ Le théorème reste-t-il vrai en remplaçant $4$ par $n \ge 4$ ?
\end{exo}

\begin{exo}
Soit $A_1,\ldots,A_n$ un polygone convexe fixé. On considère $X$ à l’intérieur du polygone. Pour tout $i$, on note $B_i$ la deuxième intersection de $(A_iX)$ avec le bord du polygone. Montrer qu’il est possible de choisir $X$ de telle manière que pour tout $i$ :
$$\frac{XA_i}{XB_i}\le 2.$$
\end{exo}

\begin{exo}
On dit qu’un ensemble de points du plan est \textit{obtus} lorsque trois points quelconques de cet ensemble sont toujours les sommets d’un triangle obtus. Prouver que tout ensemble de $n$ points du plan, trois quelconques jamais alignés, contient un sous-ensemble \textit{obtus} d’au moins $\sqrt n$ éléments.
\end{exo}

\begin{exo}
(IMO $1995,P3$) Trouver tous les entiers $n > 3$ pour lesquels il existe $n$ points $A_1, \dots, A_n$
du plan et des réels $r_1, \dots, r_n$ tels que :
\begin{enumerate}
    \item trois quelconques des points ne sont jamais alignés.
    \item pour tout ${i, j, k}$, l’aire du triangle $A_iA_jA_k$ est égale à $r_i + r_j + r_k$.
\end{enumerate}
\end{exo}

\begin{exo}
(Problème de la galerie d’art) Soit $\mathcal{P}$ un polygone non croisé (pas forcément convexe) à $n\ge 3$ sommets. Montrer qu’il existe un ensemble $\mathcal{A}$ de $\lfloor\frac n3\rfloor$ sommets de $\mathcal{P}$ tel que pour tout $X$ à l’intérieur de $\mathcal{P}$ il existe un point $C\in\mathcal{A}$ tel que le segment $[CX]$ soit entièrement à l’intérieur de $\mathcal{P}$.
\end{exo}

\begin{exo}
(IMO SL 2019 C4)

Sur une plaine plate à Camelot, le roi Arthur construit un labyrinthe $\mathfrak{L}$ constitué de $n$ droites deux-à-deux non parallèles et trois-à-trois non concourantes. Pour chaque mur, Merlin colorie un des côtés en rouge et l'autre en bleu.

A l'intersection de deux murs, il y a $4$ coins: deux coins où un côté rouge et un côté bleu s'intersectent, un coin où deux côtés rouges s'intersectent, et un coin où deux côtés bleus s'intersectent. A cette intersection, il existe une porte reliant les deux coins opposés où des côtés de couleurs différentes s'intersectent.

Après que Merlin a colorié les murs, Morgane place des chevaliers dans le labyrinthe. Les chevaliers peuvent marcher à travers les portes, mais pas à travers les murs.

Soit $k(\mathfrak{L})$ le plus grand entier $k$ tel que: quelle que soit la façon dont Merlin colorie le labyrinthe $\mathfrak{L}$, Morgane peut toujours placer $k$ chevaliers tels que deux d'entre eux ne peuvent jamais se rencontrer. A $n$ fixé, quelles peuvent être les valeurs de $k(\mathfrak{L})$ pour des labyrinthes $\mathfrak{L}$ constitués de $n$ droites ?
\end{exo}

\begin{exo}
(IMO SL 2018 C7)

Soient $2018$ cercles dans le plan, deux-à-deux concourants et non tangents, et trois-à-trois non concourants. Ces cercles divisent le plan en régions délimitées par des \textit{arêtes} circulaires qui s'intersectent en des \textit{sommets}. Chaque cercle possède un nombre pair de sommets, on les colorie alternativement en rouge et en bleu. En faisant ceci pour tous les cercles, chaque sommet est colorié deux fois, si les deux couleurs sont les mêmes le sommet reste de cette couleur, sinon le sommet devient jaune.

Montrer que si un cercle contient au moins $2061$ points jaunes, alors il existe une région dont tous les sommets sont jaunes.
\end{exo}

\begin{sol}
On utilise ici une technique classique dans les problèmes de géométrie combinatoire: quitte à faire une rotation, on peut considérer que tous les points ont une abscisse distincte, on les regarde alors par abscisse croissante.

Si les points triés comme ceci sont $A_1,\ldots,A_{4n}$, on considère des quadrilatères non croisés formés par les $A_{4k+1},A_{4k+2},A_{4k+3},A_{4k+4}$ ($1\le k\le n$). Ces quadrilatères sont compris entièrement entre les abscisses de $A_{4k+1}$ et $A_{4k+4}$ et sont alors deux-à-deux disjoints.
\end{sol}

\begin{sol}
On propose deux solutions: une solution utilisant des techniques classiques de géométrie combinatoire et une solution courte mais astucieuse.

\medskip

\underline{Solution 1:} On procède par récurrence forte. Pour $n=1$, le résultat est trivial. Supposons le vrai jusqu'au rang $n-1$ et montrons le au rang $n$. L'idée de la solution est classique et consiste à étudier l'enveloppe convexe des points. On a deux possibilités:
\begin{itemize}
    \item Si l'enveloppe convexe contient deux points de couleurs différentes, alors on peut trouver deux tels points $A$ et $B$ qui se suivent sur l'enveloppe convexe. $A$ et $B$ sont en dehors de l'enveloppe convexe des autres $2(n-1)$ points restants. On relie ces points avec l'hypothèse de récurrence et on peut ensuite relier $A$ et $B$ pour obtenir une configuration qui marche avec les $2n$ points.
    \item Sinon, l'enveloppe convexe ne contient que des points d'une même couleur, disons bleu. Encore une fois, on effectue une rotation du plan afin que tous les points aient une abscisse différente, alors les points avec les abscisses les plus petite et grande sont bleus. On fait glisser une droite verticale $(d)$ de gauche à droite de la figure. Juste après avoir dépassé le premier point, on a un point bleu de plus que de rouge à gauche de $(d)$, et juste avant de dépasser le dernier, c'est le contraire. Ainsi, il existe une position intermédiaire de $(d)$ tel que chaque côté possède autant de points bleus que rouges. On relie les points de chaque côté de $(d)$ par l'hypothèse de récurrence forte, et on a ainsi relié les $2n$ points du plan comme voulu.
\end{itemize}

\medskip

\underline{Solution 2:} Soit $\mathcal{T}$ un tracé de $n$ segments entre des points bleus et rouges, de façon à ce que chaque point appartienne à un unique segment. On définit $p(\mathcal{T})$ comme la somme des longueurs des segments de $\mathcal{T}$. L'ensemble de ces tracés $\mathcal{T}$ étant fini, il en existe un $\mathcal{T}_0$ minimisant $p(\mathcal{T})$. Mais alors si $[B_1,R_1]$ et $[B_2,R_2]$ sont deux segments distincts de $\mathcal{T}_0$ qui s'intersectent (avec les $B_i$ bleus et les $R_i$ rouges), on peut les remplacer par $[B_1,R_2]$ et $[B_2,R_1]$ qui ne s'intersectent pas et dont on vérifie aisément que la somme des longueurs est strictement inférieure à celle d'avant (faire un dessin et appliquer l'inégalité triangulaire $2$ fois). Ainsi, c'est absurde et les segments de $\mathcal{T}_0$ ne s'intersectent pas.
\end{sol}

\begin{sol}
Tout d'abord, remarquons qu'il est nécessaire d'avoir au moins $3$ couleurs pour obtenir une telle coloration. En effet, on montre facilement par récurrence sue le nombre de droites que la configuration contient au moins un triangle formé par les droites (c'est vrai pour $n=3$, et si on rajoute une droite, soit elle laisse le triangle intact soit elle le sépare en un quadrilatère et un triangle).

Montrons que $3$ couleurs suffit. Comme dans de nombreux autres exos, on effectue une rotation du plan afin qu'aucune droite ne soit verticale, on trie les points d'intersection par abscisse croissante $A_1,\ldots,A_n$ (où $n=\dbinom{2021}2$) et on les colorie dans ce même ordre. A chaque étape, on peut toujours colorier le point considéré en une couleur valide puisque s'il est le point d'intersection de $(d)$ et $(d')$, les seules contraintes sur la couleur du point sont celles données par le point à sa gauche sur $(d)$ et celui à sa gauche sur $(d')$, et dans tous les cas on peut choisir une couleur qui convient.
\end{sol}

\begin{sol}
\begin{enumerate}
    \item Soient $C_1,C_2,C_3,C_4$ les $4$ convexes. On utilise l'hypothèse de l'énoncé pour obtenir, pour chaque $1\le i \le 4$ un point $A_i$ appartenant à chacun des $C_j$ pour $j\ne i$. 
    \begin{itemize}
        \item Si $A_1,A_2,A_3,A_4$ sont les sommets d'un quadrilatère convexe, alors sans perte de généralité $[A_1A_2]$ et $[A_3A_4]$ s'intersectent en un point $X$. $A_1$ et $A_2$ appartiennent à $C_3$ et $C_4$, donc par convexité de ces ensembles, $O$ aussi. De même, $O$ appartient à $C_1$ et $C_2$ et donc à l'intersection des quatre convexes.
        \item Sinon, on suppose sans perte de généralité que $A_1$ est à l'intérieur du triangle $A_2A_3A_4$. Les trois sommets du triangle appartiennent à $C_1$ donc $A_1$ aussi par convexité de $C_1$. Comme $A_1$ apaprtient déjà aux trois autres convexes, il appartient à l'intersection des quatre.
    \end{itemize}
    
    \item La réponse est oui. Pour le montrer, on procède par récurrence sur $n$. On a traité le cas $n=4$, supposons le résultat vrai pour $n$ et montrons le au rang $n+1$. Soient donc $C_1,\ldots,C_{n+1}$ des convexes dont l'intersection de trois quelconques est non vide. Par l'hypothèse de récurrence, pour $1\le i\le 4\le n+1$, on peut trouver un point $A_i$ dans $\cap_{j\ne i}C_j$. De la même manière que dans le cas $n=4$, on termine pour trouver un point dans l'intersection de tous les convexes.
\end{enumerate}
\end{sol}

\begin{sol}
La condition $\frac{XA_i}{XB_i}\le2$ équivaut à $\frac{A_iX}{A_iB_i}\le \frac 23$, ce qui revient à dire que $X$ est dans l’image du polygone par l’homothétie de centre $A_i$ et de rapport $\frac 23$. On note $P_i$ cette image. Alors $P_i$ est un polygone convexe, et il suffit de vérifier :
$$\cap_{i=1}^nP_i \ne \emptyset. $$
D’après le théorème de Helly, il suffit de vérifier que l’intersection de trois $P_i$ n’est jamais vide. C’est vrai car pour tous $i$, $j$ et $k$ le centre de gravité de $A_iA_jA_k$ se trouve dans $P_i$, $P_j$ et $P_k$.
\end{sol}

\begin{sol}
Encore une fois, on effectue une rotation du plan pour que les abscisses des points soient toujours distinctes. Soient $(x_1,y_1),\ldots,(x_n,y_n)$ les points avec $(x_i)$ croissante. On rappelle le théorème d'Erdös-Szekeres:

\begin{thm}
Soient $a,b$ deux entiers strictement positifs. Parmi toute suite de $ab+1$ réels, on peut extraire une sous-suite croissante de $a+1$ termes ou une sous-suite décroissante de $b+1$ termes.
\end{thm}

\begin{preuve}
Si $(u_n)_{1\le n\le ab+1}$ est la suite, on définit pour tout $1\le n\le ab+1$ les entiers $a_n$ et $b_n$ qui correspondent aux tailles des plus grandes sous-suites croissante et décroissante de $(u_m)$ se terminant en $n$ respectivement. Si par l'absurde on a toujours $a_n\le a$ et $b_n\le b$, comme ces valeurs sont strictement positives, le couple $(a_n,b_n)$ ne peut prendre que $ab$ valeurs et par le principe des tiroirs, il existe $m<n$ tels que $a_m=a_n$ et $b_m=b_n$. Sans perte de généralité, $u_m\le u_n$ et une des sous-suites croissantes de taille $a_m$ se terminant en $m$ peut se prolonger en une sous-suite croissante de taille $a_m+1>a_n$ se terminant en $n$, absurde.
\end{preuve}

Notamment, ce théorème appliqué à $a=b$ donne que toute suite de taille $n^2+1$ possède une sous-suite monotone de taille $n+1$. Une suite de taille $n$ est de taille au moins $(\lceil\sqrt n\rceil-1)^2+1$ et contient donc une sous-suite monotone de taille $\lceil\sqrt n\rceil\ge \sqrt n$. On extrait donc une sous-suite monotone de $(y_i)$ de taille au moins $\sqrt n$. 

On peut vérifier facilement (par un produit scalaire par exemple) que les points correspondant forment alors le sous-ensemble \textit{obtus} cherché.
\end{sol}

\begin{sol}
Si on a quatre points $A_i,A_j,A_k,A_l$ qui forment un quadrilatère convexe dans cet ordre, en notant $\mathcal{A}$ l'aire, on a
$$\mathcal{A}(A_iA_jA_kA_l) = \mathcal{A}(A_iA_jA_k) + \mathcal{A}(A_kA_lA_i) = \mathcal{A}(A_jA_kA_l)+\mathcal{A}(A_lA_iA_j)$$
ce qui se réécrit $r_i+r_k = r_j+r_l$. Maintenant, si $A_iA_jA_ikA_lA_m$ est un pentagone convexe, le résultat précédent appliqué aux quadrilatères convexes $A_iA_jA_kA_l$ et $A_jA_kA_lA_m$ donne $r_i+r_k=r_j+r_l=r_k+r_m$ d'où $r_i=r_m$. De la même manière et en itérant, on a $r_i=r_j=r_k=r_l=r_m$. Mais alors les triangles $A_iA_jA_k,A_iA_jA_l$ et $A_iA_jA_m$ ont la même aire $3r_i$ et les points $A_k,A_l,A_m$ sont à la même distance de la droite $(A_iA_j)$. Le pentagone $A_iA_jA_kA_lA_m$ étant convexe, ils sont aussi du même côté de cette droite et sont donc alignés, ce qui est absurde. Ainsi, il n'existe pas de pentagone convexe parmi les points $A_i$. Notamment, l'enveloppe convexe de ces points contient au plus $4$ points, sans perte de généralité elle est de la forme $A_1,A_2,\ldots,A_m$ ($m\le 4$).

Maintenant si $A_i$ n'appartient pas à cette enveloppe convexe (soit $i>m$), on a
$$\mathcal{A}(A_1A_2\ldots A_m)= \mathcal{A}(A_1A_2A_i)+\mathcal{A}(A_2A_3A_i)+\ldots + \mathcal{A}(A_{m-1}A_mA_i)+\mathcal{A}(A_mA_1A_i)$$
$$=2(r_1+r_2+\ldots+r_m)+mr_i$$
et donc tous les $r_i$ pour $i>m$ sont égaux. Mais alors si $i,i'$ sont de cette forme, pour tous $k,l\le m$, les aires de $A_kA_lA_i$ et $A_kA_lA_{i'}$ sont égales et $(A_iA_{i'})$ est parallèle à $(A_kA_l)$. Ceci étant vrai pour tout couple de points de l'enveloppe convexe (dont il y en a au moins $3$), on a forcément $i=i'$. Avec l'inégalité $m\le 4$, on a forcément $n\le 5$. Pour $n=4$, on peut simplement prendre un carré et tous les $r_i$ égaux. Il reste à montrer que le cas $n=5$ est impossible.

Dans ce cas, on a $m=4$. Le point $A_5$ est à l'intérieur de deux triangles $A_iA_jA_k$ avec $i,j,k\le 4$, par exemple $A_1A_2A_3$ et $A_2A_3A_4$. On écrit
$$\mathcal{A}(A_1A_2A_3) = \mathcal{A}(A_5A_2A_3)+\mathcal{A}(A_1A_5A_3)+\mathcal{A}(A_1A_2A_5)$$
ce qui donne $r_1+r_2+r_3+3r_5=0$. De même, on a $r_2+r_3+r_4+3r_5=0$ d'où $r_1=r_4$. On en déduit que $(A_1A_4)$ est parallèle à $(A_2A_3)$ et à $(A_2A_5)$ donc $A_2,A_3,A_5$ sont alignés, absurde.

La seule valeur de $n$ qui convient est donc $n=4$.
\end{sol}

\begin{sol}
On triangule $\mathcal{P}$ en $n-2$ triangles. On montre par récurrence sur $n$ que l'on peut colorier les sommets de $\mathcal{P}$ en $3$ couleurs afin que chacun des triangles ait un sommet de chaque couleur. Pour $n=3$, c'est évident. Si le résultat est vrai pour $n-1$, on remarque que comme il y a $n$ arêtes de $\mathcal{P}$ et $n-2$ triangles, un des triangles possède deux arêtes de $\mathcal{P}$ comme côtés, elles sont nécessairement consécutives et ont un sommet $X$ en commun. On applique alors l'hypothèse de récurrence à $\mathcal{P}$ auquel on a enlevé $X$, puis lorsque l'on rajoute $X$ il n'y a que $2$ contraintes à respecter et on peut colorier $X$ de la façon voulue.

Pour un choix adéquat d'une des trois couleurs, il y a au plus $\lfloor\frac n3\rfloor$ sommets de la couleur, on prend pour $\mathcal{A}$ l'ensemble de ces sommets. $\mathcal{A}$ fonctionne car si $X$ est à l'intérieur de $\mathcal{P}$, il est à l'intérieur d'un des triangles, dont un des sommets $C$ appartient à $\mathcal{A}$, ce qui conclut.
\end{sol}

\begin{sol}
On va montrer que la seule valeur possible est $n+1$, quel que soit le labyrinthe $\mathfrak{L}$ choisi par le roi Arthur.

\begin{itemize}
    \item \underline{$k(\mathfrak{L})\ge n+1$:} Comptons le nombre de régions formées par les $n$ murs de $\mathfrak{L}$. On effectue une inversion de la figure pour se ramener à un graphe planaire, qui possède alors $S = \dbinom n2 + 1$ sommets (les $\dbinom n2$ intersections des murs et le point à l'infini), ainsi que $A = n^2$ arêtes (chaque mur est séparé en $n$ parties par les autres murs). La formule d'Euler donne alors le nombre de régions
    $$F = A + 2 - S = \dbinom n2 + n + 1.$$
    Considérons une coloration quelconque du labyrinthe par Merlin. Le nombre de portes créées est égal au nombre d'intersection des $n$ murs, soit $\dbinom n2$. Si l'on appelle composante connexe du labyrinthe un ensemble de poins connecté en prenant en compte les portes, chaque porte réduit d'au plus $1$ le nombre de composantes connexes du labyrinthe, et par conséquent $\mathfrak{L}$ contient au moins $n+1$ composantes connexes. Si Morgane place un chevalier dans chacune de ces composantes connexes, elle peut alors placer $n+1$ chevaliers qui ne peuvent pas se rencontrer dans tous les cas. Par conséquent $k(\mathfrak{L})\ge n+1$.
    
    \item \underline{$k(\mathfrak{L})\le n+1$:} Montrons maintenant que Merlin peut toujours colorier les murs de façon à obtenir au plus $n+1$ composantes connexes. Pour ceci, on effectue une rotation de la figure afin de n'avoir aucun mur horizontal, et on considère la coloration où le côté gauche de chaque mur est colorié en bleu, et le côté droit en rouge. Alors à l'intersection de deux murs, la porte se trouve du côté "vertical", et si un chevalier placé en un point arbitraire se dirige toujours vers le haut (en longeant des murs si nécessaire) il pourra continuer jusqu'à l'infini. Ainsi, chaque composante connexe du labyrinthe contient une des régions infinies "du haut" (c'est-à-dire qui ont une intersection infinie avec l'hyperplan $y>0$). Mais il n'y a que $n+1$ telles régions puisque chaque droite en créé une nouvelle. Ainsi, Morgane ne peut placer que $n+1$ chevaliers au plus.
\end{itemize}
\end{sol}

\begin{sol}
On commence par montrer que l'on peut partitionner les cercles en deux catégories intéressantes. On fait les observations suivantes:

\begin{itemize}
    \item Considérons deux cercles $\mathcal{C}_1$ et $\mathcal{C}_2$ qui s'intersectent en deux points $A$ et $B$. L'intersection des deux disques est alors une partie convexe $\mathcal{D}$ du plan délimitée par deux arcs $a_1$ et $a_2$ de $\mathcal{C}_1$ et $\mathcal{C}_2$ respectivement. Mais alors si un troisième cercle $\mathcal{C}_3$ intersecte $a_1\cup a_2$, il "rentre" dans $\mathcal{D}$ et doit en sortir par une deuxième intersection. Ainsi, le nombre de points autres que $A$ et $B$ sur $a_1\cup a_2$ est pair, ce qui signifie que les parités du nombre de points sur $a_1$ et $a_2$ sont les mêmes. Ceci signifie pour notre problème que si les deux coloriages de $A$ coincident, il en est de même pour les coloriages de $B$ et réciproquement. On dit alors que $\mathcal{C}_1$ et $\mathcal{C}_2$ sont en relation si leurs intersections sont jaunes. Par convention, on impose qu'un cercle est toujours en relation avec lui-même.
    \item On considère maintenant trois cercles $\mathcal{C}_1,\mathcal{C}_2,\mathcal{C}_3$ ainsi que des points d'intersection $A_i$ de $\cap_{j\ne i}\mathcal{C}_j$ pour $1\le i\le 3$. On considère des arcs $a_i$ de $\mathcal{C}_i$ reliant les $A_j$ sur $\mathcal{C}_i$ pour $1\le i\le 3$ ainsi que $\mathcal{D}$ la partie du plan délimitée par ces arcs. De même que dans l'observation précédente, le nombre de sommets autres que les $A_i$ sur la frontière de $\mathcal{D}$ est pair. Alors si on se balade sur la frontière de $\mathcal{D}$ en ne considérant que les coloriages donnés par les $\mathcal{C}_i$, on effectue quelque chose de la forme 
    $$R \rightarrow B \rightarrow R \rightarrow \ldots\rightarrow R$$
    sauf pour l'éventuelle perturbation où un des $A_i$ est jaune qui donne
    $$R\rightarrow J \rightarrow B\text{ ou } B\rightarrow J \rightarrow R$$
    La taille de la frontière de $D$ est impaire donc il y a un nombre impair de perturbation et \underline{un nombre impair de $A_i$ est jaune}. Ceci signifie que si $\mathcal{C}_1$ est en relation avec $\mathcal{C}_2$ et $\mathcal{C}_3$, alors $\mathcal{C}_2$ et $\mathcal{C}_3$ sont en relation, et que si $\mathcal{C}_1$ n'est pas en relation avec $\mathcal{C}_2$ et $\mathcal{C}_3$, alors $\mathcal{C}_2$ et $\mathcal{C}_3$ sont en relation.
    
    \item Ces deux observations montrent que si on fixe un cercle $\mathcal{C}$, alors $\mathcal{A}$ l'ensemble des cercles en relation avec $\mathcal{C}$ et $\mathcal{B}$ l'ensemble des autres cercles sont une partition des cercles qui vérifient que deux cercles d'un des ensembles sont en relations, et qu'un cercle de $\mathcal{A}$ et un cercle de $\mathcal{B}$ ne sont pas en relation.
\end{itemize}

Nous sommes maintenant en mesure de réécrire l'hypothèse de l'énoncé sur le nombre de points jaunes d'un cercle. En effet, si $a$ et $b$ sont les cardinaux de $\mathcal{A}$ et $\mathcal{B}$ respectivement, un cercle de $\mathcal{A}$ par exemple possède $2(a-1)$ points jaunes, qui correspondent aux intersections des $a-1$ autres cercles de $\mathcal{A}$. L'hypothèse de l'énoncé s'écrit donc $2(a-1)\ge 2061$ ou $2(b-1)\ge 2061$, soit, en supposant $a\ge b$, $a\ge 1031$. Nous allons montrer que c'est impossible. Supposons qu'il n'existe pas de région dont tous les sommets sont jaunes, et montrons que $a\le 1030$. Pour ceci, on effectue plusieurs dénombrements:

\begin{itemize}
    \item \underline{Le nombre de faces formées par les cercles de $\mathcal{A}$:} Dans le graphe planaire formé par les cercles de $\mathcal{A}$, on compte $a(a-1)$ sommets et $2a(a-1)$ arêtes d'où le nombre de faces donné par la formule d'Euler $2a(a-1)+2-a(a-1) = a^2-a+2$.
    \item \underline{Le nombre d'"arcs méchants":} Tout cercle de $\mathcal{B}$ est divisé en $2a$ arcs par les cercles de $\mathcal{A}$, on appelle ces arcs les "arcs méchants", il y en a $2ab$.
    \item \underline{Le nombre de points d'intersection de cercles de $\mathcal{B}$ dans une telle face:} Considérons une telle face formée par les cercles de $\mathcal{A}$. Cette face contient un certain nombre $k$ d'"arcs méchants". Mais ces arcs ne peuvent pas s'intersecter de telle façon à former une région à l'intérieur de la face, car sinon cette région aurait tous ses sommets coloriés en jaune. Considérons le (multi)graphe où les sommets sont ces "arcs méchants" et où deux "arcs méchants" sont reliés par une arête pour chacune de leurs intersections, alors s'il y avait plus de $k$ intersections de ces "arcs méchants", ce graphe contiendrait un cycle ce qui correspondrait à une boucle non-triviale formée par les "arcs méchants", et donc à une région à l'intérieur de la face. Ainsi, il y a au plus $k-1$ intersections de cercles de $\mathcal{B}$ dans la face.
    \item \underline{Le nombre total d'intersections d'"arcs méchants":} Ce sont précisément les intersections de cercles de $\mathcal{B}$, il y en a donc $2\dbinom b2$.
\end{itemize}

Après tout ce comptage, on est enfin en mesure de terminer la preuve. En effet, d'après les dénombrements précédents, on a:

$$2\dbinom b2\le \sum_{F} (k(F)-1)$$
où la somme porte sur les faces $F$ formées par les cercles de $\mathcal{A}$ et où $k(F)$ est le nombre d'"arcs méchants" dans $F$,
$$=\left(\sum_F k(F)\right)-\sum_F 1=2ab - (a^2-a+2).$$
Il reste à réécrire cette inégalité de manière utilisable. Si on n'a pas d'idée, on pose $b=n-a$ où $n=2018$ et on développe:
$$(n-a)(n-a-1)\le 2a(n-a)-a^2+a-2$$
$$n^2+a^2-2na-n+a\le 2na-2a^2-a^2+a-2$$
$$4a^2-4na+(n^2-n+2)\le 0$$
On se retrouve devant une équation que l'on sait résoudre, on trouve notamment
$$a\le \frac{4n+\sqrt{16n^2-16(n^2-n+2)}}{8}= \frac{n+\sqrt{n-2}}2.$$
En évaluant en $n=2018$ on trouve
$$a< 1032$$
ce qui est le résultat souhaité (on n'a pas de calculatrice le jour de l'épreuve, mais l'inégalité à montrer est équivalente à $\sqrt{2016}< 46$ soit $2016< 46^2$ ce qui se fait facilement à la main).
\end{sol}