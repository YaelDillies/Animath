L'objectif de cette séance était de regarder les comportements des polynômes dans $\Z/p\Z$ et $\Z/ n\Z$ et d'en voir des applications.

Comme pour les équations diophantiennes, regarder modulo $n$ est intéressant en arithmétique. Néanmoins, très souvent, le meilleur cadre est de regarder modulo un nombre premier ou une puissance d'un nombre premier : en effet, le lemme chinois permet de comprendre l'étude d'un polynôme modulo $n$ en le comprenant modulo des puissances de nombres premiers.\smallskip

Regardons ce qui est préservé ou non (dans la suite $n$ sera un entier strictement positif, $p$ un nombre premier) :
\begin{itemize}
\item La division euclidienne est préservée dans $\Z/p\Z$. Dans $\Z/n\Z$, on peut effectuer la division euclidienne lorsque le coefficient dominant est inversible (ceci est analogue au fait qu'on peut faire la division euclidienne dans $\Z[X]$ lorsque le coefficient dominant est unitaire). Par contre, si le polynôme n'est pas unitaire, on ne peut plus rien faire : il est impossible d'effectuer la division euclidienne de $X^2$ par $2X$.
\item Si $a$ est une racine de $P$ dans $\Z/n\Z$, on peut factoriser $P$ : il existe un polynôme $Q$ à coefficient dans $\Z/n\Z$ tel que $P(X)=(X - a)Q(X)$. Néanmoins, si $b$ est une autre racine de $P$, dans $\Z/p\Z$ on peut bien factoriser $P$ par $(X - a)(X - b)$. Dans $\Z/n\Z[X]$ par contre cela n'est pas possible : par exemple $X^2 - 1$ modulo $15$ admet pour racine $1,4,11, 14$.
\item En particulier, un polynôme de degré $d$ ayant au moins $d + 1$ racines dans $\Z/p\Z$ est nul, mais cela n'est pas le cas dans $\Z/n\Z[X]$. Un polynôme de degré $d\geq 0$ a donc au plus $d$ racines.
\item L'interpolation est toujours possible dans $\Z/p\Z$ : les polynômes interpolateurs de Lagrange sont bien définis.
\item Dans $\Z/p\Z$, $X^p - X=\prod\limits_{i=0}^{p - 1}(X - i)$. En effet, par petit Fermat, tous les éléments de $\Z/p\Z$ sont racines de $X^p - X$. C'est donc un polynôme de degré $p$ et unitaire, dont on connaît $p$ racines d'où la factorisation.

On obtient aussi que $X^{p - 1} - 1=\prod\limits_{i=1}^{p - 1}(X - i)$

\item Un polynôme dans $\Z/p\Z$ vérifie $P(x)=0$ pour tout $x\in \Z/p\Z$ si et seulement si $X^p - X$ divise $P$

\item Pour les polynômes de degré $2$, dans $\Z/p\Z$ avec $p\geq 3$, le polynôme $ax^2 + bx + c$ (avec $a\neq 0$ a une racine si et seulement si $\Delta=b^2 - 4ac$ est un carré. Les deux racines sont alors $\frac{-b\pm\sqrt{\Delta}}{2a}$. Dans $\Z/n\Z$, on peut effectuer la mise sous forme canonique d'un polynôme de degré $2$ et ensuite résoudre selon $n$ si $n$ est impair.

\item Les relations de Viète et Newton sont toujours vraies dans $\Z/p\Z$.
\item Dans $\Z/p\Z[X]$ toute l'arithmétique des polynômes persiste : pgcd, décomposition en produit d'irréductible

\item Le lemme d'Hensel permet de remonter de racines de $P$ modulo $p$ à des racines de $P$ modulo $p^k$. Si $P$ a une racine dans $\Z/p\Z$ notée $y$ telle que $P'(y)\neq 0 \pmod{p}$, alors pour tout $k\in \N^*$, il existe $z\equiv y\pmod{p}$ tel que $P(z)\equiv 0 \pmod{p^k}$.
\end{itemize}


\subsubsection{Autour des polynômes}

\begin{exo}
Soit $p$ un nombre premier, montrer que $(p - 1)!\equiv -1\pmod{p}$.
\end{exo}

\begin{exo}
Soit $p$ un nombre premier. Déterminer le reste de la division euclidienne de $1^k + \dots + (p - 1)^k$ pour tout entier $k\geq 0$  par $p$ (sans racine primitive).
\end{exo}

\begin{exo}
Soit $n,m\geq 2$ tels que pour tout entier $k\in \{1,\dots, n\}$, $k^n\equiv 1\pmod{m}$. Montrer que $m$ est premier et $n=m - 1$.
\end{exo}

\begin{exo}
Soit $p$ un nombre premier. Combien y a-t-il de polynômes unitaires dans $\Z/ p\Z$ de degré $p - 2$ admettant exactement $p - 2$ racines distinctes, et dont les coefficients sont deux à deux distincts et non nuls ?
\end{exo}


\subsubsection{Irréductibilité}


\begin{exo}
 Soit $P$ un polynôme unitaire, dont tous les coefficients sauf le coefficient dominant sont divisibles par $p$, et dont le coefficient constant n'est pas divisible par $p^2$. Montrer que $P$ est irréductible.
\end{exo}


\begin{exo}
 Soit $n\geq 2$ un entier, montrer que le polynôme $x^{n} + 5x^{n - 1} + 3$ est irréductible.
\end{exo}


\begin{exo}
 Pour $A$ un polynôme à coefficients entiers non nul, on note $c(A)$ le pgcd des coefficients entiers. Soit $P$ et $Q$ deux polynômes à coefficients entiers non nuls, montrer que $c(PQ)=c(P)c(Q)$.
\end{exo}


\subsubsection{Les exos de l'année}


\begin{exo}
Une suite $a_1,a_2,a_3,\dots$ d’entiers strictement positifs vérifie $a_1>5$ et $a_{n + 1}=5 + 6 + \dots + a_n$ pour tout entier strictement positif $n$. On suppose que, quelle que soit la valeur de $a_1$, cette suite contient toujours un multiple de $p$. Montrer que $p=2$.

Bonus : montrer que $p=2$ vérifie l'énoncé.
\end{exo}


\begin{exo}
 Soit $p$ un nombre premier. Tristan et Abigaëlle jouent au jeu suivant. Tristan écrit un entier $X\geq 1$ au tableau et donne une suite $(a_n)_{n \in \mathbb N}$ 1 d’entiers strictement positifs à Abigaëlle. Abigaëlle joue alors une
infinité de tours de jeu. Lors du  $n^{\textit{ème}}$tour de jeu,
\begin{center}
Abigaëlle remplace, selon son choix, l’entier $Y$ écrit au tableau par l’entier $Y + a_n$ ou par l’entier $Y \cdot a_n$.
\end{center}
Abigaëlle gagne si, au bout d’un nombre fini de tours de jeu, elle parvient à écrire au tableau un multiple de $p$.
Déterminer si elle peut réussir à gagner quels que soient les choix initiaux de Tristan, dans chacun des deux cas
suivants :
\begin{enumerate}[label=\alph*)]
\item $p=10^9 + 7$;
\item $p=10^9 + 9$.
\end{enumerate}
\textit{Remarque :} On admettra que $10^9 + 7$ et $10^9 + 9$ sont premiers.
\end{exo}

\begin{exo}
Pour tout nombre premier $p$, il existe un royaume de $p$-Landia, qui contient $p$ îles numérotées de $1$ à $p$. Deux villes distinctes $m$ et $n$ sont alors reliées par un pont si $p$ divise $(m^2 - n + 1)(n^2 - m + 1)$. Deux ponts peuvent se superposer, mais il est impossible de passer d'un pont à un autre directement.

Montrer qu'il existe une infinité de $p$ pour lesquels il y a deux villes du royaume de $p$-Landia qui ne sont pas connectées par une suite de ponts.
\end{exo}

\subsubsection{Pot pourri}

\begin{exo}
Soit $g,f$ deux polynômes à coefficients entiers tels que $f$ divise $g$ et $f$ et $g$ sont à coefficients dans $\{1,2022\}$. Montrer que $\deg(f) + 1$ divise $deg(g) + 1$.
\end{exo}

\begin{exo}(USA TST for EGMO 2019, P3)
Soit $n$ un entier strictement positif tel que
$$\frac{1^k + 2^k + \dots n^k}{n}$$

est un entier pour tout $k\in\{1, \dots, 99\}$. Montrer que $n$ ne possède pas de diviseurs compris entre $2$ et $100$.
\end{exo}


\begin{exo}
Déterminer tous les nombres premiers $p$ pour lesquels il existe un unique $a$ dans $\{1,\dots, p\}$ tel que $a^3 - 3a + 1$ est divisible par $p$.
\end{exo}

\begin{exo}[Existence de racines primitives]
Soit $p$ un nombre premier, $d$ un diviseur positif de $p - 1$. Montrer qu'il y a au plus $\phi(d)$ éléments dont l'ordre multiplicatif vaut $d$ modulo $p$. En déduire qu'il y a un élément d'ordre exactement $p - 1$ modulo $p$.
\end{exo}


\begin{exo}
Soit $p$ un nombre premier impair et $a$ un résidu quadratique modulo $p$, montrer que $a$ est un résidu quadratique modulo $p^k$ pour tout entier $k\geq 0$.
\end{exo}


\subsubsection{Solutions}


\begin{sol}
Comme $X^{p - 1} - 1=\prod\limits_{a\in (\Z/p\Z)^*}(X - a)$, on en déduit par Viète que $(p - 1)!\equiv \prod \limits_{a\in (\Z/p\Z)^*}a\equiv (-1)^{p - 1}\times (-1)$ donc $(p - 1)!\equiv -1$ si $p$ est impair, et sinon, $p=2$ donc $(p - 1)!\equiv 1\equiv -1\pmod{p}$ ce qui prouve le théorème de Wilson.
\end{sol}

\begin{sol}
Rappelons que $X^p - X=\prod\limits_{k=0}^{p - 1}(X - k)$ dans $\Z/p\Z$.

On utilise les relations de Newton appliquées aux éléments $0,\dots, p - 1$ dans $\Z/p\Z$. Notons $\sigma_k$ pour $k\in \N$ les polynômes symétriques élémentaires (avec $\sigma_{k}=0$ si $k\geq n$, et $S_k$ les sommes des puissances $k$-ièmes pour $k\in \N$, on a pour tout $k\geq 1$ :


$$\sum_{r=0}^{k - 1}(-1)^r\sigma_r S_{k - r} + (-1)^k k\sigma_k=0$$

Or on peut calculer facilement les $\sigma_k$ à partir du polynôme $X^p - X$ : $\sigma_0=0$, $\sigma_i=0$ pour $i\geq 1$, sauf si $i=n - 1$ : dans ce cas $(-1)^{p - 1}\sigma_{p - 1}=-1$, donc $\sigma_{p - 1}=-1$ (c'est vrai si $p$ impair, et cela se vérifie aisément si $p=2$).
En particulier les relations de Newton deviennent $S_{k} + (-1)^{p - 1}\sigma_{p - 1}S_{k - (p - 1)}$ si $k\geq p$, $S_k=0$ si $0<k<p - 1$, $S_{p - 1} + (-1)^{p - 1}(p - 1)\sigma_{p - 1}$.

Comme $S_0=0$, $S_k=S_{k - (p - 1)}$ si $k>p - 1$, $S_k=0$ si $0<k<p - 1$, et $S_{p - 1}=-1$, on obtient que par récurrence immédiate que le reste de la vision euclidienne de $1^k + \dots + (p - 1)^k$ vaut $p - 1$ si $k$ est divisible par $p - 1$, $0$ sinon (pour $k=0$, la somme vaut $S_0 - 1$).



\end{sol}

\begin{sol}
Soit $d$ un diviseur de $m$, on a bien pour tout $k\in \{1,\dots, d\}$, $k^n\equiv 1 \pmod{d}$. Ainsi si $(n,m)$ vérifie l'énoncé, $(n,d)$ aussi pour tout diviseur $d$ de $m$.

Supposons que $(n,p)$ vérifie l'énoncé pour $p$ premier. Déjà notons que $n<p$ car $p^n\equiv 0 \pmod{p}$ : comme $n\geq 2$, on obtient que $p\geq 3$. Le polynôme $X^n - 1$ a exactement $n$ racines : les éléments de $\{1,\dots, n\}$. En particulier dans $\Z/p\Z$, $X^n - 1=\prod\limits_{i=1}^n(X - i)$.

Les relations de Viète donnent que $\sum\limits_{i=1}^n i=0\pmod{p}$. Or $\sum\limits_{i=1}^n i\equiv \frac{n(n + 1)}{2}\equiv 0\pmod{p}$. Ainsi $n\equiv 0\pmod{p}$ ou $n\equiv  - 1\pmod{p}$. Comme $2\leq n <p$, on obtient que $n=p - 1$.\medskip

En particulier, cela implique que si $(n,m)$ est solution de l'énoncé, pour tout diviseur premier $p$ de $m$, $n=p - 1$. En particulier, $m$ a au plus un facteur premier : $m$ est de la forme $p^k$ avec $p$ premier et $k\geq 1$. Pour obtenir le résultat voulu, il suffit de montrer que $k<2$, i.e. par la remarque préliminaire que $(p - 1,p^2)$ n'est pas solution.

Supposons que $(n,m)=(p - 1,p^2)$ vérifie l'énoncé. Notons déjà que $p - 1\geq 2$, donc $p\geq 3$. On a alors que $(p - 1)^{p - 1}\equiv 1 \pmod{p^2}$. Or d'après le binôme de Newton,

$$(p - 1)^{p - 1}\equiv \sum\limits_{k=0}^{p - 1}\binom{p - 1}{k}p^{k}( - 1)^{p - 1 - k}\equiv (-1)^{p - 1} + (p - 1)(-1)^{p - 2}p\equiv 1 + p\pmod{p^2}$$


\end{sol}

\begin{sol}
Soit $P$ un polynôme vérifiant l'énoncé. $P$ ne peut pas avoir $0$ comme racine, donc $P$ a toutes les racines entre $1$ et $p - 1$, sauf $1$, notée $a$. Ainsi $$P_a(X)=\frac{X^{p - 1} - 1}{X - a}=\frac{X^{p - 1} - a^{p - 1}}{X - a}=X^{p - 2} + aX^{p - 3} + \dots + a^{p - 2}$$.

Parmi les $p - 1$ candidats pour vérifier l'énoncé, il reste à vérifier ceux qui ont des coefficients deux à deux distincts non nuls. Si $P_a$ a des coefficients deux à deux distincts, alors comme $P_a$ est unitaire, l'ordre de $a$ modulo $p$ vaut au moins $p - 1$, donc par Fermat, l'ordre vaut exactement $p - 1$. En particulier $a$ est une racine primitive. Réciproquement si $a$ est une racine primitive, $1,a,\dots, a^{p - 2}$ sont distincts modulo $p$, donc $P_a$ convient.

Ainsi il y a autant de polynômes que de racines primitives modulo $p$, i.e. $\phi(p - 1)$.
\end{sol}

\begin{sol}
Supposons que $P$ est réductible. Il existe alors $Q$ et $R$ des polynômes de $\Z[X]$ vérifiant $0<deg(Q)<deg(P)$ et $0<deg(R)<deg(P)$ tels que $P=QR$.

Regardons cette identité dans $\Z/p\Z[X]$ : on a $X^n=\overline{Q}\overline{R}$ où $\overline{Q}$ désigne le polynôme $Q$ vu dans $\Z/p\Z[X]$. Tous les facteurs irréductibles de $\overline{Q}$ sont de la forme $X^k$, donc $\overline{Q}=X^k$ pour $k\geq 0$. On obtient alors que $\overline{R}=X^{n - k}$.

Si $0<k<n$, on a $\overline{Q}(0)=\overline{R}(0)=0$, donc $p$ divise $R(0)$ et $Q(0)$, donc $p^2$ divise $P(0)$, ce qui est contradictoire. Ainsi $k=0$ ou $n$, donc $\overline{Q}=X^n$ ou $\overline{R}=X^n$ (on ne traitera que le premier cas par symétrie). Or $deg(Q)\geq deg(\overline{Q})=n$ ce qui est absurde.

Ainsi $P$ est bien irréductible.
\end{sol}

\begin{sol}
Supposons que le polynôme est réductible : il existe un tel couple $(g,h)$ tel que $X^{n} + 5X^{n - 1} + 3=g(X)h(X)$ avec $0<deg(g),deg(h)<n$. En passant dans $\Z/3\Z$, $X^{n - 1}(X - 1)=\overline{g}\overline{h}$ avec $\overline{g}$ et $\overline{h}$ les polynômes correspondants à $g$ et $h$ mais vu dans $\Z/3\Z[X]$. Si $0$ est racine de $\overline{g}$ et $\overline{h}$, alors $h(0)\equiv g(0)\equiv 0\pmod{3}$ donc $9$ divise $f(0)=3$ ce qui est absurde. Ainsi $X^{n - 1}$ divise $\overline{g}$ ou $\overline{h}$, supposons que $X^{n - 1}$ divise $\overline{g}$ par symétrie. Dans ce cas $g$ est de degré supérieur ou égal à $n - 1$, donc $h$ de degré inférieur ou égal à $1$ donc $1$. En particulier $h$ étant à coefficient entier et unitaire, $h$ est de coefficient dominant $\pm 1$ donc admet une racine entière, donc $f$ admet une racine entière notée $p$. Notons que $p$ divise $3$ donc $p$ est impair : $f(p)\equiv 1 + 5 + 3\equiv 1\pmod{2}$ ce qui contredit le fait que $f(p)=0$ et prouve le résultat voulu.
\end{sol}

\begin{sol}
Posons $P=c(P)P_1$ et $Q=c(Q)Q_1$ avec $P_1$ et $Q_1$ dans $\Z[X]$ : on a alors $c(P_1)=1=c(Q_1)$. Ainsi $PQ=c(P)c(Q) P_1Q_1$ : il suffit alors de montrer que $c(P_1Q_1)=1$ pour avoir $c(PQ)=c(P)c(Q)$.

Supposons que $c(P_1Q_1)\neq 1$, il existe alors un facteur premier $p$ divisant tous les coefficients de $P_1Q_1$. En particulier en passant dans $\Z/p\Z[X]$, $\overline{P_1}\overline{Q_1}=0$, donc $\overline{P_1}=0$ ou $\overline{Q_1}=0$ (sinon on peut voir que le produit des termes de coefficient dominant de $\overline{P_1}\overline{Q_1}$ apparaîtrait dans le produit). Ceci donne que $p$ divise $c(P_1)$ ou $c(Q_1)$, ce qui est contradictoire. Ainsi on a bien $c(P_1Q_1)=1$ ce qui conclut.
\end{sol}

\begin{sol}
Supposons $p\neq 2$, notons que par récurrence immédiate $a_n\geq 5$ donc l'énoncé est bien posé.
On a $a_{n + 1}=\frac{a_n(a_n + 1)}{2} - 10$

On aimerait idéalement qu'aucun terme ne soit divisible par $p$. Le plus simple serait que tous les $a_n$ soient congrus à $a_0$ modulo $p$, et que $a_1$ ne soit pas divisible par $p$.

On cherche donc à résoudre l'équation $x\equiv \frac{x(x + 1)}{2} - 10\pmod{p}$ qui est équivalent à $2x\equiv x^2 + x - 20\pmod{p}$, soit à $x^2 - x - 20\equiv 0\pmod{p}$. Or $x^2 - x - 20=(x - 5)(x + 4)$ (ce qu'on peut retrouver en appliquant les formules de résolution d'un polynôme de degré $2$ modulo $p$. En particulier si $a_1\equiv 5{p}$ ou $a_1\equiv  - 4\pmod{p}$, alors $a_n\equiv 5\pmod{p}$ pour tout $n\geq 1$, ou $a_n\equiv -4\pmod{p}$ pour tout $n\geq 1$. En prenant $a_1=10p - 4>5$, $a_n$ n'est pas divisible par $p$ car $p\neq 2$ pour tout $n\geq 1$. Ainsi on a bien $p=2$.

Pour $p=2$, supposons qu'il existe $a_1$ tel que la suite $(a_n)_{n\geq 1}$ ne contient aucun terme pair. Dans ce cas, posons $a_1=5 + 2^{l}k$ avec $l\geq 1$ (car $a_1$ est impair) et $k$ impair.

On a alors si $a_n$ est de la forme $5 + 2^b c$ avec $c$ impair et $b\geq 1$,

$$a_{n + 1}=\frac{(5 + 2^bc)(6 + 2^bc)}{2} - 10=(5 + 2^b c)(3 + 2^{b - 1}c) - 10=5 + 2^{b - 1}(5c + 6c + 2^{b - 1}c)$$

$a_{n + 1}$ est de la forme $5\times 2^{b - 1}c'$ avec $c'$ impair (valant $(5c + 6c + 2^{b - 1}c)$).

Par récurrence immédiate, on obtient que $a_{m + 1}=5 + 2^{l - m}c_m$ pour $0\leq m \leq l$ avec $c_m$ un entier impair. Ainsi $a_{l + 1}$ est pair, ce qui donne le résultat voulu.


\end{sol}

\begin{sol}
Ici Abigail a beaucoup trop de possibilité : à chaque fois elle peut potentiellement obtenir deux résultats différents, donc si Tristan a un espoir de gagner, il aimerait qu'au $n$ - ième tour, Abigail n'ait qu'une possibilité modulo $p$ : $b_n$. Au départ, il choisir $X=b_0$. Puis, à partir de $b_n$, pour n'avoir qu'une possibilité, il voudrait choisir $a_{n + 1}$ tel que $a_{n + 1}b_n\equiv a_{n + 1} + b_n\pmod{p}$, i.e. $a_{n + 1}\equiv \frac{b_n}{1 - b_n}\pmod{p}$. En particulier, on aura $b_{n + 1}\equiv \frac{b_n^2}{b_n - 1}\pmod{p}$. Ceci étant dit, on remarque qu'avoir $b_n\equiv 1\pmod{p}$ est un problème pour Tristan.

Donnons ainsi la stratégie suivante pour Tristan : Tristan un $X$ non congru ni à $0$ ni à $1$ mod $p$. Il pose $b_0=X$. Puis tant que $b_n$ ne vaut pas $1$ modulo $p$, il pose $a_{n + 1}\equiv \frac{b_n}{b_n - 1}\pmod{p}$ et $b_{n + 1}\equiv \frac{b_n^2}{b_n - 1}\pmod{p} $. On montrer par récurrence immédiate qu'Abigail, au bout de $n$ choix, aura écrit $b_n\pmod{p}$ au tableau.

On peut espérer que pour tout $n$, $a_n$ soit toujours défini et $b_n$ soit toujours différent de $0$ et $1$ modulo $p$. Notons que si $b_n\neq 0,1\pmod{p}$, $b_{n + 1} \neq 0 \pmod{p}$. De plus $b_{n + 1}\equiv 1 \pmod{p}$ équivaut à $b_n^2 - b_n + 1\equiv 0\pmod{p}$. Ce polynôme a une racine modulo $p$ si et seulement si son discriminant, $-3$ est un carré modulo $p$.

Or par réciprocité quadratique $$\left(\dfrac{-3}{p}\right)=\left(\dfrac{-1}{p}\right)\left(\dfrac{3}{p}\right)=(-1)^{\frac{p-1}{2}}(-1)^{\frac{2(p-1)}{4}}\left(\dfrac{p}{3}\right)=\left(\dfrac{p}{3}\right)$$.

En particulier cela nous donne la question $a$ : comme $p\equiv 2 \pmod{3}$, $\left(\dfrac{-3}{p}\right)=-1$ donc l'équation $x^2-x + 1\equiv 0\pmod{p}$ n'a pas de solution : ainsi $b_{n + 1}$ ne peut valoir $1$ modulo $p$, donc Tristan gagne.

Pour la question $b$, malheureusement on ne peut pas assurer que le procédé précédent marche. Une option serait que $b_n$ boucle rapidement. Comme $b_{n + 1}\equiv b_n + a_{n + 1}\equiv b_na_{n + 1}$, si $b_{n + 1}\equiv b_n$, $a_{n + 1}\equiv 0 \pmod{p}$ donc $b_{n + 1}\equiv 0\pmod{p}$. Le mieux qu'on puisse espérer est donc d'avoir une boucle de taille $2$, i.e. $b_2\equiv b_0\pmod{p}$.

Or $$b_{2}\equiv \frac{b_1^2}{b_1 - 1}\equiv \frac{\frac{b_0^4}{(b_0 - 1)^2}}{\frac{b_0^2}{b_0 - 1} - 1}$$

L'équation (avec $x\not\equiv 0 \pmod{p}$) $ x\equiv \frac{\frac{x^4}{(x - 1)^2}}{\frac{x^2}{x - 1} - 1}$ est équivalente à $1\equiv \frac{x^3}{x^2(x - 1) - (x - 1)^2}\equiv \frac{x^3}{x^3 - 2x^2 + 2x - 1}$ donc à $2x^2 - 2x + 1\equiv 0\pmod{p}$  et celle-ci a un sens sous réserve d'avoir $x$ différent de $0$ et $1$ mod $x$ et $\frac{x^2}{x - 1}$ différent de $0$ ou $1$ mod $p$.

L'équation $2x^2 - 2x + 1\equiv 0\pmod{p}$ est une équation de degré $2$, qui a une solution modulo $p$ si et seulement si son discriminant qui vaut $-4$ est un carré modulo $p$. Or $-4$ est un carré si et seulement si $-1$ en est un. Or dans la $b$, $p\equiv 1\pmod{4}$ donc $-1$ est bien un carré : il existe une racine modulo $p$ de $2x^2 - 2x + 1$ qu'on notera $y$. $0$ et $1$ n'étant pas racine $y$ est différent de $0$ ou $1$. De plus, on ne peut pas avoir $\frac{y^2}{y - 1}\equiv 0$ ou $1$ modulo $p$ : pour $0$ c'est clair, pour $1$, cela impliquerait avoir $y^2 - y\equiv 1$, donc $2y^2 - 2x\equiv 2$. Or $2y^2 - 2y\equiv -1$, et $2\not \equiv -1 \pmod{p}$. Ainsi si on prend $X=y$, on obtient par récurrence immédiate que $b_n\equiv b_0$ si $n$ est pair, $b_1$ si $n$ est impair, et est différent de $0$ et $1$ pour tout $n\geq 0$. Ainsi Abigail ne peut gagner : dans les deux cas Tristan gagne.
\end{sol}

\begin{sol}
On peut modéliser le problème par un graphe, dont les villes sont les éléments de $\Z/p\Z$ et $(m,n)$ sont reliés si et seulement si $p$ divise $(m^2 - n + 1)(n^2 - m + 1)$.

Plus simplement on définit $f : \Z/p\Z\mapsto \Z/p\Z$ qui à $x$ associe $x^2 + 1$. Deux sommets $(m,n)$ sont reliés si et seulement si $m=f(n)$ ou $n=f(m)$. On peut donc tracer les arêtes de la forme $(m,f(m))$ et obtenir le graphe voulu : il a donc $n$ arêtes.

Idéalement on aimerait montrer que pour des bons nombres premiers, le graphe n'est pas connexe. Pour cela, il suffit de trouver une infinité de nombres premiers $p$ pour lesquels il y a $n - 2$ arêtes : il suffit donc de montrer qu'il y a une infinité de nombres premiers $p$ pour lesquels on a deux arêtes de la forme $(a,a)$, i.e. $f$ a deux points fixes.


Or $f(n)\equiv n$ est équivalent à $n^2 - n + 1\equiv 0$, dont le discriminant vaut $-3$.

Or par réciprocité quadratique si $p>3$,  $$\left(\dfrac{-3}{p}\right)=\left(\dfrac{-1}{p}\right)\left(\dfrac{3}{p}\right)=(-1)^{\frac{p-1}{2}}(-1)^{\frac{2(p-1)}{4}}\left(\dfrac{p}{3}\right)=\left(\dfrac{p}{3}\right)$$

Ainsi si $p\equiv 1\pmod{3}$, $-3$ est un carré non nul, donc $f(n)\equiv n$ a deux solutions modulo $p$. Ainsi en enlevant les deux boucles, il y a $n - 2$ arêtes, donc le graphe n'est pas connexe : on a le résultat voulu.
\end{sol}

\begin{sol}
On regarde l'identité dans $\Z/2021\Z$ : on a que $\overline{f}$ divise $\overline{g}$. Or $\overline{f}=\sum\limits_{k=0}^{deg(f)}X^k$ et $\overline{g}=\sum\limits_{k=0}^{deg(g)}X^k$.
Posons la division euclidienne de $deg(g) + 1$ par $deg(f) + 1$ : on a $deg(g) + 1=q(deg(f) + 1) + r$. On a $$\overline{g}=\overline{f}\sum\limits_{k=0}^{q - 1}X^{r + k(deg(f) + 1)} + \sum\limits_{l=0}^{r - 1}X^l$$

Supposons $r\neq 0$. Ainsi $\overline{f}$ divise $\sum\limits_{l=0}^{r - 1}X^l$ dont le degré vaut $r - 1<r\leq deg(f)$. En particulier, comme $f$ est unitaire, cela est impossible. Ainsi $r=0$, donc $deg(g) + 1$ est divisible par $deg(f) + 1$
\end{sol}

\begin{sol}
Déjà essayons de voir ce qu'on peut faire avec cette hypothèse : le résultat est toujours vrai pour $k=0$. Comme la condition est linéaire on peut en déduire que $\frac{P(1) + P(2) + \dots P(n)}{n}$ est entier
ceci étant vrai pour tout polynôme $P$ à coefficients entiers de degré au plus $99$. La question maintenant est : à quel polynôme appliquer cela de façon astucieuse ?

Pour $P$ un polynôme de la forme $P(X)=Q(X) - Q(X - 1)$ pour $Q$ à coefficients entiers de degré au plus $99$. La somme précédente se téléscopage et donne que $\frac{P(n) - P(0)}{n}$ est un entier. En fait pas de chance, on sait déjà que $n - 0=n$ divise $P(n) - P(0)$. On aimerait donc avoir quelque chose de plus fort. Pour cela, on pourrait chercher à appliquer le résultat non pas à $Q(X + 1) - Q(X)$, mais à un polynôme ressemblant à $\frac{Q(X) - Q(X - 1)}{d}$ avec $d$ entier bien choisi, divisant tous les coefficients de $Q(X + 1) - Q(X)$. Le plus facile pour $d$ est de prendre $d=p$ un nombre premier, reste à trouver comment avoir un polynôme dont beaucoup de coefficients sont divisibles par $p$. Fixons $p$ un nombre premier entre $1$ et $99$, on considère alors logiquement $Q=X^p$, on a $Q(X + 1) - Q(X)=\sum \limits_{i=0}^{p - 1}\binom{p}{k}X^k=1 + pR(X)$ avec $R(X)=\sum \limits_{i=1}^{p - 1}\frac{\binom{p}{k}}{p}X^k$. Il est connu que $p$ divise $\binom{p}{k}$ si $1\leq k \leq p - 1$ (ce qui se prouve aisément par la formule comité président). Ainsi le polynôme $\frac{Q(X + 1) - Q(X) - 1}{p}=R(X)$ est à coefficient entiers.

On applique alors l'hypothèse à $R(X - 1)$ : via un téléscopage, on obtient que $\frac{Q(n) - Q(0) - n}{np}=\frac{n^p - n}{np}$ est entier. En particulier, si $p$ divise $n$, $np$ divise $n^p$ donc $np$ divise $n$ ce qui est contradictoire. Ainsi $n$ n'est divisible par aucune entier premier entre $1$ et $99$ donc aucun entier entre $2$ et $100$ (car $100$ n'est pas premier).
\end{sol}

\begin{sol}
Soit $p$ vérifiant l'énoncé, et $a$ l'unique solution de $a^3 - 3a + 1\equiv 0 \pmod{p}$. Notons que $a\neq 0$ donc en posant $b\equiv \frac{1}{a}\pmod{p}$, $b^3 - 3b^2 + 1\equiv 0$ donc $(b - 1)^3 - 3b + 2\equiv 0$ donc $(b - 1)^3 - 3(b - 1) - 1\equiv 0$. En particulier $-(b - 1)$ est aussi une racine modulo $p$ de $X^3 - 3X + 1$, donc $a\equiv -(b - 1)$, donc $a^2\equiv -(ab - a)\equiv a - 1$ donc $a^2 - a + 1\equiv 0 \pmod{p}$. Or $a^3-3a + 1=(a^2 - a + 1)(a + 1) - 3a$, donc en passant modulo $p$, $3a\equiv 0\pmod{p}$. Or clairement $a$ ne peut être nul modulo $p$, donc $p=3$.


Réciproquement si $p=3$, il y a une unique solution de $a^3 + 1\equiv 0\pmod{3}$ : $a\equiv 2$. Ainsi $p=3$ est le seul nombre premier qui convient.
\end{sol}


\begin{sol}
Soit $d$ divisant $p - 1$. S'il n'existe pas d'éléments d'ordre $d$, alors l'énoncé est vrai. Sinon, il existe $y$ élément d'ordre $d$. Tout élément d'ordre $d$ est racine de $P=X^d - 1$, un polynôme qui a au plus $d$ racines. Or les $y^k$ pour $0\leq k\leq d - 1$ sont des racines de $P$ : ainsi tout élément d'ordre $d$ est une puissance de $y$. Or si $k$ n'est pas premier avec $d$, $y^k$ est d'ordre au plus $\frac{d}{PGCD(d,k)}$ donc pas d'ordre $k$ : il y a au plus $\phi(d)$ élément d'ordre $d$.


Supposons qu'il n'y a pas d'élément d'ordre $p - 1$. Tout élément inversible est d'ordre $d$ avec $d$ divisant $p - 1$ par petit Fermat. Il y a donc au plus $\sum\limits_{d|p - 1, d\neq p - 1}\phi(d)=-\phi(p - 1) + \sum\limits_{d|p - 1}\phi(d)=-\phi(p - 1) + p - 1<p - 1$ éléments inversibles.

En effet $\sum_{d|k}\phi(d)=k$ car entre $1$ et $k$, il y a exactement $\phi(d)$ éléments dont le pgcd avec $k$ vaut $d$ pour tout $d$ divisant $k$.

Ainsi on a une contradiction : il existe bel et bien un élément d'ordre $p$.
\end{sol}

\begin{sol}
Déjà rappelons que $a$ est premier avec $p$. Il existe une racine $y\neq 0$ au polynôme $P(X)=X^2 - a$. Or $P'(y)\equiv 2y\not\equiv 0 \pmod{p}$. En particulier, d'après Hensel, $X^2 - a$ admet une racine modulo $p^k$ pour tout $k\geq 0$, d'où le résultat.
\end{sol}
