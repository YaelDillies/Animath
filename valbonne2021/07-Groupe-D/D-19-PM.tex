En introduction de ce TD ont été faits des rappels sur les résidus quadratiques. Les deux premiers exercices sont des applications de cette notion pour la mettre en pratique. Un cours plus complet sur les résidus quadratiques peut être retrouvé au chapitre 4 de
\url{https://maths-olympiques.fr/wp-content/uploads/2017/09/arith_zn.pdf}.

Par la suite, des rappels ont également été faits sur le théorème de Zsigmondy, appliqué dans les exercices 3, 4 et 5. On retrouvera un cours plus complet au chapitre 6 de \url{https://maths-olympiques.fr/wp-content/uploads/2017/09/arith_lte.pdf}.

Les exercices 1 à 7 ont été traités pendant le cours.

\subsubsection{Exercices}

\begin{exo}
Soit $n>1$ un entier et $p$ un diviseur premier de $2^{2^n}+1$. Montrer que $v_2(p-1)\ge n+2$.
\end{exo}


\begin{exo}[Corée 2012]
Trouver tous les triplets d'entiers strictement positifs $(m,n,p)$ avec $p$ premier, tels que $2^mp^2+1=n^5$.
\end{exo}


\begin{exo}
Trouver tous les $a,b,c\in\N$ tels que $2^a\cdot3^b=7^c-1$.
\end{exo}


\begin{exo}
Soient $b,m,n\in\N$ avec $b\ne 1$ et $m\ne n$. Montrer que si $b^m-1$ et $b^n-1$ ont les mêmes facteurs premiers, alors $b+1$ est une puissance de $2$.
\end{exo}


\begin{exo}
Trouver les $a,b,c\in\N$ tels que $11^a+3^b=c^2$.
\end{exo}


\begin{exo}
Trouver toutes les paires d'entiers naturels $a,b$ tels que $b^a\mid a^b-1$.
\end{exo}


\begin{exo}
Soient $a,b>1$ impairs tels que $a+b=2^l$. Trouver les $k \in\N^*$ tels que $k^2\mid a^k+b^k$.
\end{exo}


\begin{exo}
Trouver les paires d'entiers naturels non nuls $m,n$ tels que $mn\mid 3^m+1$ et $mn\mid 3^n+1$.
\end{exo}


\begin{exo}[BxMO 2010, 4]
Trouver tous les quadruplets $(a,b,p,n)$ d'entiers strictement positifs avec $p$ premier tels que $a^3+b^3=p^n$
\end{exo}


\begin{exo}[IMO 1999, P4]
Trouver tous les nombres premiers $p$ et tous les entiers $x$ tels que $1\le x\le 2p$ et $x^{p-1}$ divise $(p-1)^x+1$
\end{exo}

\subsubsection{Solutions}


\begin{sol}
Soit $p$ un diviseur premier de $2^{2^n}+1$. On a $2^{2^n}\equiv -1[p]$, donc $2^{2^{n+1}}\equiv 1[p]$, donc l'ordre de $2$ modulo $p$ est une puissance de $2$ divisant $2^{n+1}$ mais pas $2^n$, soit $2^{n+1}$. D'après le petit théorème de Fermat, $2^{p-1}\equiv 1[p]$, donc par propriété de l'ordre, $2^{n+1}\mid p-1$. Il suffit donc de montrer que 2 est un résidu quadratique modulo p, puisque $x$ vérifiant $x^2\equiv 2[p]$ sera d'ordre $2^{n+2}$, donc $2^{n+2}\mid p-1$ comme précédemmemt. Mais comme $n+1\ge 3$, on sait que $p\equiv1[8]$ par ce qui précède, donc $2$ est bien un résidu quadratique.
\end{sol}


\begin{sol}
On écrit $2^m p^2 = n^5-1 = (n-1)(n^4+n^3+n^2+n+1)$. Comme $n^4+n^3+n^2+n+1$ est impair, $2^m\mid n-1$.
\newline Ainsi, $p^2 = \frac{n-1}{2^m} (n^4+n^3+n^2+n+1)$. Or $p^2$ ne peut s'écrire comme produit de deux termes que sous la forme $p\cdot p$ ou $1\cdot p^2$, et $\frac{n-1}{2^m} < n^4+n^3+n^2+n+1$, donc $\frac{n-1}{2^m} = 1$ et $p^2 = n^4+n^3+n^2+n+1$.
\newline On a donc $n = 2^m +1$ et $p^2 = (2^m +1)^4+(2^m +1)^3+(2^m +1)^2+(2^m +1)+1$. Si $m\ge 2$, alors $p^2\equiv 5[8]$, mais $5$ n'est pas un résidu quadratique modulo $8$, absurde. Ainsi $m=1$. La seule solution est donc $(1,3,11)$.
\end{sol}


\begin{sol}
D'après le théorème de Zsigmondy, comme $7^1-1=2\cdot 3$, pour $c\ge 3$ (attention, $c=2$ est un cas d'exception de Zsigmondy car $7+1=8=2^3$), $7^c-1$ admet un diviseur premier différent de $2$ et $3$. Donc nécessairement, $c\le 2$ et les solutions sont $(1,1,1)$ et $(4,1,2)$.
\end{sol}


\begin{sol}
Les conditions ne peuvent être réalisées que dans un cas particulier de Zsigmondy.
\end{sol}


\begin{sol}
En regardant modulo $3$, on trouve que $a$ est pair. Ensuite, en posant $a=2a'$ et avec des manipulations algébriques,
$$3^b=(c-11^{a'})(c+11^{a'})$$
Ainsi par utilisation de la décomposition en facteurs premiers $c-11^{a'}\mid c+11^{a'}$ soit $c-11^{a'}\mid 2\cdot 11^{a'}$.\newline
Mais comme $c-11^{a'}=3^\beta$, on obtient $\beta=0$, donc $c-11^{a'}=1$ et $c+11^{a'}=3^b$.\newline
On soustrait et on trouve alors
$3^b-1=2\cdot 11^{a'}$. On regarde l'ordre de $3$ modulo $11$ et on trouve qu'il divise $b$ et vaut $5$, donc $3^{5b'}-1=2\cdot 11^{a'}$. Là nous pouvons utiliser Zsigmondy, car $3^5-1=2\cdot 11^2$. Si nous prenons un $b'\ge 2$, nous aurons un facteur premier dans le terme de gauche qui ne sera ni $2$ ni $11$, donc il n'y a pas d'autres solutions que $(4,5,122)$.
\end{sol}


\begin{sol}
Prenons $p$ un diviseur minimal premier de $b$. Alors on a $a^b\equiv 1$ modulo $p$ donc l'ordre de $a$ divise $b$ et $p-1$, ce qui impose qu'il vaut $1$. Ainsi $p\mid a-1$.
Dans le cas où $p$ est impair,
le LTE nous donne $av_p(b)=v_p(b^a)=v_p(a^b-1)=v_p(a-1)+v_p(b)$ donc $(a-1)v_p(b)=v_p(a-1)$ ce qui est absurde en ordre de grandeur.
Ainsi reste le cas $p=2$ ce qui signifie que $b$ est pair et
$$av_2(b)=v_2(b^a)=v_2(a^b-1)=v_2(a-1)+v_2(a+1)+v_2(b)-1$$
Si $v_2(a+1)=1$ c'est absurde. Sinon $v_2(a-1)=1$ et on a un premblème d'estimation qui nous donne que $a=3$ et ça ne marche pas non plus.
\end{sol}


\begin{sol}
Commençons par éliminer le cas $k=1$ qui fonctionne.
Déjà $k$ est impair, car si $k$ était pair, on aurait une contradiction puisque $a^k+b^k\equiv 2[4]$. Ensuite prenons $p$ diviseur premier minimal de $k$ (possible car on a éliminé le cas $k=1$). Puisqu'il est nécessaire que $a$ et $b$ soient premiers entre eux, nous avons
$$\left(\frac ab\right)^{2k}\equiv 1[p]$$
donc en posant $\omega$ l'ordre de $\frac ab$ modulo $p$, $\omega\mid 2k$ et $\omega\mid p-1$ d'où $\omega=1$ ou $2$. On étudie chacun des cas et on aboutit à chaque fois à une contradiction.\newline
Le seul $k$ qui marche est donc $1$.
\end{sol}


\begin{sol}
On suppose $m\ge n$ sans perte de généralité, et on commence par soustraire les deux quantités : on a $$mn\mid 3^{n}(3^{m-n}-1)$$
Or on a $mn\wedge 3=1$ donc
$$mn\mid 3^{m-n}-1$$
En ajoutant la divisibilité avec $3^n+1$ on trouve
$mn\mid 3^{\mid m-2n\mid}+1$.\newline
En poursuivant ce procédé de division euclidienne, on aboutit sur $$mn\mid 3^d\pm1$$ où $d=pgcd(m,n)$ (le signe dépend du nombre de soustractions effectuées dans l'algorithme d'Euclide), donc si nous prenons $p$ un diviseur premier minimal de $d$, nous avons $3^{2d}\equiv 1$ modulo $p$, donc l'ordre de $3$ divise $p-1$ et $2d$. Par choix de $p$, ceci impose que l'ordre vaut $1$ ou $2$, donc que $3\equiv 1$ ou $9\equiv 1$, ce qui n'est possible que si $p=2$. Reste enfin à regarder nos divisibilités initiales modulo 4 et on aboutit à une contradiction : en effet, $mn$ est divisible par $4$ mais $3^m+1\equiv 2$ car $m$ serait pair.
\end{sol}


\begin{sol}
On réécrit $(a+b)(a^2-ab+b^2)=p^n$. Comme $a+b\ge 2$, on sait que $p\mid a+b$. On a $a^2-ab+b^2=(a-b)^2+ab$, donc $a=b=1$ ou $a^2-ab+b^2\ge 2$ et donc $p\mid a^2-ab+b^2$. Plaçons nous dans le second cas. Alors $p$ divise également $(a+b)^2-(a^2-ab+b^2)=3ab$. On se rappelle que $p\mid a+b$, donc $p\mid a$ si et seulement si $p\mid b$.
Ainsi, soit $p=3$, soit $p\mid a$ et $p\mid b$. Dans ce cas, on réécrit $a=pa'$ et $b=pb'$, et on retrouve $a'^3+b'^3=p^{n -3}$, en constatant que $n-3>0$ car $a'\ge 1$ et $b'\ge 1$. On poursuit ce processus jusqu'à une solution $(a_0,b_0,p,n_0)$ pour laquelle $p$ ne divise pas $a_0$. D'après ce qui se précède, on se retrouve dans un des cas particuliers $a_0=b_0=1$, qui donne la solution $(1,1,2,1)$, ou $p=3$. Si $p=3$, alors $3\mid a_0+b_0$, donc $9\mid (a_0+b_0)^2$. Si $9\mid a_0^2-a_0b_0+b_0^2$, alors $9\mid 3a_0b_0$, contradiction. Ainsi, $a_0^2-a_0b_0+b_0^2=3$, donc $(a_0-b_0)^2+a_0b_0=3$, d'où $a_0=1$ et $b_0=2$ ou inversement.
\newline Ainsi on a $3$ solutions possibles quand $p$ ne divise pas $a$, qui sont $(1,1,2,1),(1,2,3,1),(2,1,3,1)$. Finalement, il existe $3$ familles de solutions à l'équation initiale : $(2^k,2^k,2,3k+1)$, $(3^k,2\cdot 3^k,3,3k+2)$, et $(2\cdot 3^k,3^k,3,3k+2)$ avec $k\in\N$, dont on vérifie facilement qu'elles conviennent.
\end{sol}


\begin{sol}
Si $p=2$, $x=1$ ou $x=2$ conviennent. Si $p=3$, $x=1$ ou $x=3$ conviennent. On suppose désormais que $p\ge 5$. Il est clair que $x=1$ est solution pour tout $p$. Si $x>1$, soit $q$ le plus petit diviseur premier de $x$. Comme $(p-1)^x+1$ est impair, $q$ est impair. Ainsi, $(p-1)^x\equiv -1[q]$ et donc $(p-1)^{2x}\equiv 1[q]$. L'ordre de $p-1$ modulo $q$ divise $2x$ mais pas $x$, donc il est pair. D'après le petit théorème de Fermat, il divise également $q-1$. Or les seuls diviseurs de $2x$ inférieurs à $q-1$ sont $1$ et $2$, donc l'ordre vaut $2$. On en déduit que $p-1\equiv -1[q]$ donc $p=q$.
Ainsi, $p\mid x$. D'après le lemme LTE, on obtient finalement
$$0<(p-1)v_p(x)\le v_p(x)+1$$
Il ne peut y avoir égalité que pour $p=3$ et $v_p(x)=1$, soit $x=3$, ce qui convient bien $(3^2\mid 2^3+1)$. Finalement, les solutions sont $(2,2)$, $(3,3)$ et $(p,1)$ pour tout nombre premier $p$.
\end{sol}