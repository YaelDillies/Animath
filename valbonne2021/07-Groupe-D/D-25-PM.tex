La théorie analytique des nombres désigne l'étude asymptotique des nombres entiers, c'est-à-dire le comportement à l'infini, \og vu de loin \fg, de certaines quantités arithmétiques. Ce domaine des mathématiques consiste ainsi en un mélange d'arithmétique et d'analyse (réelle ou complexe). Dans tout ce document, les notations $n$ et $p$ désigneront respectivement un nombre entier et un nombre premier. La somme $\sum_{p \leq x} \dots$ s'étend donc sur tous les nombres premiers $p$ inférieurs à $x$.

Pendant le cours, nous avons introduit les notations de Landau $o$, $O$ et $\sim$. Nous avons admis le théorème des nombres premiers : si $\pi(x)$ désigne le nombre de nombres premiers inférieurs ou égaux à $x$, alors on a l'équivalent
$$\pi(x) \sim \frac{x}{\ln x}$$
Nous en avons déduit que le $n$-ième nombre premier $p_n$ vérifie $p_n \sim n \ln n$.

En utilisant une comparaison série -- intégrale, nous avons prouvé que la série harmonique $H_n = \sum_{k=1}^{n} \frac{1}{k}$ diverge. Plus précisément, nous avons montré l'équivalent $H_n \sim \ln n$ à partir de l'inégalité $\ln(n+1) \leq H_n \leq 1 + \ln n $.

Pendant le TD, nous avons travaillé sur les exercices listés ci-dessous. Pour en savoir plus sur la théorie analytique des nombres, nous renvoyons à la bible d'Hardy et Wright \textit{An Introduction to the Theory of Numbers} ou encore au livre \textit{Les nombres premiers, entre l'ordre et le chaos} de Tenenbaum et Mendès France.


\begin{exo}
On note $d(n)$ le nombre de diviseurs d'un entier $n$ et on pose $F(n) = \displaystyle\sum_{k=1}^{n} d(k)$.
\begin{enumerate}
\item Montrer que $F(n) = \displaystyle\sum_{k=1}^{n} \left\lfloor \frac{n}{k} \right\rfloor$.
\item En déduire que $F(n) \sim n \ln n$. Comment interpréter ce résultat ?
\end{enumerate}
\end{exo}


\begin{exo}
\begin{enumerate}
\item Montrer que pour tout réel $-1 < x < 1$, on a
$$\displaystyle\sum_{n=0}^{+\infty} x^n = \frac{1}{1-x}$$
\item Pour une partie finie $X$ de l'ensemble des nombres premiers, on note $N(X)$ l'ensemble des nombres naturels dont la factorisation en facteurs premiers ne fait intervenir que des nombres premiers de $X$. Montrer que pour tout entier $s$
$$\prod_{p \in X} \frac{1}{1-\frac{1}{p^s}} = \sum_{n \in N(X)} \frac{1}{n^s}$$
\item Pour $s > 1$, on admet que $\zeta(s) = \displaystyle\sum_{n=1}^{+\infty} \frac{1}{n^s}$ est bien définie. Déduire de ce qui précède que
$$\frac{1}{\zeta(s)} = \prod_{p} \left(1-\frac{1}{p^s}\right)$$
Cette formule fondamentale fait le lien entre la fonction $\zeta$ (\og zêta\fg{}) de Riemann et les nombres premiers, c'est-à-dire entre l'analyse et l'arithmétique.
\end{enumerate}
\end{exo}


\begin{exo}[Preuve analytique de l'infinité des nombres premiers]
\begin{enumerate}
\item Montrer que pour tout réel $x \geq 1$ on a l'inégalité
$$\ln x \leq \prod_{p \leq x} \frac{1}{1-\frac{1}{p}}$$
\item En déduire que $\ln x \leq \pi(x) + 1$ et conclure.
\end{enumerate}
\end{exo}


\begin{exo}[dû à Erd\H{o}]
L'objectif est de démontrer que la série $\displaystyle\sum_{p} \frac{1}{p}$ diverge. On raisonne par l'absurde en supposant que cette série converge. Il existe donc un entier $k$ tel que
\[ \sum_{i \geq k+1} \frac{1}{p_i} < \frac{1}{2} \]
Appelons $p_1, \dots, p_k$ les \og petits \fg{} nombres premiers et $p_{k+1}, \dots$ les \og grands \fg{} nombres premiers. Soit $N$ un entier fixé. Notons
\begin{itemize}
	\item $N_g$ le nombre d'entiers inférieurs à $N$ divisibles par un grand nombre premier au moins
	\item $N_p$ le nombre d'entiers inférieurs à $N$ qui n'ont que des petits diviseurs premiers
\end{itemize}
\begin{enumerate}
	\item Montrer que $N_g < \frac{N}{2}$.
	\item En écrivant tous les entiers $n \leq N$ n'ayant que des petits diviseurs premiers sous la forme $n = a_nb_n^2$, avec $a_n$ sans facteur carré, prouver que $N_p \leq 2^k \sqrt{N}$.
	\item En choisissant $N$ convenablement, obtenir une contradiction à l'aide des deux questions précédentes et conclure.
\end{enumerate}
\end{exo}


\begin{exo}
\begin{enumerate}
	\item Montrer que
	\[ \ln(\ln x) - \ln \frac{\pi^2}{6} \leq \sum_{p \leq x} \frac{1}{p} \]
	\item En utilisant des techniques plus avancées (une sommation d'Abel et le résultat de l'exercice suivant), on peut en fait montrer que $\sum_{p \leq x} \frac{1}{p} \sim \ln(\ln x)$. Retrouver cet équivalent à l'aide du théorème des nombres premiers.
\end{enumerate}

\end{exo}


\begin{exo}(Premier théorème de Mertens)
\begin{enumerate}
	\item Montrer par comparaison série -- intégrale que $\ln(n!) = n \ln n - n + O(\ln n)$.
	\item Montrer la formule de Legendre
	\[ v_p(n!) = \sum_{k=1}^{+\infty} \left\lfloor \frac{n}{p^k} \right\rfloor \]
	En déduire l'inégalité
	\[ \frac{n}{p}-1 < v_p(n!) \leq \frac{n}{p} + \frac{n}{p^2}\]
	\item Soit $m$ un entier. Montrer que $\displaystyle\prod_{m+1<p\leq 2m+1} p$ divise $\binom{2m+1}{m}$. En déduire la majoration
	\[ \displaystyle\prod_{m+1<p\leq 2m+1} p \leq 4^m\]
	\item Montrer par récurrence l'inégalité $\displaystyle\prod_{p\leq n} p \leq 4^n$.
	\item En considérant $\ln(n!)$, montrer l'estimation
	\[ \sum_{p \leq x} \frac{\ln p}{p} = \ln x + O(1) \]
\end{enumerate}
\end{exo}


\begin{exo}
On souhaite affiner le résultat de l'exercice 1, dont on reprend les notations.
\begin{enumerate}
	\item Justifier l'existence de $\gamma = \lim\limits_{n \rightarrow +\infty} \left(H_n - \ln n \right)$.
	\item Montrer que $F(n) = 2 \displaystyle\sum_{k=1}^{\lfloor \sqrt{n} \rfloor} \left\lfloor \frac{n}{k} \right\rfloor - \lfloor \sqrt{n} \rfloor^2$.
	\item En déduire que $F(n) = n \ln n + (2\gamma - 1)n + O(\sqrt{n})$.
\end{enumerate}
\end{exo}


\begin{exo}
On note $q_n$ le plus grand diviseur premier de $n$.\newline
Quelle est la nature de la série de terme général $\frac{1}{nq_n}$ ?
\end{exo}


\begin{exo}
Soit $r_n$ le plus petit nombre premier ne divisant pas $n$.\newline
Montrer (sans utiliser le postulat de Bertrand) que $\lim\limits_{n \rightarrow +\infty} \frac{r_n}{n} = 0$.
\end{exo}