% Cours géométrie projective Colin 23/08/2021 Matin, groupe D

\subsubsection{Résumé du cours}

La géométrie projective est l'étude des transformations projectives. Ces transformations sont plus générales que les transformations affines, et nécessitent d'ajouter certains points à l'espace affine. Dans ce cours nous nous limiterons à la géométrie projective réelle, en dimention $1$ et $2$. Il est possible de remplacer dans tous le cours (pas forcément pour les exercices) "cercle" par "conique non dégénérée", c'est à dire ellipse ou hyperbole.

%https://alexanderrem.weebly.com/uploads/7/2/5/6/72566533/projectivegeometry.pdf}{Alexander Remorov}

\begin{dfn}
La droite projective associée à une droite affine $d$ est l'ensemble des points de $d$, auquel on ajoute un point à l'infini $\infty$.

\medskip

Le faisceau de droites en un point $O$ du plan affine est l'ensemble des droites passant par $O$.
\end{dfn}

La raison pour laquelle on définit la droite projective de cette manière est la suivante : étant donné un point $P$ en dehors d'une droite $d$, on définit la projection centrée en $P$ entre le faisceau de droites en $P$ et la droite projective $d$ comme l'appliction qui à une droite $\ell$ associe l'intersection de $\ell$ avec $d$ si elle existe, et $\infty$ sinon. Ceci est bien défini car il n'y a qu'une droite en $P$ qui n'intersecte pas $d$: celle parallèle à $d$ passant par $P$.

\medskip

Les projections ont une propriété remarquable : elle préservent une quantité appelée birapport.

\begin{dfn}

\begin{itemize}

\item[-]Soient $A,C,B,D$ quatre points distincts sur une droite projective $d$
. On définit leur \textbf{birapport} comme étant égal à :

$$b_{A,C,B,D}=\frac{AB}{CD}\times\frac{CB}{AD}$$

où les distances sont considérées avec un signe. Le choix de l'orientation de la droite $d$ ne change pas la valeur du birapport. Lorsque l'un des points est à l'infini, les deux termes où il apparait sont considérés comme égaux à $1$.

\item[-] Il existe également une notion de \textbf{birapport} pour $4$ droites $d_1,d_2,d_3,d_4$ dans le faisceau de droites d'un point $P$, qui est défini comme étant égal à :

$$b_{d_1,d_3,d_2,d_4}=\frac{\sin(d_1,d_2)}{\sin(d_3,d_4)}\times\frac{\sin(d_3,d_2)}{\sin(d_1,d_4)}$$

où le sinus entre deux droites estle sinus de l'angle orienté entre les droites.

\item[-] Soit $\Omega$ un cercle. Etant donné $4$ points distincts $A,B,C,D$ de $\Omega$ on peut définir leur \textbf{birapport} $b_{A,C,B,D}$ comme le birapport des droites $(PA)$, $(PB)$, $(PC)$, $(PD)$ pour tout point $P$ sur $\Omega$ différent de $A,B,C,D$.

\medskip

\item[-]Le \textbf{faisceau de droites} associé à un cercle $\Omega$ est l'ensemble des droites tangentes à $\Omega$. \'Etant donné $4$ droites distinctes $d_1,d_2,d_3,d_4$ de ce faisceau, on peut également définir leur birapport $b_{d_1,d_3,d_2,d_4}$ comme \textbf{le birapport} des points $d_1\cap d,d_2\cap d,d_3\cap d,d_4\cap d$ pour toute droite $d$ tangeante à $\Omega$ différente de $d_1,d_2,d_3,d_4$.

\end{itemize}
\end{dfn}

À l'aide du birapport nous pouvons définir ce qu'est une transformation projective.


\begin{dfn}
Une \textbf{transformation projective} entre deux espaces de l'un des $4$ types de la définition précédente est une application bijective qui préserve le birapport.
\end{dfn}

En utilisant un formalisme d'algèbre linéaire il serait possible d'avoir une définition plus simple, et d'obtenir

Nous avons ensuite démontré les propriétés suivantes.

\begin{pro}
La projection en $A$ entre une droite projective $d$ ne passant pas par $A$ et le faisceau de droites en $A$ est une transformation projective.

\smallskip

La même opération entre les points d'un cercle $\Omega$ et un faisceau de droites en un point $P$ est une transformation projective \textbf{si et seulement si} $P$ est sur le cercle.
\end{pro}
\medskip

La construction de la droite projective réelle peut être généralisée. Il est possible de voir les points du plan affine réel $P$ comme représentant presque toutes les droites de l'espace passant par un point $O$ en dehors du plan $P$. Cependant un cercle de droites n'intersecte pas $P$. En les ajoutant au plan affine on obtient la définition suivante.

\begin{dfn}

Le \textbf{plan projectif (réel)} est l'ensemble des points du plan affine, auquel on ajoute les points à l'infini de toutes les droites $d$ incluses dans $P$, en considérant que deux droites parallèles ont le même point à l'infini.

\medskip

Une \textbf{droite projective} dans le plan projectif est soit la droite projective associée à une droite affine, soit l'ensemble des points à l'infini, appelé "droite à l'infini".

\end{dfn}

Autrement dit on ajoute un point à l'infini pour chaque direction (le sens ne compte pas).

\begin{dfn}
Une \textbf{transformation projective} du plan projectif réel est une bijection du plan projectif réel dans lui même tel que chaque droite projective soit envoyée sur une droite projective.
\end{dfn}

À présent, voyons les résultats qui ont été évoqués.

\begin{pro}
Les transformations projectives du plan projectif préservent le birapport.

\medskip

Pour toute droite et tout cercle disjoint de la droite il existe une transformation projective qui stabilise le cercle et envoie la droite à l'infini.

\medskip

Une transformation projective de la droite projective qui fixe $3$ points est l'identité.
\end{pro}

Lorsque le birapport de quatre points vaut $-1$ on dit que ces points sont \textit{harmoniques}. Nous avons étudié divers configurations dans lesquelles on retrouve des points harmoniques, notamment le quadrilatère complet. Nous avons également défini la transformation polaire associée à un cercle qui induit une dualité entre points et droites.

\begin{dfn}
La \textit{polaire} associée à un cercle $C$ associée au point $E$ est la droite contenant les points $X$ tels que si $A,B$ sont les intersections de $(EX)$ et $(AB)$ alors $E,X,A,B$ sont harminoques. C'est la droite passant par les points de tangence des droites passant par $E$ tangentes à $C$.
\end{dfn}

\subsubsection{Préservation du birapport, points harmoniques}

\begin{thm}[Théorème de Pascal]
Soient $A,B,C,D,E,F$ six points sur un cercle $\Omega$. Soient $G$ l'intersection de $(AC)$ et $(BD)$, $H$ l'intersection de $(FD)$ et $(AE)$ et enfin $I$ l'intersection de $G$ et $H$
\end{thm}

\begin{rem}
Comme d'habitude, une droite entre deux points confondus est considérée comme étant la tangente à la conique en ce point.
\end{rem}

\begin{proof}

\begin{center}
\definecolor{uuuuuu}{rgb}{0.26666666666666666,0.26666666666666666,0.26666666666666666}
\definecolor{xdxdff}{rgb}{0.49019607843137253,0.49019607843137253,1.}
\definecolor{ududff}{rgb}{0.30196078431372547,0.30196078431372547,1.}
\begin{tikzpicture}[line cap=round,line join=round,>=triangle 45,x=1.0cm,y=1.0cm]
\clip(-0.9081181960526863,-6.075000188753791) rectangle (12.164265383589996,2.9584077333913172);
\draw [line width=1.pt] (4.188257888823294,-1.5672047761421417) circle (3.8509180317591993cm);
\draw [line width=1.pt] (0.54,-2.8)-- (7.877968265841787,-0.4646590759327571);
\draw [line width=1.pt] (3.5014518377380863,2.2219728570021386)-- (4.96,-5.34);
\draw [line width=1.pt] (0.54,-2.8)-- (7.701406036329229,-3.144341352664215);
\draw [line width=1.pt] (7.701406036329229,-3.144341352664215)-- (3.5014518377380863,2.2219728570021386);
\draw [line width=1.pt] (1.64,1.32)-- (4.96,-5.34);
\draw [line width=1.pt] (1.64,1.32)-- (7.877968265841787,-0.4646590759327571);
\draw [line width=1.pt,dash pattern=on 3pt off 3pt] (3.771264861266099,-2.9553686674795827)-- (4.948114216340589,0.3735610620619734);
\begin{scriptsize}
\draw [fill=ududff] (1.64,1.32) circle (2.5pt);
\draw[color=ududff] (1.4805233218222225,1.541509014833701) node {$A$};
\draw [fill=ududff] (0.54,-2.8) circle (2.5pt);
\draw[color=ududff] (0.6119264062313464,-2.5951837956678454) node {$B$};
\draw [fill=ududff] (4.96,-5.34) circle (2.5pt);
\draw[color=ududff] (5.030913214299927,-5.135829773771157) node {$C$};
\draw[color=black] (2.457694851861958,1.5849388606132448) node {$\Omega$};
\draw [fill=xdxdff] (7.701406036329229,-3.144341352664215) circle (2.5pt);
\draw[color=xdxdff] (7.940712881529361,-3.246631482361002) node {$D$};
\draw [fill=xdxdff] (7.877968265841787,-0.4646590759327571) circle (2.5pt);
\draw[color=xdxdff] (7.9515703429742475,-0.2608295850173664) node {$E$};
\draw [fill=xdxdff] (3.5014518377380863,2.2219728570021386) circle (2.5pt);
\draw[color=xdxdff] (3.5760133806852106,2.4209633918694626) node {$F$};
\draw [fill=uuuuuu] (4.242791780223141,-1.6215701043696304) circle (2.0pt);
\draw[color=uuuuuu] (3.9885969155908767,-1.4551503439548208) node {$I$};
\draw [fill=uuuuuu] (3.771264861266099,-2.9553686674795827) circle (2.0pt);
\draw[color=uuuuuu] (3.5977283035749825,-3.127199406467257) node {$G$};
\draw [fill=uuuuuu] (4.948114216340589,0.3735610620619734) circle (2.0pt);
\draw[color=uuuuuu] (5.020055752855042,0.5534800233490798) node {$H$};
\end{scriptsize}
\end{tikzpicture}
\end{center}

Considérons $X$ l'intersection de $(BD)$ et $(FC)$ ainsi que $Y$ l'intersection de $(BE)$ et $(FD)$. Le théorème est vrai si et seulement $b_{B,X,G,D}=b_{Y,F,H,D}$: en effet la projection centrée en $I$ entre les droites projectives $(BD)$ et $(FD)$ envoie $G$ sur l'unique point $H'$ satisfiant cette condition.

\medskip

Utilisons donc à présent la cocyclicité de nos points. La projection de centre $C\in \Omega$ entre la droite projective $(BD)$ et $\Omega$ envoie $B,G,X,D$ sur $B,A,F,D$. Ainsi $b_{B,X,G,D}=b_{B,F,A,D}$. De même en considérant la projection de centre $E\in \Omega$ entre la droite projective $(FD)$ et $\Omega$, on obtient $b_{Y,F,H,D}=b_{B,F,A,D}$, d'où le résultat.

\medskip

Cette preuve s'adapte parfaitement au cas où $\Omega$ est une conique quelconque.
\end{proof}

\begin{exo}
Soit $ABC$ un triangle. Soit $P,Q$ des points sur $(BC)$ tels que $\angle ABC=\angle  CAQ$ et $\angle ACB=\angle BAP$. Soient $M$ et $N$ tels que $P$ et $Q$ soient les milieux respectifs de $[AM]$ et $[AN]$. Montrer que l'intersection $D$ de $CN$ et $BM$ est sur le cercle circonscrit à $ABC$.
\end{exo}

\begin{sol}
Soit $\Omega$ le cercle circonscrit à $ABC$.Soit $D_1$ l'intersection de $BM$ etde $\Omega$, et $D_2$ l'intersection de $CN$ et de $\Omega$.

\medskip

L'égalité d'angle donne des droites parallèles à $AN$ et $AM$ respectivement : les droites tangentes à $\Omega$ en $C$ et $B$ respectivement. Sur la droite projective associée )à $AN$, avec $\infty_{AN}$ à l'infini, les points $ANQ\infty_{AN}$ sont harmoniques, ainsi la projection de centre $C$ sur $\Omega$ envoie ces points sur des points harmoniques. Ainsi $AD_1BC$ sont harmoniques. De même $AD_2BC$ sont harmoniques, ainsi $D_1=D_2$, d'où le résultat.
\end{sol}

Voici le théorème de Desargues :

\begin{thm}
Soit $A_1,B_1,A_2,B_2,A_3,B_3$ des points distincts tels que les droites $(A_1B_1)$,$(A_2B_2)$ et $(A_3B_3)$ sont distinctes et concourantes en $O$. Soient $X,Y,Z$ les intersections respectivement de $(A_2B_3)$ et $(A_3B_2)$, $(A_3B_1)$ et $(A_1B_3)$ et enfin $(A_1B_2)$ et $(A_2B_1)$. Montrer que $X,Y,Z$ sont alignés.
\end{thm}

\begin{proof}

Une première approche est de considérer une transformation projective qui envoie une droite passant par $O$ à l'infini. On se ramène alors à un problème avec $3$ segments parallèles, qui se résous en utilisant des homothéties.

\medskip

Mais il existe aussi une approche projective directe. Soit $d$ la droite $(XY)$, et soient $c_1,C_2,C_2$ ses intersections avec $(A_1B_1)$,$(A_2B_2)$ et $(A_3B_3)$ respectivement. Pour montrer que $Z$ est sur cette droite il suffit de montrer que $b_{OC_1A_1B_1}=b_{OC_2A_2B_2}$. Or les projections de centres $X$ et $Y$ montrent que ces birapports sont égaux à $b_{OC_3A_3B_3}$, ce qui conclut.
\end{proof}

\begin{exo}%{[M.L, Problem 20]}
Soit $ABCD$ un quadrilatère cyclique. Les droites $(AB)$ et $(CD)$ se coupent en $E$, et les diagonales $AC$ et $BD$ en $F$. Le cercle circonscrit des triangles $AFD$ et $BFC$ s'intersectent à nouveau en $H$. Montrer que l'anlge $\angle EHF$ est un angle droit.
\end{exo}

\begin{sol}

Soit $C_1$ le cercle circonscrit à $BCF$. On introduit le point $F'$ diamètralement opposé à $F$ sur $C_1$. Le problème se ramène à montrer que $H,F',E$ sont alignés.

\medskip

Soit $X$ l'intersection de $(AD)$ et $(BC)$. Soit $Y$ l'intersection de $(DE)$ et $(FH)$. La droite $(FH)$ est l'axe radical des cercles circonscrits à $ADF$ et $BCH$ donc $X,Y,H,F$ sont alignés.

Grâce au quadrilatère complet $ABCD$ les points $DCYE$ sont harmoniques. Les points $H,F',E$ sont donc alignés si et seulement si les droites $HD,HC,HF'$ et $HX$ sont harmoniques, avec $X$ l'intersection de $(AD)$ et $(BC)$. Montrons donc cela.

\medskip

Cela se ramène à montrer que sur le cercle circonscrit à $BCF$, les points $F,C,F',B'$ sont harmoniques, avec $B'$ l'intersection du cercle avec $DH$, différente de $H$.

Or l'angle $\angle FHB'$ est égal à $\angle DAC$ grâce au cercle passant par $AFHD$, et cet angle est égal à l'angle $\angle DBC$ grâce au cercle passant par $ABCD$. Ainsi l'arc $B'F$ et $FC$ ont même longueur dans le cercle passant par $BCFH$.

En particulier $(B'C)$ et $(FF')$ sont orthogonales avec $FF'$ un diamètre, donc ces points sont harmoniques, ce qui conclut.

\begin{center}
\definecolor{qqwuqq}{rgb}{0.,0.39215686274509803,0.}
\definecolor{uuuuuu}{rgb}{0.26666666666666666,0.26666666666666666,0.26666666666666666}
\definecolor{xdxdff}{rgb}{0.49019607843137253,0.49019607843137253,1.}
\definecolor{ududff}{rgb}{0.30196078431372547,0.30196078431372547,1.}
\begin{tikzpicture}[line cap=round,line join=round,>=triangle 45,x=.35cm,y=.35cm]
\clip(-13.66,-11.96) rectangle (30.42,14.58);
\draw[line width=1.pt,color=qqwuqq,fill=qqwuqq,fill opacity=0.10000000149011612] (8.11939335949667,2.433791094296911) -- (7.844066261199087,2.7565835801806964) -- (7.521273775315301,2.4812564818831144) -- (7.796600873612883,2.158463995999329) -- cycle;
\draw [line width=1.pt] (3.73788720609154,-1.3034315120073592) circle (7.059463753656761);
\draw [line width=1.pt,domain=-13.66:30.42] plot(\x,{(--33.0768-0.18*\x)/6.54});
\draw [line width=1.pt,domain=-13.66:30.42] plot(\x,{(-89.21813731417265--10.954881315694555*\x)/6.21839267031114});
\draw [line width=1.pt] (0.64,5.04)-- (9.62,2.6);
\draw [line width=1.pt] (7.18,4.86)-- (3.4016073296888596,-8.354881315694556);
\draw [line width=1.pt] (8.385105893979615,3.713919637747902) circle (1.6630635623584293);
\draw [line width=1.pt] (2.1090801172543587,-1.6392408028229144) circle (6.838892753436521);
\draw [line width=1.pt,domain=-13.66:30.42] plot(\x,{(--22.491224983676368-13.394881315694555*\x)/2.7616073296888595});
\draw [line width=1.pt,domain=-13.66:30.42] plot(\x,{(-28.0852--2.26*\x)/-2.44});
\draw [line width=1.pt,dash pattern=on 5pt off 5pt,domain=-13.66:30.42] plot(\x,{(-13.695079727705739--2.600648169969808*\x)/3.0489904295084926});
\draw [line width=1.pt,domain=-13.66:30.42] plot(\x,{(--97.7873449770027-10.14636670615311*\x)/8.654382693635865});
\begin{scriptsize}
\draw [fill=ududff] (0.64,5.04) circle (2.5pt);
\draw[color=ududff] (0.78,5.41) node {$A$};
\draw [fill=ududff] (7.18,4.86) circle (2.5pt);
\draw[color=ududff] (7.1,5.51) node {$B$};
\draw [fill=ududff] (9.62,2.6) circle (2.5pt);
\draw[color=ududff] (10.06,2.69) node {$C$};
\draw [fill=xdxdff] (3.4016073296888596,-8.354881315694556) circle (2.5pt);
\draw[color=xdxdff] (3.74,-8.65) node {$D$};
\draw[color=black] (-13.5,5.29) node {$f$};
\draw [fill=uuuuuu] (10.845591303121376,4.759112165969137) circle (2.0pt);
\draw[color=uuuuuu] (10.58,5.15) node {$E$};
\draw [fill=uuuuuu] (6.756298805142431,3.378110346932347) circle (2.0pt);
\draw[color=uuuuuu] (6.34,3.03) node {$F$};
\draw [fill=uuuuuu] (7.796600873612883,2.158463995999329) circle (2.0pt);
\draw[color=uuuuuu] (7.72,1.69) node {$H$};
\draw[color=black] (-8.3,-11.69) node {$l$};
\draw [fill=uuuuuu] (10.013912982816786,4.0497289285634475) circle (2.0pt);
\draw[color=uuuuuu] (9.56,4.29) node {$F'$};
\draw [fill=uuuuuu] (-0.8577818200229825,12.304830702152438) circle (2.0pt);
\draw[color=uuuuuu] (-0.72,12.63) node {$X$};
\draw[color=black] (-2.42,14.39) node {$m$};
\draw [fill=xdxdff] (5.427879834377444,4.935543903308783) circle (2.5pt);
\draw[color=xdxdff] (5.56,5.31) node {$Y$};
\end{scriptsize}
\end{tikzpicture}
\end{center}
\end{sol}



\begin{exo}

Dans un quadriltère cyclique $ABCD$, soit $E$ l'intersection de $(AD)$ et $(BC)$ (de sorte que $C$ soit entre $B$ et $E$), et $F$ l'intersection de $(BD)$ et $(AC)$. Soit $M$ le milieu de $[CD]$, et $N\neq M$ un point du cercle circonscrit au triangle $ABC$ tel que $\frac{AM}{MB}=\frac{AN}{NB}$. Montrer que $E,F,N$ sont alignés.

\end{exo}

\begin{sol}

\begin{center}
\definecolor{qqqqff}{rgb}{0.,0.,1.}
\definecolor{uuuuuu}{rgb}{0.26666666666666666,0.26666666666666666,0.26666666666666666}
\definecolor{xdxdff}{rgb}{0.49019607843137253,0.49019607843137253,1.}
\definecolor{ududff}{rgb}{0.30196078431372547,0.30196078431372547,1.}
\begin{tikzpicture}[line cap=round,line join=round,>=triangle 45,x=0.35cm,y=0.35cm]
\clip(-7.2165131217305305,-17.817964317871674) rectangle (36.54176525918287,8.528331240872502);
\draw [line width=1.pt] (10.960518094080774,-3.073144908072124) circle (6.2701166045712595);
\draw [line width=1.pt,domain=-7.2165131217305305:36.54176525918287] plot(\x,{(-15.765321577666427-1.1282250123746778*\x)/2.938085969725724});
\draw [line width=1.pt,domain=-7.2165131217305305:36.54176525918287] plot(\x,{(--86.97363318771907-5.523324263869746*\x)/1.9767448461799848});
\draw [line width=1.pt,domain=-7.2165131217305305:36.54176525918287] plot(\x,{(--135.5599282312279-9.73094073173159*\x)/-7.874070398864936});
\draw [line width=1.pt,domain=-7.2165131217305305:36.54176525918287] plot(\x,{(--119.0238795571982-5.335841480236521*\x)/-6.912729275319197});
\draw [line width=1.pt] (15.189782782980101,1.5558595665584092)-- (10.253798353840889,-9.303306177547858);
\draw [line width=1.pt] (7.315712384115165,-8.17508116517318)-- (17.166527629160086,-3.967464697311337);
\draw [line width=1.pt,domain=-7.2165131217305305:36.54176525918287] plot(\x,{(-23.47570191075158--6.884912969628594*\x)/-8.883757361472039});
\draw [line width=1.pt,color=qqqqff] (9.109102180115437,-1.5750185042595948) circle (6.83937673245038);
\begin{scriptsize}
\draw [fill=ududff] (15.189782782980101,1.5558595665584092) circle (2.5pt);
\draw[color=ududff] (15.25824691783116,2.065845028083332) node {$A$};
\draw [fill=ududff] (7.315712384115165,-8.17508116517318) circle (2.5pt);
\draw[color=ududff] (7.276927539824269,-8.635476227503036) node {$B$};
\draw [fill=ududff] (10.253798353840889,-9.303306177547858) circle (2.5pt);
\draw[color=ududff] (10.433718040080725,-9.826717925713021) node {$C$};
\draw [fill=xdxdff] (17.166527629160086,-3.967464697311337) circle (2.5pt);
\draw[color=xdxdff] (17.561314201037128,-4.227881944126091) node {$D$};
\draw[color=black] (-0.30731127211262627,-17.549934935774427) node {$i$};
\draw [fill=uuuuuu] (20.481808866560346,-13.23086221443213) circle (2.0pt);
\draw[color=uuuuuu] (20.618834559776086,-12.904092312755486) node {$E$};
\draw [fill=uuuuuu] (13.710162991500487,-6.635385437429598) circle (2.0pt);
\draw[color=uuuuuu] (14.047151191317676,-6.828759651884559) node {$M$};
\draw [fill=uuuuuu] (11.598051505088307,-6.345949244803537) circle (2.0pt);
\draw[color=uuuuuu] (11.485981540166211,-5.677226010281573) node {$F$};
\draw[color=black] (-7.057680895302534,7.9426374059192595) node {$l$};
\draw[color=qqqqff] (5.80772944536529,4.011539801826308) node {$C'$};
\draw [fill=uuuuuu] (12.83909037712848,-7.307754512356173) circle (2.0pt);
\draw[color=uuuuuu] (12.756639351590195,-7.742044953845547) node {$X$};
\draw [fill=uuuuuu] (2.6270065296877094,0.6066116355875701) circle (2.0pt);
\draw[color=uuuuuu] (2.770063114929832,0.934165414783846) node {$N$};
\end{scriptsize}
\end{tikzpicture}

\end{center}

La condition sur le point $N$ est équivalent au fait que les points $A,B,M,N$ sur le cercle $C'$, qui est le cercle circonscrit au triangle $ABC$, soient harmoniques. Soit $N'$ l'intersection de $C'$ et $(EF)$. Montrons que $A,B,M,N$ sont harmoniques sur $C'$.

\medskip

Soit $X$ l'intersection de $(NF)$ et $(CD)$. Considérons la projection de centre $X$ entre $C'$ et la droite $(BD)$, puis celle de centre $A$ entre $(BD)$ et $(CD)$. Elle envoie $A,B,M,N$ sur $X,Y,D,C$ avec $Y$ l'intersection de $AB$ et $CD$. Il reste à montrer que $X,Y,C,D$ sont harmoniques.

\medskip

Plaçons un repère sur la droite $CD$ de sorte que $C=1$, $D=-1$, $M=0$, $X=x$ et $Y=y$. Montrons que $xy=1$.

Or $Y$ est sur l'axe radical des deux cercles de la figure, ainsi $YC\times YD=YX\times YM$. En d'autres termes $(y-x)y=(y-1)(y+1)$, ce qui implique que $xy=1$. Cela conclut.
\end{sol}


\subsubsection{Transformations projectives}


\begin{exo}
Soit $ABC$ un triangle. Soient $F$ et $G$ les milieux de $AB$ et $AC$. Soit $P$ un point sur la hauteur issue de $A$. Soit $d_1$ la droite passant par $F$ et orthogonale à $(BP)$ et $d_2$ la droite passant par $G$ et orthogonale à $(PC)$. Montrer que l'intersection de $d_1$ et $d_2$, si est équidistante de $B$ et $C$.
\end{exo}

\begin{sol}
\begin{center}
\definecolor{qqwuqq}{rgb}{0.,0.39215686274509803,0.}
\definecolor{xdxdff}{rgb}{0.49019607843137253,0.49019607843137253,1.}
\definecolor{uuuuuu}{rgb}{0.26666666666666666,0.26666666666666666,0.26666666666666666}
\definecolor{zzttqq}{rgb}{0.6,0.2,0.}
\definecolor{ududff}{rgb}{0.30196078431372547,0.30196078431372547,1.}
\begin{tikzpicture}[line cap=round,line join=round,>=triangle 45,x=0.5cm,y=0.5cm, scale=1]
\clip(-6.94,-6.94) rectangle (13.58,8.58);
\draw[line width=1.pt,color=qqwuqq,fill=qqwuqq,fill opacity=0.10000000149011612] (-0.6388224001026219,1.4277845180930384) -- (-0.9928994996302749,1.1940551712191707) -- (-0.7591701527564073,0.8399780716915178) -- (-0.40509305322875433,1.0737074185653854) -- cycle;
\draw[line width=1.pt,color=qqwuqq,fill=qqwuqq,fill opacity=0.10000000149011612] (3.8185629359836453,0.36038163870242546) -- (3.9356665234068213,0.7681643624814286) -- (3.5278837996278183,0.8852679499046046) -- (3.4107802122046422,0.4774852261256014) -- cycle;
\draw [line width=1.pt,color=zzttqq] (0.2,5.54)-- (-2.88,-0.56);
\draw [line width=1.pt,color=zzttqq] (-2.88,-0.56)-- (7.72,-0.76);
\draw [line width=1.pt,color=zzttqq] (7.72,-0.76)-- (0.2,5.54);
\draw [line width=1.pt,domain=-6.94:13.58] plot(\x,{(-1.012--10.6*\x)/0.2});
\draw [line width=1.pt,dash pattern=on 2pt off 2pt,domain=-6.94:13.58] plot(\x,{(-25.784--10.6*\x)/0.2});
\draw [line width=1.pt,domain=-6.94:13.58] plot(\x,{(--4.026421637010675--1.9818434163701064*\x)/3.0022989323843414});
\draw [line width=1.pt,domain=-6.94:13.58] plot(\x,{(--26.126-0.2*\x)/10.6});
\draw [line width=1.pt,domain=-6.94:13.58] plot(\x,{(-0.9117095373665487--3.0022989323843414*\x)/-1.9818434163701064});
\draw [line width=1.pt,domain=-6.94:13.58] plot(\x,{(-11.06957836298932--2.181843416370106*\x)/-7.597701067615658});
\draw [line width=1.pt,domain=-6.94:13.58] plot(\x,{(--1623.5636322388605-495.9475348896797*\x)/-142.42201084697504});
\begin{scriptsize}
\draw [fill=ududff] (0.2,5.54) circle (2.5pt);
\draw[color=ududff] (0.34,5.91) node {$A$};
\draw [fill=ududff] (-2.88,-0.56) circle (2.5pt);
\draw[color=ududff] (-2.74,-0.19) node {$B$};
\draw [fill=ududff] (7.72,-0.76) circle (2.5pt);
\draw[color=ududff] (7.86,-0.39) node {$C$};
\draw[color=black] (0.52,6.85) node {$h$};
\draw[color=black] (1.96,7.09) node {$m$};
\draw [fill=uuuuuu] (2.42,-0.66) circle (2.0pt);
\draw[color=uuuuuu] (2.56,-0.33) node {$M$};
\draw [fill=xdxdff] (0.12229893238434165,1.4218434163701064) circle (2.5pt);
\draw[color=xdxdff] (0.26,1.79) node {$P$};
\draw [fill=xdxdff] (-1.34,2.49) circle (2.5pt);
\draw[color=xdxdff] (-1.86,2.87) node {$F$};
\draw[color=black] (-6.78,2.45) node {$i$};
\draw [fill=uuuuuu] (3.96,2.39) circle (2.0pt);
\draw[color=uuuuuu] (4.1,2.71) node {$G$};
\draw [fill=uuuuuu] (2.373296582523146,-3.1352811262732523) circle (2.0pt);
\end{scriptsize}
\end{tikzpicture}
\end{center}

Ici on a un point qui varie sur une droite : le point $P$. À un tel point sur la hauteur $h$ issue de $A$ on peut associer $K_1(P)$ l'intersection de $d_1$ et la médiatrice $m$ de $[BC]$. De même on peut définir $K_2(P)$ comme l'intersection de $d_2$ et la médiatrice $m$ de $[BC]$. Ceci définit deux applications entre les points de la droite projective $h$ et la droite projective $m$.

\medskip

Ces transformations s'avèrent être projectives. En effet montrons le pour $K_1$. La transformation qui à $P$ associe la droite $(BP)$ passant par $B$ est projective. Ensuite l'opération qui consiste à associer à une droite passant par $B$ la droite orthogonale passant par $F$ est également projective. Enfin l'opération qui à une droite passant par $F$ associe son intersection avec $m$ est projective.

\medskip

Une transformation projective est uniquement déterminée par trois points. Il suffit de trouver trois points "simples" $P_1$, $P_2$ et $P_3$ et de montrer dans les trois cas que $K=K_1(P)=K_2(P)$. Pour $P_1=A$, les deux points sont sur le centre du cercle circonscrit, pour $P_2$ le pied de la hauteur, ils sont à l'infini, et pour $P_3$ à l'infini, ils sont à l'intersection de $m$ et $(FG)$.
\end{sol}


\subsubsection{Pôles et polaires}


\begin{exo}
Soient $C_1$ et $C_2$ deux cercles tangeants extérieurement en $M$, avec $C_2$ de rayon strictement supérieur à $C_1$. Soit $A$ un point sur $C_2$ en dehors de la droite joignant les centres des deux cercles. Soient $B$ et $C$ deux points sur $C_1$ tels que $AB$ et $AC$ soient tangeant à $C_1$. Les droites $(BM)$ et $(CM)$ intersectent $C_2$ à nouveau en $E$ et $F$ respectivement. Soit $D$ l'intersection de la tangeante à $C_2$ en $A$ et de $(EF)$. Montrer que le lieu du point $D$ lorsque $A$ varie est une droite.
\end{exo}


\begin{sol}
\begin{center}
\definecolor{uuuuuu}{rgb}{0.26666666666666666,0.26666666666666666,0.26666666666666666}
\definecolor{xdxdff}{rgb}{0.49019607843137253,0.49019607843137253,1.}
\definecolor{ududff}{rgb}{0.30196078431372547,0.30196078431372547,1.}
\begin{tikzpicture}[line cap=round,line join=round,>=triangle 45,x=0.5cm,y=0.5cm]
\clip(-10.248360201656851,-7.919165543807547) rectangle (19.54795199179514,10.020810082649268);
\draw [line width=1.pt] (-1.14,1.38) circle (4.819880549805019);
\draw [line width=1.pt] (6.6,1.4) circle (2.92014528994513);
\draw [line width=1.pt,domain=-10.248360201656851:19.54795199179514] plot(\x,{(--26.245476914488904-6.576826591004165*\x)/5.096290418266154});
\draw [line width=1.pt,domain=-10.248360201656851:19.54795199179514] plot(\x,{(--48.48608077791878-1.9493218487590136*\x)/8.088693860284343});
\draw [line width=1.pt] (7.284149173690642,4.238870976731267)-- (4.291745731672452,-0.3886337655138852);
\draw [line width=1.pt,domain=-10.248360201656851:19.54795199179514] plot(\x,{(-1.0048465016947024-7.637983006556839*\x)/-4.939147102377251});
\draw [line width=1.pt,dash pattern=on 3pt off 3pt,domain=-10.248360201656851:19.54795199179514] plot(\x,{(--14.673813761147976-3.8210774709845805*\x)/0.44006221528841305});
\begin{scriptsize}
\draw [fill=ududff] (-1.14,1.38) circle (2.5pt);
\draw[color=ududff] (-1.0417864250049416,1.6321477795471933) node {$X$};
\draw [fill=ududff] (6.6,1.4) circle (2.5pt);
\draw[color=ududff] (6.691194779466266,1.645666977457108) node {$Y$};
\draw [fill=xdxdff] (3.6798644588368834,1.3924544301261936) circle (2.5pt);
\draw[color=xdxdff] (3.771048030924691,1.645666977457108) node {$M$};
\draw[color=black] (-3.4820016477445446,5.356686803728695) node {$C_2$};
\draw[color=black] (4.913420254312483,3.4775182942505505) node {$C_{1}$};
\draw [fill=xdxdff] (-0.8045446865937015,6.1881928254902805) circle (2.5pt);
\draw[color=xdxdff] (-0.703806477257074,6.444982235476829) node {$A$};
\draw [fill=uuuuuu] (7.284149173690642,4.238870976731267) circle (2.0pt);
\draw[color=uuuuuu] (7.380673872871915,4.457660142719367) node {$B$};
\draw [fill=uuuuuu] (4.291745731672452,-0.3886337655138852) circle (2.0pt);
\draw[color=uuuuuu] (4.555161509699744,-0.2740591257507803) node {$C$};
\draw [fill=uuuuuu] (-2.2692305580790677,-3.305732265194967) circle (2.0pt);
\draw[color=uuuuuu] (-2.5153789971856435,-3.05901389519321) node {$F$};
\draw [fill=uuuuuu] (2.669916544298183,4.332250741361872) circle (2.0pt);
\draw[color=uuuuuu] (2.8923001667802355,4.376544955259879) node {$E$};
\draw [fill=uuuuuu] (3.2398022435484704,5.213531901110774) circle (2.0pt);
\draw[color=uuuuuu] (2.973415354239724,5.485119183872885) node {$D$};
\end{scriptsize}
\end{tikzpicture}
\end{center}

Il est important ici de conjecturer la droite sur laquelle $D$ varie. On se convainc à l'aide d'une figure que c'est l'axe radical $d$ de $C_1$ et $C_2$. Nous allons réduire le problème à montrer qu'une transformation projective est triviale.

\medskip

À un point $D$ sur $d$ on associe le seul point $A$ de $C_2$ tel que $AD$ est tangeant à $C_2$ et $A\neq M$ si $D\neq M$. On lui associe ensuite la droite $MA$ passant par $M$, Puis on lui asscie la polaire $D'$ à $MA$ via à vis de $C_1$, qui se trouve sur la droite $d$. Enfin on applique l'homothétie de centre $M$ qui envoie $C_1$ sur $C_2$. Cette suite de transformations envoie $D$ sur l'intersection de $(EF)$ et $d$.

\medskip

Il nous reste à montrer que cette transformation, qui est projective est l'identité. On peut considérer $D$ à l'infini, ce qui revient à placer $A$ au point diamétralement opposé à $M$ (l'énoncé l'interdit, mais avec le formalisme de la géométrie projective on peut le faire, et la transformatione st bien définie).

On place ensuite $D$ en $M$. Pour conclure, on place $D$ comme sur la figure au millieu de la tangeante commune à $C_1$ et $C_2$. Soit $B'$ le symétrique de $B$ par rapport à $(XY)$. Ce point se situe sur la polaire de $D$ vis à vis de $C_2$, donc sa polaire vis à vis de $C_2$ passe par $D$. Or elle passe aussi par $E$ et $F$, donc les point sont alignés.


\medskip

Ainsi la transformation projective est l'identité, ce qui conclut.

\end{sol}

%\begin{exo}
%Les cercles $ w_{1}$ et $ w_{2}$ de centres $ O_{1}$ et $ O_{2}$ sont tangeants extérieurement en $ D$ et intérieurement tangeants à un cercle $ w$ aux points $ E$ et $ F$ respectivement. Soit $ t$ la tangeante commune à $ w_{1}$ et $ w_{2}$ en $ D$. Soit $ AB$ le diamètre de $ w$ perpendiculaire à $ t$, tel que $ A, E, O_{1}$ sont sur le même côté de $ t$. Montrer que les droites $ AO_{1}$, $ BO_{2}$, $ EF$ et $ t$ sont concourrantes.
%\end{exo}
%
%%\begin{sol}
%%
%%Soit $P$ l'intersection de $t$ et $(EF)$, et $Q$ l'intersection de $(EF)$ et $(AB)$. Soit $X$ l'intersection de la tangeante à $w$ en $E$ et celle en $F$. Le point $X$ appartient à l'axe radical des cercles $w_1$ et $w_2$, donc à $t$. Le point $Q$ est sur la polaire de $X$ par rapport à $w$, , donc la polaire de $Q$ vis à vis de $w$ passe par $D$, et c'est donc la droite $t$. Ainsi $Q,P,E,F$ sont harmoniques. On projete la droite projective $(EF)$ sur $(O_1O_2)$ depuis le point $O$, et on obtient donc que les points $O_1,O_2,Y,\infty$ sont harmoniques avec $\infty$ le point à l'infini de la droite $(O_1O_2)$ et $Y$ l'intersection de $(OD)$ et $O_1O_2$. Ainsi $Y$ est le milieu de $[O_1O_2]$. Par le théorème de la droite des milieux, le point d'intersection de $AO_1$ et
%%
%%\end{sol}
