
En $2020$, après $8$ ans d'absence (autant dire une éternité), surgissait dans le sujet des IMO un ennemi que l'on ne craignait plus : un problème d'inégalité, une menace fantôme.

Dans le sujet de $2021$, confirmant nos pire craintes, un nouveau problème d'inégalité, particulièrement redoutable, faisait à nouveau office de problème $2$ . Voilà qui marquait définitivement l'avènement de l'attaque des inégalités.

Notre objectif ici est, et celles et ceux qui ont vu la prélogie de Star Wars l'auront compris, d'éviter d'avoir à subir la revanche des inégalités, et que nos jeunes olympistes français, pas assez préparés à cet exercice, soient contraints de s'incliner devant les problèmes.

\subsubsection{Exercices}
\begin{exo}
Soient $a,b,c$ les longueurs des côtés d'un triangle. Montrer que $\frac{a}{b+c-a}+\frac{b}{c+a-b}+\frac{c}{a+b-c} \geq 3$.
\end{exo}

\begin{exo}
(BXMO 2014) Soient $a,b,c$ et $d$ des entiers strictement positifs. Déterminer la plus petite valeur que peut prendre l'expression :

\[S=\left\lfloor \frac{a+b+c}{d}\right\rfloor + \left\lfloor \frac{a+b+d}{c}\right\rfloor + \left\lfloor \frac{a+c+d}{b}\right\rfloor +\left\lfloor \frac{b+c+d}{a}\right\rfloor\]
\end{exo}

\begin{exo}
(TST belge 2010)
Soient $a_1,a_2,\ldots a_n$ des réels vérifiant $\sum_{i=1}^{n} a_i=0$ et $\sum_{i=1}^{n} |a_i| =1$. Montrer que

\[\left| \sum_{i=1}^{n} ia_i \right| \leqslant \frac{n-1}2\]
\end{exo}


\begin{exo}
(IMO 2012 P2)
Soit $n \geq 3$ un entier et soient $a_2,\dots,a_n$ des réels strictement positifs tels que : $a_2 \cdot \dots \cdot a_n =1$.

Montrer que
\[
(1+a_2)^2(1+a_3)^3 \cdot \cdot \cdot (1+a_n)^n > n^n.
\]
\end{exo}

\begin{exo}
(IMO SL 2017 A1)
Soient $a_1, \ldots, a_n, k$ et $M$ des entiers strictement positifs. On suppose que

\[\begin{array}{lllllll}
\frac1{a_1}+ \ldots + \frac1{a_n} &=& k &\text{ et }& a_1 a_2 \ldots a_n &=& M\\
\end{array}\]

On suppose que $M>1$. Montrer que le polynôme $M(1+X)^k-(X+a_1) \ldots (X+a_n)$ ne possède pas de racine strictement positive.
\end{exo}

\begin{exo}
(IMO SL 2019 A2)

Soient $u_1, \ldots , u_{2019}$ des réels satisfaisant

\[\begin{array}{lllllll}
u_1 + \ldots + u_{2019} &=& 0 & \text{ et } & u_1^2+ \ldots + u_{2019}^2 &=& 1 \\
\end{array}\]

On note $a= \max (u_1, \ldots , u_{2019})$ et $b= \min (u_1, \ldots , u_{2019})$. Montrer que

\[ab \leqslant - \frac1{2019}\]
\end{exo}

\begin{exo}
(IMO 2020 P2)
Soit $a,b,c,d$ des réels tels que $a\geqslant b \geqslant c\geqslant d >0$ et $a+b+c+d=1$. Montrer que

\[(a+2b+3c+4d)a^ab^bc^cd^d<1\]

\end{exo}

\begin{exo}
(EGMO 2014 P1)
Déterminer tous les réels $t$ tels que pour tout triplet $(a,b,c)$ désignant les longueurs des côtés d'un triangle, $a^2+bct$, $b^2+cat$ et $c^2+abt$ sont également les longueurs des côtés d'un triangle.
\end{exo}

\begin{exo}
(EGMO 2016 P1)
Soit $n$ un entier positif impair, et soit $x_1,x_2,\dots,x_n$ des nombres réels positifs ou nuls. Montrer que :
\[
\min_{i=1,\dots,n}(x_i^2+x_{i+1}^2) \leq \max_{j=1,\dots,n}(2x_jx_{j+1}),
\]
où $x_{n+1}=x_1$.
\end{exo}


\begin{exo}
(IMO SL 2020 A3)
Soient $a,b,c,d$ des réels strictement positifs vérifiant $(a+c)(b+d)=ac+bd$. Déterminer la plus petite valeur que peut prendre

\[\frac{a}{b}+\frac{b}{c}+\frac{c}{d}+\frac{d}{a}\]
\end{exo}

\begin{exo}
(BXMO 2012)
Déterminer tous les quadruplets $(a,b,c,d)$ de réels strictement positifs vérifiant $abcd=1$ et :

\[\begin{array}{lllllll}
a^{2012}+2012b &=& 2012c+d^{2012} &\text{ et } & 2012a+b^{2012} &=& c^{2012}+2012d\\
\end{array}\]
\end{exo}

\begin{exo}
(BXMO 2019) Soit $0\leqslant a,b,c,d\leqslant 1$ des réels.

1) Montrer que

\[ab(a-b)+bc(b-c)+cd(c-d)+da(d-a) \leqslant \frac{8}{27}\]

2) Déterminer les cas d'égalité.
\end{exo}




\begin{exo}
(IMO SL 2015 A1)
Soit $(a_k)$ une suite de réels strictement positifs telle que pour tout entier $k$ :

\[a_{k+1} \geqslant \frac{ka_k}{a_k^2 +(k-1)}\]

Montrer que pour tout entier $n\geqslant 2$, $a_1+a_2+\ldots +a_n \geqslant n$.
\end{exo}



\subsubsection{Solutions}

\setcounter{exo}{0}
\begin{exo}
Soient $a,b,c$ les longueurs des côtés d'un triangle. Montrer que $\frac{a}{b+c-a}+\frac{b}{c+a-b}+\frac{c}{a+b-c} \geq 3$.
\end{exo}

\begin{sol}
Nous faisons le changement de variables dit de Ravi : $a=y+z$, $b=z+x$ et $c=x+y$, avec la seule propriété vérifiée par $x,y,z$ est $x,y,z>0$. Ce changement de variables est facile à comprendre une fois qu'on trace le cercle inscrit du triangle. Ainsi, l'inégalité à montrer devient :
\[
\frac{y+z}{x}+\frac{z+x}{y}+\frac{x+y}{z} \geq 6.
\]
Ensuite, on utilise le fait que pour tout $X>0$, on a :
\[
X+\frac1{X}\geq 2.
\]
Ce qui termine l'exercice.
\end{sol}



\begin{exo}
(BXMO 2014) Soient $a,b,c$ et $d$ des entiers strictement positifs. Déterminer la plus petite valeur que peut prendre l'expression :

\[S=\left\lfloor \frac{a+b+c}{d}\right\rfloor + \left\lfloor \frac{a+b+d}{c}\right\rfloor + \left\lfloor \frac{a+c+d}{b}\right\rfloor +\left\lfloor \frac{b+c+d}{a}\right\rfloor\]
\end{exo}

\begin{sol}
Avec des parties entières, il n'y a pas grand chose à faire : on applique l'inégalité $\lfloor x \rfloor >x-1$ pour trouver, en réarrangeant les termes

\[S > \left(\frac{a}{b}+ \frac{b}{a}\right) + \left(\frac{a}{c}+ \frac{c}{a}\right)+\left(\frac{a}{d}+ \frac{d}{a}\right) +\left(\frac{c}{b}+ \frac{b}{c}\right) +\left(\frac{d}{b}+ \frac{b}{d}\right) + \left(\frac{c}{d}+ \frac{d}{c}\right)-4\]

et chaque terme entre parenthèse est minoré par $2$ par IAG. On a donc $S>8$, soit $S\geqslant 9$.

En tatonnant, on trouve la construction $(5,5,5,4)$.
\end{sol}

\begin{exo}
(TST belge 2010)
Soient $a_1,a_2,\ldots a_n$ des réels vérifiant $\sum_{i=1}^{n} a_i=0$ et $\sum_{i=1}^{n} |a_i| =1$. Montrer que

\[\left| \sum_{i=1}^{n} ia_i \right| \leqslant \frac{n-1}2\]
\end{exo}

\begin{sol}
On peut imaginer au moins deux manières de résoudre cet exercice. À vous de choisir laquelle vous trouvez la plus naturelle ou la plus élégante.

\underline{\textit{Solution n$^\circ 1$ : échanger les signes somme}}

Si on utilise l'inégalité triangulaire à tours de bras, cela ne va pas donner quelque chose de concluant. En revanche, la forme de la somme donne envie de séparer les termes comme suit, puis d'échanger l'ordre des signe somme (on a le droit de le faire car le résultat ne dépend pas de l'ordre de sommation) :

\[
\sum_{i=1}^{n} ia_i = \sum_{i=1}^n \sum_{j=1}^i a_i = \sum_{j=1}^n \sum_{i=j}^n a_i
\]

Par inégalité triangulaire on a donc :

\[\left| \sum_{i=1}^{n} ia_i \right| \leqslant \sum_{j=1}^n \left|\sum_{i=j}^n a_i\right|\]

Pour rajouter un peu de symétrie, on est tenté d'introduire un terme de la forme $|\sum (n-i)a_i|$. Cela est d'ailleurs bien commode puisque

\[|\sum (n-i)a_i| = |n\sum a_i - \sum ia_i| = |\sum ia_i|\]

Or ce terme vérifie de même que plus haut :

\[\left|\sum_{i=1}^n (n-i)a_i\right| \leqslant \sum_{j=1}^n \left|\sum_{i=1}^j a_i \right|\]

En résumé :

\[\begin{array}{lll}
\displaystyle 2 \left| \sum_{i=1}^{n} ia_i \right|
&=&\displaystyle \left| \sum_{i=1}^{n} (n-i)a_i \right|+\left| \sum_{i=1}^{n} ia_i \right| \\
&&\\
& \leqslant & \displaystyle\sum_{j=1}^n \left|\sum_{i=j}^n a_i\right| + \sum_{j=1}^n \left|\sum_{i=1}^j a_i \right| \\
&&\\
&=& \displaystyle\sum_{j=1}^n \left(\left|\sum_{i=j}^n a_i\right|+\left|\sum_{i=1}^j a_i \right|\right) \\
&&\\
&\leqslant & \displaystyle\sum_{j=2}^n \sum_{i=1}^n |a_i|\\
&&\\
&=&n-1\\
\end{array}\]

ce qui est l'inégalité voulue.

%\newline

\underline{\textit{Solution n$^\circ 2$ : Séparer les positifs des négatifs}}

On peut utiliser une technique souvent utile quand on a affaire à des réels de signe quelconque qui est de les séparer en deux équipes : les positifs et les négatifs. Notez que ceux qui sont nuls peuvent rejoindre l'équipe qu'ils veulent, cela ne change rien. Ainsi, on note $P$ et $N$ deux ensemble d'indices (i.e. $P \subset \{1,\dots,n\}$ et $N \subset \{1,..,n\}$) disjoints (i.e. $P \cap N = \emptyset$) et tels que tout indice est soit dans $P$ soit dans $N$ (i.e. $P \cup N =\{1,\dots,n\}$), tels que

\[
\forall i \in P, \quad a_i \geq 0,
\]

et

\[
\forall i \in N, \quad a_i \leq 0.
\]

Maintenant, qu'on a séparé les réels selon leur signe, la seule information qui nous intéresse sur chacun d'entre eux est leur valeur absolue. C'est pour cela qu'il est commode d'introduire de nouvelles variables qui nous donnent la valeur ansolue de chaque réel qui nous intéresse. On pose, pour tout $i \in \{1,\dots,n\}$,
\[
b_i := \vert a_i \vert.
\]

Nous pouvons donc réécrire les deux hypothèses de l'énoncé avec les nouvelles variables. Le fait que la somme des réels est nulle veut dire la même chose que la somme des positifs est égale à la somme des négatifs. Ainsi, la première hypothèse s'écrit comme :

\[
\sum_{i \in P} b_i = \sum_{i \in N} b_i.
\]

La deuxième hypothèse peut se réécrire comme :

\[
\sum_{i \in P} b_i + \sum_{i \in N} b_i =1.
\]

Finalement, on peut résumer les deux hypothèses en une seule :

\[
\sum_{i \in P} b_i = \sum_{i \in N} b_i = \frac12.
\]

On peut en particulier en déduire que $P$ et $N$ sont non vides.

Enfin, ce qu'on nous demande de démontrer peut se réécrire comme :
\[
-\frac{n-1}2 \leq \sum_{i \in P} ib_i - \sum_{i \in N} ib_i \leq \frac{n-1}2.
\]

Démontrons l'inégalité à droite, il est clair que l'inégalité de gauche se démontre de la même manière.

Observons pour cela que
\[
\sum_{i \in P} 2b_i =1.
\]

Ainsi, on peut voir les $2b_i$, pour $i \in P$ comme des poids d'une moyenne (ils sont tous positifs et leur somme fait $1$). Ainsi, $2\sum_{i \in P} ib_i$ est une moyenne pondérée de $\{i,i \in P\}$. En particulier, elle est majorée par le plus grand des $i \in P$. Chose qu'on peut résumer par la ligne suivante :
\[
2\sum_{i \in P} ib_i \leq \max P.
\]

Autrement dit,
\[
\sum_{i \in P} ib_i \leq \frac{\max P}2 \leq \frac{n}2,
\]
car $\max P$ est évidemment majoré par $n$.

De même, $2 \sum_{i \in N} ib_i$ est une moyenne pondérée de $\{i,i \in N\}$. En particulier, elle est minorée par le plus petit des $i \in N$, ce qu'on peut écrire comme suit :
\[
2\sum_{i \in N} ib_i \geq \min N.
\]

Autrement dit,
\[
\sum_{i \in N} ib_i \geq \frac{\min N}2 \geq \frac12,
\]
car $\min N$ est évidemment majoré par $1$.

On est maintenant à deux pas de conclure :
\[
\sum_{i \in P} ib_i - \sum_{i \in N} ib_i \leq \frac{n}2 - \frac12 = \frac{n-1}2.
\]
\end{sol}

\begin{exo}
(IMO 2012 P2)
Soit $n \geq 3$ un entier et soient $a_2,\dots,a_n$ des réels strictement positifs tels que : $a_2 \cdot \dots \cdot a_n =1$.

Montrer que
\[
(1+a_2)^2(1+a_3)^3 \cdot \cdot \cdot (1+a_n)^n > n^n.
\]
\end{exo}

\begin{sol}
Voilà un exercice qui montre l'importance de penser à l'inégalité arithmético-géométrique, même dans un exercice de niveau olympique ! Les puissances qu'on voit dans l'énoncé nous soufflent les poids à utiliser dans les moyennes.

D'après l'inégalité arithmético-géométrique pondérée,
\[
\frac1{k}(1+a_k) = \frac{k-1}{k} \frac1{k-1} + \frac1{k} a_k \geq \left(\frac1{k-1}\right)^{\frac{k-1}{k}}a_k^{\frac1{k}},
\]
avec égalité si et seulement si $a_k = \frac1{k-1}$.

Autrement dit,
\[
(1+a_k)^k \geq k^k \frac1{(k-1)^{k-1}} a_k.
\]

En faisant le produit (on obtient un produit téléscopique), on obtient
\[
(1+a_2)^2(1+a_3)^3 \cdot \cdot \cdot (1+a_n)^n \geq n^n a_2 \cdot \dots \cdot a_n = n^n,
\]
avec égalité si et seulement si pout tout $k$, $a_k = \frac1{k-1}$. Ce qui est exclus grâce à l'hypothèse $a_2 \cdot \dots \cdot a_n =1$, d'où l'inégalité stricte.
\end{sol}


\begin{exo}
(IMO SL 2017 A1)
Soient $a_1, \ldots, a_n, k$ et $M$ des entiers strictement positifs. On suppose que

\[\begin{array}{lllllll}
\frac1{a_1}+ \ldots + \frac1{a_n} &=& k &\text{ et }& a_1 a_2 \ldots a_n &=& M\\
\end{array}\]

On suppose que $M>1$. Montrer que le polynôme $M(1+X)^k-(X+a_1) \ldots (X+a_n)$ ne possède pas de racine strictement positive.
\end{exo}

\begin{sol}

\underline{\textit{Première solution :}}
Il suffit de montrer que pour tout $x>0$, on a $M(1+x)^k > (x+a_1) \ldots (x+a_n)$. Or, d'après l'inégalité de Bernouilli : $(1+y)^a > 1+ya$ si $y>0$ :

\[ (x+a_1) \ldots (x+a_n) < a_1(x+1)^{1/a_1} a_2 (1+x)^{1/a_2} \ldots a_n (1+x)^{1/a_n} = M(1+x)^k\]

ce qui donne le résultat voulu.

\underline{\textit{Deuxième solution :}}
La première solution est peut-être trop astucieuse, mais cet exercice peut être vu comme un deuxième exemple de l'application de l'IAG au niveau olympique.

Commençons par remarquer que $0$ est bien une racine du polynôme. Nous allons donc montrer que pour tout $x>0$, on a $M(1+x)^k > (x+a_1) \ldots (x+a_n)$, ou ce qui revient au même qu'on a $a_1(1+x)^{1/a_1} \cdot a_2(1+x)^{1/a_2} \cdot \cdot \cdot a_n(1+x)^{1/a_n} >(x+a_1) \ldots (x+a_n)$.

Il suffit donc de montrer que pour tout $1 \leq j \leq n$, on a $a_j(1+x)^{1/a_j} > (x+a_j)$ pour tout $x>0$. D'après l'inégalité arithmético-géométrique pondérée, on a
\[
x+a_j=(x+1)+(a_j-1)=a_j\left[\frac1{a_j}(x+1)+\frac{a_j-1}{a_j}1\right] \geq a_j(x+1)^{1/a_j},
\]
avec égalité si et seulement si $x+1=1$, i.e. $x=0$, ce qui est exclus. D'où, l'inégalité stricte.
\end{sol}


\begin{exo}
(IMO SL 2019 A2)

Soient $u_1, \ldots , u_{2019}$ des réels satisfaisant

\[\begin{array}{lllllll}
u_1 + \ldots + u_{2019} &=& 0 & \text{ et } & u_1^2+ \ldots + u_{2019}^2 &=& 1 \\
\end{array}\]

On note $a= \max (u_1, \ldots , u_{2019})$ et $b= \min (u_1, \ldots , u_{2019})$. Montrer que

\[ab \leqslant - \frac1{2019}\]
\end{exo}

\begin{sol}
Encore un exercice où il est une bonne idée de séparer les positifs des négatifs ! On peut d'ailleurs remarquer que $b< 0 < a$ puisqu'il y a au moins un réel non nul par la deuxième égalité, et donc un réel strictement positif (resp négaitf) parmi les $u_i$.

Quitte à renuméroter les $u_i$, on peut donc supposer que $u_1 \leqslant , u_k \leqslant 0$ et $0<u_{k+1} \leqslant \ldots \leqslant u_{2019}$, avec $1\leqslant k \leqslant 2019$. On s'attend donc à devoir raisonner avec les réels positifs d'un côté, dont notera $\sum_P u_i$ la somme, et les négatifs de l'autre, dont on note $\sum_N u_i$ la somme. On obtient par la première égalité que $\sum_P u_i = - \sum_N u_i$.

La deuxième chose est qu'il faut, dans le calcul, faire apparaître d'une façon ou d'une autre les nombres $a$ et $b$. On n'a pas encore utilisé la deuxième égalité sur la somme des carrés. Majorer brutalement chaque terme par $a^2$ ou $b^2$ ne fonctionne pas, il faut donc être plus fin :

\[\begin{array}{lll}
1 &=&\displaystyle \sum_P u_i^2 + \sum_N u_i^2 \\
&&\\
&\leqslant & \displaystyle \sum_P u_i a + \sum_N (-u_i) (-b)\\
&&\\
&=&\displaystyle a(-\sum_N u_i) + (-b) \sum_P u_i \\
&&\\
&\leqslant&\displaystyle -abk + (-b) a(2019-k) \\
&&\\
&=& \displaystyle -2019ab\\
\end{array}\]

ce qui est le résultat voulu.

\underline{\textit{Alternative pour finir l'exercice}}
Il est possible que vous trouviez que la deuxième partie de la solution précédente est trop astucieuse. En effet, il faut y penser à faire les majorations qui marchent ! On voudrait proposer ici une façon alternative de terminer l'exercice après avoir séparé les positifs des négatifs. Il est possible de résoudre cet exercice avec les multiplicateurs de Lagrange. Celui qui voudrait apprendre les multiplicateurs de Lagrange pourrait consulter le cours de Jean-François Martin dans le poly de 2014, il s'agit d'une très bonne introduction.

Vous pouvez remarquer que $2019$ est choisi arbitrairement dans l'énoncé. On se permet donc de remplacer $2019$ par $n$ et ceci nous autorise à réaliser un raisonnement par récurrence forte sur $n \geq 2$ (il est assez clair que l'énoncé est vide pour $n=1$, c'est pour cela qu'on peut exclure cette situation). On rappelle au lecteur qu'une récurrence forte ne demande pas de faire une initialisation. Essayons de formuler l'assertion qu'on veut montrer par récurrence.

On peut reformuler le problème de façon suivante. Soient $P \geq 1$ et $N \geq 1$ tels que $n=P+N$. Soient $x_1\geq\dots\geq x_P \geq 0$ et $y_1 \geq \dots\geq y_N \geq 0$ tels que
\[
x_1 + \dots + x_P = y_1 + \dots + y_N,
\]
et
\[
x_1^2 + \dots + x_P^2 + y_1^2 + \dots + y_N^2 = 1.
\]
On veut montrer que
\[
x_1 y_1 \geq \frac1{n}.
\]

Vous l'avez compris : les $x_i$ désignent les positifs et les $y_j$ désignent les négatifs ! Remarquez qu'on a fixé le nombre des variables positives et le nombre des variables négatives dans cet énoncé : ceci n'est pas gênant car la preuve marche pour une répartition quelconque. Observons que l'on peut jouer sur l'homogénéité de cet énoncé en le reformulant de façon suivante :

Soient $P \geq 1$ et $N \geq 1$ tels que $n=P+N$. Soient $x_1\geq\dots\geq x_P \geq 0$ et $y_1 \geq \dots\geq y_N \geq 0$ tels que
\[
x_1 + \dots + x_P = y_1 + \dots + y_N = 1.
\]
On veut montrer que
\[
\frac{x_1 y_1}{x_1^2 + \dots + x_P^2 + y_1^2 + \dots + y_N^2} \geq \frac1{n}.
\]

C'est ce dernier énoncé que nous allons montrer avec les multiplicateurs de Lagrange par récurrence forte sur $n$, on le désigne donc par $\mathcal{P}_n$. On suppose que pour $k<n$, $\mathcal{P}_k$ est établi.

On définit la fonction
\[
f(x_1,\dots,x_P,y_1,\dots,y_N)=\frac{x_1 y_1}{x_1^2 + \dots + x_P^2 + y_1^2 + \dots + y_N^2}.
\]

Notez que cette fonction est bien définie et continue sur $K=\{(x_1,\dots,x_P,y_1,\dots,y_N) \in [0,1]^{n},\ x_1 \geq x_2 \geq \dots \geq x_P,\ y_1 \geq y_2 \geq \dots \geq y_N,\ x_1 + \dots + x_P = y_1 + \dots + y_N = 1 \}$. C'est l'ensemble sur lequel on voudrait la minorer par $\frac1{n}$. Cet ensemble est ce qu'on appelle en analyse un compact, i.e. il contient ses bords (ce qui s'obtient avec des conditions qui sont des inégalités larges) et il est borné (en effet, il est inclus dans $[0,1]^{n}$). Un théorème général en analyse nous apprend qu'une fonction continue définie sur un compact atteint son minimum en un point donné de ce compact (pas forcément unique). Notre travail maintenant est de trouver un point en lequel le minimum est atteint. On peut déjà vérifier que si $x_1 = \dots = x_P = \frac1{P}$ et $y_1 = \dots =y_N = \frac1{N}$, alors la fonction $f$ atteint $\frac1{n}$. Ainsi, le minimum de $f$ ne pourra être que plus petit ou égal que $\frac1{n}$. On veut donc montrer que le minimum de $f$ vaut $\frac1{n}$.

Grâce à l'hypothèse de récurrence, on peut tout de suite dire que $f$ ne peut pas atteindre son minimum en un point où certains $x_i$ ou $y_j$ sont nuls. En effet, l'hypothèse de récurrence nous apprend que dans ce cas, $f$ est minorée par $\frac1{n-1}$. Ainsi, le point où $f$ atteint son minimum est nécessairement dans $M=\{(x_1,\dots,x_P,y_1,\dots,y_N) \in ]0,1]^{n},\ x_1 \geq x_2 \geq \dots \geq x_P,\ y_1 \geq y_2 \geq \dots \geq y_N,\ x_1 + \dots + x_P = y_1 + \dots + y_N = 1 \}$.

Distinguons maintenant trois cas en fonction de la partition choisie. D'abord, on traite le cas où $P \geq 2$ et $N \geq 2$, puis le cas où $P=1$ et $N \geq 2$ (ce qui est équivalent au cas où $P \geq 2$ et $N=1$), enfin le cas où $P=1$ et $N=1$ (c'est le cas où $n=2$, ce qui peut être vu comme l'initialisation de cette récurrence).

\textit{Premier cas :}

Soit $(x_1,\dots,x_P,y_1,\dots,y_N)$ un élément de $M$ où $f$ atteint son minimum. Dans ce cas, tout $x_i$ et tout $y_j$ est strictement inférieur à $1$. Parmi les inégalités $x_1 \geq x_2 \geq \dots \geq x_P$ et $y_1 \geq y_2 \geq \dots \geq y_N$, il y en a qui sont des inégalités strictes et il y en a qui sont des égalités. Supposons qu'il y a $P' \geq 1$ variables distinctes parmi les $x_i$ et $N' \geq 1$ variables distinctes parmi les $y_j$. On peut donc voir ce minimum comme le minimum d'une fonction $g$ de $P'+N'$ variables définie et régulière sur l'ouvert $O=\{(a_1,\dots,a_{P'},b_1,\dots,b_{N'} \in ]0,1[^n,\ a_1>\dots>a_{P'},\ b_1>\dots>b_{N'}\}$ (un ouvert s'obtient avec des inégalités strictes, il est très important de se placer sur un ouvert pour appliquer les multiplicateurs de Lagrange). Par exemple, si $x_1=x_2$ et $y_1>y_2$ avec $P=N=2$ désigne le point où le minimum de $f$ est atteint, on a $P'=1$ et $N'=2$, et $g$ est une fonction de $3$ variables définie comme suit : $g(a_1,b_1,b_2)=f(a_1,a_1,b_1,b_2)$. On généralise cette définition pour le cas général. Pour chaque $a_i$, on note $\alpha_i>0$ la multiplicité de $a_i$, i.e. le nombre de variables de $f$ qu'il désigne ; de même, on note $\beta_j>0$ la multiplicité de $b_j$.

Ainsi, on a un extrémum local de $g$ en $(a_1,\dots,a_{P'},b_1,\dots,b_{N'})$ sur $O$, lorsqu'on impose les deux conditions suivantes : $\alpha_1 a_1 +\dots+ \alpha_{P'} a_{P'} =1$ et $\beta_1 b_1 +\dots+ \beta_{n'} b_{N'} =1$. On a ainsi deux multiplicateurs de Lagrange $\lambda$ et $\nu$ tels que les égalités suivantes sont vérifiées après simplifications (pour les obtenir, on a dérivé la fonction par rapport à chacune de ses variables, cf. le cours de Jean-François Martin pour davantage de détails) :
\[
-\alpha_1 a_1^2 + \alpha_2 a_2^2 +\dots+ \alpha_{P'} a_{P'}^2 + \beta_1 b_1^2 +\dots+\beta_{N'} b_{N'}^2 = \alpha_1 \lambda,
\]
pour $2 \leq i \leq P'$,
\[
-2 a_1 a_i = \lambda,
\]
\[
\alpha_1 a_1^2 +\dots+ \alpha_{P'} a_{P'}^2 -\beta_1 b_1^2 +\beta_2 b_2^2 +\dots+\beta_{N'} b_{N'}^2 = \beta_1 \mu,
\]
et pour $2 \leq j \leq N'$,
\[
-2 b_1 b_j =\mu.
\]

On en déduit ainsi que les $a_i$, pour $i \geq 2$ sont tous égaux. De même pour les $b_j$ pour $j \geq 2$. Ce qui est exclus d'après la définition de $O$, à moins que $P' \leq 2$ et $N' \leq 2$. On en déduit donc que $P' \leq 2$ et $N' \leq 2$. Supposons par l'absurde que $P'=2$ et $N'=2$. Dans ce cas, les égalités obtenues grâce aux multiplicateurs de Lagrange se réécrivent en les deux lignes suivantes :
\[
\alpha_2 a_2^2 + \beta_1 b_1^2 + \beta_2 b_2^2 + 2\alpha_1 a_1 a_2 = \alpha_1 a_1^2,
\]
\[
\alpha_1 a_1^2 + \alpha_2 a_2^2 + \beta_2 b_2^2 +2\beta_1 b_1 b_2 = \beta_1 b_1^2.
\]

En particulier, on en déduit que $\beta_1 b_1^2 <\alpha_1 a_1^2$ et $\alpha_1 a_1^2 < \beta_1 b_1^2$. Ce qui est exclus et est une contradiction.

Supposons maintenant par l'absurde que $P'=1$ et $N'=2$ (cette configuration est bien entendu équivalente à $P'=2$ et $N'=1$). Dans ce cas, $a_1 = \frac1{P}$, $\alpha_1 = P$ (car toutes les variables $x_i$ sont égales) et on écrit $\beta$ à la place de $\beta_1$ et $N-\beta$ à la place de $\beta_2$, on écrit $y$ à la place de $b_1$ et $\frac{1-\beta y}{N-\beta}$ à la place de $b_2$, ce qu'on peut faire car $\beta b_1 + (N-\beta) b_2 =1$. Comme $b_1 > b_2$, on a que $\frac1{N} < y < \frac1{\beta}$. On pourrait à nouveau utiliser les multiplicateurs de Lagrange ou tout simplement chercher les extrémums d'une fonction d'une variable (ici, cela revient au même). On va utiliser la deuxième méthode ici. On veut trouver $y \in ]1/N,1/\beta[$ qui minimise la fonction $g$ qui a l'expression suivante :
\[
\frac{\frac1{P}y}{\frac1{P}+\beta y^2 +\frac1{N-\beta}-2\frac{2\beta y}{N-\beta}+\frac{\beta^2 y^2}{N-\beta}}.
\]

Sa dérivée par rapport à $y$ a le même signe que
\[
\frac1{P}-\beta y^2 + \frac1{N-\beta} - \frac{\beta^2 y^2}{N-\beta}.
\]

On en déduit la valeur de $y$ où cette dérivée s'annule :
\[
y=\sqrt{\frac{N+P-\beta}{\beta}\frac1{NP}}.
\]

Pour savoir si ce point correspond à un minimum local ou un maximum local, il faut regarder le signe de la dérivée seconde de $g$ par rapport à $y$. On voit que cette dernière a le même signe que :
\[
-2\beta y -2\frac{\beta^2 y}{N-\beta} <0.
\]

On a donc affaire à une fonction concave et $y$ correspond à un maximum local de cette fonction. Cette fonction n'admet donc pas de minimum dans l'ouvert considéré, ce qui est une contradiction.

On en déduit donc que $P'=1$ et $N'=1$, ce qui correspond au cas où tous les $x_i$ sont égaux et tous les $y_j$ sont égaux. Par élimination des cas, ceci correspond au minimum de $f$. On a déjà vérifié que dans ce cas, $f$ vaut $\frac1{n}$. Ce qui termine la démonstration.

\textit{Deuxième cas : $P=1$ et $N \geq 2$}

On démontre de la même manière que ci-dessus que $N' \leq 2$, puis que $N'=1$. Ainsi, le minimum qu'on trouve vaut bien $\frac1{n}$.

\textit{Troisième cas : $P=1$ et $N=1$}

Il est immédiat que dans ce cas le minimum vaut $\frac12$ (sachant qu'on est dans le cas $n=2$).
\end{sol}

\begin{exo}
(IMO 2020 P2)
Soit $a,b,c,d$ des réels tels que $a\geqslant b \geqslant c\geqslant d >0$ et $a+b+c+d=1$. Montrer que

\[(a+2b+3c+4d)a^ab^bc^cd^d<1\]

\end{exo}

\begin{sol}
Le voilà donc, cet exercice responsable de tous nos maux. Néanmoins, il n'est pas dénué d'intérêt.

On présente deux solutions avec deux philosophies différentes. Ne pas s'y tromper, aucune de ces solutions n'est vraiment facile.

\underline{\textit{Solution n$^\circ 1$ : la méthode élégante}}

On regarde le terme $a^ab^bc^cd^d$ comme un terme complètement artificiel, et on décide de s'en débarasser.

$a,b,c,d$ peuvent être vus comme des poids dans une moyenne car ils sont positifs et $a+b+c+d=1$. Ainsi, d'après l'inégalité arithmético-géométrique

\[a^ab^bc^cd^d \leqslant a*a+b*b+c*c+d*d = a^2+b^2+c^2+d^2\]

Il nous suffit donc de montrer que

\[(a+2b+3c+4d)(a^2+b^2+c^2+d^2)<1\]

La deuxième idée, assez classique, est d'homogénéiser l'inégalité en utilisant l'hypothèse que $a+b+c+d=1$ (ainsi, on se débarasse de cette hypothèse). On désire donc montrer que

\[ (a+2b+3c+4d)(a^2+b^2+c^2+d^2)< (a+b+c+d)^3\]

Cette inégalité n'est pas aussi facile à montrer qu'elle en a l'air. Il existe deux approches pour la montrer, la première est plus astucieuse, la deuxième est plus généralisable.

\textit{Première approche :}
On n'a pas encore utilisé l'ordre donné aux variables. Cela va venir maintenant. Un peu d'analyse : si on développe tout à droite, on aura $4^3=64$ termes, tandis que si on développe tout à gauche, on aura $4\cdot 10=40$ termes. On peut donc espérer qu'on pourra simplement minorer certains termes de $(a+b+c+d)^3$ par $0$. C'est ainsi que l'on va se contenter de considérer uniquement les termes du développement qui contiennent un facteur $a^2, b^2, c^2$ ou $d^2$ :

\[(a+b+c+d)^3 > a^3+b^3+c^3+d^3 + 3a^2 (b+c+d) + 3b^2(a+c+d) + 3c^2(a+b+d) + 3d^2(a+b+c)\]

On a donc

\[(a+b+c+d)^3 > a^2(a+3b+3c+3d)+b^2(3a+b+3c+3d) + c^2(3a+3b+c+3d)+d^2(3a+3b+3c+d) \]

Dû à l'ordre des variables, chaque somme entre parenthèse est minorée par $a+2b+3c+4d$, ce qui conclut.

\textit{Deuxième approche :}
On applique le principe "peu de variables petites et positives", c'est pourquoi on pose le changement de variables suivant :
\[
a=x+y+z+t,\quad b=x+y+z, \quad c=x+y,\quad d=x,
\]
et la seule hypothèse vérifiée par $x,y,z,t$ est qu'elles sont positives (ainsi, on a réduit au minimum le nombre d'hypothèses sur nos variables). On peut réécrire l'inégalité qu'on veut montrer avec les nouvelles variables
\[
(10x+6y+3z+t)((x+y+z+t)^2+(x+y+z)^2+(x+y)^2+x^2) < (4x+3y+2z+t)^3.
\]
On laisse le lecteur développer cette expression. La fin de la preuve ne pose alors plus de difficulté.

\medskip

\underline{\textit{Solution n$^\circ 2$ : la méthodé brutale}}

L'idée ici est de se ramener à une seule variable en utilisant les majorations à disposition. Par exemple, on a

\[a+2b+3c+4d \leqslant a +3b+3c+3d = 3-2a\]

et

\[a^ab^bc^cd^d \leqslant a^a a^ba^ca^d =a\]

Et pour $a<1/2$, on a bien $(3-2a)a<1$.

Il faut donc traiter le cas où $a\geqslant 1/2$. On est encore plus bourrin puisque l'on écrit

\[b^bc^cd^d < (1-a)^b (1-a)^c(1-a)^d = (1-a)^{1-a}\]

On veut donc montrer que $(3-2a)a^a(1-a)^{1-a} <1$. On appelle $f(a)$ le membre de gauche, et on désire étudier les variations de $f$. Une façon sympathique de procéder est d'étudier les variations du logarithme de $f$, noté $g$, plus facile à dériver et donc à étudier.

Une étude exhaustive de $g''$ montre que $g$ est convexe, elle atteint donc son maximum au bord de l'intervalle $]1/2, 1]$. Puisque $g(1/2)=0$ et que $g(1^-) =0$, on déduit que $g\leqslant 0$, donc $f\leqslant 1$, comme voulu.
\end{sol}

\begin{exo}
(EGMO 2014 P1)
Déterminer tous les réels $t$ tels que pour tout triplet $(a,b,c)$ désignant les longueurs des côtés d'un triangle, $a^2+bct$, $b^2+cat$ et $c^2+abt$ sont également les longueurs des côtés d'un triangle.
\end{exo}

\begin{sol}
Quand on voit un exercice qui parle des longueurs des côtés d'un triangle, on pense au changement de variables, dit de Ravi :
\[
a=y+z, \quad b=z+x, \quad c=x+y,
\]
où $x,y,z$ n'ont pour seule propriété que d'être positives.

Pour un $t \in \R$, on se demande si quelque soient $x,y,z > 0$, le triplet $(A,B,C)=(a^2+bct,b^+cat,c^2+abt)$ désigne les longueurs des côtés d'un triangle. On exprime $A,B,C$ en fonction de $(x,y,z)$ :
\[
A=y^2+z^2+x^2t+zxt+xyt+zy(2+t),
\]
\[
B=z^2+x^2+y^2t+xyt+yzt+xz(2+t),
\]
\[
C=x^2+y^2+z^2t+yzt+zxt+yx(2+t).
\]

Par symétrie des rôles, il suffit qu'on cherche la condition nécessaire et suffisante sur $t$ pour que $A+B-C>0$ pour tout $x,y,z>0$.

Tout d'abord,
\[
A+B-C=x^2t+y^2t+z^2(2-t)+xy(t-2)+yz(2+t)+zx(2+t).
\]

Étudions plusieurs cas de figure. Dans le cas où $x=y>0$ et $z=0$,
\[
A+B-C=2x^2t+x^2(t-2)=(3t-2)x^2.
\]
Ainsi, $A+B-C<0$ si et seulement si $t<\frac2{3}$. Ainsi, pour $t<\frac2{3}$, par continuité, en choisissant des $z$ petits, on trouve des $A+B-C$ strictement négatifs.

Dans le cas où $x=y=0$ et $z>0$,
\[
A+B-C=z^2(2-t).
\]
Ainsi, $A+B-C<0$ si et seulement si $t>2$. Ainsi, pour $t>2$, par continuité, en choisissant des $x,y$ suffisament petits, on trouve des $A+B-C$ strictement négatifs.

Ainsi, les seuls $t$ qui ont une chance de marcher sont tels que $\frac2{3} \leq t \leq 2$. Vérifions que $t\in[2/3,2]$ convient. On a
\[
A+B-C > x^2t+y^2t+xy(t-2) = t\left(x^2+y^2+\frac{t-2}{t}xy\right) \geq t(x^2+y^2-2xy) = (x-y)^2 \geq 0,
\]
car $-2 \leq \frac{t-2}{t} \leq 0$ pour tout $t\in[2/3,2]$.
\end{sol}






\begin{exo}
(EGMO 2016 P1)
Soit $n$ un entier positif impair, et soit $x_1,x_2,\dots,x_n$ des nombres réels positifs ou nuls. Montrer que :
\[
\min_{i=1,\dots,n}(x_i^2+x_{i+1}^2) \leq \max_{j=1,\dots,n}(2x_jx_{j+1}),
\]
où $x_{n+1}=x_1$.
\end{exo}

\begin{sol}
L'inégalité est contre-intuitive puisqu'on a un produit d'un côté et une somme de carrés de l'autre.

On est forcé de travailler avec l'hypothèse de $n$ impaire. a bonne façon de voir les variables en utilisant $n$ impaire est de placer les $x_i$ sur un cercle. entre $x_i$ et $x_{i+1}$, on mettra un $+$ si $x_{i+1}\geq x_i$ et un $-$ sinon. On obtient un nouveau cercle composé de $+$ et de $-$ contenant exactement $n$ signes. Puisque $n$ est impaire, on a deux signes consécutifs identiques, par exemple deux signes $+$ côte-à-côte ou deux signes $-$. Quitte à renverser l'ordre des $x_i$, on peut supposer que les deux signes identiques côte-à-côte sont identiques et donc que $x_0\leq x_1\leq x_2$.

Alors $$min _{1\leq i\leq n} (x_i^2+x_{i+1}^2)\leq x_0^2+x_1^2\leq x_1^2+x_1^2=2x_1^2\leq 2x_1x_2\leq max_{1\leq j\leq n}(2x_jx_{j+1})$$ et on a bien l'inégalité.
\end{sol}


\begin{exo}
(IMO SL 2020 A3)
Soient $a,b,c,d$ des réels strictement positifs vérifiant $(a+c)(b+d)=ac+bd$. Déterminer la plus petite valeur que peut prendre

\[\frac{a}{b}+\frac{b}{c}+\frac{c}{d}+\frac{d}{a}\]
\end{exo}

\begin{sol}
L'hypothèse suggère de mettre ensemble les termes en $a$ et $c$ d'une part, et les termes en $b$ et $d$ d'autre part. On peut donc regrouper comme suit (et cela conserve une certaine symétrie en $a$ et $c$ et en $b$ et $d$) :

\[\left(\frac{a}{b} +\frac{c}{d}\right) + \left(\frac{b}{c}+\frac{d}{a}\right)\]

Pour faire apparaître des facteurs $ac$ et $bd$, on utilise l'IAG :

 \[\left(\frac{a}{b} +\frac{c}{d}\right) + \left(\frac{b}{c}+\frac{d}{a}\right) \geqslant 2\sqrt{\frac{ac}{bd}}+ 2 \sqrt{\frac{bd}{ac}}\]

 On voudrait alors utiliser l'IAG à nouveau en reconnaissant un terme de la forme $x+1/x$. Mais cela est très gourmand : on n'a pas encore utilisé l'hypothèse de l'énoncé et un petit raisonnement montre qu'il n'existe pas de réels $a,b,c,d$ vérifiant $ac=bd$ et $(a+c)(b+d)=ac+bd$. En revanche, on peut tout de même poser $x=\sqrt{ac/bd}$ et déterminer la plus petite valeur que peut prendre $x+1/x$. La fonction en $x$ étant convexe, il suffit de déterminer l'ensemble des valeurs que peut prendre $x$ pour pouvoir minorer $f$.

\medskip

Or d'après IAG et l'hypothèse

\[ac+bd= (a+c)(b+d) \geqslant 2\sqrt{ac}\cdot 2 \sqrt{bd} \geqslant 4 \sqrt{acbd}\]

En divisant des deux côtés par $\sqrt{acbd}$, on obtient que $x+\dfrac1{x} \geqslant 4$. Ainsi, la somme est toujours supérieure ou égale à $8$.

\medskip

Pour vérifier que la valeur est bien atteignable, on utilise les cas d'égalité établis précédemment. On voit donc qu'il faut $a=c$ et $b=d$. Injecté dans l'hypothèse, cela donne $4ab=a^2+b^2$. On déduit une équation quadratique en $a/b$, qui donne pour solution $a/b= 2\pm \sqrt{3}$. La valeur $8$ est donc atteinte par exemple pour $b=d=1$ et $a=c=2+\sqrt{3}$.
\end{sol}


\begin{exo}
(BXMO 2012)
Déterminer tous les quadruplets $(a,b,c,d)$ de réels strictement positifs vérifiant $abcd=1$ et :

\[\begin{array}{lllllll}
a^{2012}+2012b &=& 2012c+d^{2012} &\text{ et } & 2012a+b^{2012} &=& c^{2012}+2012d\\
\end{array}\]
\end{exo}

\begin{sol}
Face à un tel exercice, la stratégie est la suivante :

\begin{enumerate}
\item Déterminer des solutions simples et se douter qu'il n'y en a pas d'autres.
\item Réinterpréter le problème comme un problème d'inégalité, et imaginer que les réels vérifiant les hypothèses sont en fait des réels vérifiant le cas d'égalité d'une certaine inégalité.
\item Démontrer une inégalité dans le cas général et conclure.
\end{enumerate}

Ici, on voit que les quadruplets vérifiant $a=d$ et $b=c$, couplé avec $abcd=1$, c'est-à-dire les quadruplets de la forme $(t, 1/t, 1/t, t)$, sont bien solutions. On devine qu'il n'y en a pas d'autres.

On réinterprète le problème comme un problème d'inégalité : on suppose que le quadruplet $(a,b,c,d)$ vérifie $a^{2012}+2012b=2012c+d^{2012}$ et $2012a+b^{2012}=c^{2012}+2012d$ et on souhaite comparer $abcd$ avec $1$.

\medskip

Dans la suite, on suppose que $a\neq d$ et $b\neq c$. On réécrit les deux égalités comme $a^{2012}-d^{2012}=2012(c-b)$ et $b^{2012}-c^{2012}=2012(d-a)$.

En multipliant les deux égalités, on trouve

\[(a^{2012}-d^{2012})(b^{2012}-c^{2012})= 2012^2 (d-a)(c-b)\]

que l'on réécrit :

\[1= \frac{a^{2011}+a^{2010}d+ \ldots +ad^{2010}+d^{2011}}{2012} \cdot \frac{b^{2011}+b^{2010}c+ \ldots +bc^{2010}+c^{2011}}{2012}\]

L'inégalité des moyennes donne alors,

\[ \frac{a^{2011}+a^{2010}d+ \ldots +ad^{2010}+d^{2011}}{2012} > (ad)^{2011/2}\]

L'inégalité est stricte car les variables ne sont pas égales. De même

\[\frac{b^{2011}+b^{2010}c+ \ldots +bc^{2010}+c^{2011}}{2012}> (bc)^{2011/2}\]

Le produit est donc strictement plus grand que $1$, contredisant l'égalité établie plus haut. Les solutions sotn donc bien celles annoncées.

\end{sol}

\begin{exo}
(BXMO 2019) Soit $0\leqslant a,b,c,d\leqslant 1$ des réels.

1) Montrer que

\[ab(a-b)+bc(b-c)+cd(c-d)+da(d-a) \leqslant \frac{8}{27}\]

2) Déterminer les cas d'égalité.
\end{exo}

\begin{sol}
Il s'agit d'un exercice de factorisation.

1) L'idée est de rassembler ensemble les termes en $a$ et $c$. Si on appelle $S$ le membre de gauche de l'inégalité :

\[\begin{array}{lll}
S &=& a[b(a-b)+d(d-a)] + c[b(b-c)+d(c-d)]\\
&&\\
&=& a(b-d)(a-b-d) + c(d-b)(c-b-d)\\
&&\\
&=& (b-d)(c-a)(b+d-a-c)\\
\end{array}\]

Sans perte de généralité, on peut supposer que $b+d\geqslant a+c, b\geqslant d$ et $c\geqslant a$. Alors par IAG

\[S\leqslant \left(\frac{b-d+c-a+b+d-c-a}{3}\right)^3 = \frac{8}{27}(b-a)^3\leqslant \frac{8}{27}\]

2) On regarde le cas dégalité du raisonnement précédent. Il faut notamment que $b=1$ et $a=0$. Il faut ensuite pour l'IAG que $c-a=b-d$, soit $c+d=1$ et il faut $b+d-a-c= c-a$ soit $b+d=2c$. On déduit que le cas d'égalité est réalisé pour $(0,1,2/3,1/3)$ et ses variantes.
\end{sol}





\begin{exo}
(IMO SL 2015 A1)
Soit $(a_k)$ une suite de réels strictement positifs telle que pour tout entier $k$ :

\[a_{k+1} \geqslant \frac{ka_k}{a_k^2 +(k-1)}\]

Montrer que pour tout entier $n\geqslant 2$, $a_1+a_2+\ldots +a_n \geqslant n$.
\end{exo}

\begin{sol}
Un exercice plus difficile qu'il n'y paraît. Il faut bien entendu commencer par regarder l'énoncé pour des petites valeurs.

Pour $n=2$, on a $a_2\geqslant \dfrac1{a_1}$, si bien que par IAG la somme est bien $\geqslant 2$.

Pour $n=3$, cela se complique. Une observation facile est que si $a_3\geqslant 1$, alors $a_1+a_2+a_3\geqslant 2+a_3 \geqslant 3$ comme voulu. Cette remarque nous encourage à procéder par récurrence.

Ainsi, pour l'hérédité, si $a_n\geqslant 1$, on a déjà gagné. Dans le cas contraire, il est désormais nécessaire de mettre se salir les mains. En passant la relation à l'inverse, on a

\[\frac{k}{a_{k+1}}- \frac{k-1}{a_k} \leqslant a_k\]

En sommant et en télescopant (c'est magnifique !) et en utilisant que $\frac1{a_n}\geqslant 1$. :

\[a_1+ \ldots + a_n \geqslant \frac{n-1}{a_n} + a_n \geqslant n-2 + \frac1{a_n}+a_n \geqslant n\]

ce qui achève la récurrence.
\end{sol}