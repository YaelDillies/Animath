\begin[nosol,poly]{document}

%Le cours s'inspirait du cours de Timothée Rocquet durant le stage d'été 2019 :
%https://maths-olympiques.fr/wp-content/uploads/2019/08/PolycopieV1-1.pdf p.245
%et du cours de Margaret Bilu durant le stage d'été 2013 :
%http://www.math.ens.fr/~budzinski/polys/Combinatoire/Avancé/2013_courstd.pdf
\titre {Séries génératrices}

\begin{exo}
Déterminer le nombre de manières de se servir $n$ aliments à la cantine sachant que les pommes sont prises par $3$, les yaourts vont par $2$, et qu'on n'a le droit qu'à $2$ morceaux de pain et un bol de céréales au plus pour cause de changement de prestataire.
\end{exo}

\begin{exo}
Les $80$ stagiaires du stage Animath choisissent chacun une activité pour l’après-midi libre parmi $5$ activités proposées. On sait que :
. La piscine a été au moins aussi populaire que le foot ;
. Les élèves allaient au shopping par groupe de $5$ ;
. Au plus $4$ élèves ont joué aux cartes;
. Au plus un élève est resté dans sa chambre.
En écrivant la liste des activités par ordre alphabétique, on écrit le nombre d’élèves correspondant à chaque activité. Combien de listes de nombres différentes a-t-on pu écrire ?
\end{exo}

\begin{exo}
Déterminer la série génératrice du nombre de manières payer $n$ centimes 
\end{exo}

\subsubsection{Partitions}

\begin{exo}
Montrer que le nombre de partitions de $n$ en parts impaires vaut le nombre de partitions de $n$ en parts distinctes ? \\
Montrer que le nombre de partitions of $n$ en parts non divisibles par $k$ vaut le nombre de partitions de $n$ où chaque part apparait au plus $k − 1$ fois.
\end{exo}

\begin{exo}
Peut-on partitionner $\mathbb Z$ en un nombre fini de suites arithmétiques de raisons distinctes ?
\end{exo}

\begin{exo}
Montrer que le nombre de partitions de $[\![1, n]\!]$ vaut
$$\frac 1e \sum_{k=0}^\infty \frac{k^n}{k!} $$
\end{exo}
\begin{sol}
Posons $B_n$ ce nombre de partitions. On a $a_{n+1} = \sum_{k=0}^n {n \choose k} a_{n - k}$. et $A$ la série génératrice associée
$\frac{B_{n+1}}{n!} = \sum_{k+l=n} \frac 1{k!}\frac{B_l}{l!}$
\end{sol}

\subsubsection{Suites arithmétiques}

\begin{exo}
On a $n$ suites arithmétiques qui couvrent les entiers de $1$ à $2^n$. Montrer qu'elles couvrent tous les entiers.
\end{exo}

\end{document}