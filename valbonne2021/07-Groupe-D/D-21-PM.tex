%Cours Combinatroire.graphes Groupe D 24/08/2021 Après midi Colin

L'objet de ce TD fut d'étudier des exercices variés en théorie des graphes sans essayer d'appliquer de théorie générale.

\subsubsection*{Lemme des mariages de Hall}

Voici le lemme des mariages de Hall. On en trouve de nombreuses preuves dans la littérature.

\begin{lem}
Considérons un ensemble $X$ de $n$ filles et un ensemble $Y$ de $n$ garçons. Chaque fille $x$ apprécie un ensemble $A(x)$ de garçons.
\smallskip

 On suppose qu'il n'y a pas de groupes de filles trop exigentes, c'est à dire que pour tout ensemble $X_0\subset X$ de filles alors $|X_0|\leq |\cup_{x\in X_0} A(x)|$. Alors il est possible de marier chaque fille avec un garçon de façon bijective, et de sorte que toutes les paires fille/garçon s'apprécient.
\end{lem}

Avant de prouver ce lemme remarquons que si il est possible d'organiser les mariages, alors l'hypothèse du lemme doit également être vérifiée.

\begin{proof}

Nous allons procéder par récurrence forte sur $n=1$. Le cas $n=1$ est clair. 

\medskip

Si $n>1$, supposons qu'il existe un ensemble $X_0\subset X$ de taille $n>k>0$ qui soit saturé, c'est à dire que $|X_0|=|\cup_{x\in X_0}A(x)|$. Alors si on retire les garçons et filles de ces deux ensembles, il reste $0<n-k<n$ garçons et autant de filles. De plus on peut vérifier que l'hypothèse du lemme est vérifiée pour tous ceux qui restent. Ainsi on peut marier ces $n-k$ garçons et filles, par hypothèse de récurrence forte, ainsi que les $k$ garçons et filles oubliés.

Il reste à traiter le cas où aucun ensemble n'est saturé. Dans ce cas, on marie tout simplement n'importe quel garçon avec n'importe quelle fille, et l'hypothèse reste encore vérifiée puisqu'aucunb ensemble n'est saturé. On conclut par récurrence.
\end{proof}

\begin{exo}
Le petit Nicolas découpe deux feuilles carrées de côté 2019 en $2019^2$ polygones d'aire 1 de deux facons différentes. Ensuite il remet les morceaux ensemble et superpose les deux feuilles. Montrer qu'il peut planter $2019^2$ punaises à travers les deux feuilles de façon à ce tous les polygones soient percés.
\end{exo}

\begin{sol}

On applique le lemme des mariages de Hall avec les polygones de la première feuille à la place des filles, les polygones de la seconde feuille à la place des garçons et deux polygones sont compatibles si lorsque l'on superpose les feuilles il ont un point d'intersection. Pour un ensemble $X_0$ de polygones, l'union des polygones compatibles avec ceux-ci englobe l'union des polygones de $X_0$, et par conséquent leur aire vaut au moins le cardinal de $X_0$ Par conséquent le lemme peut être appliqué, et on en déduit l'existence d'une bijection entre les $2019^2$ polygones de chaque côté.

\smallskip
 Il reste à prendre pour chaque paire un point où elles se chevauchent, et à planter une punaise à cet endroit.

\end{sol}

\begin{exo}
Dans une grille $8\times 8$ des cailloux sont posés sur certaines cases de sorte que sur chaque ligne et chaque colonne il y a exactement $3$ pierres. Montrer qu'il existe un ensemble de $n$ pierres telles que deux d'entre elles ne sont jamais sur la même ligne ou colonne. 
\end{exo}

\begin{sol}
On applique encore le lemme des mariages de Hall. Les colonnes ici jouent le rôle des filles, et les lignes jouent celui des garçons. Une ligne et une colonne sont compatibles si il y a un cailloux à leur interection. 

Si on considère $k$ lignes, elles contiennent $3k$ pierres réparties sur $\ell$ colonnes. Or chaque colonne contient au plus $3$ pierres parmis les $3k$ pierres, donc $\ell \geq k$. 

\smallskip

Ainsi le lemme des mariage de Hall nous donne $n$ paires de lignes et colonnes dont les points d'intersections contiennent des pierres : ces $n$ pierres conviennent.
\end{sol}

\begin{exo}

Sur une planète lointaine, il y a $n$ hommes, $n$ femmes et $n$ matheux. Une famille heureuse est une famille composée d'un homme, une femme et un matheux. Cependant pour former une famille, il faut que les trois membres s'apprécient (le fait de s'aprécier est toujours réciproque). Est-t-il toujours possible de former $n$ familles disjointes sachant que chaque membre de chacun des trois groupes apprécie au moins $k$ personnes dans chacun des deux autres groupes (donc $2k$ personnes en tout), et si :

\begin{itemize}
\item[(i)] $2k=n$ ?
\item[(ii)] $4k=3n$?
\end{itemize} 
\end{exo}

\begin{sol}

Dans le cas $(i)$ il est possible qu'aucune famille ne puisse être créée. En effet séparons chaque groupe en deux moitiés $A$ et $B$. Si les hommes $A$ apprécient les femmes $A$ et les matheux $B$ alors que les femmes $A$ apprécient les matheux $A$ et que à l'inverse les hommes $B$ apprécient les femmes $B$ et les matheux $A$ alors que les femmes $B$ apprécient les matheux $B$, aucune famille ne peut être créée.

\medskip

Dans le cas $(ii)$, il est toujours possible de créer des familles. On commencer par créer des couples homme/femme à l'aide du lemme de Hall. C'est possible car si on a un groupes de $\ell\geq 1$ femmes, et si $\ell\leq k$ alors apprécient à elles toutes au moins $k$ hommes. Si $\ell>k\geq n-k$, alors tous les hommes doivent apprécier au moins une femme de ce groupe. Ainsi par le lemme de Hall on peut créer ces couples.

\medskip
Ensuite on cherche à appareiller chaque couple avec un matheux. Puisque $k>\frac{3}{4}n$, alors un homme et une femme ont au moins $\frac{n}{2}$ matheux qu'ils apprécient tous deux. De plus un matheux apprécie aussi au moins $\frac{n}{2}$ couples. Ainsi par le même raisonnement, on peut appareiller les couples et les matheux pour créer les triplets souhaités.
\end{sol}

\begin{exo}
On considère une grille $n\times m$ sur laquelle sont posés des pierres. Soit $k$ le plus petit nombre de lignes et colonnes qu'il faut supprimer pour supprimer toutes les pierres. Montrer qu'il existe $k$ pierres telles que deux de ces pierres ne soient jamais sur la même ligne ou colonne.
\end{exo}

\begin{sol}
Si $k=n=m$ on retrouve exactement le lemme des mariages de Hall. Cet exercice est un lemme attribué à König.

\medskip

Nous allons reformuler le problème comme suit. On considère un graphe bipartite dont l'ensemble de sommets est partitionné en deux ensembles $A$ et $B$ et avec un ensemble d'arrêtes $E$. On suppose que le plus grand ensemble d'arrêtes tel que deux arrêtes choisies n'ont pas de sommet en commun est de taille $k$. Soit $E_0\subset E$ un tel ensemble.

\medskip

 Etant donné une arrête $e\in E_0$ on dit qu'elle est de type $A$ si il existe une suite de sommets distincts $a_0, b_0,a_1,b_1,\cdots b_x,a_x$ où $a_x$ n'appartient à aucune arrête de $E_0$ et tel que pour tout $0\leq y\leq x$, $\{a_i,b_i\}\in E_0$ et $\{b_i,a_{i+1}\}\in E$ avec $e=\{a_0,b_0\}$. 
 De même on dit que $e$ est de type $B$ si il existe une suite de sommets distincts $b_0, a_0,b_1,a_1,\cdots a_x,b_x$ où $b_x$ n'appartient à aucune arrête de $E_0$ et tel que pour tout $0\leq y\leq x$, $\{b_i,a_i\}\in E_0$ et $\{a_i,b_{i+1}\}\in E$ avec $e=\{a_0,b_0\}$.
 
 \medskip
 
 Si une arrête était de type $A$ et $B$, on aurait un chemin alternant entre $A$ et $B$. Si on enlève les arrête de $E_0$ de ce chemin et que l'on ajoute les arrêtes de ce chemin qui ne sont pas dans $E_0$, on obtient un ensemble $E_0$ plus grand contredisant la maximalité de $E_0$. Ainsi les arrêtes de $E_0$ ne sont pas des deux types.
 
 \medskip
 
 Pour une arrête de type $A$, on choisis le sommet dans $B$, pour une arrête de type $B$ on choisis le sommet dans $A$ (si une arrête n'a pas de type on choisi l'un des deux). Ces sommets choisis couvrent bien toutes les arrêtes, et sont au nombre de $k$ au plus, ce qui conclut.

\end{sol}

\begin{exo}%{[IMO C2]}

Sur une planête il y a $2^N$ pays $(N \geq 5).$ Chaque pays a un drapeau composé d'une ligne de $N$  carrés de côté $1$, chacun étant en jaune ou bleu. Les drapeaux sont $2$ à $2$ distincts. Un ensemble de $N$ drapeaux est divers si ils peuvent être placés dans un certain ordre pour former un carré $N \times N$ tel que les $N$ carrés sur la diagonale principale soient de la même couleur. Déterminez le plus petit entier $M$ tel que tout ensemble de $M$ drapeaux contient un ensemble divers de $N$ drapeaux.
\end{exo}


\begin{sol}
Si $M=2^{N-2}$, on peut choisir tous kes drapeaux qui finissent par jaune puis bleu. Il est impossible d'avoir un sous ensemble divers de cet ensemble de drapeaux.

\medskip

Montrons que $M=2^{N-2}+1$ est optimal. Considérons la grille $M\times N$ obtenue en mettant les drapeaux les uns sur les autres. Par la contraposée de l'exercice précédent, si à permutation près il n'est pas possible d'observer de diagonale monochrome, c'est qu'il existe $a$ lignes et $a'$ colonnes avec $a+a'<N$ qui recouvrent toutes les cases jaunes, et $b$ lignes et $b'$ colonnes avec $b+b'<N$ qui recouvrent toutes les cases bleues.

\medskip


Soit $A$ et $B$ les ensembles de colonnes choisies, de taille $a'$ et $b'$. Supposons que $A\cup B$ n'est pas égal à l'ensemble des colonnes. Puisque toutes les cases de la colonne restante sont bleues ou jaunes, $a+b\geq M$, donc $2N-2\geq 2^{N-2}+1$, ce qui est impossible car $N\geq 5$. Ainsi $A\cup B$ recouvre toutes les colonnes. 

\medskip

En particulier il n'y a que $2^{|A \cap B|}+a+b$ drapeaux possibles dans cet ensemble, car en dehors des $a+b$ lignes choisies, les couleurs sur les colonnes hors de $A\cap B$ sont déjà fixées. Ainsi $2^{|A \cap B|}+a+b> 2^{N-2}$. Or $|A\cap B|<N-(N-a')-(N-b')\leq N-2-a-b$. Ainsi si $x=a+b$, $2^{N-2-x}+x> 2^{N-2}$, ce qui est impossible, d'où le résultat.

\end{sol}

%\begin{exo}
%Une compagnie aérienne propose des vols entre toutes les capitales de l'union européenne. Le coût d'un vol entre deux villes est le même dans les deux sens. Anna et Bella partent chacune d'une capitale, pas forcément la même. Anna prend systématiquement le vol le moins cher parmis les vols qui la mènent à une capitale non encore visitée, et Bella prend systématiquement le vol le moins cher parmis les vols qui la mènent à une capitale non encore visitée. De plus les prix des vols sont tous distincts. Lorsqu'elles ont visité toutes les capitales, Anna et Bella comparent le prix total qu'elles ont payé. Est-t-il possible qu'Anna ait plus dépensé que Bella ?
%
%\end{exo}

\subsection*{Graphes orientés.}

\begin{exo}
Il faut attribuer $n$ maisons à $n$ personnes. Chaque personne range les maisons dans un ordre, sans égalités. Après les attributions, on remarque que toute autre attribution attribuerai à au moins une personne une maison moins bien classée. Montrer qu'il existe une personne qui a eu son premier choix de maison.
\end{exo}

\begin{sol}

Supposons que les personnes  et les maisonssont numérotées de $1$ à $n$ , et que la maison $i$ est attribuée à la personne $i$. On trace le graphe orienté avec une flèche entre $i$ et $j$ si la personne $i$ a placé la maison $j$ en première position. Ce graphe orienté a la propriété que chaque sommet a un degré sortant de $1$, donc il existe un cycle. Si ce cycle est de taille $1$, on a gagné. 

\medskip

Si on obtient un cycle $c_0,c_1,\cdots,c_k=c_0$, alors en associant à $c_i$ la maison $c_{i+1}$, on donnerai à chacun une maison qu'il préfère à celle qu'il avait avant, contredisant la condition de l'énoncé.
\end{sol}

\begin{exo}
Une grille $n\times n$ est remplie avec les nombres $-1,0,1$ de sorte que chaque ligne et colonne contienne exactement un $1$ et un $-1$. Montrer que l'on peut permuter les lignes et les colonnes de sorte à obtenir le même tableau avec les signes échangés. 
\end{exo}

\begin{sol}
Ce tableau décrit un graphe orienté: si dans la ligne $i$ il y a un $1$ en position $a$ et un $-1$ en position $b$ on place une arrête entre $a$ et $b$. Il y a exactement une arrête entrante et sortante à chaque sommet, donc le graphe est une union disjointe de cycles. Pour chaque cycle $c_0,c_1,\cdots,c_k=c_0$, on peut permuter les lignes et colonnes correspondantes de sorte à envoyer $c_1,c_2,\cdots, c_k$ sur $c_k,c_{k-1},\cdots,c_1$. Si on effectue cette permutation pour chaque cycle, on obtient le graphe opposé, ce qui revient à échanger les $+1$ et $-1$.
\end{sol}

\begin{exo}
Dans une élection, il y a $2042$ candidats et des votants. Chaque votant ordonne les candidats dans l'ordre de son choix.

On forme un graphe orienté de $2042$ sommets, avec une flèche de  $U$ vers $V$ lorsque le candidat $U$ est placé au dessus de $V$ pour strictement plus de la moitié des votes. En cas d'égalité il n'y a pas d'arrêtes.

Est-il possible d'obtenir tous les graphes complets orientés connexes ?
\end{exo}

\begin{sol}
Pour chaque paire de sommets $a,b$ on note $e_{a,b}$ le nombre de vote pour lequel $a$ est mieux classé que $b$ moins le nombre des autres votes. Il est possible d'ajouter des votants de sorte à faire grandir arbitrairement $e_{a,b}$ ou $e_{b,a}=-e_{a,b}$ sans changer les autres écarts: il suffit d'ajouter un votant pour chaque ordre dans lequel $a$ est au dessus de $b$ (ou l'inverse).

\medskip

Avec cette opération on peut construire le gaphe orienté que l'on souhaite.
\end{sol}


\begin{exo}
Un tournois est un graphe complet où chaque arrête a une orientation. Un coloriage propre des arrêtes Est un coloriage des arêtes tel que si $a,b,c$ sont 3 sommets distinct du graphe, et si l'arrête entre $a$ et $b$ pointe vers $b$, et l'arrête entre $b$ et $c$ pointe vers $c$, alors ces deux arrêtes sont de couleur différente. 

Le nombre chromatique des arrêtes d'un graphe orienté complet est le plus petit nombre de couleurs nécessaire pour obtenir un coloriage propre des arrêtes. Quel est le plus petit nombre chromatique des arrêtes possible pour un graphe orienté complet à $n$ sommets ?
\end{exo}

\begin{sol}

Soit $k$ l'unique entier tel que $2^{k-1}<n\leq 2^k$. Supposons que les sommets du graphe soient numérotés de $0$ à $n-1$, et que les arrêtes pointent toujours vers le sommet au plus petit numéro. On colore de la couleur $1\leq i\leq k$ l'arrête entre les sommets numéroté $a$ et $b$ si le plus grand chiffre dans l'écriture en base $2$ de $a$ et $b$ qui diffère est le $k$-ième. Ce coloriage en $k$ couleurs est bien propre.

\medskip

Montrons qu'aucun graphe orienté complet à $n$ sommets ne peut être colorié avec $k-1$ couleurs. En effet pour la couleur $1\leq i\leq k-1$, on peut définir l'ensemble $A_i$ des sommets qui sont point d'arrivée d'une arrête de couleur $i$, et $B_i$ les autres. Etant donnée deux sommets $x$ et $y$, il existe au moins un $i$ (la couleur de l'arrête entre $x$ et $y$ tel que $x,y$ ne soient pas dans la même partie de la partition $A_i\cup B_i$. Ainsi l'ensemble des $i$ pour lesquels $x\in A_i$ est différent pour chaque sommet $x$, il y a donc au plus $2^{k-1}$ sommets, ce qui est absurde car $n>2^{k-1}$, d'où le résultat.

\end{sol}

\begin{exo}%{[Olympiades Russes]}
Soit $G$ un graphe orienté (sans arrêtes pointant d'un sommet vers lui même), tel que chaque sommet admet au moins deux arrêtes sortantes et deux arrêtes entrantes. Supposons que $G$ est fortement connexe. Montrer qu'il existe un cycle dans le graphe tel qu'en le supprimant le graphe reste fortement connexe. 
\end{exo}

\begin{sol}
On considère le plus grand cycle $c$ qui ne passe pas deux fois sur la même arrête. Ce cycle contient au moins un sous-cycle $c'$ qui ne passent pas deux fois par le même sommet. Notons $c_0,c_1,c_2,\cdots,c_k=c_0,c_{k+1},\cdots,c_n=c_0$ le cycle $c$ et 
le cycle $c'$ le cycle contenant les $k$ premières étapes.

\medskip

Montrons que si on supprime ce cycle, le graphe reste fortement connexe. Pour cela montrons que tout sommet du cycle $c'$ peut mener à un sommet du cycle $c\setminus c'$. De même on pourra montrer en inversant le même raisonement que tout sommet du cycle $c'$ peut être atteint par un sommet du cycle $c\setminus c'$, et on en déduira alors que le graphe est fortement connexe.

\medskip

Soit $1\leq\ell_1<k$, et supposons que le sommet $c_{\ell_1}$ ne puisse pas mener à $c\setminus c'$. Ce sommet dispose d'au moins deux arrêtes sortantes, donc il mène à un sommet $b$ par une arrête qui n'est pas dans le cycle. En continuant d'utiliser des arrêtes qui ne sont pas dans le cycle, et puisque le graphe était fortemment connexe, il est possible de trouver un chemin disjoint du cycle qui va de $c_{\ell_1}$ à $c_{\ell_2}$ pour un certain $1\leq\ell<k$. On peut recommencer à partir de $\ell_2$, et définir $\ell_3$, $\ell_4,\cdots$, juqu'à ce qu'il existe $i,j$ tels que $\ell_i=\ell_j$. 


Ainsi on dispose d'un chemin de $c_{\ell_i}$ à $c_{\ell_j}$ qui est disjoint du cycle initial. Il est possible de faire en sorte que ce chemin ne passe pas deux fois par la même arrête, quitte à le réduire, sans le rendre trivial. Ainsi on peut alonger le cycle initial, ce qui est une contradiction.

\medskip

Cela conclut la solution.


\end{sol}
