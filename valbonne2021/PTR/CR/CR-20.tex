\begin{center}
{\textbf{Jeudi 20 août 2020 : derniers cours de la première période}}
\end{center}
\vspace{2mm}


Suivant la routine désormais rodée, 9h sonne le début d’une matinée d’arithmétique pour le groupe D avec Vincent, d’équations fonctionnelles pour le groupe C avec Rémi, et de combinatoire pour le groupe B avec Aline. Pendant ce temps, le groupe A se régale d’un TD pot-pourri concocté par Pierre-Marie et Martin.


%Oups, pas si vite !
Cependant, notre bénévole infirmier est à nouveau sollicité, pour gérer promptement une suspicion d’appendicite : Antoine doit quitter l’équipe d’animatheurs plus tôt que prévu  ; décidément le travail du responsable santé n’aura jamais été aussi intense et nous devons une fière chandelle au professionnalisme de Pierre-Marie.

Il n’est d’ailleurs pas au bout de ses peines puisqu’il lui reste à prendre en charge une petite foulure au poignet à montrer au médecin, résultat d’une partie de football enthousiaste la veille. Espérant pouvoir se concentrer à présent pleinement sur de paisibles mathématiques, il retrouve après le déjeuner Mathieu B., qui donne une deuxième séance de formation aux nouveaux animatheurs (ceux-ci apprenant à leurs dépends que la stratégie gagnante au jeu de PacMan n’est pas évidente).


%Étape 1 : trouver une bonne idée
Le programme de l’après-midi est plus pratique que théorique : le groupe A travaille les inégalités avec Paul, les lycéens ont à leur tour un pot-pourri par Matthieu B. et Auguste, le groupe C fait de l’arithmétique avec Matthieu P. et c’est sur de la combinatoire que Colin et Omid occupent les plus avancés.


%Étape 2 : convaincre qu’on a raison
Avant la séance d’entraînement de demain, rien de tel qu’un petit tour sur les terrains de sport pour se détendre : ce sont Rémi et Mathieu B. qui accompagnent les nombreux volontaires (sans rien de cassé cette fois !).


%Le sport, c’est la santé !

%Quelle est la stratégie gagnante au jeu de ballon ?
Le renouvellement de l’équipe encadrante ente la première et la deuxième période du stage commence : nous disons dès ce soir au-revoir à Ilyes et Paul.

Quelle est la bonne dose de mathématiques avant un entraînement ? L’heure optimale pour aller dormir ? Y a-t-il plus de moustiques dedans ou dehors ? Pour que chacun passe la soirée comme il l’entend, les conférences reprendront après-demain. Mais les stagiaires sont incorrigibles, il est presque difficile de trouver du monde pour occuper une table de jeu… une grande partie de Sporz et quelques jeux de cartes finissent tout de même par donner un peu de répit à la muraille, ouf. Pour profiter de la journée de demain, qui promet d’être riche en surprises (mais chut) il reste à faire le plein de sommeil.