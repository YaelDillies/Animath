\begin{center}
{\textbf{Mardi 18 août : c’est parti !}}
\end{center}
\vspace{2mm}

Serait-ce un petit air de rentrée ? Les cigales chantent toujours, le soleil est résolument aoûtien, pourtant dès 9h, nos 80 stagiaires ont retrouvé tableau noir, bureaux, cahiers et stylos. Auguste initie les collégiens à la façon un peu exotique de faire des mathématiques à Animath, Paul et le principe de récurrence triomphent rapidement de la timidité du groupe B. Pendant ce temps chez les plus avancés, les uns approfondissent leur connaissance des polynômes et les deniers discutent de combinatoire et de graphes avec Omid.


%Le groupe B pratique la récurrence avec Paul


Il n’y a pas qu’en classe que l’on travaille : les animatheurs ont fort à faire avec la quantité prolifique de solutions (ou tentatives) déposées au pied de la muraille, même avec l’aide nouvellement arrivée de Colin.


%Plus on en corrige, plus il en arrive…
Malheureusement notre infirmier volontaire se trouve lui aussi sollicité (pas de virus en vue, non) et un élève doit rentrer avec ses parents en attendant de se sentir mieux ; nous espérons le revoir vite en forme.

Le trombinoscope se remplit peu à peu, les habitudes reviennent comme celle de prolonger le cours des avancés une bonne demi-heure, mais tout le monde a le temps d’une pause déjeuner réconfortante avant de reprendre le travail. Nouvelle demi-journée, nouvelle musique : on découvre la géométrie avec Auguste en groupe A, et tous les autres font de l’arithmétique avec Ilyes, Théo ou Raphaël.

A 17h, c’est le départ de la première équipée courses avec une dizaine d’élèves pour un petit ravitaillement déjà longuement attendu. Toutefois, suite à une panne de courant à l’unique supérette des environs, le groupe en sera quitte pour une demi-heure de randonnée dans les pins et un peu de patience… Pendant ce temps, Mathieu Barré nous a rejoint et tout le monde peut aller dîner avant d’aller écouter l’exposé passionnant que Pooran nous propose sur les triangulations.


%Connaissez-vous le théorème de la galerie d’art ?


La soirée se réorganise sur un modèle similaire à celui d’hier qui a remporté un franc succès ; les stagiaires ont apporté de quoi agrandir et diversifier encore notre stock de jeux, et à 23h il est temps de se retourner sous les étoiles trouver un repos bien mérité.

