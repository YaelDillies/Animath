\begin{center}
{\textbf{Lundi 17 août : bienvenue en Provence}}
\end{center}
\vspace{2mm}


Venus des quatre coins de France voire de quelques contrées aussi exotiques que la Belgique ou la Pologne, la plupart des heureux lauréats de la coupe Animath de printemps sont arrivés sans encombre jusqu’à lieu de villégiature. Certains malheureusement devront suivre à distance lorsque la mobilité transfrontalière était trop contrariante. Mais en dépit de quelques menus détails comme le port du masque, ceci ressemble à n’importe quel bon début de stage : les élèves explorent le domaine et s’attaquent à la mythique muraille de 144 exercices…


%A peine installée, la muraille d’exercices commence à subir des assauts !
Avant de commencer les cours, chacun passe déjà un petit questionnaire suivi d’un cours entretien pour la constitution des groupes : collégiens débutants (A), lycéens débutants (B), élèves avec déjà une certaine expérience des mathématiques olympiques (C) et élèves avancés (D). Le groupe E (IMO) est une nouveauté de cette année, puisque le report de la compétition offre à l’équipe de France une occasion de se perfectionner ensemble.

La première soirée est dédiée à la traditionnelle conférence de présentation d’Animath et de la POFM, par Vincent Jugé, et à la remise de leur récompense aux premiers prix de la coupe Animath.


%Maris David (cinquième), Serge Bidallier (quatrième) et Inès Soua (troisième). Bravo à eux !

%Beau sérieux, non ?
Enfin, c’est le moment tant attendu où l’on découvre la couleur du nouveau t-shirt de la collection 2020, après le violet et le bleu, place au orange !


%Demandez votre taille !
Nous avons ensuite jusqu’à 23h30 pour profiter de la première nuit, qui s’organise en de nombreux groupes de jeux de société avant d’aller dormir ; il s’agit d’être en forme demain car les choses sérieuses commencent…


%En aurons-nous assez pour 10 jours ?
