\begin{center}
{\textbf{Mercredi 19 août 2020 : température de l’air 30$^\circ$ C, température de l’eau…}}
\end{center}
\vspace{2mm}


L’atmosphère estivale s’annonce pesante aujourd’hui, mais le programme continue comme prévu. Ce matin, il n’y a pas que les élèves qui ont cours : combinatoire en groupes B et D avec Pierre-Marie et Matthieu P., arithmétique pour les groupes C et E avec Rémi et Vincent, pendant que le groupe A poursuit sa découverte des disciplines olympiques avec une première séance d’inégalités par Raphaël. En effet, Mathieu B. s’occupe d’une séance de formation animatheurs, ce qui laisse peu de monde pour faire face à la montagne de copies de la muraille…


%Tout est dans la figure, il s’agit de se concentrer
Pas de prolongations ce matin, puisque la séance de photos-souvenirs empiète déjà sur la pause. C’est la seule occasion d’enlever son masque (aaah on y respire tout de même mieux) pour 30 secondes, à bonne distance bien sûr.


%Tout compte fait, même avec masque, on est photogéniques non ?
Mathieu B. reprend le groupe A pour parler de géométrie cet après-midi, Antoine et Théo font de l’arithmétique avec les groupes B et D pendant que le groupe C passe de la théorie à la pratique des polynômes, en TD avec Ilyes et Aline. Au milieu de l’après-midi, Paul et Colin qui ont apparemment réussi à trouver le magasin ouvert, ramènent des gourdes pour tous les élèves. Inutile de le dire deux fois, par cette chaleur étouffante !


%Bon, passons aux exercices moyens après ce petit apéritif…
Les rigueurs climatiques et une journée studieusement remplie ne semblent pas avoir entamé l’énergie des stagiaires, qui accourent en nombre à la séance de volley/foot (masqué s’il vous plaît !) proposée par Rémi. Autre activité phare, la razzia sur le rayon goûter de la supérette ne rentre pas bredouille, loin de là.


%A la recherche de la stratégie gagnante

%Spra…gue…Grun…dy…
Juste après le dîner, la conférence de ce soir est animée par Colin et nous parle de stratégies de jeux, du fameux théorème de Sprague-Grundy, et d’autres sommes de Nim. Malgré un exposé passionnant, l’appel de l’air frais du dehors – en comparaison de l’atmosphère péniblement ventilée de d’amphithéâtre – nous pousse à écourter les questions. Mais puisqu’il fait décidément bon sur la terrasse, même ceux qui préparent ou corrigent encore des cours et des exercices y rejoignent les tables de joueurs. De toute façon, les moustiques s’activent autant dedans, dehors, sous le soleil, les lampadaires, les étoiles ou les néons… Fait remarquable, les stagiaires cette année observent aussi scrupuleusement les consignes d’horaires que les contraintes sanitaires : la soirée animée se termine à 23h30 comme toujours.


%Ici un Sporz hautement prometteur…