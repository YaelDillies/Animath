\begin{center}
{\textbf Jeudi 19 août 2021 : Veille de l’entraînement}
\end{center}
\vspace{2mm}

Sans que l’on s’en soit aperçu, on est déjà presque arrivé à la moitié du stage ! Il va bientôt être temps de prendre un peu de repos. Mais, en attendant, on apporte les dernières touches au contenu mathématique de la première partie du séjour.

9h. Les groupes A et B doivent à présent appliquer toutes leurs nouvelles connaissances en géométrie dans un TD avec Raphaël et Aurélien respectivement. Les élèves du groupe C assiste à leur dernier cours d’arithmétique du séjour avec Arthur. Martin et Alexander entraînent les membres du groupe D en inégalités.

13 h 30. Cette fois-ci, c’est aux groupes A et B de se heurter aux inégalités avec Auguste et Victor. Combinatoire avec Anna dans le groupe C, et arithmétique dans le groupe D avec Rémi et Pierre -Marie.

L’après-midi, les diverses activités sportives et ludiques continuent. Courses, volley, piscine, foot, jeux de sociétés…

Photo A
L’eau tiède ne refroidit pas l’enthousiasme des baigneurs. Une partie de jeux de balles ne tarde pas de s’organiser.

Hormis le stage olympique de mathématiques, le CIV accueille aussi d’autres évènements et rassemblements : stages de judo, chorales d’enfants et cette année : une convention nationale de science-fiction ! Pour quelques jours, le grand Hall de l’agora se remplit de stands avec livres et T-shirts thématiques.

Et, surprise, les initiés offrent, le temps d’une soirée, la possibilité à une quinzaine d’élèves d’infiltrer leurs rangs et participer à une conférence sur la physique dans la science-fiction à 18h.

Photo B

Il n’y a pas de présentation la veille de l’entraînement de demain. Pas besoin de se faire de soucis pour les élèves tout autant : ces derniers savent parfaitement comment s’occuper.

Photo C
Quel air d’insouciance ! S’ils savaient l'activité intellectuelle acharnée qui se déroule en salle anim pendant ce temps…