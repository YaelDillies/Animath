\subsubsection{Définition des nombres premiers}


\begin{dfn}[Nombre premier]
Un nombre premier est un entier naturel qui a exactement $2$ diviseurs naturels : $1$ et lui-même.
\end{dfn}

\begin{rem}
$1$ n'est pas un nombre premier car qu'il n'a qu'un seul diviseur naturel. Cela peut paraitre bizarre de choisir une définition de premier qui exclut $1$, mais on verra que ça nous simplifie la vie.
\end{rem}


\begin{exo}
Trouver tous les premiers entre $1$ et $15$.
\end{exo}


\begin{exo}
Trouver tous les premiers pairs.
\end{exo}


\begin{exo}
Combien y a-t-il de paires de premiers consécutifs ?
\end{exo}


\begin{exo}
Trouver tous les premiers de la forme $a^2 - 1$, avec $a\ge 2$ naturel.
\end{exo}


\begin{exo}
Est-ce que tous les nombres de la forme $n^2 + n + 41$, avec $n$ naturel, sont premiers ?
\end{exo}


\begin{exo}
Soit $n$ un entier naturel. On veut tester si $n$ est premier.
On peut tester si $n$ est divisible par $2, 3, 4, \dots, n-1$. On sait alors que $n$ est premier si et seulement s’il n'est divisible par aucun de ses nombres. Montrer qu'en fait, il suffit de tester que les diviseurs plus petits que $\sqrt n$, ce qui donne une façon plus rapide de tester si un nombre $n$ est premier.
\end{exo}


\begin{exo}[Difficile]
Soit $n$ naturel. Montrer que si $2^n - 1$ est premier, alors $n$ est premier.
\end{exo}


\begin{exo}
Montrer que si un premier $p$ divise un premier $q$, alors $p=q$.
\end{exo}


\begin{exo}
Soit $p$ premier et $n$ naturel. Montrer que soit $p$ divise $n$, soit $p$ et $n$ sont premiers entre eux.
\end{exo}


\subsubsection{Lemme d'Euclide}


\begin{lem}[Lemme d'Euclide]
Soit $p$ premier et $a, b$ naturels. Si $p$ divise $ab$, alors soit $p$ divise $a$, soit $p$ divise $b$
\end{lem}

\begin{rem}
La condition $p$ premier est indispensable. De fait, $6$ divise $12 = 3 \times 4$, mais $6$ ne divise ni $3$ ni $4$. Cela arrive parce que $6$ n'est pas premier.
\end{rem}

On va prouver le lemme d'Euclide plus tard, puisque la preuve est assez compliquée et technique.


\begin{exo}
Soit $a$ entier.
Montrer que $5$ divise $a^2$ si et seulement si $5$ divise $a$.
\end{exo}


On peut légèrement généraliser le lemme d'Euclide :

\begin{thm}[Lemme d'Euclide (généralisé)]
Soit $p$ premier et $a_1, a_2, \dots, a_n$ des entiers. Si $p$ divise $a_1 \times a_2 \times \dots \times a_n$, alors il divise un des $a_1, a_2, \dots, a_n$.
\end{thm}

\begin{preuve}
On va prouver cette généralisation en appliquant le lemme d'Euclide plusieurs fois.
Si un premier $p$ divise $a_1 \times a_2 \times \dots \times a_n$, alors par le lemme d'Euclide, $p$ divise soit $a_1$ soit $a_2 \times a_3 \times \dots a_n$. Dans le premier cas, on a terminé. Dans le deuxième cas, on réapplique le lemme d'Euclide de la même manière. On a alors que $p$ divise soit $a_2$ soit $a_3 \times a_4 \times \dots \times a_n$. Dans le premier cas, on a terminé, dans le deuxième cas, on refait le même raisonnement. Si on refait ce raisonnement assez de fois, on aura terminé au bout d'un moment.
\end{preuve}

\begin{rem}
Ce raisonnement peut être rendu plus rigoureux grâce à une récurrence.
\end{rem}


\begin{exo}
Soit $a$ entier.
Montrer que $5$ divise $a^3$ si et seulement si $5$ divise $a$.
\end{exo}


On prouve maintenant le lemme d'Euclide. Cette preuve peut paraitre parachutée, il existe une preuve plus naturelle, mais elle utilise les modulos.

\begin{preuve}[Lemme d'Euclide]
Soit $p$ premier et $a, b$ entiers tels que $p$ divise $ab$, écrivons $ab = kp$ avec $k$ entier. Supposons par l'absurde que $p$ ne divise ni $a$ ni $b$. Alors $p$ est premier avec $a$ et avec $b$ donc par le théorème de Bézout, il existe des entiers $w, x, y, z$ tels que
$$wa + xp = 1\text{ et }yb + zp = 1$$
soit
$$wa = 1 - xp\text{ et }yb = 1 - zp$$
On a alors
$$wykp = wyab = \left(1 - xp\right)\left(1 - zp\right) = 1 - xp - zp + xzp^2$$
soit
$$1 = p(wyk + x + z - xzp)$$
Donc $p$ divise $1$, ce qui est une contradiction.
\end{preuve}


\subsubsection{Décomposition en facteurs premiers}


\begin{thm}[Théorème fondamental de l'arithmétique]
Soit $n\ge 1$ naturel. $n$ peut s'écrire comme produit de facteurs premiers d'une manière unique. Cette écriture s'appelle la décomposition de $n$ en facteurs premiers.
\end{thm}

Par exemple, $6 = 2 \times 3$, $2 \times 3$ est la décomposition de $6$ en facteurs premiers. $12 = 2 \times 2 \times 3, 15 = 3 \times 5, 14 = 2 \times 7, 8 = 2 \times 2 \times 2$.

\begin{rem}
\begin{itemize}
\item Pour que ce théorème s'applique aux nombres premiers et à $1$, un produit d'aucun nombre premier (le \textit{produit vide}) vaut $1$, et un produit d'un seul nombre premier vaut lui-même.
\item Bien sûr, $12 = 2 \times 2 \times 3 = 2 \times 3 \times 2 = 3 \times 2 \times 2$. Quand on parle de décomposition en facteurs premiers, on la considère implicitement unique \textit{à permutation près}.
\item On a défini les nombres premiers de manière à ce que $1$ ne soit pas premier. On peut maintenant voir en quoi ça nous simplifie la vie : si $1$ était premier, la décomposition en facteurs premiers ne serait pas unique : $6 = 2 \times 3 = 1 \times 2 \times 3 = 1 \times 1 \times 2 \times 3 = \dots$.
\end{itemize}
\end{rem}


\begin{exo}
Montrer qu'un premier divise un entier naturel si et seulement s’il est présent dans sa décomposition en facteurs premiers.
\end{exo}


\begin{exo}
Montrer que si un entier naturel est divisible par $2$ et par $3$, alors il est divisible par $6$.
\end{exo}


\begin{exo}
Montrer qu’un entier naturel est un carré parfait si et seulement si chaque premier présent dans sa décomposition en facteurs premiers y apparait un nombre pair de fois.

Montrer qu’un entier naturel est une puissance $k$-ième si et seulement si chaque premier présent dans sa décomposition en facteurs premiers y apparait un nombre divisible par $k$ de fois.
\end{exo}


\begin{exo}
Un entier est à la fois un carré parfait et un cube parfait. Est-ce forcément une puissance $6$-ième ?
\end{exo}


\begin{exo}
Montrer que deux naturels sont premiers entre eux si et seulement s’ils n'ont pas de facteur premier en commun.
\end{exo}


\begin{exo}
Soient $a$ et $b$ naturels. Montrer que $a$ divise $b$ si et seulement si tous les premiers présents dans la décomposition en facteurs premiers de $a$ sont aussi présent dans la décomposition en facteurs premiers de $b$ et chacun d'entre eux y apparait au moins autant de fois que dans celle de $a$.
\end{exo}


\begin{exo}
Combien de diviseurs naturels a $121$ ?
Combien de diviseurs naturels a $1\ 000$ ?
Combien de diviseurs naturels a $1\ 000\ 000\ 000$ ?
\end{exo}

\begin{exo}
Soit $p$ premier. Montrer qu'un entier naturel est une puissance de $p$ si et seulement si $p$ est son seul facteur premier.
\end{exo}


\subsubsection{Lemme de Gauss}


\begin{thm}[Lemme de Gauss]
Soient $n, a, b$ des naturels tels que $n$ et $a$ sont premiers entre eux. Si $n$ divise $ab$, alors $n$ divise $b$.
\end{thm}

\begin{preuve}
$n$ divise $ab$ et $nb$, donc $n$ divise leur pgcd. Or $\pgcd(ab, nb) = \pgcd(a, n)b = b$. Donc $n$ divise $b$, comme voulu.
\end{preuve}

\begin{rem}
L'unicité de la décomposition en facteurs premiers découle du lemme de Gauss. On ne peut donc pas utiliser cette dernière pour le prouver.
\end{rem}


\begin{exo}
Montrer le lemme d'Euclide en utilisant le lemme de Gauss.
\end{exo}


\begin{exo}
Soient $x$ et $y$ des entiers. Montrer que si $2x+1$ divise $8y$, alors $2x+1$ divise $y$.
\end{exo}


\begin{exo}
Soient $a$ et $b$ deux naturels premiers entre eux et $n$ un naturel. Montrer que si $a$ divise $n$ et $b$ divise $n$, alors $ab$ divise $n$.
\end{exo}


\begin{exo}
Soient $x$ et $y$ des naturels. Montrer que si $x^2$ divise $x^2 + xy + x + y$, alors $x^2$ divise $x+y$.
\end{exo}


\subsubsection{Infinitude des nombres premiers}


\begin{thm}
Il y a une infinité de nombres premiers.
\end{thm}

\begin{preuve}
Supposons par l'absurde qu'il n'y a qu'un nombre fini de nombres premiers. Disons que $p_1, p_2, \dots, p_n$ sont tous les nombres premiers. Mais alors, $p_1 p_2 \dots p_n + 1$ n'est divisible par aucun nombre premier puisque, si $p_i$ divise $p_1 p_2 \dots p_n + 1$, alors il divise $p_1 p_2 \dots p_n + 1 - p_i \times p_1 p_2 \dots p_{i-1} p_{i+1} \dots p_n = 1$. Donc $p_1 p_2 \dots p_n + 1 = 1$, ce qui signifie que l'un des $p_i$ est $0$. Absurde.
\end{preuve}


\subsubsection{Solutions}


\begin{sol}
Ce sont $2, 3, 5, 7, 11, 13$. Un calcul montre que ces nombres ne sont bien divisibles que par $1$ et eux-mêmes. Réciproquement, $4$ n'est pas premier, car il est divisible par $2$ et $2\ne 1$ et $2\ne 4$. De même, $6$ est divisible par $2$, $8$ est divisible par $2$, $9$ est divisible par $3$, $10$ est divisible par $2$, $12$ est divisible par $2$, $14$ est divisible par $2$ et $15$ est divisible par $3$.
\end{sol}


\begin{sol}
Si un premier est pair, alors il est divisible $2$ en est un diviseur. Donc, comme $2\ne 1$, il vaut $2$. Réciproquement, $2$ est bien premier.
\end{sol}


\begin{sol}
Montrons que $2, 3$ est la seule paire de premiers consécutifs. Si deux premiers sont consécutifs, alors l'un d'eux est pair et donc égal à $2$. $1, 2$ ne sont pas deux premiers consécutifs car $1$ n'est pas premier, et $2, 3$ sont bien deux premiers consécutifs.
\end{sol}


\begin{sol}
$3$ est le seul premier de cette forme. De fait, si un premier $p$ est de la forme $a^2 - 1 = (a - 1)(a + 1)$, il est divisible par $a - 1$ et $a + 1$. Donc ceux-ci doivent être $1$ et $p$. Comme ils sont distincts et $a - 1$ est le plus petit d'entre eux, $a - 1 = 1$ et $p = a^2 - 1 = 2^2 - 1 = 3$. Réciproquement, $2^2 - 1 = 3$ est bien premier.
\end{sol}


\begin{sol}
On veut montrer que la réponse est non. On veut qu'un nombre de cette forme ait un diviseur autre que $1$ et lui-même. Notamment, si on veut que ce diviseur soit $41$, on n'a qu'à prendre $n=41$ et on aura alors $n^2 + n + 41 = 41^2 + 41 + 41 = 41 \left(41 + 1 + 1\right)$ est le produit de deux naturels $>1$ et donc pas premier.
\end{sol}


\begin{sol}
On va montrer que si $n$ n'est pas premier, alors il a un diviseur naturel autre que $1$ et $n$ qui vaut au plus $\sqrt n$. Prenons n'importe quel diviseur $d$ de $n$. Si $d \le \sqrt n$, on a fini. Sinon, si $d > \sqrt n$, on a que $\frac nd$ est aussi un diviseur de $n$, mais alors $\frac nd < \frac n{{\sqrt n}} = \frac{{\sqrt n}^2 }{\sqrt n} = \sqrt n$, ce qui conclut.
\end{sol}


\begin{sol}
Supposons que $n$ n'est pas premier. Écrivons donc $n = ab$ avec $a, b > 1$. Alors
$$2^n - 1 = 2^{ab} - 1 = (2^a - 1)(2^{a(b - 1)} + 2^{a(b - 2)} + \dots + 2^{2a} + 2^a + 1)$$
est divisible par $2^a - 1$ tandis que $1 = 2^1 - 1 < 2^a - 1 = 2^{a\cdot 1} - 1 < 2^{ab} - 1$ et n'est donc pas premier.
\end{sol}


\begin{sol}
Comme $p$ est un diviseur de $q$ et $q$ est premier, on a que $p$ vaut soit $q$ soit $1$. Le deuxième cas n'étant pas possible (car $1$ n'est pas premier), on a fini.
\end{sol}


\begin{sol}
Soit $d$ le plus grand diviseur commun de $p$ et de $n$. Comme $d$ divise $p$, $d$ vaut soit $1$ soit $p$. Dans le premier cas, $p$ et $n$ sont premiers entre eux, dans le second cas, $p = d$ divise $n$.
\end{sol}


\begin{sol}
$5$ est premier. Donc, par le lemme d'Euclide, si $5$ divise $a^2 = a \times a$, alors soit $5$ divise $a$, soit $5$ divise $a$, on a donc conclu.
\end{sol}


\begin{sol}
$5$ est premier. Donc par la généralisation du lemme d'Euclide, on a que $5$ divise soit $a$, soit $a$, soit $a$.
\end{sol}


\begin{sol}
Dans un sens, si $p$ est présent dans la décomposition en facteurs premiers de $n$, alors on peut ordonner cette décomposition comme $n = p \times q_1 q_2 \dots q_k$, donc $p$ divise bien $n$.

Dans l'autre sens, si $p$ divise $n$, alors décomposons $n$ en facteurs premiers : $n = q_1 q_2 \dots q_k$. $p$ divise $q_1 q_2 \dots q_k$ et donc par la généralisation du lemme de Gauss il divise $q_i$ pour un certain $i$. Donc, comme $p$ et $q_i$ sont premiers, on a $q_i = p$, donc $p$ est bien présent dans la décomposition de $n$ en facteurs premiers.
\end{sol}


\begin{sol}
$2$ et $3$ sont premiers. Donc, si un entier naturel $n$ est divisible par $2$ et par $3$, $2$ et $3$ sont présents dans sa décomposition en facteurs premiers. Alors cette décomposition peut être réordonnée comme $n = 2 \times 3 \times q_1 q_2 \dots q_k = 6 \times q_1 q_2 \dots q_k$. $n$ est donc bien divisible par $6$.
\end{sol}


\begin{sol}
Pour la première sous-question, dans un sens, si un entier naturel $n$ est un carré parfait, posons $n = a^2$. Soit alors $a = p_1^{a_1} p_2^{a_2} \dots p_l^{a_l}$ la décomposition de $a$ en facteurs premiers. On a
$$n = a \times a = \left(p_1^{a_1} \dots p_l^{a_l}\right) \times \left(p_1^{a_1} \dots p_l^{a_l}\right) = p_1^{2a_1} \dots p_l^{2a_l}$$
On voit que chaque nombre premier apparait un nombre pair de fois.

Dans l'autre sens, si chaque facteur premier apparait un nombre pair de fois dans la décomposition en facteurs premiers de $n$, on peut écrire
$$n = p_1^{2a_1} \dots p_l^{2a_l} = \left(p_1^{a_1} \dots p_l^{a_l}\right)^2$$
comme voulu.

Pour la deuxième sous-question, on procède de la même manière.
Dans un sens, si un entier naturel $n$ est une puissance $k$-ième, posons $n = a^k$ et décomposons $a$ en facteurs premiers : $a = p_1 \dots p_l$. On a alors
$$n = a^k = \left(p_1 \dots p_l\right)^k = p_1^{ka_1} \dots p_l^{ka_l}$$
Chaque facteur apparait bien un multiple de $k$ fois dans cette décomposition.

Dans l'autre sens, si chaque facteur premier apparait un multiple de $k$ fois dans la décomposition de $n$, alors on peut écrire
$$n = p_1^{ka_1} \dots p_l^{ka_l} = \left(p_1^{a_1} \dots p_l^{a_l}\right)^k$$
comme voulu.
\end{sol}


\begin{sol}
La réponse est oui. Soit $n$ un naturel qui est à la fois un carré parfait et un cube parfait. Comme $n$ est un carré parfait, chaque facteur premier présent dans sa décomposition en facteurs premiers est présent un nombre divisible par $2$ de fois dans sa décomposition en facteurs premiers. Comme $n$ est un cube parfait, chaque facteur premier présent dans sa décomposition en facteurs premiers y est présent un nombre divisible par $3$ de fois. Donc, chaque facteur premier de $n$ est présent un nombre divisible par $2 \times 3 = 6$ de fois dans la décomposition de $n$ en facteurs premiers, donc $n$ est bien une puissance $6$-ième.
\end{sol}


\begin{sol}
Soient $a, b$ ces deux naturels.

Si $p$ premier divise $a$ et $b$, alors il divise leur pgcd qui ne peut donc pas valoir $1$. $a$ et $b$ ne sont pas premiers entre eux.

Réciproquement, supposons que $\pgcd(a, b) \ne 1$. Alors ce pgcd possède au moins un facteur premier $p$. $p$ divise $\pgcd(a, b)$ qui divise $a$ et $b$, donc $a$ et $b$ ont $p$ comme facteur premier en commun.
\end{sol}


\begin{sol}
Supposons que dans la décomposition de $a$ en facteurs premiers, $p_1$ apparait $k_1$ fois, $p_2$ apparait $k_2$ fois, $\dots$, $p_n$ apparait $k_n$ fois, avec $p_1, p_2, \dots, p_n$ des premiers distincts. On a alors
$$a = p_1^{k_1} p_2^{k_2} \dots p_n^{k_n}$$
Supposons que $p_1, \dots, p_n$ apparaissent aussi dans la décomposition de $b$ en facteurs premiers, $l_1$ fois, $\dots$, $l_n$ fois respectivement, avec $k_1 \le l_1, \dots, k_n \le l_n$. On a alors, pour certains premiers $q_1, q_2, \dots q_m$,
$$b = p_1^{l_1} \dots p_n^{l_n} q_1 \dots q_m = p_1^{k_1} \dots p_n^{k_n} \times p_1^{l_1 - k_1} \dots p_n^{l_n - k_n} q_1 \dots q_m = a \times p_1^{l_1 - k_1} \dots p_n^{l_n - k_n} q_1 \dots q_m$$
Donc $a$ divise bien $b$ (on utilise $k_i \le l_i$ pour que $l_i - k_i$ soit positif et donc que $p_i^{l_i - k_i}$ soit entier).

Réciproquement, supposons que $a$ divise $b$ et qu'un premier $p$ apparait $k$ fois dans la décomposition de $a$ en facteurs premiers. On a que $p^k$ divise $a$ et $a$ divise $b$, donc $p^k$ divise $b$. $p$ apparait donc aussi (au moins) $k$ fois dans la décomposition de $b$ en facteurs premiers.
\end{sol}


\begin{sol}
$121 = 11^2$. Donc les diviseurs naturels de $121$ sont exactement $1$, qui n'a aucun facteur premier, et les naturels dont $11$ est le seul facteur premier et ce facteur premier est présent $1$ ou $2$ fois dans la décomposition en facteurs premiers, ce sont $11$ et $121$. Donc, $121$ a exactement $3$ diviseurs naturels.

$1\ 000 = 2^3 \times 5^3$. Les diviseurs naturels de $1\ 000$ sont donc exactement les naturels avec que des $2$ et $5$ dans la décomposition en facteurs premiers et qui y ont $0, 1, 2$ ou $3$ fois $2$ et $0, 1, 2$ ou $3$ fois $5$. Il y a quatre possibilités de combien de facteurs $2$ il y a et quatre possibilités de combien de facteurs $5$ il y a, ça donne donc en tout $4 \times 4 = 16$ diviseurs.

$1\ 000\ 000\ 000 = 2^9 \times 5^9$. Donc, de même que dans la question précédente, il a $(9 + 1)\times (9 + 1) = 100$ diviseurs.
\end{sol}


\begin{sol}
Si $p$ est le seul facteur premier de $n$, alors, en décomposant $n$, $n = p \times p \times \dots \times p$ est bien une puissance de $p$. Réciproquement, si $n$ est une puissance de $p$, alors on peut écrire $n = p \times p \times \dots \times p$, un produit de nombres premiers, donc $p$ est le seul facteur premier de $n$.
\end{sol}


\begin{sol}
Supposons qu'un premier $p$ divise $ab$. Si $p$ divise $a$, alors on a gagné. Si $p$ ne divise pas $a$, alors $p$ est premier avec $a$ et le lemme de Gauss donne alors que $p$ divise $b$.
\end{sol}


\begin{sol}
$2x+1$ est premier avec $8 = 2 \times 2 \times 2$ car $2x+1$ est impair et $2$ n'en est donc pas un facteur premier. Donc, par le lemme de Gauss, $2x+1$ divise $y$ car il divise $8y$.
\end{sol}


\begin{sol}
$b$ divise $n = a \cdot \frac n a$ or $b$ est premier avec $a$ et $\dfrac n a$ est entier. Par le lemme de Gauss, $b$ divise donc $\dfrac n a$, et cela revient à dire que $ab$ divise $n$.
\end{sol}


\begin{sol}
\textit{Première solution, factorisation} \\
On factorise $x^2 + xy + x + y = x\left(x+y\right) + x+y = (x + 1)(x + y)$. $x+1$ est premier avec $x^2$ car ils n'ont pas de facteurs premiers en commun. De fait, si $p$ est un facteur premier de $x^2$, il est aussi un facteur premier de $x$, donc il ne divise pas $x+1$ car sinon il diviserait $x+1-x=1$. Donc, par le lemme de Gauss, $x^2$ divise $x+y$, ce qu'il fallait démontrer. \\
\textit{Seconde solution, divisibilité} \\
$x$ divise $x^2$ qui divise $x^2 + xy + x + y$. Donc $x$ divise $x^2 + xy + x + y$. Donc $x$ divise $y$ (on retire tout ce qui est divisible par $x$). Donc $x^2$ divise $xy$. Donc $x^2$ divise $x + y$ (car $x^2$ divise $x^2 + xy + x + y$ et on retire tout ce qui est divisible par $x^2$).
\end{sol}