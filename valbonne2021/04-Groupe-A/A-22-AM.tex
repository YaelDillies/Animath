\subsubsection{Définition des nombres premiers}
\dfn[Nombre premier]
Un nombre premier est un nombre naturel qui a exactement \(2\) diviseurs naturels - \(1\) et lui-même.
\endgroup

\exo
Trouver tous les premiers entre \(1\) et \(15\).

\exo
Trouver tous les premiers pairs.

\exo
Combien y a-t-il de paires de premiers consécutifs ?

\exo
Trouver tous les premiers de la forme \(a^2 - 1\), avec \(a\geqslant2\) naturel.

\exo
Est-ce que tous les nombres de la forme \(n^2 + n + 41\), avec \(n\) naturel, sont premiers ?

\exo
Soit \(n\) un nombre naturel. On veut tester si \(n\) est premier.
On peut tester si \(n\) est divisible par \(2, 3, 4, \ldots, n-1\). On sait alors que \(n\) est premier si et seulement s’il n'est divisible par aucun de ses nombres. Montrer qu'en faite, il suffit de tester que les diviseurs plus petits que \(\sqrt{n}\), ce qui donne une façon plus rapide de tester si un nombre \(n\) est premier.

\exo
Soit \(n\) naturel. Montrer que si \(2^{n-1}\) est premier, alors \(n\) est premier.

\exo
Montrer que si un premier \(p\) divise un premier \(q\), alors \(p=q\).

\exo
Soit \(p\) premier et \(n\) naturel. Montrer que soit \(p\) divise \(n\), soit \(p\) et \(n\) sont premiers entre eux.

\subsubsection{Lemme d'Euclide}
Le lemme d'Euclide s'énonce comme suit :


\thm[Lemme d'Euclide]

Soit \(p\) premier et \(a, b\) naturels. Si \(p\) divise \(ab\), alors soit \(p\) divise \(a\), soit \(p\) divise \(b\)

\endgroup

Remarquons que la condition \(p\) premier est indispensable. De faite, \(6\) divise \(12 = 3 \times 4\), mais \(6\) ne divise ni \(3\) di \(4\). C'est possible par ce que \(6\) n'est pas premier.

On va le prouver plus tard, puis ce que la preuve est assez compliquée et technique.

\exo
Soit \(a\) entier.
Montrer que \(5\) divise \(a^2\) si et seulement si \(5\) divise \(a\).

On peut un peut généraliser le lemme d'Euclide :

\thm[Lemme d'Euclide (généralisé)]

Soit \(p\) premier et \(a_2 , a_2 , \ldots , a_{n}\) des entiers. Si \(p\) divise \(a_2 \times a_2 \times \cdots \times a_{n}\), alors il divise un des \(a_2 , a_2 , \ldots , a_{n}\).

\endgroup

\preuve

On va prouver cette généralisation en appliquant le lemme d'Euclide plusieurs fois.
Si un premier \(p\) divise \(a_2 \times a_2 \times \cdots \times a_{n}\), alors par le lemme d'Euclide, \(p\) divise soit \(a_2\) soit \(a_2 \times a_{3} \times \cdots a_{n}\). Dans le premier cas, on a terminé. Dans le deuxiemme cas, on reapplique le lemme d'Euclide de la meme manière. On a alors que \(p\) divise soit \(a_2\) soit \(a_{3} \times a_{4} \times \cdots \times a_{n}\). Dans le premiér cas, on a terminé, dans le deuxiemme cas, on refait le meme raisonnement. Si on refait ce raisonnement assez de fois, on aura terminé au bound d'un moment.

Remarque : ce raisonnement peut etre rendu plus rigoureux grace a une recurence.

\exo
Soit \(a\) entier.
Montrer que \(5\) divise \(a^{3}\) si et seulement si \(5\) divise \(a\).

\preuve[Lemme d'Euclide]

Soit \(p\) premier et \(a, b\) entiers tels que \(p\) divise \(ab\), écrivons \(ab = kp\) avec \(k\) entier. Supposons par l'absurde que \(p\) ne divise ni \(a\) ni \(b\). Alors \(p\) est premier avec \(a\) et avec \(b\) donc par le théoréme de Bézout, il existe des entiers \(w, x, y, z\) tels que
\[wa + xp = 1\]
et
\[yb + zp = 1\]
soit
\[wa = 1 - xp\]
et
\[yb = 1 - zp\]
on a alors
\[wyab = \left(1 - xp\right)\left(1 - zp\right)\]
\[wykp = 1 - xp - zp + xzp^2\]
\[1 = p\left(wyk + x + z - p\right)\]
Donc \(p\) divise \(1\), ce qui est une contradiction.

\subsubsection{Décomposition en facteurs premiers}
\thm[Théorème fondamental de l'arithmétique]
Soit \(n\geqslant1\) naturel. \(n\) peut s'ecrir comme produit de facteurs premiers d'une manière unique. Cette ecriture s'appéle la décomposition de \(n\) en facteurs premiers.

\endgroup

Par example, \(6 = 2 \times 3\), \(2 \times 3\) est la décomposition de \(6\) en facteurs premiers. \(12 = 2 \times 2 \times 3, 15 = 3 \times 5, 14 = 2 \times 17, 8 = 2 \times 2 \times 2\).
Pour que ce théoréme s'applique aux nombres premiers, on dit qu'un nombre premier est produit de juste lui-même (meme si c'est pas vraiment un produit, ca nous simplifie la vie). On dit aussi que \(1\) est produit de zero nombres premiers.

Bien sur, \(12 = 2 \times 2 \times 3 = 2 \times 3 \times 2 = 3 \times 2 \times 2\). On dit que la décomposition en facteurs premiers reste la meme quand on change l'ordre des facteurs, donc quand on dit unique, on veut dire unique a réordonnement pret.

Remarque : on a défini les nombres premiers de manière a ce que \(1\) ne soit pas premier. On peut maintenant en quoi ca nous simplifie la vie : si \(1\) était premier, la décompositoin en facteurs premiers ne serait pas unique : \(6 = 2 \times 3 = 2 \times 3 \times 1 = 2 \times 3 \times 1 \times 1 = 2 \times 3 \times 1 \times 1 \times 1 = \ldots\).

\exo
Montrer qu'un premier divise un nombre naturel si et seulement s’il est présent dans sa décomposition en facteurs premiers.

\exo
Montrer que si un nombre naturel est divisible par \(2\) et par \(3\), alors il est divisible par \(6\).

\exo

Montrer qu’un nombre naturel est un carré parfait si et seulement si chaque premier présent dans sa décomposition en facteurs premiers y apparait un nombre pair de fois.

Montrer qu’un nombre naturel est une puissance \(k^{e}\) parfait si et seulement si chaque premier présent dans sa décomposition en facteurs premiers y apparait un nombre divisible par \(k\) de fois.

\exo
Un entier est à la fois un carré parfait et un cube parfait.
Est-il alors forcement une puissance \(6^{e}\) parfaite ?

\exo

Montrer que deux naturels sont premiers entre eux si et seulement s’ils n'ont pas de facteur premier en commun.


\exo
Soient \(a\) et \(b\) naturels.
Montrer que \(a\) divise \(b\) si et seulement si tous les premiers présents dans la décomposition en facteurs premiers de \(a\) sont aussi présent dans la décomposition en facteurs premiers de \(b\) et chacun d'entre eux y apparait au moins au temps de fois que dans celle de \(a\).

\exo
Combien de diviseurs naturels a \(121\) ?
Combien de diviseurs naturels a \(1\ 000\) ?
Combien de diviseurs naturels a \(1\ 000\ 000\ 000\) ?

\exo
Soient \(a\) et \(b\) deux naturels premiers entre eux et \(n\) un naturel. Montrer que si \(a\) divise \(n\) et \(b\) divise \(n\), alors \(ab\) divise \(n\).

\exo
Soit \(p\) premier. Montrer qu'un nombre naturel est une puissance de \(p\) si et seulement si \(p\) est son seul facteur premier.

\subsubsection{Lemme de Gauss}
\thm[Lemme de Gauss]

Soient \(n \not= 0, a, b\) des naturels tels que \(n\) et \(a\) sont premiers entre eux. Si \(n\) divise \(ab\), alors \(n\) divise \(b\).

\endgroup

\preuve

Comme \(n\) et \(a\) sont premiers entre eux, ils n'ont pas de facteurs preimers en commun. Or comme \(n\) divise \(ab\), tous les facteurs premiers de \(n\) sont présents dans la décomposition en facteurs premiers de \(ab\) au moins autemps de fois que dans celle de \(n\). Donc, comme \(n\) et \(a\) n'ont pas de facteurs premiers en commun, ils sont tous présents dans la décomposition de \(b\) en facteurs premiers, ce qui conclut.

\exo
Montrer le lemme d'Euclide en utilisant le lemme de Gauss.

\exo
Soient \(x\) et \(y\) des entiers. Montrer que si \(2x+1\) divise \(8y\), alors \(2x+1\) divise \(y\).

\exo
Soient \(x\) et \(y\) des naturels. Montrer que si \(x^2\) divise \(x^2 + xy + x + y\), alors \(x^2\) divise \(x+y\).

\subsubsection{Infinitude de nombres premiers}
\thm

Il y a une infinité de nombres premiers.

\endgroup

\preuve

Supposons par l'absurde qu'il y a un nombre fini de nombres premiers. Disons que \(p_2 , p_2 , \ldots , p_{n}\) sont tous les nombres premiers. Mais alors, \(p_2 p_2 \cdots p_{n} + 1\) n'est divisible par aucun des \(p_2 , p_2 , \ldots , p_{n}\) puis ce que si \(p_{i}\) divise \(p_2 p_2 \cdots p_{n} + 1\), il divise \(p_2 p_2 \cdots p_{n} + 1 - p_{i} \times p_2 p_2 \cdots p_{i-1} p_{i+1} \cdots p_{n} = 1\). Mais, \(p_2 p_2 \cdots p_{n} + 1\) est un entier plus grand que \(2\), il a donc au moins un facteur premier. Comme ce facteur ne peut etre aucun des \(p_2 , p_2 , \ldots , p_{n}\), il y a un autre premier que \(p_2 , p_2 , \ldots , p_{n}\). Contradiction.

\subsubsection{Solutions des exercices}
\sol

Ce sont \(2, 3, 5, 7, 11, 13\). Un calcul montre que ses nombres sont bien divisibles que par \(1\) et eux memes. Réciproquement, \(4\) n'est pas premier car il est divisible par \(2\) et \(2\not=1\) et \(2\not=4\). De meme, \(6\) est divisible par \(2\), \(8\) est divisible par \(2\), \(9\) est divisible par \(3\), \(10\) est divisible par \(2\), \(12\) est divisible par \(2\), \(14\) est divisible par \(2\) et \(15\) est divisible par \(3\).

Remarque : \(1\) n'est pas un nombre premier parce qu'il n'a qu'un seul diviseur naturel. Ca peut paraitre bizare de choisir une définition de premier qui exclut \(1\), mais on vera aprés que ca nous simplifie en faite la vie.

\sol

\(2\) est le seul premier pair. De faite, si un premier est pair, alors \(2\) en est un diviseur, donc, comme \(2\not=1\), \(2\) est égal a ce premier. Réciproquement, \(2\) est bien premier.

\sol

\(2, 3\) est la seule paire de premiers consécutifs. De faite, si deux premiers sont consécutifs, alors un d'entre eux est pair, il est donc égal a \(2\). Réciprquement, \(1, 2\) ne sont pas deux premiers consécutifs car \(1\) n'est pas premier, et \(2, 3\) sont bien deux premiers consecutifs.

\sol

\(3\) est le seul premier de cette forme. De faite, si un premier \(p\) est de la forme \(a^2 - 1 = \left(a-1\right)\left(a+1\right)\), il est divisible par \(a-1\) et \(a+1\). Or, comme \(a\geqslant2\), \(a-1\) et \(a+1\) sont naturels, donc chacun d'entre eux vaut soit \(1\) soit \(p\). Comme ils sont distincts et \(a-1\) est le plus petit d'entre eux, c'est \(a-1\) qui doit valoir \(1\), donc \(a\) doit valoir \(2\). Réciproquement, \(2^2 - 1 = 3\) est bien premier.

\sol

On veut montrer que la réponse est non. On veut qu'un nombre de cette forme ait un diviseur autre que \(1\) et lui-même. Notament, si on veut que ce diviseur soit \(41\), on n'a qu'a prendre \(n=41\) et on aura alors \(n^2 + n + 41 = 41^2 + 41 + 41 = 41 \left(41 + 1 + 1\right)\) est le produit de deux naturels \(>1\) et donc pas premier.

\sol

On va montrer que si \(n\) n'est pas premier, alors il a un diviseur naturel autre que \(1\) et \(n\) qui vaut au plus \(\sqrt{n}\). Prenons n'importe quel diviseur \(d\) de \(n\). Si \(d \leqslant \sqrt{n}\), on a fini. Sinon, si \(d > \sqrt{n}\), on a que \(\frac{n}{d}\) est aussi un diviseur de \(n\), mais alors \(\frac{n}{d} < \frac{n}{{\sqrt{n}}} = \frac{{\sqrt{n}}^2}{{\sqrt{n}}} = \sqrt{n}\), soit \(\frac{n}{d} < \sqrt{n}\), ce qui conclut.

\sol

Supposons par l'absurde que \(2^{n} - 1\) est premier et \(n\) n'est pas premier. \(n\) a alors un diviseur naturel \(k\not=1, n\) et on peut alors ecrir \(n = kl\) avec \(l = \frac{n}{k} \not= 1, n\) entier naturel. Alors \(2^{n} - 1 = 2^{kl} - 1 = \left(2^{k} - 1\right)\left(2^{\left(l-1\right)k} + 2^{\left(l-2\right)k} + \cdots + 2^{2k} + 2^{k} + 1\right)\) est un produit de deux entiers \(>1\) et donc pas premier.

\sol

Comme \(p\) est un diviseur de \(q\) et \(q\) est premier, on a que \(p\) vaut soit \(q\) soit \(1\). Le deuxiemme cas n'étant pas possible (car \(1\) n'est pas premier), on a bien fini.

\sol

Soit \(d\) le plus grand diviseur commun de \(p\) et de \(n\). Comme \(d\) divise \(p\), on a que \(d\) vaut soit \(1\) soit \(p\). Dans le premier cas, \(p\) et \(n\) sont premiers entre eux, dans le second cas, \(p\) divise \(n\) car \(d\) divise \(n\).

\sol

\(5\) est premier. Donc, par le lemme d'Euclide, si \(5\) divise \(a^2 = a \times a\), alors soit \(5\) divise \(a\), soit \(5\) divise \(a\), on a donc conclu.

\sol

\(5\) est premier. Donc par la généralisation du lemme d'Euclide, on a que \(5\) divise soit \(a\), soit \(a\), soit \(a\).

On prouve maintenant le lemme d'Euclide. Cette preuve peut paraitre parachutée, il existe une preuve plus naturelle mais elle utilise les modulos qui sont au programme du cours de demain.

\sol

Dans un sens, si \(p\) est présent dans la décomposition en facteurs premiers de \(n\), alors on peut ordonner cette decomposition comme \(n = p \times q_2 q_2 \cdots q_{k}\), donc \(p\) divise bien \(n\).

Dans l'autre sens, si \(p\) divise \(n\), alors décomposons \(n\) en facteurs premiers : \(n = q_2 q_2 \cdots q_{k}\). \(p\) divise \(q_2 q_2 \cdots q_{k}\) et donc par la généralisation du lemme de Gauss il divise \(q_{i}\) pour un certain \(i\). Donc, comme \(p\) et \(q_{i}\) sont premiers, on a \(q_{i} = p\), donc \(p\) est bien présent dans la décomposition de \(n\) en facteurs premiers.

\sol

\(2\) et \(3\) sont premiers. Donc, si un nombre naturel \(n\) est divisible par \(2\) et par \(3\), \(2\) et \(3\) sont présents dans sa décomposition en faceturs premiers. Alors cette décomposition peut etre reordonnée comme \(n = 2 \times 3 \times q_2 q_2 \cdots q_{k} = 6 \times q_2 q_2 \cdots q_{k}\), \(n\) est donc bien divisible par \(6\).

\sol

Pour la premiére sous question, dans un sens, si un nombre naturel \(n\) est un carré parfait, posons \(n = a^2\). Soit alors \(a = p_2 p_2 \cdots p_{l}\) la décomposition de \(a\) en facteurs premiers. On a
\[n = a \times a = \left(p_2 p_2 \cdots p_{l}\right) \times \left(p_2 p_2 \cdots p_{l}\right)\]. Pour chaque premier qui apparait dans cette décomposition, soit \(x\) le nombre de fois qu'il apparait dans un des facteurs. Il apparait alors \(2x\) fois dans toute la décomposition, un nombre pair de fois.

Dans l'autre sens, si chaque facteur premier apparait un nombre pair de fois dans la décomposition en facteurs premiers de \(n\), on peut regrouper ses facteurs premiers :
\[n = \left(p_2 p_2\right)\left(p_2 p_2\right) \cdots \left(p_{l} p_{l}\right)\]
n a alors
\[n = \left(p_2 p_2 \cdots p_{l}\right) \left(p_2 p_2 \cdots p_{l}\right) = \left(p_2 p_2 \cdots p_{l}\right)^2\]
comme voulu.

Poul la deuxiemme sous question, on procéde de la meme manière.
Dans un sens, si un nombre naturel \(n\) est
une puissance \(k^{e}\), posons \(n=a^{k}\).
Décomposons \(a\) en facteurs premiers : \(a = p_2 p_2 \cdots p_{l}\). On a alors \(n = a^{k} = a \times a \times \cdots \times a = \left(p_2 p_2 \cdots p_{l}\right) \left(p_2 p_2 \cdots p_{l}\right) \cdots \left(p_2 p_2 \cdots p_{l}\right)\), avec \(k\) fois \(p_2 p_2 \cdots p_{l}\). Si un facteur premier apparait dans cette décomposition, il apparait dans un des \(k\) facteurs. Soit \(x\) le nombre de fois qu'il apparait dans ce facteur. Il apparait alors \(kx\) fois dans toute la décomposition, c'est bien un nombre divisible par \(k\).

Dans l'autre sens, si chaque facteur premier apparait un nombre pair de fois dans la décomposition de \(n\), alors cette décomposition peut etre écrite comme \[n = \left(p_2 p_2 \cdots p_2\right) \left(p_2 p_2 \cdots p_2\right) \cdots \left(p_{l} p_{l} \cdots p_{l}\right)\] et puis réarangée comme \[n = \left(p_2 p_2 \cdots p_{l}\right) \left(p_2 p_2 \cdots p_{l}\right) \cdots \left(p_2 p_2 \cdots p_{l}\right) = \left(p_2 p_2 \cdots p_{l}\right)\] n est donc bien une puissance \(k^{e}\), ce qu'il falait démontrer.

\sol

La réponse est oui. Supposons que \(n\) est un naturel qui est a la fois un carré parfait et un cube parfait. Comme \(n\) est un carré parfait, chaque facteur premier présent dans sa décomposition en facteurs premiers est présent un nombre divisible par \(2\) de fois dans sa décomposition en facteurs premiers. Comme \(n\) est un cube parfait, chaque facteur premier présent dans sa décomposition en facteurs premiers y est présent un nombre divisible par \(3\) de fois. Donc, chaque facteur premier de \(n\) est présent un nombre divisible par \(2 \times 3 = 6\) de fois dans la décomposition de \(n\) en facteurs premiers, d'ou \(n\) est bien une puissance \(6^{e}\).

\sol

Si deux naturels ont un facteur premier en commun, alors ce facteur en est un diviseur commun. Comme il n'est pas égal a \(1\) (car premier), ils ne sont pas premiers entre eux.

Réciproquement, si supposons que deux naturels n'ont pas de facteurs premiers en commun. Supposons par l'absurde qu'ils ne sont pas premiers entre eux, c'est a dire que leur pgcd ne vaut pas \(1\). Ce pgcd a alors au moins un diviseur premier (n'importe quel facteur de sa décomposition en facteurs premiers). C'est alors un facteur premier en commun des deux naturels, contradiction.

\sol

Supposons que dans la décomposition de \(a\) en facteurs premiers, \(p_2\) apparait \(k_2\) fois, \(p_2\) apparait \(k_2\) fois, \(\ldots\), \(p_{n}\) apparait \(k_{n}\) fois, avec \(p_2 , p_2 , \ldots , p_{n}\) des premiers distincts. On a alors
\[a = p_2 ^{k_2} p_2 ^{k_2} \cdots p_{n} ^{k_{n}}\]
Supposons que \(p_2 , p_2 , \ldots , p_{n}\) apparaissent aussi dans la décomposition de \(b\) en facteurs premiers, \(l_2\) fois, \(l_2\) fois, \(\ldots\), \(l_{n}\) fois respectivement, avec \(k_2 \leqslant l_2 , k_2 \leqslant l_2 , \ldots , k_{n} \leqslant l_{n}\). On a alors
\[b = p_2 ^{l_2} p_2 ^{l_2} \cdots p_{n} ^{l_{n}} q_2 q_2 \cdots q_{m}\]
pour certains premiers \(q_2 , q_2 , \ldots q_{m}\). On a alors
\[b = p_2 ^{k_2} p_2 ^{k_2} \cdots p_{n} ^{k_{n}} \times p_2 ^{{l_2 - k_2}} p_2 ^{{l_2 - k_2}} \cdots p_{n} ^{{l_{n} - k_{n}}} q_2 q_2 \cdots q_{m}\]
\[b = a \times p_2 ^{{l_2 - k_2}} p_2 ^{{l_2 - k_2}} \cdots p_{n} ^{{l_{n} - k_{n}}} q_2 q_2 \cdots q_{m}\]
Donc \(a\) divise bien \(b\) (Remarque qu'on a besoin que \(k_{i} \leqslant l_{i}\) pour que \(l_{i} - k_{i}\) soit positif et donc que \(p_{i} ^{{l_{i} - k_{i}}}\) soit entier).

Réciproquement, supposons que \(a\) divise \(b\) et qu'un preimer \(p\) apparait \(k\) fois dans la décomposition de \(a\) en facteurs premiers. On a que \(p\) divise \(a\) et \(a\) divise \(b\), donc \(p\) divise \(b\). \(p\) apparait donc aussi dans la décomposition de \(b\) en facteurs premiers. Soit \(l\) le nombre de fois que \(p\) apparait dans la décomposition de \(b\) en facteurs premiers. On peut ecrire \(b = p^{l} \times q_2 q_2 \cdots q_{n}\). Supposons alors par l'absurde que \(k > l\). On peut alors écrir \(p^{k} = p^{l-k} p^{l}\), avec \(l-k \geqslant 1\). Comme \(p^{k}\) divise \(a\) et \(a\) divise \(b\), on a que \(p^{k}\) divise \(b\). Autrement dit, \(p^{l-k} p^{l}\) divise \(p^{l} \times q_2 q_2 \cdots q_{n}\), donc \(p^{l-k}\) divise \(q_2 q_2 \cdots q_{n}\), donc \(p\) divise \(q_2 q_2 \cdots q_{n}\), ce qui n'est pas possible car \(p\) est premier mais il n'est pas un des \(q_2 , q_2 , \cdots , q_{n}\).

\sol

\(121 = 11^2\). Donc les diviseurs naturels de \(121\) sont exactement \(1\), qui n'a aucun facteur premier, et les nombres naturels dont \(11\) est le seul facteur premier et ce facteur premier est présent \(1\) ou \(2\) fois dans la décomposition en facteurs premiers, ce sont \(11\) et \(121\). Donc, \(121\) a exactement \(3\) diviseurs naturels.

\(1\ 000 = 2^{3} \times 5^{3}\). Les diviseurs naturels de \(1\ 000\) sont donc exactement les nombres naturels avec que des \(2\) et \(5\) dans la décomposition en facteurs premiers et qui y ont \(0, 1, 2\) ou \(3\) fois \(2\) et \(0, 1, 2\) ou \(3\) fois \(5\). Il y a quatre possibilités de combien de facteurs \(2\) il y a et quatre possibilités
de combien de facteurs \(5\) il y a, ca donne donc en tout \(4 \times 4 = 16\) diviseurs.

\(1\ 000\ 000\ 000 = 2^{9} \times 5^{9}\). Donc, de meme que dans la question précédente, il a \(\left(9+1\right)\times\left(9+1\right) = 100\) diviseurs.

\sol

Comme \(a\) et \(b\) sont premiers entre eux, ils n'ont pas de facteur premier en commun. Par l'exercice précédent, tous les facteurs premiers de \(a\) apparaissent dans la décomposition en facteurs premiers de \(n\) au moins autemps de fois que dans celle de \(a\). De meme, tous les facteurs preimers de \(b\) apparaissent dans la décomposition de \(n\) en facteurs premiers au moins autemps de fois que dans celle de \(b\). Donc, tous les facteurs premiers de \(ab\) apparaissent dans la décomposition de \(n\) au moins autemps de fois que dans celle de \(n\) (par ce que chaque facteur premier de \(ab\) est soit un facteur premier de \(a\) et il apparait dans la décomposition en facteurs premiers de \(ab\) autemps de fois que dans celle de \(b\), soit la meme chose mais c'est un facteur premier de \(b\)).

\sol

Si \(p\) est le seul facteur premier de \(n\), alors \(n = p \times p \times \times \times p\) est bien une puissance de \(p\). Réciproquement, si \(n\) est une puissance de \(p\), alors on peut ecrir \(n = p \times p \times \times \times p\), un produit de nombres premiers, donc \(p\) est le seul facteur premier de \(n\).

\sol

Supposons qu'un premier \(p\) divise \(ab\) mais il ne divise ni \(a\) ni \(b\), c'est a dire il est premier avec \(a\) et avec \(b\). Par le lemme de Gauss, il divise alors \(b\), ce qui n'est pas possible car il est premier avec \(b\).

\sol

\(2x+1\) est premier avec \(8 = 2 \times 2 \times 2\) car \(2x+1\) est impair et \(2\) n'en est donc pas un facteur premier. Donc, par le lemme de Gauss, \(2x+1\) divise \(y\) car il divise \(8y\).

\sol

On factorise \(x^2 + xy + x + y = x\left(x+y\right) + x+y = \left(x+1\right)\left(x+y\right)\). \(x+1\) est premier avec \(x\) car ils n'ont pas de facteurs premiers en commun. De faite, si \(p\) est un facteur premier de \(x^2\), il est aussi un facteur premier de \(x\), donc il ne divise pas \(x+1\) car sinon il diviserais \(x+1-x=1\). Donc, par le lemme de Gauss, \(x^2\) divise \(x+y\), ce qu'il fallait démontrer.
