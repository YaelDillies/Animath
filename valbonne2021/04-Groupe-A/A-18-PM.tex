\pro [Inégalité arithmético géométrique, cas $n=2$]

Soit $a, b >0$. $\frac{a+b}{2} \geq \sqrt{ab}$ 

\preuve 

$(\sqrt{a}-\sqrt{b} )^2 \geq 0$ donc en développant, $a+b \geq 2\sqrt{ab}$
\pro [Inégalité arithmético géométrique, cas général]Soit $n\in \mathbb{N^*}$, $a_1,\dots, a_n >0$. Alors $\sum_{k=1}^n a_k \geq n\sqrt[n]{\prod_{k=1}^na_k}$

\preuve Avec une récurrence. On note $I_n$ l'inégalité de Cauchy-Schwarz au rang $n$. 

INITITALISATION : On a déjà montré que $I_2$ est vraie. 

HEREDITE : Soit $n\geq 2$ On suppose $I_n$. Montrons $I_{2n}$. \\ $\frac{a_1+ \dots + a_n + a_{n+1} + \dots + a_{2n}}{2n} \geq \frac{1}{2} \cdot (\sqrt[n]{a1\dots a_n} + \sqrt[n]{a_{n+1}\dots a_{2n}})$ en utilisant l'hypothèse de récurrence. 
On applique ensuite l'IAG au rang $2$ ce qui démontre $I_{2n}$. \\ Montrons à présent $I_{n-1}$. Posons $M := \frac{a_1+\dots + a_{n-1}}{n-1}$. \\ Alors, $M = \frac{(n-1)M+M}{n} = \frac{a_1+\dots+a_{n-1} + M}{n}$. Par hyspothèse de récurrance, on obtient \\ $M \geq \sqrt[n]{a_1 \dots a_{n-1} M} $ On élève à la puissance $n$ième puis on simplifie, pour obtenir $M^{n-1} \geq a_1\dots a_{n-1}$, ce qui conclut. 

\pro [Inégalité de Cauchy Schwarz]

Soit $n \in \mathbb{N}$. On considère deux listes de réels $a_1, \dots, a_n$ et $b_1, \dots, b_n$. \\ Alors, $(a_1^2+\dots + a_n^2)(b_1^2+ \dots b_n^2) \geq (a_1b_1 + \dots + a_nb_n)^2$, avec égalité si les listes sont proportionnelles. 

\preuve On peut démontrer cette inégalité par le calcul. \\ On calcule ainsi la différence des deux termes. 
\begin{align*}
(a_1^2+\dots + a_n^2)(b_1^2+ \dots b_n^2)-(a_1b_1 + \dots + a_nb_n)^2 &= \sum_{1\leq i, j \leq n} a_i^2b_j^2 - \sum_{1 \leq i, j \leq n} a_ia_jb_ib_j \\
&= \frac{1}{2} \cdot (\sum_{1\leq i, j \leq n} a_i^2b_j^2 + \sum_{1\leq i, j \leq n} a_j^2b_i^2 - \sum_{1 \leq i, j \leq n} 2a_ia_jb_ib_j \\
&= \frac{1}{2} \cdot \sum_{1\leq i, j \leq n} (a_ib_j-a_jb_i)^2 \geq 0 
\end{align*}
Le cas d'égalité en découle naturellement. 

\pro[Inégalité des mauvais élèves] Soit $n\in\mathbb{N}$, $a_1, \dots, a_n$ et $b_1, \dots, b_n$ des réels.\\ Alors, $\sum_{k=1}^n\frac{a_k^2}{b_k} \geq \frac{(\sum_{k=1}^na_k)^2}{\sum_{k=1}^n b_k}$

\preuve Montrons cette inégalité dans les cas de réels positifs. \\ D'après l'inégalité de Cauchy-Schwarz, $(\sum_{k=1}^n\frac{a_k^2}{b_k})\cdot (\sum_{k=1}^nb_k) \geq (\frac{a_k}{\sum_{k=1}^n\sqrt{b_k}}\cdot \sqrt{b_k})^2$

\exo Montrer que $\forall a, b >0, a^3 + b^3 + a+b \geq 4ab$

\exo Montrer que $\forall a, b, c, a^2 + b^2+c^2 \geq ab+bc+ac$
\exo[Inégalité de Nesbitt]

Montrer que $\frac{a}{b+c} + \frac{b}{a+c} + \frac{c}{a+b} \geq \frac{3}{2} $

\exo Soit$x, y$ des réels. Montrer que $x^2+y^2+1> x\sqrt{y^2+1} + y\sqrt{x^2+1}$

\exo Soit $a, b$ deux réels strictement supérieurs à $1$. Montrer que $\frac{a^2}{b-1} + \frac {b^2}{a-1} \geq 8$

\exo Soit $a, b$ deux réels strictement positifs. Montrer que $\frac{1}{a} + \frac{1}{b} \geq \frac{4}{a+b}$

\exo Soit $x>0$. Montrer que $x+\frac{1}{x} \geq 2$

\exo $x, y, z >0$. Montrer que $\frac{2}{x+y} + \frac{2}{y+z} + \frac{2}{x+z} \geq \frac{9}{x+y+z}$

\exo $a, b, x, y, z \geq 0$. Montrer que $\frac{x}{ay+bz} + \frac{y}{az+bx} + \frac{z}{ax+by} \geq \frac{3}{a+b}$

\exo Soit $n\in \mathbb{N}$, $x_1, \dots, x_n$ des réels strictement positifs. Montrer que $(x_1+\dots+x_n)(\frac{1}{x_1}+\dots+\frac{1}{x_n}) \geq n^2$

\exo Trouver la valeur minimale de $a^2+b^2+c^2+d^2$ sachant que $a+2b+3c+4d =12$ 

\exo Soit $a, b, c$ strictement positifs. Montrer que $\frac{a}{bc}+\frac{b}{ac}+\frac{c}{ab} \geq \frac{2}{a}+\frac{2}{b}-\frac{2}{c}$

\exo Soit $a, b, c, d$ tels que $0\leq a\leq c \leq d$. Montrer que $ab^3+bc^3+cd^3+da^3 \geq a^2b^2 + b^2c^2+c^2d^2+a^2d^2$
\sol On applique l'IAG, ici dans le cas $n=4$
\sol On écrit $a^2+b^2+c^2 = \frac{a^2+b^2}{2}+\frac{b^2+c^2}{2}+\frac{a^2+c^2}{2}$, puis on applique trois fois l'IAG dans le cas $n=2$. 

\sol On a la chaîne d'inégalités suivante : \\ $\frac{a}{b+c} + \frac{b}{a+c} + \frac{c}{a+b} = \frac{a^2}{ab+ac} + \frac{b^2}{ab+bc} + \frac{c^2}{ac+bc} \geq \frac{(a+b+c)^2}{2(ab+bc+ac)} = 1 + \frac{a^2+b^2+c^2}{2(ab+ac+bc)} \geq 1+\frac{1}{2} = \frac{3}{2}$

\sol On va montrer dans le cas de réels positifs (les autres cas en découlent naturellement)
$x^2+y^2+1 = \frac{(x^2+1)+y^2}{2} + \frac{x^2+(y^2+1)}{2} \geq y\sqrt{x^2+1}+x\sqrt{y^2+1}$\\ On suppose par l'absurde qu'il existe un cas d'égalité. Alors, d'après le cas d'égalité de l'IAG, $y^2=x^2+1$ et $x^2 = y^2+1$ ce qui est absurde. 

\sol On utilise directement l'inégalité des mauvais élèves : $\frac{a^2}{b-1} + \frac {b^2}{a-1} \geq \frac{(a+b)^2}{a+b-2}$\\ Il suffirait de montrer que $(a+b)^2 \geq 8a+8b-16$, c'est-à-dire que $a^2+b^2+16-8a-8b+2ab$, ce qui est vrai car $(a+b-4)^2 \geq 0$

\sol En remarquant que $1 = 1^2$, il s'agit d'une application directe de l'inégalité des mauvais élèves. On peut aussi tout développer.

\sol On utilise l'IAG

\sol On peut utiliser l'inégalité des mauvais élèves directement

\sol idem

\sol Il s'agit ici d'une application directe de l'inégalité de Cauchy-Schwarz, en retournant bien à la définition

\sol D'après l'inégalité de Cauchy-Schwarz, $$(a^2+b^2+c^2+d^2)(1+4+9+16) \geq (a+2b+3c+4d)^2 = 144$$ ce qui permet de trouver un minorant. On montre que c'est un minimum avec le cas d'égalité.

\sol On développe tout et on trouve l'expression d'un carré

\sol D'après l'inégalité de Cauchy-Schwarz, $$(ab^3 + bc^3+cd^3+da^3)(a^3b+b^3c+c^3d+d^3a) \geq (a^2b^2+b^2c^2+c^2d^2+a^2d^2)^2$$

Or, 
\begin{align*}
ab^3 + bc^3+cd^3+da^3 - (a^3b+b^3c+c^3d+d^3a) & = a^3(d-b) +d^3(c-a)+c^3(b-d)+b^3(a-c) \\
& = (d^3-b^3)(c-a)- (c^3-a^3)(d-b) \\
& = (d-b)(c-a)(d^2+bd+b^2-c^2-ac-a^2) \\
& \geq 0
\end{align*}

On en déduit l'inégalité proposée.
