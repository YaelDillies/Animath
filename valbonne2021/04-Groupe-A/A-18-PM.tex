\subsubsection{Quelques inégalités classiques}
\begin{pro} [Inégalité arithmético géométrique, cas $n=2$]
Soit $a, b$ des réels positifs. Alors $\frac{a+b}{2} \geqslant \sqrt{ab}$, avec égalité si et seulement si $a=b$.
\end{pro}

Interprétation géométrique : la moyenne géométrique correspond au côté d'un carré de même aire que le rectangle de côtés $a$ et $b$.

\begin{preuve}
$(\sqrt{a}-\sqrt{b} )^2 \geqslant 0$ car c'ets un carré donc en développant, $a+b \geqslant 2\sqrt{ab}$. Il y a égalité si et seulement si $(\sqrt{a} - \sqrt{b})^2 = 0$, si et seulement si $\sqrt{a} = \sqrt{b}$, si et seulement si $a=b$.
\end{preuve}

\begin{pro} [Inégalité arithmético géométrique, cas général]
Soit $n\in \mathbb{N^*}$, $a_1,\dots, a_n >0$. Alors $\sum_{k=1}^n a_k \geqslant n\sqrt[n]{\prod_{k=1}^na_k}$, avec égalité si et seulement si $a_1=\dots=a_n$
\end{pro}

Nous allons en particulier démontrer les cas $n=3$ et $n=4$.

\begin{preuve} : cas $n=4$.
Soient $a, b, c, d$ des réels strictement positifs. Montrons que $\frac{a+b+c+d}{4} \geqslant \sqrt[4]{abcd}$. \\ Pour cela, on utilise deux fois l'IAG au rang $2$ : $\frac{a+b+c+d}{4} \geqslant \frac{\sqrt{ab}+\sqrt{bc}}{2} \geqslant \sqrt{\sqrt{ab}\sqrt{bc}} = \sqrt[4]{abcd}$\\ Le cas d'égalité s'obtient à partir du cas d'égalité de l'IAG au rang $2$.
\end{preuve}

\begin{preuve} : cas $n=3$.

Soient $a, b, c$ des réels positifs. Montrons que $\frac{a+b+c}{3} \geqslant \sqrt[3]{abc}$. \\ Cela est équivalent à montrer que pour tout $x, y,z$ réels strictement positifs, $x^3+y^3+z^3 \geqslant 3xyz$. On commence par démontrer que $x^2+y^2+z^2 \geqslant xy+yz+xz$ (lemme du tourniquet) (voir exercice 2)\\ Ceci démontré, on remarque que $x^3+y^3+z^3-3xyz = (x+y+z)(x^2+y^2+z^2-xy-yz-xz) \geqslant 0$ comme produit de termes positifs. Ainsi, $x^3+y^3+z^3 \geqslant 3xyz$. \\ Il y a égalité si et seulement si $x+y+z = 0$ ou $x^2+y^2+z^2 - xy-xz-yz = 0$, si et seulement si $x=y=z=0$ (car ce sont des termes positifs dont la somme est nulle), ou $x=y=z$; si et seulement si $x=y=z$. \\ On en déduit l'IAG au rang $3$
\end{preuve}

\begin{preuve} Avec une récurrence. Cette preuve nécessite une très bonne maîtrise du principe de récurrence avant de l'aborder. On note $I_n$ l'IAG au rang $n$.

\textbf{INITIALISATION} : On a déjà montré que $I_2$ est vraie.

\textbf{HÉRÉDITÉ} : Soit $n\geqslant 2$ On suppose $I_n$. Montrons $I_{2n}$. \\ $\frac{a_1+ \dots + a_n + a_{n+1} + \dots + a_{2n}}{2n} \geqslant \frac{1}{2} \cdot (\sqrt[n]{a1\dots a_n} + \sqrt[n]{a_{n+1}\dots a_{2n}})$ en utilisant l'hypothèse de récurrence.
On applique ensuite l'IAG au rang $2$ ce qui démontre $I_{2n}$. \\ Montrons à présent $I_{n-1}$. Posons $M := \frac{a_1+\dots + a_{n-1}}{n-1}$. \\ Alors, $M = \frac{(n-1)M+M}{n} = \frac{a_1+\dots+a_{n-1} + M}{n}$. Par hyspothèse de récurrance, on obtient \\ $M \geqslant \sqrt[n]{a_1 \dots a_{n-1} M} $ On élève à la puissance $n$ième puis on simplifie, pour obtenir $M^{n-1} \geqslant a_1\dots a_{n-1}$, ce qui conclut.
\end{preuve}

\begin{pro} [Inégalité de Cauchy Schwarz]
Soit $n \in \mathbb{N}$. On considère deux listes de réels $a_1, \dots, a_n$ et $b_1, \dots, b_n$. \\ Alors, $(a_1^2+\dots + a_n^2)(b_1^2+ \dots b_n^2) \geqslant (a_1b_1 + \dots + a_nb_n)^2$, avec égalité si et seulement si les listes sont proportionnelles.
\end{pro}

\begin{preuve} On peut démontrer cette inégalité par le calcul. \\ On calcule ainsi la différence des deux termes.
\begin{align*}
(a_1^2+\dots + a_n^2)(b_1^2+ \dots b_n^2)-(a_1b_1 + \dots + a_nb_n)^2 &= \sum_{1\leqslant i, j \leqslant n} a_i^2b_j^2 - \sum_{1 \leqslant i, j \leqslant n} a_ia_jb_ib_j \\
&= \frac{1}{2} \cdot (\sum_{1\leqslant i, j \leqslant n} a_i^2b_j^2 + \sum_{1\leqslant i, j \leqslant n} a_j^2b_i^2 - \sum_{1 \leqslant i, j \leqslant n} 2a_ia_jb_ib_j \\
&= \frac{1}{2} \cdot \sum_{1\leqslant i, j \leqslant n} (a_ib_j-a_jb_i)^2 \geqslant 0
\end{align*}
Il y a égalité si, et seulement si, $\forall (i, j) \in \llbracket 1, n \rrbracket ^2$, $a_ib_j-a_jb_i = 0$\\ On peut alors distinguer deux cas : \\
CAS 1 : la liste $(a_1, \dots, a_n)$ est la liste nulle. Alors, $\forall i \in \llbracket 1, n \rrbracket, a_i = 0 \cdot b_i$, donc les listes sont bien proportionnelles. \\
CAS 2 : les réels $(a_1, \dots, a_n)$ sont non tous nuls. Alors il existe $ 1 \leqslant j \leqslant n$ tel que $a_j \neq 0$. $\forall i \in \llbracket 1, n \rrbracket, a_ib_j = a_jb_i$ donc $a_i \cdot \frac{b_j}{a_j} = b_i$. Or, $\frac{b_j}{a_j}$ est une constante. Donc les deux listes sont porportionnelles.

\end{preuve}

\begin{pro}[Inégalité des mauvais élèves] Soit $n\in\mathbb{N}$, $a_1, \dots, a_n$ et $b_1, \dots, b_n$ des réels positifs.\\ Alors, $\sum_{k=1}^n\frac{a_k^2}{b_k} \geqslant \frac{(\sum_{k=1}^na_k)^2}{\sum_{k=1}^n b_k}$
\end{pro}

\begin{preuve} D'après l'inégalité de Cauchy-Schwarz, $(\sum_{k=1}^n\frac{a_k^2}{b_k})\cdot (\sum_{k=1}^nb_k) \geqslant (\frac{a_k}{\sum_{k=1}^n\sqrt{b_k}}\cdot \sqrt{b_k})^2$ \\ Il y a égalité si et seulement si les listes auxquelles on a appliqué Cauchy-Schwarz sont proportionnelles, c'est-à-dire que les listes $(a_1, \dots, a_n)$ et $(b_1, \dots, b_n)$ sont proportionnelles.

\end{preuve}

\subsubsection{ Exercices d'application plus ou moins directe }

\begin{exo} Montrer que $\forall a, b >0, a^3 + b^3 + a+b \geqslant 4ab$
\end{exo}

\begin{exo} Montrer que $\forall a, b, c, a^2 + b^2+c^2 \geqslant ab+bc+ac$
\end{exo}

\begin{exo} Soit $a, b$ deux réels strictement positifs. Montrer que $\frac{1}{a} + \frac{1}{b} \geqslant \frac{4}{a+b}$
\end{exo}

\begin{exo} Soit $x>0$. Montrer que $x+\frac{1}{x} \geqslant 2$
\end{exo}


\begin{sol} D'après l'IAG appliquée au rang $n=4$, $a^3+b^3+a+b \geqslant 4\sqrt[4]{a^3b^3ab} = 4 ab $
\end{sol}

\begin{sol} L'idée est d'appliquer l'IAG au rang $n=2$.Pour cela, on fait apparaître de manière artificielle des termes en réorganisant la somme astucieusement : $a^2+b^2+c^2 = \frac{a^2+b^2}{2}+\frac{b^2+c^2}{2}+\frac{a^2+c^2}{2}$. On applique ensuite l'IAG dans le cas $n=2$ : $\frac{a^2+b^2}{2}+\frac{b^2+c^2}{2}+\frac{a^2+c^2}{2} \geqslant \sqrt{a^2b^2} + \sqrt{b^2c^2} + \sqrt{a^2c^2}$. On en déduit que $a^2+b^2+c^2 \geqslant ab+ac+bc$.
\end{sol}

\begin{sol} On remarque que $1 = 1^2$. Alors, d'après l'inégalité des mauvais élèves, \\ $\frac{1}{a} + \frac{1}{b} \geqslant \frac{(1+1)^2}{a+b} = \frac{4}{a+b}$
\end{sol}

\begin{sol} D'après l'IAG au rang $2$, $x+\frac{1}{x} \geqslant 2 \sqrt{x \cdot \frac{1}{x}} = 2$. Il y a égalité si et seulement si $x=\frac{1}{x}$, dons si $x=1$.
\end{sol}
