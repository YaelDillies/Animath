\subsubsection{Quelques inégalités classiques}


\begin{pro}[Inégalité arithmético-géométrique, cas $n=2$]
Soit $a, b$ des réels positifs. Alors $\frac{a+b}2 \ge \sqrt{ab}$, avec égalité si et seulement si $a=b$.
\end{pro}

Interprétation géométrique : la moyenne géométrique correspond au côté d'un carré de même aire que le rectangle de côtés $a$ et $b$.

\begin{preuve}
$(\sqrt a-\sqrt b)^2 \ge 0$ car c'est un carré donc en développant, $a+b \ge 2\sqrt{ab}$. Il y a égalité si et seulement si $(\sqrt a - \sqrt b)^2 = 0$, si et seulement si $\sqrt a = \sqrt b$, si et seulement si $a=b$.
\end{preuve}

\begin{pro} [Inégalité arithmético-géométrique, cas général]
Soit $n\in \N^*$, $a_1,\dots, a_n >0$. Alors $\sum_{k=1}^n a_k \ge n\sqrt[n]{\prod_{k=1}^na_k}$, avec égalité si et seulement si $a_1=\dots=a_n$
\end{pro}

Nous allons en particulier démontrer les cas $n=3$ et $n=4$.

\begin{preuve} : cas $n=4$.
Soient $a, b, c, d$ des réels strictement positifs. Montrons que $\frac{a+b+c+d}{4} \ge \sqrt[4]{abcd}$. \\ Pour cela, on utilise deux fois l'IAG au rang $2$ : $\frac{a+b+c+d}{4} \ge \frac{\sqrt{ab}+\sqrt{bc}}2 \ge \sqrt{\sqrt{ab}\sqrt{bc}} = \sqrt[4]{abcd}$\\ Le cas d'égalité s'obtient à partir du cas d'égalité de l'IAG au rang $2$.
\end{preuve}

\begin{preuve} : cas $n=3$.

Soient $a, b, c$ des réels positifs. Montrons que $\frac{a+b+c}{3} \ge \sqrt[3]{abc}$. \\ Cela est équivalent à montrer que pour tout $x, y,z$ réels strictement positifs, $x^3+y^3+z^3 \ge 3xyz$. On commence par démontrer que $x^2+y^2+z^2 \ge xy+yz+xz$ (lemme du tourniquet) (voir exercice 2)\\ Ceci démontré, on remarque que $x^3+y^3+z^3-3xyz = (x+y+z)(x^2+y^2+z^2-xy-yz-xz) \ge 0$ comme produit de termes positifs. Ainsi, $x^3+y^3+z^3 \ge 3xyz$. \\ Il y a égalité si et seulement si $x+y+z = 0$ ou $x^2+y^2+z^2 - xy-xz-yz = 0$, si et seulement si $x=y=z=0$ (car ce sont des termes positifs dont la somme est nulle), ou $x=y=z$; si et seulement si $x=y=z$. \\ On en déduit l'IAG au rang $3$.
\end{preuve}

\begin{preuve}
Avec une récurrence. Cette preuve nécessite une très bonne maîtrise du principe de récurrence avant de l'aborder. On note $I_n$ l'IAG au rang $n$. \\
\textbf{Initialisation :} On a déjà montré que $I_2$ est vraie. \\
\textbf{Hérédité :} Soit $n\ge 2$ On suppose $I_n$. Montrons $I_{2n}$. \\ $\frac{a_1+ \dots + a_n + a_{n+1} + \dots + a_{2n}}{2n} \ge \frac{1}2 \cdot (\sqrt[n]{a1\dots a_n} + \sqrt[n]{a_{n+1}\dots a_{2n}})$ en utilisant l'hypothèse de récurrence.
On applique ensuite l'IAG au rang $2$ ce qui démontre $I_{2n}$. \\ Montrons à présent $I_{n-1}$. Posons $M := \frac{a_1+\dots + a_{n-1}}{n-1}$. \\ Alors, $M = \frac{(n-1)M+M}{n} = \frac{a_1+\dots+a_{n-1} + M}{n}$. Par hypothèse de récurrence, on obtient \\ $M \ge \sqrt[n]{a_1 \dots a_{n-1} M} $ On élève à la puissance $n$-ième puis on simplifie, pour obtenir $M^{n-1} \ge a_1\dots a_{n-1}$, ce qui conclut.
\end{preuve}

\begin{pro}[Inégalité de Cauchy-Schwarz]
Soit $n \in \N$. On considère deux listes de réels $a_1, \dots, a_n$ et $b_1, \dots, b_n$. Alors
$$(a_1^2 + \dots + a_n^2)(b_1^2 + \dots + b_n^2) \ge (a_1b_1 + \dots + a_nb_n)^2$$
avec égalité si et seulement si les listes sont proportionnelles.
\end{pro}

\begin{preuve} On peut démontrer cette inégalité par le calcul. \\
  On calcule ainsi la différence des deux termes.
\begin{eqnarray*}
(a_1^2 +\dots + a_n^2)(b_1^2 + \dots + b_n^2) - (a_1b_1 + \dots + a_nb_n)^2
& = & \sum_{1\le i, j \le n} a_i^2b_j^2 - \sum_{1 \le i, j \le n} a_ia_jb_ib_j \\
& = & \frac 12 \cdot (\sum_{1\le i, j \le n} a_i^2b_j^2 + \sum_{1\le i, j \le n} a_j^2b_i^2 - \sum_{1 \le i, j \le n} 2a_ia_jb_ib_j \\
& = & \frac 12 \cdot \sum_{1\le i, j \le n} (a_ib_j - a_jb_i)^2 \ge 0
\end{eqnarray*}
Il y a égalité si, et seulement si, pour tout $i, j \in [\![1, n]\!]$, on a $a_ib_j - a_jb_i = 0$, ce qui signifie exactement que les deux listes sont porportionnelles.
\end{preuve}

\begin{pro}[Inégalité des mauvais élèves]
Soit $n \in \N$, $a_1, \dots, a_n$ et $b_1, \dots, b_n$ des réels positifs. Alors
$$\sum_{k = 1}^n\frac{a_k^2}{b_k} \ge \frac{(\sum_{k = 1}^n a_k)^2}{\sum_{k = 1}^n b_k}$$
\end{pro}

\begin{preuve}
D'après l'inégalité de Cauchy-Schwarz,
$$(\sum_{k=1}^n\frac{a_k^2}{b_k})\cdot (\sum_{k=1}^nb_k) \ge (\frac{a_k}{\sum_{k=1}^n\sqrt{b_k}}\cdot \sqrt{b_k})^2$$
Il y a égalité si et seulement si les listes auxquelles on a appliqué Cauchy-Schwarz sont proportionnelles, c'est-à-dire que les listes $(a_1, \dots, a_n)$ et $(b_1, \dots, b_n)$ sont proportionnelles.
\end{preuve}


\subsubsection{Exercices d'application}


\begin{exo}
Montrer que pour tous $a, b > 0$,
$$a^3 + b^3 + a + b \ge 4ab$$
\end{exo}


\begin{exo}
Montrer que $a^2 + b^2 + c^2 \ge ab+bc+ac$
\end{exo}


\begin{exo}
Soit $a, b$ deux réels strictement positifs. Montrer que $\frac 1a + \frac 1b \ge \frac 4{a+b}$
\end{exo}


\begin{exo}
Soit $x>0$. Montrer que $x+\frac{1}{x} \ge 2$
\end{exo}


\subsubsection{Solutions}


\begin{sol}
D'après l'IAG appliquée au rang $n=4$, $a^3+b^3+a+b \ge 4\sqrt[4]{a^3b^3ab} = 4 ab $
\end{sol}


\begin{sol}
L'idée est d'appliquer l'IAG au rang $n=2$.Pour cela, on fait apparaître de manière artificielle des termes en réorganisant la somme astucieusement : $a^2+b^2+c^2 = \frac{a^2+b^2}2+\frac{b^2+c^2}2+\frac{a^2+c^2}2$. On applique ensuite l'IAG dans le cas $n=2$ :
$$\frac{a^2+b^2}2+\frac{b^2+c^2}2+\frac{a^2+c^2}2 \ge \sqrt{a^2b^2} + \sqrt{b^2c^2} + \sqrt{a^2c^2}$$
On en déduit que $a^2+b^2+c^2 \ge ab+ac+bc$.
\end{sol}


\begin{sol}
On remarque que $1 = 1^2$. Alors, d'après l'inégalité des mauvais élèves,
$$\frac 1a + \frac 1b \ge \frac{(1+1)^2}{a+b} = \frac 4{a+b}$$
\end{sol}

\begin{sol}
D'après l'IAG au rang $2$, $x + \frac 1x \ge 2 \sqrt{x \cdot \frac 1x} = 2$. Il y a égalité si et seulement si $x = \frac 1x$, donc si $x = 1$.
\end{sol}
