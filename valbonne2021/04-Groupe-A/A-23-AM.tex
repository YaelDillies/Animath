L'objectif de cette séance était de découvrir les congruences.

\smallskip

\begin{dfn}
Soient $a$, $b$ deux entiers, et $n$ un entier strictement positif. On dit que $a$ et $b$ sont congrus modulo $n$ si $a$ et $b$ ont le même reste dans la division euclidienne par $n$, ou si $n$ divise $a - b$.
On note $a \equiv b \pmod{n}$, ou encore $a \equiv b [n]$.
\end{dfn}
\smallskip


\begin{pro}
Soient $a$, $b$, $c$, $d$ des entiers, $n$ un entier strictement positif et $m$ un entier positif. Supposons $a \equiv b [n]$ et $c \equiv d [n]$. Alors :
\begin{itemize}
    \item $a + c \equiv b + d [n]$\
    \item $ac \equiv bd [n]$\
    \item $a^m\equiv b^m [n]$
\end{itemize}
\end{pro}
\smallskip

\textbf{Démonstration} :\\
Par hypothèse, $n|a-b$ et $n|c-d$.
\begin{itemize}
    \item On a alors $n|a-b+c-d$, c'est-à-dire $a-b+c-d\equiv 0[n]$, d'où $a+c\equiv b+d [n]$\
    \item $ac - bd = a(c - d + d) - bd = a(c - d) + ad - bd = a(c - d) + (a - b)d$ qui est divisible par $n$, donc $ac \equiv bd [n]$.\
    \item Il suffit d'appliquer $m$ fois la proposition précédente.\
\end{itemize}

\bigskip

\textbf{Remarques} :
\begin{itemize}
    \item La relation de congruence modulo $n$ est compatible avec l’addition, la soustraction, la multiplication mais \textbf{pas} la division en général.\
    \item Soient $a$, $b$, $c$ des entiers.
    \newline
    $a\equiv b[n]$ n'implique pas $c^a\equiv c^b[n]$ !\
    \newline
    Par exemple, $0\equiv 3[3]$, mais $2^0\equiv 1\not\equiv 2\equiv 2^3[3]$
\end{itemize}

\bigskip

\begin{ex}[Critère de divisibilité par 3]
Tout nombre peut s'écrire sous la forme $10^kn_k+10^{k-1}n_{k-1}...+10n_1+n_0$.
\newline
Or $10\equiv 1[3]$ donc $10^n\equiv 1[3]$
\newline
On a alors $10^kn_k+10^{k-1}n_{k-1}...+10n_1+n_0\equiv n_k+n_{k-1}...+n_1+n_0[3]$, ce qui signifie qu'un nombre est divisible par 3 si et seulement si la somme de ses chiffres l'est.
\end{ex}


\begin{exo}
Montrer que pour tout entier $y>1$, $y-1$ divise $y^{y^2-y+2}-4y+y^{2021}+3y^2-1$.
\end{exo}


\begin{exo}[Rappel factorisation]
Trouver tous les couples d'entiers $(x,y)$ solutions de l'équation $x^2=7+y^2$.
\end{exo}


%\begin{exo}
%Calculer la somme des chiffres de la somme des chiffres de la somme des chiffres de $4444^{4444}$.
%\end{exo}


\vspace{0.5cm}
\begin{sol}
$y-1\equiv 0[y-1] \Rightarrow y\equiv 1 [y-1]$
\newline Donc $y^{y^2-y+2}\equiv 1^{y^2-y+2}\equiv 1[y-1]$, $4y\equiv 4[y-1]$ et $3y^2\equiv 2 [y-1]$.
\newline Ainsi en particulier :

$$y^{y^2-y+2}-4y+y^{2021}+3y^2-1\equiv 1-4+1+3-1\equiv 0[y-1]$$

Donc $y-1$ divise cet entier.
\end{sol}

\begin{sol}
$x^2=7+y^2 \iff x^2-y^2=7 \iff (x-y)(x+y)=7$
\newline $x+y$ et $x-y$ sont donc des diviseurs associés de 7.
\newline En résolvant les systèmes on obtient 4 solutions : $(4,3), (4,-3), (-4,-3), (-4,3)$.
\end{sol}

\bigskip


\subsubsection{Carrés modulo $n$}


\begin{dfn}
Un entier $a$ est un \textbf{carré modulo $n$} s'il existe $b$ tel que $a \equiv b^2[n]$.
\end{dfn}

Tous les nombres ne sont pas des carrés modulo $n$, ce qui nous permet de résoudre certains exercices en trouvant une contradiction.

\smallskip
On retiendra par exemple que :
\newline
— Un carré est toujours congru à 0 ou à 1 modulo 4.
\newline
— Un carré est toujours congru à 0 ou 1 modulo 3.
\newline
— Les cubes modulo 7 sont 0, 1, -1.

\bigskip
\textbf{Démonstration :}
\newline
Faisons un tableau de congruences modulo 4 :
$$\begin{array}{|c|c|c|c|c|}
\hline
\ n[4]&0&1&2&3 \\
\hline\ n^2[4]&0&1&4\equiv0&9\equiv1 \\
\hline
\end{array}$$
\newline

De même, on obtient le tableau suivant pour les carrés modulo 3 :
$$\begin{array}{|c|c|c|c|}
\hline
\ n[3]&0&1&2 \\
\hline\ n^2[3]&0&1&9\equiv0 \\
\hline
\end{array}$$
\newline

Pour les cubes modulo 7 :
$$\begin{array}{|c|c|c|c|c|c|c|c|}
\hline
\ n[7]&0&1&2&3&4&5&6 \\
\hline\ n^3[7]&0&1&8&27&64&125&216 \\
\hline\ n^3[7]&0&1&1&-1&1&-1&-1 \\
\hline
\end{array}$$

\begin{ex}
Montrer que si $(x, y, z)$ est une solution de l’équation $x^2+y^2=z^2$ alors $x$ ou $y$ est un multiple de 2.
\smallskip
\newline

On regarde l’équation modulo 4. Si $x$ et $y$ sont tous les deux impairs, alors $x^2\equiv y^2\equiv 1$ donc $z^2\equiv 2$. Or un carré n'est jamais congru à 2 modulo 4. Absurde.
\end{ex}

\smallskip


\begin{exo}
Trouvez tous les entiers positifs ou nuls $x,y$ tels que:
 $$2^x=y^2+y+1$$
\end{exo}


\begin{exo}
Existe-t-il des entiers $a$ et $b$ tels que $a^2 + b^2 = 10^{100} + 3$ ?
\end{exo}


\begin{exo} Trouver tous les triplets d'entiers positifs $(x,y,n)$ tels que:
$x^2+y^2+41=2^n$.
\end{exo}


\begin{exo}
Trouver les entiers $a$ et $b$ tels que : $3a^2=b^2+1$
\end{exo}


\begin{exo}
Existe-t-il un entier $n\ge 0$ tel que $6^n+19$ est premier ?
\end{exo}


\begin{exo}
Soit $a_1,a_2,\ldots$ la suite d'entiers telle que $a_1 = 1$ et,
pour tout entier $n \ge 1$,
\[a_{n+1} = a_n^2+a_n+1.\]
Démontrer, pour tout entier $n \ge 1$, que
$a_n^2+1$ divise $a_{n+1}^2+1$.
\end{exo}


\begin{exo}
Trouver tous les entiers $n$ tels que $2^n + 3$ est un carré parfait. Même question avec $2^n + 1$.
\end{exo}


\begin{exo} Trouver tous les quadruplets d'entiers positifs $(x,y,z,n)$ tels que: $x^2+y^2+z^2+1=2^n$
\end{exo}


\begin{exo}
Déterminer toutes les $x,y$ entiers positifs solution de $x^2-2\cdot y!=2021$.
\end{exo}


\begin{sol}
Le côté gauche de l'équation est pair si $x>0$. Le côté droit est toujours impair: en effet $y$ et $y^2$ sont de même parité, leur somme est donc paire, donc $1+y+y^2$ est toujours impair.

\smallskip

Si $x=0$, alorsil faut et il suffit que $1+y+y^2=1$ avec $y\ge 0$ donc la seule solution est $x=y=0$.
\end{sol}


\begin{sol}
Nous avons vu que pour tout entier $a$, $a^2$ est soit congru à 0 soit à 1 modulo 4. Donc $a^2 + b^2$ est congru à 0, 1 ou 2 modulo 4. Or $10^{100} + 3\equiv 3[4]$ donc ne peut pas être égal à $a^2 + b^2$.
\end{sol}


\begin{sol}
On regarde modulo 4, un carré est congru à 0 ou 1. Ainsi si $n\ge 2$, on n'a pas de solutions car le membre de gauche est congru à 1,2 ou 3 modulo 4, alors que le membre de droite est congru à 0 modulo 4. Ainsi il reste seulement le cas où $n=0$ ou $n=1$, et on vérifie alors qu'il n'y a pas de solutions.
\end{sol}


\begin{sol}
On regarde modulo 3 : le membre de gauche est congru à $0$, tandis que le membre de droite est congru à $1$ ou $2$ modulo $3$, car aucun carré modulo $3$ n'est congru à $2$. Ainsi il n'ya pas de solution.

\smallskip

On peut aussi regarder modulo $4$ pour montrer qu'il n'y a pas de solutions.
\end{sol}


\begin{sol}
En testant les petits cas on remarque les nombres de la forme $6^n+19$ sont des multiples de 5. Regardons alors modulo 5 :
\newline
$6^n+19\equiv 1+19\equiv 0[5]$, donc $6^n+19$ est toujours divisible par $5$. Le seul nombre premier divisible par $5$ est $5$, mais $6^n+19=5$ ne possède pas de solution pour $n\ge 0$.
\end{sol}


\begin{sol}
Modulo $a_n^2+1,$ on a $a_{n+1}\equiv a_n[a_n^2+1],$ donc $a_{n+1}^2+1\equiv a_n^2+1\equiv 0[a_n^2+1]$. $a_{n+1}^2+1$ est bien divisible par $a_n^2+1$.
\end{sol}


\begin{sol}
\begin{itemize}
\item Un carré n’est jamais congru à 3 mod 4, donc si $n > 2$, $2^n + 3$ ne peut jamais être un carré. Pour $n = 1$, $2^1 + 3 = 5$ n’est pas le carré d’un entier. Pour $n = 0$, $2^0 + 3 = 4$ est le carré de 2.
\item On cherche deux entiers $n$ et $x$ tels que $2^n + 1 = x^2$. On réécrit cette équation $2^n =x^2-1 = (x + 1)(x - 1)$. $x + 1$ et $x - 1$ sont tous les deux des puissance de 2. Les seules puissances de 2 à distance 2 l’une de l’autre sont 2 et 4. Donc $2^n = 2 \cdot 4 = 8 = 2^3$. $n = 3$ est la seule solution.
\end{itemize}
\end{sol}


\begin{sol}
On regarde modulo $8$. Pour $n<3$ les seules solutions sont :
$$(0,0,0,0),(0,0,1,1),(0,1,0,1),(1,0,0,1),(1,1,1,2)$$
Si $n\ge 3$, on regarde modulo $8$. Un carré est congru à $0,1,4$ modulo $8$, donc le membre de droite vaut $1,2,3,4,5,6$ ou $7$, mais jamais $0$ modulo $8$. Ainsi il n'y a que les $5$ solutions citées précédemment.
\end{sol}


\begin{sol}
En regardant modulo $8$, $2021\equiv 5,$ et $8\mid 2\cdot y!$ dès que $y\ge 4.$ Ainsi, on aurait $x^2\equiv 5[8],$ contradiction !\\
Donc $y<4$. En traitant les cas restants, on trouve le couple $(45,2)$ comme seule solution.
\end{sol}


\subsubsection{Inverse modulo $n$}


\begin{dfn}
Un entier $b$ est appelé \textbf{inverse de $a$ modulo $n$} si $ab \equiv 1 [n]$. Si $a$ possède un inverse modulo $n$, alors on dit que $a$ est \textit{inversible modulo $n$}.
\end{dfn}


\begin{ex}
$2$ est son propre inverse modulo $3$ car $2 \cdot 2 = 4 \equiv 1 \mod 3$.
\end{ex}

\begin{pro}
Un entier $a$ possède un inverse modulo $n$ si et seulement si $a$ et $n$ sont premiers entre eux. L'inverse de $a$ est alors unique modulo $n$.
\end{pro}

\begin{preuve}
Par le théorème de Bézout, il existe des entiers $u$ et $v$ tels que $au + nv = 1$ si et seulement si $a$ et $n$ sont premiers entre eux. Dans ce cas, $u$ est l'inverse de $a$ modulo $n$.

Pour montrer l'unicité, supposons que $b$ et $b'$ sont deux inverses de $a$ modulo $n$. Nous avons alors
$$b=b\cdot 1\equiv b(ab')\equiv (ba)b'\equiv b'[n]$$

On a donc $b\equiv b'[n]$, d’où l’unicité modulo $n$.
\end{preuve}