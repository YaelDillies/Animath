% Auguste de Lambilly
\subsubsection{Exercices}


\begin{exo}
Soit $a, b$ deux nombres réels tels que
$$\frac a{1 + b} + \frac b{1 + a} = 1$$
Montrer que $a^3 + b^3 = a + b$
\end{exo}


\begin{exo}
Montrer que pour tous réels positifs $a,b,c$, on a : $ (a+b)(b+c)(c+a)\ge 8abc$
\end{exo}


\begin{exo}
Trouver tous les $n$ entiers tels que $4n^4+1$ soit un nombre premier.
\end{exo}


\begin{exo}
Trouver la valeur minimale de $a^2+b^2+c^2+d^2$ sachant que $a+2b+3c+4d=12$.
\end{exo}


\begin{exo}
Pour $x, y, z > 0$ montrer que
$$ \frac xy + \frac yz + \frac zx + \frac xz + \frac zy + \frac yx \ge 6$$
\end{exo}


\begin{exo}
Trouver $k$ tel que, pour tous $a, b, c$ réels,
$$(a+b)(b+c)(c+a) = (a+b+c)(ab+bc+ca) + k\cdot abc $$
\end{exo}


\begin{exo}
Pour tous réels positifs $a,b,c$, montrer que
$$(a^2b+b^2c+c^2a)(ab^2+bc^2+ca^2)\ge 9a^2b^2c^2$$
\end{exo}


\subsubsection{Solutions}


\begin{sol}
On a $a(1+a)+b(1+b)=(1+a)(1+b)$. Donc $a+a^2+b+b^2=1+a+b+ab$ puis $a^2+b^2=1+ab$ ie $a^2-ab+b^2 = 1$.

Il suit
$$a^3+b^3=(a+b)(a^2-ab+b^2)=a+b$$
\end{sol}


\begin{sol}
Par IAG on a : $a+b\ge 2\sqrt{ab}$, $b+c\ge 2\sqrt{bc}$, $c+a\ge 2\sqrt{ca}$.

Comme chaque terme est positif, on peut multiplier les inégalités, ce qui donne le résultat voulu.
\end{sol}


\begin{sol}
Par l'identité de Sophie-Germain :

$1+4n^4=(1-2n+2n^2)(1+2n+2n^2)$
Donc il faut nécessairement $1-2n+2n^2=1$, donc $n=1$.
\end{sol}


\begin{sol}
Par Cauchy-Schwarz,
$$a^2+b^2+c^2+d^2\ge \frac{(a+2b+3c+4d)^2}{1+4+9+16}=\frac{144}{30}=\frac{24}5$$
Réciproquement, cette valeur est atteinte pour $a = \frac 25$, $b = \frac 45$, $c = \frac 65$, $d = \frac 85$.
\end{sol}


\begin{sol}
On a $a + \frac 1a \ge 2$ par IAG. Donc avec $a = \frac xy,\frac yz,\frac zx$, en les sommant on a le résultat voulu.
\end{sol}


\begin{sol}
En développant on trouve $k=-1$
\end{sol}


\begin{sol}
Par IAG,
$$a^2b+b^2c+c^2a\ge 3abc \text{ et } ab^2+bc^2+ca^2\ge 3abc$$
En multipliant on obtient le résultat voulu.
\end{sol}
