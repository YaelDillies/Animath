
Pour la 5ième fois, le Centre International de Valbonne (CIV) nous a accueilli du lundi 16 août vers 15h au jeudi 26 août vers 8h, avec un effectif final de 79 stagiaires et 28 animatheurs.

Parmi le millier de candidats à la Coupe Animath, 700 ont franchi le cap du premier tour. Sur la base des résultats du second tour, nous devions accueillir 79 stagiaires : 31 de fin de première, 18 de seconde, 18 de troisième, 11 de quatrième et 1 de cinquième. En prévision des EGMO, Olympiades Européennes Féminines de Mathématiques, et de la JBMO, Olympiades Balkaniques Junior de Mathématiques, des bonifications ont été ajoutées pour favoriser les filles et les plus jeunes.

Le stage était structuré comme ceux des années précédentes : deux périodes de quatre jours (17 - 20 août et 21 - 24 août), les trois premiers dédiés aux cours et exercices et le dernier à un entraînement de type olympique le matin (de 9h à 12h, ou, pour le groupe D, de 8h à 12h) et un après-midi récréatif. Les élèves étaient répartis en 4 groupes A, B, C, et D en fonction de leur expérience en mathématiques olympiques. \\
Le programme mathématique suit les disciplines évaluées lors des compétitions internationales : Algèbre, Combinatoire, Géométrie et Arithmétique.

Au delà des cours, les élèves assistèrent, le soir, à des conférences à vocation culturelle, donnant à découvrir des domaines mathématiques non olympiques. Merci à Victor Vermès pour son explication sur comment empiler des briques (ici modélisées par les innombrables jeux de cartes fournis par notre sponsor) pour construire un pont mathématiquement rapide ; Raphaël Ducatez pour son exposé sur la théorie de l'information, jusqu'à l'entropie de Shannon ; et Jean Rax pour avoir présenté le lien entre approcher des solutions et résoudre des équations approchées.

\vfill
\pagebreak
