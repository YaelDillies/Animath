%Martin Rakovsky
Dans le triangle $ABC$, soit $D$ le pied de la bissectrice de l'angle $\widehat{BAC}$ et $E$ et $F$ les centres respectifs des cercles circonscrits aux triangles $ABD$ et $ACD$ respectivement. Soit $\omega$ le cercle circonscrit à $DEF$ et soit $X$ l'intersection de $(BF)$ et $(CE)$. Les droites $(BE)$ et $(BF)$ coupent $\omega$ en $P$ et $Q$ respectivement et les droites $(CE)$ et $(CF)$ recoupent $\omega$ en $R$ et $S$ respectivement. Soit $Y$ le second point d'intersection des cercles circonscrits à $PQX$ et $RSX$. Montrer que $Y$ est sur $(AD)$.