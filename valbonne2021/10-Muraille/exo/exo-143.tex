%Théodore Fougereux
% Iran Mo 3rd round 2016 NT P3
Soit $(a_n)_{n\in \N^*}$ une suite d'entiers. On dit que $(a_n)_{n\in \N^*}$ est une permutation de $\N^*$ si pour tout $m\in \N^*$, il existe un unique $n\in \N^*$ tel que $a_n = m$. Montrer qu'il existe une suite $(P_i)_{i\in \N^*}=((a_{i, j})_{j\in \N^*})_{i\in \N^*}$ de permutations de $\N^*$ vérifiant pour tout $k\in \N^*$ et tous $1 \le i_1 < i_2$ :
$$S_k(P_{i_1})~|~S_k(P_{i_2})$$
où $S_k(P_i)$ désigne la somme des $k$ premiers éléments de $P_i$: $S_k(P_i)=\sum\limits_{j=1}^{k} a_{i, j}$.