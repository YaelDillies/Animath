% Théodore Fougereux
%Tournament of towns, 1989, P1
% L'aire du quadrilatère est $\frac12 AC\cdot BD$, et si on note $M$ le milieu de $BD$ et $H$ l'intersection des diagonales, on a $|ACD|=AC\cdot HD $ et $|AOC|=\frac12HM\cdot AC$, puis on somme.
Soit $ABCD$ un quadrilatère cyclique. On note $O$ le centre de son cercle circonscrit. On suppose de plus que les diagonales de ce quadrilatère sont perpendiculaires. Montrer que la ligne brisée $AOC$ (composée des segments $[AO]$ et $[OC]$) coupe le quadrilatère $ABCD$ en deux portions d'aires égales.