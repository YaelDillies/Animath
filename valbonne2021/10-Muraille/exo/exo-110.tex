%Martin Rakovsky
%On considère $t_1$ et $t_2$ les solutions de l'équation $X^2-m(n+1)X-mn=0$. On obtient que $m$ divise $t_1+t_2, t_1t_2$ et on déduit plus généralement que $m$ divise $t_1^\ell+t_2^\ell$.
%Ensuite, si $t_1>t_2$, on peut montrer que $t_1\in ]-1, 0[$. Si bien que
%$$\lfloor t_2^{2k+1} \rfloor = t_1^{2k+1}+t_2^{2k+1} \equiv 0 \mod m$$
%$$\lfloor t_2^{2k+1} \rfloor = t_1^{2k+1}+t_2^{2k+1}-1 \equiv -1 \mod m$$
Soit $m\ge 2$ un entier. Montrer qu'il existe une infinité de nombres irrationnels $x$ tels que $\lfloor x^{2k+1}\rfloor \equiv 0 \mod m$ et $\lfloor x^{2k} \rfloor \equiv -1 \mod m$ pour tout entier $k$.