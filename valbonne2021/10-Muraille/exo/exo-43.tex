%Victor Vermès
% https://images.math.cnrs.fr/Algebriser.html
% Pour un test
On fixe $n$ un entier pair. Étant donné un sous-ensemble $S$ des entiers $\{0, 1, \dots, n - 1\}$, on va définir $D(S)$ l'ensemble des différences d'éléments de $S$, prises modulo $n$ et regardées avec leurs multiplicité. Par exemple, si $n = 6$, et $S = \{1, 2, 5\}$, alors $D(S) = \{0, 0, 0, 1, 2, 3, 3, 4, 5\}$.\\

On note $\overline S = \{0, 1, \dots, n - 1\} \backslash S$ le complémentaire de $S$. \\
Montrer que si $S$ a exactement $\frac n2$ éléments, alors $D(S) = D(\overline S)$.