%Martin Rakovsky
%ELMO SL 2012 G3
%\begin{sol}
%Voici la première solution venue à mon esprit dérangé :
%Soient $E$ et $F$ les autres points de contact du cercle inscrit. Soit $X$ le centre de la similitude envoyant $E$ sur $C$ et $F$ sur $B$. Notons que $A, X, E, F, I, P$ sont cocycliques. On a par des rapports
%$$\frac{XB}{XC}=\widehat{FB}{EC}=\widehat{DB}{DC}$$
%Donc $\widehat{BXD}=\widehat{DXC}$. Il reste donc à montrer que $P$ est sur le $X$-cercle d'Appolonius de $BXC$. Soit ensuite $Y$ le point d'intersection de $(EF)$ avec $(BC)$. Ménélaüs nous dit que
%$$\frac{YB}{YC}=\frac{EA}{EC}\cdot \frac{FB}{FA}=\frac{DB}{DC}$$
%Il reste donc à montrer que $D, P, X$ et $Y$ sont cocycliques. Notons que $YD^2=YE\cdot YF$ et $IE^2=ID^2$ donc $I$ et $Y$ sont sur l'axe radical du cercle inscrit et du cercle de centre $A$ de rayon $AE$. Donc les droites $(IY)$ et $(AD)$ sont perpendiculaires, donc $I, P$ et $Y$ sont alignés. Donc $\widehat{DPY}=90^\circ$ donc $P$ appartient au cercle de diamètre $[DY]$, ie le cercle d'Appolonius annoncé, c'est gagné.
%\end{sol}
Soit $ABC$ un triangle, $I$ le centre de son cercle inscrit. Soit $D$ le projeté orthogonal de $I$ sur $[BC]$ et $P$ le projeté orthogonal de $I$ sur $(AD)$. Montrer que $\widehat{BPD} = \widehat{CPD}$.