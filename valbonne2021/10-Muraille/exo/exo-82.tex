%Martin Rakovsky
%Canada MO 2018 P1
%\begin{sol}
%Si $n=2^k$, on procède par récurrence sur l'exposant $k$. Si c'est possible pour $k$, alors dans le cas $k+1$, on partitionne en $2^k$ paires de jetons. Puis on itère (l'initialisation étant triviale).
%\medskip
%Si $n\ne 2^k$, on prend la configu avec $n-1$ jetons en $(0, 0)$, puis le dernier en $(1, 0)$. Le barycentre est $(1/n, 0)$ et est invariant. Les abscisses des jetons seront toujours des nombres dyadiques, et $1/n$ n'est pas un nombre dyadique. Dommage.
%\end{sol}
On considère une configuration de $n$ jetons dans le plan, pas forcément à des positions distinctes. On s'autorise l'opération suivante : on choisit deux jetons $A$ et $B$ et on les déplace tous les deux vers l'emplacement du milieu du segment joignant les jetons $A$ et $B$. On dit qu'une configuration \textit{collisionne} s'il est possible, après une suite finie d'opérations, que tous les jetons se retrouvent au même emplacement. Montrer que toutes les configurations à $n$ jetons collisionnent si et seulement si $n$ est une puissance de $2$.