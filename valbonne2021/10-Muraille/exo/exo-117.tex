%Théodore Fougereux
%Olympiades Australienne 2020, P7
%https://artofproblemsolving.com/community/c6h2149323p15841722
On appelle tétramino une pièce formée de quatre carré $1\times 1$ collés par leurs côtés. On considère une salle de bain en forme de rectangle de dimensions $2\times 2n$, et on note $T_n$ le nombre de manières de paver cette salle de bain avec des tétraminos. Montrer que $T_n$ est toujours un carré parfait.

\textit{Par exemple, pour $n=2$, on a $T_2=4$ car on peut réaliser les $4$ pavages suivants:}
\begin{center}
\begin{tikzpicture}[line cap=round,line join=round,>=triangle 45,x=1cm,y=1cm,scale=0.5]
\clip(-13,-4) rectangle (11,0);
\draw [line width=2pt] (-12,-1)-- (-12,-3);
\draw [line width=2pt] (-12,-3)-- (-8,-3);
\draw [line width=2pt] (-8,-3)-- (-8,-1);
\draw [line width=2pt] (-8,-1)-- (-12,-1);
\draw [line width=2pt] (-6,-1)-- (-6,-3);
\draw [line width=2pt] (-6,-3)-- (-2,-3);
\draw [line width=2pt] (-2,-3)-- (-2,-1);
\draw [line width=2pt] (-2,-1)-- (-6,-1);
\draw [line width=2pt] (0,-1)-- (0,-3);
\draw [line width=2pt] (0,-3)-- (4,-3);
\draw [line width=2pt] (4,-3)-- (4,-1);
\draw [line width=2pt] (4,-1)-- (0,-1);
\draw [line width=2pt] (6,-1)-- (6,-3);
\draw [line width=2pt] (6,-3)-- (10,-3);
\draw [line width=2pt] (10,-3)-- (10,-1);
\draw [line width=2pt] (10,-1)-- (6,-1);
\draw [line width=2pt] (-12,-2)-- (-8,-2);
\draw [line width=2pt] (-5,-1)-- (-5,-2);
\draw [line width=2pt] (-5,-2)-- (-3,-2);
\draw [line width=2pt] (-3,-2)-- (-3,-3);
\draw [line width=2pt] (1,-3)-- (1,-2);
\draw [line width=2pt] (1,-2)-- (3,-2);
\draw [line width=2pt] (3,-2)-- (3,-1);
\draw [line width=2pt] (8,-1)-- (8,-3);
\end{tikzpicture}\\
\end{center}