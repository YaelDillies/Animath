% Raphaël Ducatez
% Réponse : oui
% La stratégie est la suivante : On prend un entier $N$ (fixé par la suite, mais disons une puissance de $2$ pour que les quantités suivantes soient définies). Après $N$ coups de Raphaël, la bille se trouvera dans une case de coordonnées $(N, k_0)$ avec $k_0\in \llbracket -2021\cdot N, 2021 \cdot N\rrbracket$. Pendant les $N/2$ premiers coups, Yaël bloque une case sur $4096$ parmi ces cases (car $4096\ge 2\cdot 2021+1$). Lorsque la bille arrive sur une case de la forme $(N/2, y_1)$, on sait qu'elle arrivera sur une case de la forme $(N, y_1+k_1)$ avec $k_1\in \llbracket -2021\cdot N/2, 2021 \cdot N/2\rrbracket$, pendant les $N/4$ prochains coups, on peut bloquer une case sur $4096$ parmi ces cases, qui n'ont pas déjà été bloquées. En répétant ce processus, on peut arriver à bloquer la case d'arrivée après $N$ coups, en prenant $N\ge 2^{4096}$ par exemple.
Raphaël et Yaël jouent à un jeu sur un échiquier infini, chaque case étant repérée par deux coordonnées entières $(x, y)$. Initialement Raphaël pose une bille sur la case de coordonnées $(0, 0)$. À chaque coup, si la bille se trouve sur un point de coordonnées $(x, y)$, Raphaël peut la déplacer sur une case de coordonnées $(x + 1, y + k)$ avec $k\in \llbracket - 2021, 2021 \rrbracket$. Pour l'en empêcher, à chaque fois que Raphaël déplace sa bille, Yaël peut choisir de \textit{bloquer} une case, qui ne sera plus jamais accessible pour Raphaël. Yaël peut-il faire en sorte que Raphaël ne puisse plus déplacer sa bille ?