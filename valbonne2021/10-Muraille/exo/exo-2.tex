%Martin Rakovsky
%Pan African MO 2019 P2
%\begin{sol}
%Manifestement $2$ et $5$ sont dans la liste. Donc on peut simplifier $p_3+\dots + p_k +7 = p_3\cdot p_k$. Si $p_k$ est le plus grand nombre premier, on a
%$$(k-2)p_k+7 \ge p_3+\dots + p_k+7 = p_3\cdot p_k \ge 2^{k-3}p_k$$
%On déduit que $7 \ge (2^{k-3}-(k-2))p_k \ge 2(2^{k-3}-(k-2))$. Cela impose que $k\le 5$. Puisque par ailleurs $k=3$ ne convient pas, on a $k = 4, 5$.
%Réciproquement, pour $k=4$, on prend $\{2, 3, 5, 5\}$ qui convient. Pour $k=5$, il faut résoudre l'équation $a + b + c + 7 = abc$.
%Or si $a, b, c$ sont impairs, le LHS est pair mais pas le RHS. Si $a=2$ et les autres sont impaires, on a un pb. Donc il faut $a=b=2$ (on ne peut avoir $c=2$). Mais alors $3c=11$ ce qui n'est pas possible. Donc $k\ne 5$.
%\end{sol}
Soit $k$ un entier strictement positif. On considère $k$ nombres premiers (pas forcément distincts) tels que leur produit vaut $10$ fois leur somme. Déterminer toutes les valeurs de $k$ possibles.