%Théodore Fougereux
%Albania TST 2009 P4
% On factorise en $(n-1)(n+1)=5\cdot 2^m$, $n$ doit être impair ($m=0$ ne marche pas), disons $n=2k+1$, on a alors $k(k+1)=5\cdot 2^{m-2}$, comme $k$ et $k+1$ sont premiers entre eux, on a $2^{m-2}$ divise $k$ ou $k+1$. Donc $k=5\cdot 2^{m-2}, k=2^{m-2}, k+1=5\cdot 2^{m-2}$ ou $k+1=2^{m-2}$ et on conclut suivant les cas.
Trouver les entiers $m, n\in \N$ vérifiant :
$$1+5\cdot 2^m=n^2$$