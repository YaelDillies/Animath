%Martin Rakovsky
%Puisque la droite $(FG)$ est tangente au cercle $(BCH)$, $\widehat{CGF}=180^\circ - \widehat{HCA}-\widehat{GHC}=180^\circ -\widehat{HBC}-\widehat{HBA}= \widehat{ABC}$ donc les points $F, G, C$ et $B$ sont cocycliques.
%Soit $J$ le point d'intersection des droites $(XG)$ et $(YF)$. Alors
%$$\widehat{XJY}=180^\circ -\widehat{JGF}-\widehat{JFG}=\widehat{HGX}-\widehat{HAY}=\widehat{XAY}$$ donc $J$ appartient au cercle $ABC$.
%Il suffit dès lors de montrer que les points $J, H$ et $M$ sont alignés.
%On reconnaît en $M$ le point de Miquel du quadrilatère $AGCBEF$. Etant donné que $FGBC$ est cyclique, $M$ est sur $[AE]$.
%Par puissance d'un point, $EH^2=EB\cdot EC=EF\cdot EG= EM\cdot EA$ donc la droite $(FG)$ est tangente au cercle $AHM$. Il vient que $\widehat{AMH}=\widehat{AHG}=\widehat{AXG}=\widehat{AXJ}=\widehat{AMJ}$ ce qui donne bien que les points $J, M$ et $H$ sont alignés.
Soit $ABC$ un triangle, $H$ son orthocentre. La tangente au cercle circonscrit au triangle $CBH$ en $H$ coupe $[AC]$ en $G$, $[AB]$ en $F$ et $(BC)$ en $E$. Soit $M$ le point d'intersection des cercles circonscrits aux triangles $ABC$ et $AFG$. Soit $X$ le point d'intersection des cercles circonscrits aux triangles $ABC$ et $AGH$. Soit $Y$ le point d'intersection des cercles circonscrits aux triangles $ABC$ et $AFH$. Montrer que les droites $(YF), (XG)$ et $(HM)$ sont concourantes en un point du cercle circonscrit au triangle $ABC$.