% Théodore Fougereux
%TOT 2013 A level
% Réponse : $2$, si il y en a au moins $3$ disons $a<b<c$, supposons $a+b=2^m, a+c=2^n, b+c=2^p$, alors $a+b=2^m<a+c=2^n<b+c=2^p<2a+b+c=2^m+2^n<2^{n+1}$, donc $n<p<n+1$, absurde.
$N$ nombres entiers distincts strictement positifs sont écrits au tableau. Si $a$ et $b$ sont deux entiers distincts écrits au tableau, $a+b$ est une puissance de $2$. Trouver la valeur maximale possible pour $N$.