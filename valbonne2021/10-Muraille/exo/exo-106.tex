%Martin Rakovsky
%Iran 2020 round 2 P5
%Si $n$ est pair, on peut répartir les entiers en $n$ paires parfaites de la forme $(\ell, \ell+2)$.
%Dans l'autre sens, remarquons qu'un entier $a=2, 6\mod 8$ doit être rangé avec un entier $b=0, 4 \mod 8$. Donc le nombre d'entiers congrus à $2$ ou 6 $\mod 8$ doit être inférieur au nombre d'entiers congrus à $0, 4$. Cela impose que $n$ soit pair.
On dit qu'une paire d'entiers $(a, b)$ est \textit{parfaite} si $ab+1$ est un carré parfait.

Pour quels entiers $n$ est-il possible de diviser l'ensemble $\{1, \dots, 2n\}$ en $n$ paires parfaites ?