%Martin Rakovsky
%Balkan MO SL 2016 G1
%\begin{sol}
%Soit $X$ le point d'intersection de $(BE)$ avec $(AEF)$. Soit $M$ le point d'intersection de $(CEX)$ avec $(BC)$. Le théorème de Miquel nous assure que $FBMX$ est cyclique. Par puissance d'un point on a ensuite :
%$$CE\times CA = BC^2-BF\times BA = BC^2-BC\times BD = BC\times CD$$
%donc $C$ est sur l'axe radical de $(AEF)$ et $(BFX)$, si bien que $C, F$ et $X$ sont alignés. Soit désormais $P$ le second point d'intersection de $(AEF)$ et $(BXC)$. $\widehat{BPC}=\widehat{BXC}=\widehat{EXF}=180^\circ-\widehat{BAC}$ donc $P$ est sur un cercle fixe. Puis
%$$\widehat{PAC}=\widehat{BAC}-\widehat{BAP}=180^\circ-\widehat{BXC}-\widehat{FXP}= \widehat{PXB}=\widehat{PCB}$$
%donc $P$ est sur le cercle passant apr $A$ et $C$ tangent à $(BC)$. est donc l'intersection d'une droite fixe, d'un cercle fixe, il y a deux positions possibles, l'une est $C$, donc $P$ est fixe et c'est gagné.
%\end{sol}
Soit $ABC$ un triangle. Soient $E$ et $F$ des points sur les segments $[AB]$ et $[AC]$ tels que
$$BC^2 = BA\times BF + CA \times CE$$

Montrer que pour tout tel choix de $E$ et $F$, le cercle circonscrit au triangle $AEF$ passe par un point fixe.