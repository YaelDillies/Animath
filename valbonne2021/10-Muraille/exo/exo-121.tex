%Martin Rakovsky
%Dutch IMO TST 2016
%\begin{sol}
%La droite recherchée est l'axe radical des deux cercles. On va montrer que $T, S$ et $N$ sont alignés. Si c'est le cas, puisque $(AQ)$ et $(ST)$ sont perpendiculaires, on aura $NS\cdot NT = NT^2 = AN\cdot NQ$ (relations d'Euclide dans le triangle $ATQ$ rectangle en $T$, avec $N$ pied de la hauteur issue de $T$). On aura donc gagné.
%\medskip
%On veut donc montrer que $T, S$ et $N$ sont alignés, ou plutôt que $T, M$ et $S$ sont alignés. On remarque alors que les points $A, S, P$ et $M$ sont cocycliques ($\widehat{ASP}=90^\circ=\widehat{AMP}$). De même pour $A, S, T$ et $Q$.
%Soit $F$ le point de $\Gamma_2$ tel que $[AF]$ est un diamètre de $\Gamma_2$.
%On a donc
%$$\widehat{MSP}=\widehat{MAP}=\widehat{FAP}=180^\circ-\widehat{PQF}$$
%donc $(SM)$ et $(QF)$ sont parallèles.
%On a aussi
%$$\widehat{TSQ}=\widehat{TAQ}=90^\circ-\widehat{AQT} = 90^\circ-\widehat{PQA}=90^\circ-\widehat{PFA}=\widehat{PAF}=180^\circ-\widehat{PQF}$$
%et les droites $(ST)$ et $(QF)$ sont parallèles aussi.
%Donc on a l'alignement et le résultat.
%\end{sol}
Soit $\Gamma_1$ un cercle de centre $A$ et $\Gamma_2$ un cercle de centre $B$ passant par $A$. Soit $P$ un point variable sur $\Gamma_2$. Une tangente à $\Gamma_1$ issue de $P$ touche $\Gamma_1$ en $S$ et recoupe $\Gamma_2$ en $Q$, et on suppose que $S, P$ et $Q$ sont du même côté de $[AB]$. Une tangente à $\Gamma_1$ différente de la droite $(SP)$ issue de $Q$ touche $\Gamma_1$ en $T$. Soit $M$ le pied de la hauteur issue de $P$ dans le triangle $APB$. Soit $N$ le point d'intersection des droite $(TM)$ et $(AQ)$. Montrer que lorsque le point $P$ varie sur le cercle $\Gamma_2$, le point $N$ reste sur une droite ne dépendant pas du point $P$.