%Martin Rakovsky
%Taiwan 2021 round 2
%\begin{sol}
%On effectue une involution de centre $A$ fixant $(ABC)$. Tous les cercles deviennent des droites, et en faisant un dessin on tombe sur une configuration de projective qui nous indique que les points $B^\star, C^\star, M^\star$ et $X^\star$ sont harmoniques. Donc en revenant à la figure initiale, cela veut dire que $B, C, X$ et $M$ sont harmoniques. On projette depuis $S$ sur la droite $BC$, le résultat à démontrer est alors équivalent au fait que $(XS)$ coupe $[BC]$ en son milieu.
%\medskip
%Le fait que $B, C, X, M$ soient harmoniques implique que $M$ est sur la symédiane issue de $X$ dans $BCX$. On note $S'$ le point d'intersection de la médiane issue de $X$ dans $BCX$ avec le cercle $(BCX)$. On montrer que $N$ est sur $(XS')$, ce qui concluera. Pour cela, il suffit de montrer que $N, S'$ et le milieu de $[BC]$ sont alignés. Il suffit pour cela de montrer que $BNCS'$ est un parallélogramme. Or
%$$\widehat{NCB}=\widehat{MCB}=\widehat{CBS'}$$
%car $BCMS'$ est un trapèze isocèle. On a de même $\widehat{NBC}=\widehat{S'CB}$, donc on a gagné.
% \end{sol}
Soit $ABC$ un triangle, $E$ un point sur le segment $[AC]$ et $F$ un point sur le segment $[AB]$. Les cercles circonscrits aux triangles $ABC$ et $AEF$ se recoupent au point $X$. Les cercles circonscrits aux triangles $AEB$ et $AFC$ se recoupent au point $K$. La droite $(AK)$ recoupe le cercle circonscrit au triangle $ABC$ au point $M$. Soit $N$ le symétrique du point $M$ par rapport au segment $[BC]$. La droite $(XN)$ recoupe le cercle circonscrit au triangle $ABC$ au point $S$. Montrer que les droites $(BC)$ et $(SM)$ sont parallèles.