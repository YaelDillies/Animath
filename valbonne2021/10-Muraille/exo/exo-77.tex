%Théodore Fougereux
Théo et Paul jouent à un jeu. Théo a face à lui $n$ enveloppes indistinguables et fermées. L'une d'elles contient $n$ roubles. Théo choisit une enveloppe. Pour aider Théo, Paul ouvre une enveloppe (autre que celle de Théo) ne contenant rien s'il y a au moins $3$ enveloppes non ouvertes sur la table. Théo peut alors choisir soit d'ouvrir son enveloppe, soit de recommencer le même processus avec une nouvelle enveloppe (potentiellement la même) mais en enlevant $1$ rouble de l'enveloppe contenant de l'argent. Soit $E_n$ l'espérance du gain de Théo en jouant optimalement.

Calculer $\lim_{n\to \infty}(E_n)$.