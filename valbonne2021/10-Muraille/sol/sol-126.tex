On sait que $(ST)$ est la polaire de $A$ par rapport à $\mathcal{C}$. Comme $P$ est sur la polaire et sur $(AC)$, on obtient par définition de la polaire :
$$(A, P; B, C)=-1 $$
Il suffit donc de montrer que les points $(A, P; B, C)$ sont harmoniques et on aura
$$Ap\cdot BC = 2AB\cdot PC $$
Comme $A, P; B, C$ sont harmoniques, on peut supposer que $A=-1, P=1, B=x, C=\frac{1}{x}$ avec $1>x>0$.
\\
On souhaite démontrer que $AP\cdot BC=2Ab\cdot PC$ avec $AP=2, BC=\frac{1}{x}-x$, $AB=x+1$, $DC=\frac{1}{x}-1$. On veut donc montrer que
$$2\left(\frac{1}{x}-x\right)=2(x+1)\left(\frac{1}{x}-1\right) \iff \left(\frac{1}{x}-x\right)=\left(1+\frac{1}{x}-x-1\right) $$
Et donc
$$2\left(\frac{1}{x}-x\right)=2(x+1)\left(\frac{1}{x}-1\right) \iff \frac{1}{x}-x=\frac{1}{x}-1$$
Donc on a bien montré que
$$Ap\cdot BC=2AB\cdot PC $$
