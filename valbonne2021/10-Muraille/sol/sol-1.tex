On note $O_1,O_2,O_3,O_4$ les centres respectifs des quatre cercles de l'énoncé.

Alors $O_1P=O_1X$ et $O_2P=O_2X$ donc $(O_1O_2)$ est la médiatrice de $[PX]$. C'est aussi celle de $[AB]$ et $[CD]$ par définition du cercle circonscrit.

On a de même $(O_3O_4)$ médiatrice commune à $[PY],[AD]$ et $[BC]$.

$(PY)$ est alors perpendiculaire à $(O_3O_4)$ donc parallèle à $(AD)$, donc perpendiculaire à $(PX)$. 

Définissons alors $Q$ le point tel que $PYQX$ est un rectangle. Alors le centre de $PYQX$ est l'intersection des médiatrices dans le rectangle, donc celui de $ABCD$. Comme c'est aussi l'intersection des diagonales, le centre de $ABCD$ est sur la droite $(XY)$.