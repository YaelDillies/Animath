Sans perte de généralité, $d = 1$. Montrons d'abord que n'importe quels trois points de $A, B, C \in T$ sont inclus dans un cercle de rayon $\frac 1{\sqrt3}$. Sans perte de généralité, $AB$ est le plus grand côté du triangle $ABC$. Sans perte de généralité, $AB=1$, quitte à faire une homothétie. Soit $C'$ le point du même côté de $(AB)$ que $C$ tel que $ABC'$ soit équilatéral. Le cercle $\Omega$ circonscrit à $ABC$ est de rayon $\frac{AB}{2\sin(60)}=\frac 1{\sqrt3}$ par la loi des sinus. Il est alors suffisant de montrer que $C$ est à l'intérieur de $\Omega$. Sans perte de généralité, $C$ est du même côté que $B$ de la hauteur de $ABC'$ issue de de $C'$. Comme $AC\leqslant AB=1$, $C$ est à l'intérieur du cercle de centre $A$ et de rayon $1$, qui passe par $A$ et $ C'$, et donc à l'intérieur de $\Omega$.

Maintenant, pour tout point $P\in T$, on trace le cercle $\Gamma_A$ de centre $A$ et de rayon $\frac 1{\sqrt 3}$. Pour tous trois points $A, B, C \in T$, $\Gamma_A, \Gamma_B, \Gamma_C$ ont une intersection non vide car le centre du cercle de rayon $\frac 1{\sqrt 3}$ contenant $A, B, C$ est à distance au plus $\frac 1{\sqrt 3}$ de $A, B, C$ et donc dans cette intersection. Donc, par le théorème de Helly, et puisque les cercles sont convexes, l'ensemble des cercles $\Gamma_P$ avec $P\in T$ a une intersection non vide. Si on prend alors $O$ un point dans cette intersection et on trace le cercle de centre $O$ et de rayon $\frac 1{\sqrt 3}$, tout point de $T$ est à distance au plus $\frac 1{\sqrt 3}$ de $O$ et donc à l'intérieur de ce cercle, ce qui conclut.