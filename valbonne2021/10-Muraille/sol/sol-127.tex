Soit $K$ l'intersection de $(EF)$ et de $(BC)$. Comme $(EF)$ est la polaire de $A$, $A$ se trouve sur la polaire de $K$. Or $D$ se trouve aussi sur la polaire de $K$. Donc $(AD)$ est la polaire de $K$, donc $(AD)$ et $(IK)$ sont perpendiculaires. Ainsi les points $I, P, K$ sont alignés.

Comme $\widehat{KPD}$ est droit, et qu'il faut prouver que $\widehat{BPD} = \widehat{CPD}$, il suffit de montrer que $K, D, B, C$ sont harmoniques. On note $G$ le point de Gergonne de $ABC$ et on considère le quadrilatère complet $AEGFBC$. On a que $K$ est l'intersection de $(EF)$ et $(BC)$, que $D$ est l'intersection de $(AG)$ et $(BC)$, donc $K,D,B,C$ sont harmoniques ce qui conclut.