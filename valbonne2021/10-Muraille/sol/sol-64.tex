On note $|XYZ|$ l'aire du triangle $XYZ$.

Si $Q, A, B$ sont alignés dans cet ordre, alors on a $|QCB| = |QCA| + |ABC|$, d'où $|QCA| = 0$. On suppose donc que $A$,$B$, $Q$ sont alignés dans cet ordre.

Les triangles $ABC$ et $CBQ$ partagent la hauteur issue de $C$ et ils ont la même aire, ils ont donc la même base, d'où $B$ est le milieu de $AQ$. Soit $K$ le milieu de $CQ$. Par Thalès, $AC$ est parallèle à $BK$. Par angle tangent, $\widehat{CAB} = \widehat{BCK}$ et par angles alternes-internes, on a $\widehat{ACB} = \widehat{KBC}$, donc les triangles $ABC$ et $CKB$ sont semblables. Les triangles $APC$ et $ACQ$ partagent la hauteur issue de $A$ et on a $|ACQ| = |ABC| + |CBQ| = 2|APC|$ d'où $CQ = 2PC$ soit $PC = CK = KQ$. Soit $L$ le pied de la hauteur de $PCA$ issue de $P$. On a $LC = \frac12 AC$ donc par Thalès $LC = BK$. Par angles alternes-externes, $\widehat{PCL} = \widehat{CKB}$. Donc les triangles $PCL$ et $CKB$ sont isométriques. $ABC$ étant semblable à $CKB$ et $PCL$ étant rectangle par la définition de $L$, ces trois triangles sont rectangles, ce qui conclut.