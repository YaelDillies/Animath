Dans un sens, supposons que $AB = BC$. La symétrie axiale le long de la médiatrice de $[BC]$ échange $A$ et $C$ et fixe $I$. Ainsi, elle échange $AI$ et $CI$ et donc aussi $E$ et $D$. Donc $BE = BD$.

Dans l'autre sens, supposons que $BE = BD$. On a, par Pythagore, $EI = \sqrt{BI^2 - BE^2} = \sqrt{BI^2 - BD^2} = DI$. Donc les triangles $BEI$ et $BDI$ sont isométriques. En particulier, $\widehat{BIE} = \widehat{BID}$. Soient $E'$ et $D'$ les intersections respectives de $(BA)$ avec $(IE)$ et $(BC)$ avec $(ID)$. On a $\widehat{IBE'} = \widehat{IBD'}$ car $I$ est un centre inscrit et donc $(BI)$ est une bissectrice. Les triangles $BIE'$ et $BID'$  ont donc un côté et les deux angles adjacents en commun et sont isométriques. Donc $\widehat{BE'I} = \widehat{BD'I}$ et donc $\widehat{IE'A} = \widehat{ID'C}$ par angles complémentaires. De plus, $\widehat{D'IC} = \widehat{E'IA}$ car opposés par sommets. Donc les triangles $E'IA$ et $D'IC$ sont semblables, d'où $\widehat{E'AI} = \widehat{D'CI}$. Finalement, on a
$$\widehat{BAC} = 2\widehat{E'AI} = 2\widehat{D'CI} = \widehat{ACB}$$
Ainsi, $ABC$ est isocèle et on conclut $AB = BC$.