Soit $ u : \N \rightarrow \N^*$ une suite de naturels non nuls. On dit qu'un indice $i$ de cette suite est sympa si $u_1 + \dots + u_i$ et $u_1 + \dots + u_{i + 1}$ sont premiers entre eux.

Comme la divisibilité est transitive, il est nécessaire et suffisant de trouver $P : i \rightarrow \N^*$ tel que $S_k(P_i) | S_k(P_{i + 1})$ pour tous $i, k$. On montre qu'il existe une permutation de $\N^*$ sympa et que pour toute permutation $P_i$ sympa, on peut construire une permutation $P_{i + 1}$ sympa telle que $S_k(P_i) | S_k(P_{i + 1})$ pour tous $i, k$.

Premièrement, trouvons une permutation $a_1, a_2, \dots$ de $\N^*$ sympa. On la définit par récurrence. $a_1 = 1$. Si $i$ est pair, on prend $a_{i + 1} = p$ un premier $\ne a_1, \dots, a_i$ assez grand pour que $\gcd(a_1 + \dots + a_i, a_i + \dots + a_i + a_{i + 1}) = \gcd(a_1 + \dots + a_i, p) = 1$ d'où $i$ est un indice sympa. Si $i \ge i$ est impair, on prend $a_{i + 1} = \min (\N^* \setminus \{a_1, \dots, a_i\})$, d'où la suite est une permutation de $\N^*$.

Deuxièmement, soit $a_1, a_2, \dots$ une suite sympa. Construisons une suite $ b_1, b_2, \dots$ sympa telle que $a_1 + \dots + a_i | b_1 + \dots + b_i$ pour tout $i$. On construit la suite $b$ pa récurrence. On prend $b_1 = a_1$. Puis, pour construire $b_i$, si $i$ n'est pas sympa dans la suite $a$, on prend $b_i$ tel que $b_1 + \dots + b_i \equiv 0 \pmod{a_1 + \dots + a_i}$ et $b_i\ne b_1, \dots, b_{i-1}$. Si $i$ est un indice sympa dans $a$, on va construire à la fois $b_i$ et $b_{i + 1}$. Soit $A = a_1 + \dots + a_i$ et $A' = a_1 + \dots + a_{i + 1}$, $A$ et $A'$ sont premiers entre eux car $i$ est un indice sympa. On va montrer que si $x\ne b_1, b_2, \dots a_{i-1}$, on peut trouver $b_i, b_{i + 1}$ tels que toutes les conditions de divisibilité soient respectées et $b_{i + 1} = x$. Il suffira alors de prendre $x = \min(\N^*\setminus \{b_1, \dots, b_{i-1})$ une infinité de fois afin que la suite $b$ soit une permutation de $\N^*$ et $x$ un premier assez grand pour que $i$ soit un indice sympa dans la suite $b$ une infinité de fois afin que la suite $b$ soit sympa. Si on fixe $b_{i + 1} = x$, il est suffisant de trouver $b_i$ tel que $b_1 + \dots + b_i\equiv 0 \pmod A$ et $b_1 + \dots + b_i + x\equiv 0\pmod{A'}$. Il existe bien par le théorème des restes chinois et on le prend assez grand pour qu'il soit différent de $a_1, \dots, a_{i - 1}, x$.