Deux nombres sont premiers entre eux si leur seul diviseur commun positif est $1$, ou encore s'ils ne possèdent pas de diviseurs premiers en commun.

Pour étudier nos $10$ nombres consécutifs, il suffit d'étudier les nombres premiers $2,3,5,7$ car si l'on considère un autre nombre premier $p$, on a $p>10$ et si l'un des $10$ nombres est divisible par $p$, aucun autre ne peut l'être.

Maintenant, parmi les $10$ nombres, il y en a exactement $5$ pairs qui ne peuvent donc pas être le nombre cherché. Il reste donc $5$ candidats qui peuvent marcher. Ensuite, parmi les $10$ nombres, on en a au plus $4$ divisibles par $3$. Mais un multiple de $3$ sur deux est pair, donc il y a au plus $2$ multiples de $3$ impairs parmi les $10$ nombres. Ces nombres ne peuvent pas non plus être premiers avec tous les autres donc il nous reste encore au moins $3$ candidats. Parmi les $10$ nombres, on a aussi $2$ multiples de $5$, dont au plus $1$ impair, donc il reste encore au moins $2$. Enfin, ces $2$ candidats ne peuvent pas être tous les deux multiples de $7$ car sinon un serait pair. Il y a donc un nombre parmi les $10$ qui n'est pas divisible par $2,3,5$ ou $7$, et qui est donc premier avec tous les autres.