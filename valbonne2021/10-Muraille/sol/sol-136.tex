On voit que l'identité est une solution.

Soit $f$ une solution : $\forall x, y > 0,~f(xf(y)) + f(y^3f(x)) = xy + xy^3$.

On fait les substitutions
\begin{itemize}
\item $x = y = 1 : f(f(1)) = 1$
\item $x = f(1), y = 1 : f(f(1)^2) = f(1)$
\item $x = f(1)^2, y = 1 : f(f(1)^3) = 2f(1)^2 - 1$
\item $x = y = f(1) : f(1)^2 = f(1)^4$ donc $f(1) = 1$.
\end{itemize}

On a donc, en substituant $y = 1$, $\forall x > 1, ~f(x) + f(f(x)) = 2x$.

On montre alors par récurrence que si $x > 0$ est fixé, on a
$$\forall n\in\N,~f^n(x) = \frac{2 + (-2)^n}3 x + \frac{1 - (-2)^n}3 f(x)$$
où $f^n$ est la $n$-ième itérée de $f$.
\medskip

Ainsi, puisque $f > 0$,
$$0 < f^{2n}(x) = \frac{2^{2n} + 2}3 x - \frac{2^{2n} - 1}3 f(x) \text{ et } 0 < f^{2n + 1}(x) = \frac{2^{2n + 1} + 1}3 f(x) - \frac{2^{2n + 1} - 2}3 x$$
Donc pour tout $n \ge 1$, on a
$$\frac{2^{2n + 1}-2}{2^{2n + 1} + 1} < \frac{f(x)}x < \frac{2^{2n} + 2}{2^{2n} - 1}$$
En faisant tendre $n$ vers l'infini, on trouve $1 \le \frac{f(x)}x \le 1$ donc $f(x) = x$.