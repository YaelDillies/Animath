
Dans la configuration initiale, il y a $2019$ élèves (sans Alice et Bob) dans le cercle, donc Alice en a un nombre impair à sa gauche et pair ($\ge2$) à sa droite (ou l'inverse). On remarque alors qu'Alice peut se ramener à un nombre impair de voisins à sa gauche et à sa droite : Bob sera alors forcé de laisser à Alice une configuration où d'un côté un nombre pair d'élèves la sépare de Bob et de l'autre un nombre impair d'élèves l'en sépare : ainsi, soit Alice est voisine de Bob et gagne la partie, soit elle peut toujours maintenir un nombre impair d'élèves entre Bob et elle après son tour.

De cette manière, Bob est toujours à distance au moins $1$ d'Alice après qu'elle a joué, donc ne peut pas gagner. En appliquant cette stratégie, Alice est donc certaine d'être gagnante, vu que le jeu a nécessairement un gagnant et se termine en au plus $2020$ tour d'élimination : c'est donc Alice (qui commence) qui dispose d'une stratégie gagnante.