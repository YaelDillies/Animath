Attribuons des points à chaque élève comme suit : les élèves du groupe A valent 1 point, ceux du groupe B valent 2 points, ceux du C 4 points, ceux du D 8 points et les animatheurs 16 points. Ainsi, les élèves valent 15 points en tout dans la situation initiale, et Raphaël cherche à atteindre une situation où les élèves valent au moins 16 points en tout. Montrons que les élèves peuvent toujours empêcher Raphaël d'atteindre son but.

Chaque jour, Raphaël divise les élèves en deux groupes, supposons que les points au sein de chaque groupe soient $a$ et $b$, avec $a\le b$. Si les élèves choisissent d'envoyer à la piscine ceux du second groupe et de promouvoir ceux du premier, la situation à la fin de la journée vaudra $2a$ points, puisque chaque promotion double la valeur de l'élève promu. On est passé d'une situation à $a+b$ points à une situation à $2a \le a + b$ points.

Ainsi, les élèves peuvent toujours faire leur choix de telle sorte que le nombre de points dans la situation n'augmente jamais, rendant la tâche de Raphaël impossible.