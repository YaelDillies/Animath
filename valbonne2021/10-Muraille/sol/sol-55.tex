On place dans le plan un premier point $P$, on se donne son projeté orthogonal $H$ sur une droite $d$ telle que $P\notin d$ et $PH$ est entier.

Il suffit alors pour placer ces points de construire $p = \left\lceil\frac n2 - 1\right\rceil$ triangles rectangles de côtés entier dont l'un est $PH$ ($p$ de chaque côté de $(PH)$) : tous les points seront alors à distance entière.

Mais on sait qu'il existe une infinité de triplets pythagoriciens primitifs, donnons-nous en $p$ deux à deux distincts notés $(a_i, b_i, c_i)_{1 \le i \le p}$ avec $a_i^2 + b_i^2 = c_i^2$ pour tout $i$.

On choisit alors $PH = a_1\dots a_p$ et on place les points à distance $\frac{PH\cdot b_i}{a_i}$ de $H$ sur $d$. On a donc bien construit (au moins) $n$ points satisfaisant l'énoncé.