\begin{center}
\small{\textbf{Instructions}}
\end{center}
\begin{small}
Les exercices $1$ à $42$ sont dits de Niveau $1$.\\
Les exercices $43$ à $96$ sont dits de Niveau $2$.\\
Les exercices au-delà de $97$ sont dits de de Niveau $3$.\\
Un exercice est décoré de $n$ étoiles lorsqu’il est resté sans solution à la muraille durant $n$ stages.\\
Les élèves du groupe A cherchent les exercices de Niveau $1$ (ou au-dessus). Les élèves du groupe B cherchent les exercices de Niveau $2$ et les exercices étoilés de niveau Niveau $1$ (ou au-dessus). Les élèves du groupe C cherchent les exercices de Niveau $3$ et les exercices étoilés de niveau Niveau $2$ (ou au-dessus). Les élèves du groupe D cherchent les exercices de Niveau $3$ (mais pas au-dessus, vu qu’il n’y en a pas).\\
- Une fois un exercice résolu, la solution doit être rédigée et donnée à une animathrice ou un animatheur. Le nom de la personne ayant résolu un exercice sera écrit dans le polycopié.\\
- Il est possible de résoudre les exercices à plusieurs, le but est d’avoir tout résolu à la fin du stage !
\end{small}
%\newpage
\begin{center}
\Large{\textbf{Les prix de la muraille}}
\end{center}
À la fin du stage, quatre Grand Prix Mystère seront décernés aux quatre élèves ayant obtenu le plus de points en résolvant des exercices de la Muraille dans chacun des quatre groupes A, B, C, D.\\
À la fin du stage, un autre Grand Prix Mystère sera décerné à l’équipe (constituée d’au moins deux élèves et d’au plus quatre élèves) ayant obtenu le plus de points en résolvant des exercices de la Muraille.\\~~\\
\textbf{Barème} : Un exercice à $n$ étoiles résolu rapporte $n + 1$ points (sauf pour les élèves du groupe B qui résolvent des exercices étoilés de Niveau $1$ et les élèves du groupe C qui résolvent des exercices étoilés de Niveau $2$, pour lesquels un exercice à $n$ étoiles rapporte $n$ points). Dans une équipe, on prend en compte le groupe de l’élève le plus avancé.