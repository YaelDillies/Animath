\textbf{Introduction:} Le cours qui suit est une introduction à la théorie des équations fonctionnelles. Elle aborde les technique les plus courantes comme la substitution et l'étude des propriétés des solutions. Le cours est accompagné d'exercices corrigés de difficulté indiquée par une lettre ($F$ facile, $M$ moyen et $D$ difficile).

\subsubsection{Généralités et définitions}

\emph{Une équation fonctionnelle est une équation dont l'inconnue est une fonction. Nous allons détailler dans ce cours certaines méthodes classiques de résolution d'équations fonctionnelles. Commençons par donner des exemples.}
\begin{ex}
L'équation fonctionnelle $f(x)=f(-x)$ a pour solution les fonctions paires.
\\
L'équation fonctionnelle $f(x)=-f(-x)$ a pour solution les fonctions impaires.
\\
L'équation fonctionnelle $f\circ f(x) = x$ a pour solution les fonctions involutives.
\\
L'équation fonctionnelle $f(x+T)=f(x) $ a pour solution les fonctions $T$-périodiques.
\end{ex}

\begin{rem}
Sur ces trois premiers exemples, l'équation fonctionnelle sert de définition à certaines propriétés remarquables que peut avoir une fonction. Dans ce qui suit, on se donnera des équations fonctionnelles et on cherchera à obtenir une expression des solutions la plus explicite possible.
\end{rem}

\emph{Nous voyons dans un premier temps comment on peut résoudre une équation fonctionnelle uniquement grâce à des manipulations algébriques puis nous verrons comment on peut s'intéresser à certaines propriétés de la solution pour simplifier la résolution.}


\subsubsection{Substitution}

\emph{L'idée est assez simple, comme la relation fonctionnelle est vérifiée pour tout élément de l'ensemble de définition, on peut chercher à fixer des paramètres pour obtenir d'autres équations. On cherchera surtout à simplifier des termes en utilisant des subtitutions du type $x=y$ ou $x=0$ ou $x=-y$. \textbf{Attention:} en procédant ainsi, on raisonne par analyse synthèse, c'est à dire que l'on suppose que $f$ est une solution puis on trouve des conditions nécéssaires que $f$ doit vérifier jusqu'à obtenir un candidat possible de solution de l'équation de départ puis on \textbf{doit} vérifier que ce candidat est effectivement solution. }

La théorie  des équations fonctionnelles étant basée sur la pratique: voyons quelques exemples.
\begin{ex}
Trouver les fonctions $f:\R\to \R$ telle que pour tout $x$ et pour tout $y$ on ait
$$f(x+y)=f(x)+y $$
\end{ex}
\begin{preuve}
\textbf{Analyse :}
On commence par supposer que $f$ est une solution de l'équation.
\\

Comme la relation tient pour toute paire de réels $(x,y)$, on peut chercher à fixer des valeurs de $(x,y)$ pour simplifier la relation. Une idée naturelle est de poser $x=0$ ou $y=0$. Si on pose $y=0$, on obtient $f(x)=f(x)$ ce qui ne nous avance pas. On pose donc $x=0$ ce qui donne 
$$f(y)=y+f(0) $$
$f(0)$ est inconnu mais il est constant, on a donc montré que si $f$ était solution, alors elle doit être de la forme 
$$f(x)=x+a, \: a\in \R $$
\textbf{Synthèse :} Vérifions si ces fonctions sont solution:
$$f(x+y)=f(x)+y\iff x+y+a=x+a+y $$
Donc elles sont bien solutions et les solutions de l'équation de base sont donc les 
$$f(x)=x+a, \: a\in \R $$
\end{preuve}
\\
Voyons un deuxième exemple.
\begin{ex}
Trouver les fonctions $f:\R\to \R$ telle que pour tout $x$ et pour tout $y$ on ait
$$f(x-f(y))=1-x-y $$
\end{ex}

\begin{preuve}
\textbf{Analyse :}
On commence par supposer que $f$ est une solution de l'équation.
\\

Comme la relation tient pour toute paire de réels $(x,y)$, on peut chercher à fixer des valeurs de $(x,y)$ pour simplifier la relation. Ici, c'est l'argument du membre de gauche qui est difficile à appréhender, on va donc le simplifier en posant $x=f(y)$.
\\
L'équation devient alors
$$f(0)=1-f(y)-y\iff f(y)=1-y-f(0) $$
Ainsi, si $f$ est une solution, elle doit être une fonction affine. Plus précisément, elle doit s'écrire 
$$f(y)=-y+a, \; a\in \R $$
\textbf{Synthèse :}
On a trouvé les seuls candidats possibles, il faut vérifier lesquels sont vraiment solutions. On injecte donc la forme $f(y)=-y+a$ dasn l'équation de base et 
$$f(x-f(y))=1-x-y\iff f(x+y-a)=1-x-y\iff -x-y+a+a=1-x-y $$
Et donc on doit forcément avoir
$$a=\frac{1}{2} $$
Finalement, la seule solution de l'équation fonctionnelle est 
$$f(y)=-y+\frac{1}{2} $$
\end{preuve}

Voici quelques exercices à chercher qui repose sur cette idée.
\begin{exo}[F]
Trouver les fonctions $f:\R\to \R$ à pente constante. C'est à dire telle qu'ile xiste un réel $a$ tel que pour chaque paire de réels distincts $x,y$, on ait:
$$\frac{f(x)-f(y)}{x-y}=a $$
\end{exo}

\begin{proof}
\textbf{Analyse :} Supposons que $f$ est solution. On pose $y=0$ ce qui donne pour tout $x$ réel
$$f(x)=ax+f(0) $$
Donc $f$ doit être affine de pente $a$.
\\
\textbf{Synthèse :} Réciproquement, si $f(x)=ax+b$ on vérifie que pour toute paire de réels $x$ et $y$ distincts, on a
$$\frac{f(x)-f(y)}{x-y}=\frac{(ax+b)-(ay+b)}{x-y}=a $$
Donc les fonctions de pente constante sont les fonctions affines.
\end{proof}
\begin{exo}[F]
Trouver les fonctions $f:\R\to \R$ telles que pour toute paire de réels $(x,y)$ on ait
$$f(x)f(y)=f(xy)+x+y $$
\end{exo}
\begin{preuve}
\textbf{Analyse :} Supposons que $f$ est solution. On pose $y=0$ ce qui donne pour tout $x$ réel
$$f(x)f(0)=x+f(0) $$
Pour $x=1$ on a
$$f(1)f(0)=1+f(0) $$ ce qui implique en particulier que $f(0)\neq 0$. On en déduit donc que 
$$f(x)=\frac{x}{f(0)}+1 $$
\textbf{Synthèse :} On considère une fonction $f$ de la forme $f(x)=ax+1$. On veut
$$(ax+1)(ay+1)=axy+1+x+y\iff a^2xy+a(x+y)+1=axy+x+y+1\Rightarrow a=1 $$
Donc la seule solution est $f(x)=x+1$.
\end{preuve}
\begin{exo}[F]
Trouver les fonctions $f:\R\to \R$ telles que pour toute paire de réels $(x,y)$ on ait
$$f(xy)=xf(x)+yf(y) $$

\end{exo}
\begin{preuve}
\textbf{Analyse :} Supposons que $f$ est solution. On pose $y=0$ ce qui donne pour tout $x$ réel
$$f(0)=xf(x) $$
En particulier, pour $x=0$ on en déduit que $f(0)=0$. On a donc que pour tout $x\in \R$
$$0=f(x)x $$ Cela assure que si $x\neq 0$, on a $f(x)=0$.
\\
Comme on avait $f(0)=0$, on a en fait montré que la solution nulle était la seule fonction candidate pour être solution.
\textbf{Synthèse :} Soit $f$ la fonction nulle, on  vérifie que 
$$f(xy)=0=xf(x)+yf(y)=0+0 $$
\end{preuve}
\begin{exo}[M]
Trouver les fonctions $f:\R_*^+\to \R_*^+$ telles que pour toute paire de réels $(x,y)$ on ait
$$f(yf(x))(x+y)=x^2(f(x)+f(y)) $$

\end{exo}
\begin{preuve}
\textbf{Analyse :} Supposons que $f$ est solution. On pose $x=y=1$ ce qui donne 
$$f(f(1))=f(1) $$
On pose alors $x=f(1)$ et $y=1$ ce qui donne 
$$f(f(f(1)))(1+f(1))=f(1)^2(f(f(1))+f(1))\iff f(1)(1+f(1))=2f(1)^2f(1)$$ $$\iff 1+f(1)=2f(1)^2\iff f(1)=1 \textit{ ou } f(1)=-\frac{1}{2} $$
Comme $f(1)>0$, on a $f(1)=1$. De là, on pose $x=1$ ce qui donne pour tout $y$ réel
$$f(y)(1+y)=(1+f(y))\iff f(y)y=1 $$
Donc la seule fonction possiblment solution est $y\mapsto \frac{1}{y}$.
\\
\textbf{Syntèse:}
Considérons la fonction $f$ définie par $f(x)=\frac{1}{x}$. On vérifie que $$f(yf(x))(x+y)=\frac{x}{y}(x+y)=\frac{x^2}{y}+x=x^2\left(\frac{1}{x}+ \frac{1}{y}\right) $$
Et donc la seule la solution est la fonction inverse.
\end{preuve}
\begin{exo}[M]
Trouver les fonctions $f:\R_*\to \R$ vérifiant
$$x \cdot f\left(\frac{x}{2}\right) - f\left(\frac{2}{x}\right) = 1 \ \text{ pour tout } x \in \R_*. $$
\end{exo}
\begin{preuve}
\textbf{Analyse :} on suppose que $f$ est solution de l'équation:
\\

On commence par poser $2y=x$ pour se débarrasser de la division par 2, on a donc
$$2yf(y)-f\left(\frac{1}{y}\right)=1$$
En substituant avec $1/y$ on a:
$$\frac{2}{y}f\left(\frac{1}{y}\right)-f(y)=1$$
Soit après multiplication par $\frac{y}{2}$
$$f\left(\frac{1}{y}\right)-\frac{y}{2}f(y)=\frac{y}{2}$$
La somme des deux équations donne
$$\left(2y-\frac{y}{2}\right)f(y)=1+\frac{y}{2}$$
Soit $$f(y)=\frac{2+y}{3y}$$
\textbf{Synthèse :}
Réciproquement, on vérifie que
pour $f(y)=\frac{2+y}{3y}$
On a :
$$xf\left(\frac{x}{2}\right)-f\left(\frac{2}{x}\right)=\frac{x(4+x)}{3x}-\frac{1+1/x}{3/x}=\frac{4+x}{3}-\frac{x+1}{3}=1$$
Donc c'est bien solution est c'est la seule. 
\end{preuve}

\begin{exo}[D]
Trouver les fonctions $f:\R\to \R$ telles que pour toute paire de réels $x,y$ on ait:
$$f(xy-1)+f(x)f(y)=2xy-1 $$
\end{exo}

\begin{proof}
\textbf{Analyse :} Supposons que $f$ soit une solution de l'équation.
\\
On pose $x=0$ ce qui donne 
$$f(-1)+f(x)f(0)=-1 $$ Si $f(0)\neq 0$, on peut diviser et donc $f$ est constante. C'est absurde car aucune fonction constante n'est solution. Donc $f(0)=0$ ce qui donne 
$$f(-1)=-1 $$
Pour $x\neq 0$ on pose $y=\frac{1}{x}$ ce qui donne 
$$f(0)+f(x)f\left(\frac{1}{x}\right)=2x\frac{1}{x}-1\iff f(x)f\left(\frac{1}{x}\right)=1 $$
L'idée maintenant est de forcer une factorisation, pour cela, on voudrait que $xy-1=y\iff y=\frac{1}{x-1}$. Pour $x\neq 1$, on a donc 
$$f\left(\frac{1}{x-1}\right)\left(1+f(x)\right)=\frac{2x}{x-1}-1=\frac{x+1}{x-1} $$
Pour utiliser l'égalité précédente, on multiplie par $f(x-1)$ ce qui donne 
$$f(x)+1=\frac{x+1}{x-1}f(x-1) $$
On pose maintenant $y=1$ ce qui donne 
$$f(x-1)+f(x)f(1)=2x-1 $$
On réinjecte l'expression de $f(x-1)$ obtenue.
$$(f(x)+1)\frac{x-1}{1+x} +f(x)f(1)=2x-1$$
Cela donne donc
$$f(x)=\frac{2x^2}{x-1+f(1)(x+1)}$$
On pose $x=1$ dans cette expression ce qui donne 
$$f(1)^2=1 $$
Si $f(1)=1$ on a $f(x)=-x^2$.
\\
Si $f(1)=-1$ on a $f(x)=x$.
\\
\textbf{Synthèse :} Si $f(x)=x$ on a 
$$f(xy-1)+f(x)f(y)=xy-1+xy=2xy-1 $$
Si $f(x)=-x^2$ on a 
$$f(xy-1)+f(x)f(y)=-(x^2y^2-2xy+1)-x^2y^2=-2xy-1 $$
Donc les deux seules solutions sont $f(x)=x$ ainsi que $f(x)=-x^2$.
\end{proof}



\subsubsection{Particularités d'une fonction}
\emph{On va maintenant se concentrer sur certaines propriétés remarquables que peut avoir une fonction. Lors de la résolution d'une équation fonctionnelle, on peut toujours s'intéresser à si les solutions présentent ce genre de propriétés. }

Voici les principales propriétés.

\begin{dfn}[Parité]
Pour une fonction $f:\R\to \R$, on dit que
\begin{itemize}
    \item $f$ est une fonction paire si $f(x)=f(-x)$ pour tout réel $x$.
    \item $f$ est une fonction impaire si $f(x)=-f(-x)$ pour tout réel $x$.
\end{itemize}

\end{dfn}

\begin{rem}
Attention, contrairement à ce que semble impliquer la terminologie, une fonction peut être ni paire ni impaire.
\\
Pour une telle fonction, une substitution du type $x\gets-x$ peut donner des informations importantes.
\end{rem}

Voici un exercice qui utilise cette idée.



\begin{exo}[M]
Trouver les fonctions $f:\R\to \R$ telles que pour tous $x,y$ réels on ait:
$$f(x+yf(x)) +f(xf(y)-y) = f(x)-f(y)+2xy^2 \ \text{ pour tous } x, y \in \R.$$
\end{exo}
\begin{preuve}
\textbf{Analyse :} On suppose que $f$ est solution de l'équation.
\\
On pose tout d'abord $x=y=0$
ce qui donne $2f(0)=0$ soit $f(0)=0$.
\\
On pose seulement $x=0$ ce qui donne $f(-y)=-f(y)$. Ainsi, $f$ est impaire.
\\
On pose $y=-w$ ce qui donne $$f(x-wf(x))+f(w-xf(w))=f(w)+f(x)+2xw^2$$
Soit encore $$f(x-wf(x))+f(w-xf(w))-f(w)-f(x)=2xw^2$$
Mais le membre de gauche est symétrique ce qui implique que le membre de droite aussi, donc que $$2xw^2=2wx^2$$ pour tous réels $x,w$ soit encore que $x=w$ pour tous réels non nuls, ce qui est absurde, il n'y a donc pas de telle fonction. 
\end{preuve}

\emph{Deux propriétés essentielles sont la surjectivité et l'injectivité.}

\begin{dfn}[Surjectivité]
Soit $f:A\to B$ une fonction entre deux ensemble. On dit que $f$ est surjective si elle atteint tous les éléments de $B$. Autrement dit, si pour tout $b\in B$, on peut trouver un $a\in A$ tel que $f(a)=b$.

\end{dfn}
\begin{rem}
En pratique, la surjectivité de $f$ permet des subtitution du type $f(y)=z$.
\end{rem}
Voici un exemple.
\begin{ex}
Trouver les fonctions $f:\R\to \R$ telles que pour toute paire de réels $x,y$ on ait:
$$f(x^2+f(y))=2x-f(y) $$
\end{ex}
\begin{preuve}
\textbf{Analyse :} Supososon que $f$ est solution de l'équation.
\\
Montrons que $f$ est surjective. Pour cela, on pose $y=0$ ce qui donne 
$$f(x^2f(0))=2x-f(0) $$
Ainsi, pour tout réel $x$, on trouver un antécédant de $x$ en posant
$z=\left( \frac{x+f(0)}{2} \right) ^2+f(0)$. Donc $f$ est surjective. On peut donc poser $z=f(y)$ avec $z$ qui varie dans $\R$. Cela simplifie la relation initiale
$$f(x^2+z)=2x-z $$
On pose alors $x=0$ ce qui donne 
$$f(z)=-z $$
\textbf{Synthèse :}
Si $f(z)=-z$ on a d'une part $$f(x^2+f(y))=y-x^2$$ et d'autre part
$$ 2x-f(y)=2x+y$$
Donc l'équation n'a pas de solutions.
 \end{preuve}
\begin{dfn}[Injectivité]
Soit $f:A\to B$ une fonction entre deux ensemble. On dit que $f$ est injective si deux élements distincts de $A$ ont des images par $f$ différentes.
\end{dfn}
\begin{rem}
Cela s'écrit que $a\neq a'\Rightarrow f(a)\neq f(a')$. Néanmoins, pour montrer qu'une fonction est injective, on raisonne par contraposée. On montre en fait que 
$f(a)=f(a')\Rightarrow a=a'$. On commence donc par supposer que $f(a)=f(a')$, on substitue dans l'équation et on espère montrer que $a=a'$.
\\
En pratique, l'injectivité permet de "simplifier les $f$". Ainsi, si $f$ est injective, une relation du type
$$f(A)=f(B) $$ devient simplement $A=B$.
\end{rem}
Voici un exemple d'utilisation de l'injectivité.
\begin{ex}
Trouver les fonctions $f:\R^+\to \R^+$ telles que pour toute paire de réels $x,y$ on ait:
$$f(f(x)+y)=2x+f(-f(f(x)))+f(y)) $$
\end{ex}
\begin{preuve}
\textbf{Analyse :} On suppose que $f$ est une solution de l'équation.
\\
Commençons par poser $x=0$ ce qui donne 
$$f(f(0)+y)=f(-f(f(0))+f(y)) $$
On aimerait avoir $f$ injective pour se débarasser des $f$. On suppose donc que $f(a)=f(b)$. Alors avec $x=a$ on a
$$f(f(a)+y)-f(-f(f(a))+f(y))=2a $$
Mais aussi
$$f(f(b)+y)-f(-f(f(b))+f(y))=2b $$
De là, on a donc $a=b$ et donc $f$ est bien injective. On en déduit que 
$$f(y)=y+f(0)+f(f(0)) $$
\textbf{Synthèse :} Si $f(y)=y+a$ on écrit que 
$$f(f(x)+y)=x+y+2a=2x+f(-f(f(x))+f(y))=2x+(-x+y-a)=x+y-a $$
Et donc $a=0$. Ainsi, la seule solution est l'identité.
\end{preuve}
\begin{dfn}[Bijectivité]
Soit $f:A\to B$ une fonction entre deux ensemble. On dit que $f$ est bijective si elle est injective et surjective.
\end{dfn}
\begin{exo}[F-M]
Trouver les $f:\R\to \R$ telles que pour toute paire de réels $x,y$ on ait $$f(f(x)f(y))=f(x)y $$

\end{exo}
\begin{preuve}
\textbf{Analyse :} Supposons que $f$ est une solution.
\\
La fonction nulle est solution. Sinon, supposons que $f$ est une solution non constamment nulle et donnons nous donc $a\in \R$ tel que $f(a)\neq 0$. On pose alors $x=a$ ce qui donne pour tout $y\in \R$
$$f(f(a)f(y))=f(a)y $$
Montrons que $f$ est injective. Supposons donc qu'il existe $b,c$ tels que $f(b)=f(c)$.
\\
Alors avec $y=b$ on a 
$$f(f(a)f(b))=f(a)b $$
Et avec $y=c$ on a 
$$f(f(a)f(c))=f(a)c $$
Et donc on a 
$$f(a)b=f(a)c $$ et comme $f(a)\neq 0$ on a $b= c$. Donc $f$ est injective.
Pour utiliser l'injectivité, on pose $y=1$ ce qui donne 
$$f(f(x)f(1))=f(x)\Rightarrow f(x)f(1)=x $$
Avec $x=1$, on a$$ f(1)^2=1\Rightarrow f(1)=\pm 1$$
Donc les seules solutions possibles sont $f(x)=x$ et $f(x)=-x$.
\\
\textbf{Synthèse :} On suppose que $f(x)=x$ ce qui donne 
$$f(f(x)f(y))=xy=f(x)y $$
Si $f(x)=-x$ on vérifie que
$$f(f(x)f(y))=-xy=f(x)y $$
Donc les deux seules solutions sont $f(x)=x$ et $f(x)=-x$.
\end{preuve}
\begin{exo}[M]
Trouver les $f:\R\to \R$ telles que pour toute paire de réels $x,y$ on ait $$f(x^2+f(y))=y+f(x)^2$$
\end{exo}
\begin{preuve}
\textbf{Analyse :} Supposons que $f$ est une solution.
\\
Montrons que $f$ est bijective. On pose $x=0$ et on a $f(f(y))=y+f(0)^2$ puis on pose $x=f(0)$ et $f(f(0)^2+f(y))=f^3(y)=y+f(f(0))^2$ puis avec $x=y=0$ on a $f(f(0))=f(0)^2$ et donc $f^3(y)=y+f(0)^4$. Avec $y=0$ on a $f(x^2+f(0))=f(x)^2$ et donc $f(x)=0$ implique $f(-x)=0$. Mais par bijectivité on doit alors avoir $f(0)=0$. de là, on a $f^3=f^2$ et par injectivité $f=Id$.
\end{preuve}
\\

\emph{Une autre idée importante est d'utiliser la symétrie. En effet, si le membre de gauche de l'équation est symétrique en $(x,y)$, alors celui de droite aussi. Ainsi, en exploitant la symétrie, on peut obtenir de nouvelles équations. Voyons une exemple.}
\begin{ex}
Trouver les $f:\R\to \R$ telles que pour toute paire de réels $x,y$ on ait
$$f(f(x)+f(y))=f(-x-f(y))+2(x+y) $$
\end{ex}

\begin{preuve}
\textbf{Analyse :} On suppose que $f$ est solution de l'équation. Etudions pour commencer l'injectivité et la surjectivité de $f$. On pose pour cela $x=0$ ce qui donne 
$$f(f(0)+f(y))-f(-f(y))=2y $$
Dès lors, si $f(y)=f(z)$ alors on en déduit que 
$$2z=f(f(0)+f(z))-f(-f(z))=f(f(0)+f(y))-f(-f(y))=2y\Rightarrow z=y $$
En d'autres termes, on a montré que $f$ était injective. 
\\
Utilisons maintenant la symétrie. On commence par mettre tous les termes symétriques du même côté:
$$ f(f(x)+f(y))-2(x+y)=f(-x-f(y))$$
La symétrie à gauche implique celle à droite ce qui donne 
$$f(-x-f(y))=f(-y-f(x)) $$
Puis l'injectivité permet de se débarasser des $f$ ce qui donne 
$$-x-f(y)=-y-f(x) $$
Avec $x=0$ on a montré que pour tout réel $y$ on a 
$$f(y)=y $$
\textbf{Synthèse :} Réciproquement, si $f$ est l'identité, on écrit que 
$$f(f(x)+f(y))=x+y=f(-x-f(y))+2(x+y)=-x-y+2(x+y) $$
Donc la seule solution est l'identité.
\end{preuve}
\\
\emph{La dernière propriété à laquelle on va s'intéresser est la périodicité.}
\begin{dfn}[Périodicité]
Soit $f : \R\to \R$ une fonction et $T>0$. On dit que $f$ est $T-$périodique si pour tout réel $x$, on a $f(x)=f(x+T)$.

\end{dfn}
\begin{rem}
En pratique, la périodicité permet des substitution dy type $x\gets x+T$. 
\\
La périodicité est souvent reliée aux notions de bornitude. C'est à dire telle qu'il existe une constante $M>0$ vérifiant 
$|f(x)|<M$ pour tout réel $x$.
\\
On verra qu'elle est souvent reliée à l'injectivité. Ce point est un peu plus technique et ne sera pas trop abordé dans le cours.
\end{rem}
\begin{exo}[M]
Trouver toutes les fonctions $f : \Z\to \Z$ telles que pour toute paire d'entiers $x,y$ on ait
$$f(x+f(y))=f(x)-y $$

\end{exo}
\begin{preuve}
\textbf{Analyse :} On suppose que $f$ est solution de l'équation.
\\
On pose $x=0$ ce qui donne 
$$f(f(y))=f(0)-y $$
On en déduit aisément que $f$ est bijective.
\\
On pose $y=0$, on a alors
$$f(x+f(0))=f(x) $$
et par injectivité on a 
$$x+f(0)=x\Rightarrow f(0)=0 $$
Puis en posant $y=f(x)$ on a
$$f(f(f(x))+x)=0 $$ ce qui donne par injectivité
$$f(f(x))+x=0\iff f(f(x))=-x $$
On pose alors $y=f(z)$ ce qui donne 
$$f(x-z)=f(x)-f(z) $$
On obtient alors que pour tous entiers, $u,v$ :
$$f(u)+f(v)=f(u+v) $$
En particulier, on a pour $u\geq 0$
$$f(u)=f(\underbrace{1+\dots+1}_{u \textit{ fois}})=\underbrace{f(1)+\dots+f(1)}_{u \textit{ fois}} =uf(1)$$
Et pour $u\leq 0$, on écrit que 
$$f(u)+f(-u)=f(0)=0\Rightarrow f(u)=-(-uf(1))=uf(1) $$
\textbf{Synthèse :} Si $f(u)=au$ on vérifie que 
$$ f(x+f(y))=ax+a^2y=f(x)-y=ax-y$$ cela donne $a^2=-y$ donc l'équation n'a pas de solution.
\end{preuve}
\begin{rem}
L'équation fonctionnelle
$$f(x+y)=f(x)+f(y) $$
est l'équation de Cauchy. On vient de montrer que les solutions de l'équation de Cauchy de $\Z\to \Z$ sont les fonctions linéaires. C'est aussi vrai pour les fonctions de $\Q\to \Q$. Pour étendre le résultat aux fonctions de $\R \to \R$, il faut une hypothèse de régularité supplémentaire. Par exmeple, si $f$ est continue ou monotone, on montre que $f$ est linéaire. L'étude des solutions de l'équation de Cauchy est faite dans la plupart des cours d'équations fonctionnelles. Nous ne la traiterons pas ici.
\end{rem}
\begin{exo}[M]
Trouver les fonctions $f : \R\to \R$ telles pour toute paire de réels $x,y$ on ait:
$$f(f(x)+9y)=f(y)+9x+24y  $$
\end{exo}
\begin{preuve}
\textbf{Analyse :} On suppose que $f$ est une solution de l'équation.
\\
On pose tout d'abord $y=0$ ce qui nous donne $f(f(x))=f(0)+9x$,$f(0)$ étant constant, en faisant varier $x$, on peut atteindre toutes les valeurs réelles, en particulier, on en déduit que $f$ est surjective.
\\

Il existe donc au moins un réel $k$ tel que $f(k)=0$,or, on a donc $f(f(k))=f(0)+9k$ soit $f(0)=f(0)+9k$ d'où $k=0$ soit encore $f(0)=0$
\\
En particulier, on a $$f(f(x))=9x$$ pour tout $x$.
\\
On pose maintenant $x=0$ ce qui donne $f(9y)=f(y)+24y$ soit encore $$f(f(f(y)))=f(y)+24y$$ $f$ étant surjective, il existe donc un $w$ tel que $f(w)=y$ en substituant on a $$f(f(f(f(w))))=f(f(9w))=81w=f(f(w))+24f(w)=9w+24f(w)$$
Soit encore $$24f(w)=(81-9)w=72w$$
D'où $$f(w)=3w$$ pour tout $w\in \R$ (en faisant varier $y\in \R$.
\\
\textbf{Synthèse :}
On vérifie que $f$ est bien solution:
$f(f(x)+9y)=9x+27y=f(y)+9x+24y$
Ce qui conclut. 
\end{preuve}
\begin{exo}[D]
Trouver les fonctions $f:\R_+^*\to \R_+^*$ telles que pour tout couple de réels positifs $x,y$ on ait 
$$f(x)f(y) = 2f(x+yf(x)) $$
\end{exo}
\begin{preuve}
\textbf{Analyse :} On commence par utiliser la symétrie à gauche pour écrire que 
$$f(x+yf(x))=f(y+xf(y)) $$
Ce genre d'équation nous pousse à étudier l'injectivité de $f$. Supposons donc que 
$0<a<b$ tels que $f(a)=f(b)$. Comme $f(x)>0$ pour tout réel $c>0$ on peut poser $y=\frac{c}{f(x)}$ ce qui donne 
$$f(x)f\left(\frac{c}{f(x)}\right)=2f(x+c) $$
L'astuce est que le membre de gauche ne dépend que de $f(x)$, en particulier, avec $x=a$ et $x=b$ on déduit que 
$$2f(a+c)=2f(b+c)\iff f(a+c)=f(b+c)$$
On a donc montré que $f$ était $b-a$ périodique à partir de $a$. 
\\
\textbf{Attention:} Ici, $c$ est positif donc on a la périodicité seulement pour $x\geq a$.

Ceci nous sera très utile si l'on arrive à trouver une façon de créer des paires de réels positifs de même image. On commence par écarté le cas injectif.
\\
\textbf{Si $f$ est injective:}
Alors la relation $$f(x+f(x)y)=f(y+f(y)x) $$ devient donc 
$$x+f(x)y=y+f(y)x $$
Donc avec $x=1$ on a 
$$f(y)=(f(1)-1)y+1 $$
\textbf{Si $f$ n'est pas injective:}
Alors par ce qui précède, on se donne $a>0$ te $T=b-a$ tels que $f$ est $T-$périodique pour $x\geq a$. Pour forcer des simplifications, on part de la relation 
$$f(x)f(y)=2f(x+yf(x)) $$ et on pose $y_0=\frac{T}{f(x)}$ pour $x\geq a$. Cela donne donc
$$f(x)f(y_0)=2f(x+T)=2f(x)\Rightarrow f(y_0)=2 $$
On a donc trouvé un antécédant de $2$. On pose alors $x=y_0$ ce qui donne 
$$f(y)=f(y_0+2y) $$
Ainsi, par ce qui précède, on a montré que pour tout $y>0$, $f$ est $(y_0+y)$ périodique pour $x\geq y$.
\\
Montrons que cela implique que $f$ est constane. Soit donc $0<a<b$. On va montrer que $f(a)=f(b)$.
\\
On sait que pour tout $0<y<a$, on a 
$f(a+y+y_0)=f(a)$ ainsi que $f(b)=f(b+y+y_0)$.
On prend donc $y_1=\frac{b-a}{n}$ où l'entier $n$ est suffisamment petit pour que $\frac{b-a}{n}<a$. On prend alors $y_2>0$ suffisamment petit pour que $0<y_2<y_2+y_1<a$. Alors on écrit que 
$$f(a)=f(a+n(y_1+y_2+y_0))=f\left(a+\frac{b-a}{n}n+n(y_0+y_2)\right)=f(b+n(y_0+y_2))=f(b) $$
Donc $f$ est constante.
\\
\textbf{Synthèse :} Si $f$ est constante en $c$ on doit avoir $c^2=2c$ et comme $c>0$ on a $c=2$.
\\
Si $f(x)=(f(1)-1)x-1=ax-1$ on écrit que 
$$f(x)f(y)=a^2xy-a(x+y)+1=2f(x+yf(x))=2f(x+axy-y) $$
Soit encore
$$a^2xy-a(x+y)+1=2a(x+axy-y)-2 $$
Cela implique que $a=0$ ce qui est aburde. Donc la seule solution est la fonction constante en $2$.

\end{preuve}
\begin{rem}
Il n'est pas rare d'obtenir un résultat du type "soit $f$ est injective, soit elle est périodique (à partir d'un rang)". C'est une idée plus avancée mais qu'il est bon de garder en tête.
\end{rem}













