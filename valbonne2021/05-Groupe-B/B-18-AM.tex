%Le cours portait sur les triangles semblables. On trouvera toute la théorie nécessaire dans l'excellent polycopié suivant https://maths-olympiques.fr/wp-content/uploads/2017/09/geom_base.pdf à la page 14.

\subsubsection{Exercices}


\begin{exo}
(Puissance d'un point)
Soit $ABCD$ un quadrilatère cyclique et $X$ l’intersection de $AB$ et $CD$. Montrer que $XBC$ est semblable à $XDA$.
\end{exo}

\begin{exo}
Soit $ABC$ un triangle rectangle en $A$ et $H$ le pied de la hauteur issue de $A$. Montrer que $AH^2 = HB \cdot HC$, $AB^2=BH\times BC$ et $AC^2=CH\times BC$.
\end{exo}

\begin{exo}
Soit $ABC$ un triangle et $D,E$ les pieds des hauteurs issues de $A$ et $B$ respectivement. Montrer que les triangles $CDE$ et $CAB$ sont semblables.
\end{exo}

\begin{exo}
Soit $ABCD$ un parallèlogramme, $M$ un point du segment $[AC]$. Soit $E$ le projeté orhtogonal de $M$ sur le segment $[AB]$ et $F$ le projeté orhtogonal sur le segment $[AD]$. Montrer que $\frac{ME}{MF}=\frac{AD}{AB}$.
\end{exo}

\begin{exo}
Soit $ABCD$ un parallélogramme. Soit $M$ le milieu du segment $[BC]$ et $N$ le milieu du segment $[CD]$. Les droites $(AN)$ et $(BD)$ se coupent en $Q$ et les droites $(AM)$ et $(BD)$ se coupent en $P$. Montrer que $BP=PQ=QD$.
\end{exo}

\begin{exo}
Soit $ABCD$ un losange. Soit $F$ un point du segment $[AD]$ et $E$ un point du segment $[AB]$. Les droites $(FC)$ et $(BD)$ se coupent en $L$, les droites $(EC)$ et $(BD)$ se coupent en $K$. Les droites $(FK)$ et $(BC)$ se coupent en $Q$ et les droites $(EL)$ et $(DC)$ se coupent en $P$. Montrer que $CP=CQ$.
\end{exo}

\begin{exo}
%concours 15 MT P4
Soit $ABC$ un triangle et soit $D,E$ et $F$ les milieux respectifs des segments $[BC], [CA]$ et $[AB]$. Soit $X$ un point à l'intérieur du triangle $ABC$. Soient $A',B'$ et $C'$ les symétriques du point $X$ par rapport respectivement aux points $D,E$ et $F$. Montrer que les droites $(AA'),(BB'),(CC')$ sont concourrantes.
\end{exo}

\begin{exo}
%Prasolov
Soit $ABC$ un triangle rectangle isocèle en $C$, et $D,E$ deux points sur $[CA]$ et $[CB]$ tel que $CD=CE$. Puis, soit $U$ et $V$ deux points sur $[AB]$ tels que $(DU)$ et $(CV)$ sont perpendiculaires à $(AE)$. Montrer que $UV=VB$.
\end{exo}

\begin{exo}
%problem.ru pb 108646

Soit $ABCD$ trapèze avec $(AD) \parallel (BC)$, $M$ l’intersection de ses diagonales et $P$ un point de $[BC]$ tel que $\widehat{APM}=\widehat{DPM}$. Montrer que la distance de $B$ à $(DP)$ est égale à la distance de $C$ à $(AP)$.
\end{exo}

\begin{exo}
%Iran Round 3 2018
Soit $ABC$ un triangle et soient $D,E$ et $F$ les points de contact du cercle inscrit respectivement avec les côtés $[BC], [CA]$ et $[AB]$. Soit $P$ un point tel que $PF=FB$, les droites $(FP)$ et $(AC)$ sont parallèles et les points $P$ et $C$ sont dans le même demi-plan délimité par la droite $(AB)$. Soit $Q$ un point tel que $QE=EC$, les droites $(EQ)$ et $(AB)$ sont parallèles et les points $Q$ et $B$ sont dans le même demi-plan délimité par la droite $(AC)$. Montrer que les points $P, D$ et $Q$ sont alignés.
\end{exo}

\begin{exo}
Soit $ABCD$ un trapèze avec $AB<CD$ et les droites $(AB)$ et $(CD)$ parallèles. Soit $P$ un point appartenant au segment $[CB]$. La parallèle à le droite $(AP)$ passant par le point $C$ coupe le segment $[AD]$ en le point $R$ et la parallèle à la droite $(DP)$ passant par le point $B$ coupe le segment $[AD]$ en le point $R'$. Montrer que $R=R'$.
\end{exo}

\begin{exo}
%Pierre Haas
Soit $ABC$ un triangle et $X$ un point situé à l'intérieur du triangle $ABC$. Les droites $(AX), (BX)$ et $(CX)$ recoupent le cercle circonscrit au triangle $ABC$ en les points $P,Q$ et $R$ respectivement. Soit $U$ un point appartenant au segment $[XP]$. Les parallèles à $(AB),(AC)$ passant par $U$ coupent les droites $(XQ)$ et $(XR)$ en les points $V$ et $W$ respectivement. Montrer que les points $R,W,V,Q$ sont cocycliques.
\end{exo}

\newpage

\subsubsection{Solutions}

\setcounter{exo}{0}

\begin{exo}
(Puissance d'un point)
Soit $ABCD$ un quadrilatère cyclique et $X$ l’intersection de $AB$ et $CD$. Montrer que $XBC$ est semblable à $XDA$.
\end{exo}

\begin{sol}
\begin{center}
\begin{tikzpicture}
[scale=1]
\tkzInit[ymin=-4,ymax=4,xmin=-5.5,xmax=4]
\tkzClip

\tkzDefPoint(0,0){O}
\tkzDefPoint(-5,0){X}
\tkzDefPoint(-4,0.5){Y}
\tkzDefPoint(-4,-0.25){Z}
\tkzDefPoint(-3,0){W}
\tkzInterLC(X,Y)(O,W) \tkzGetPoints{A}{B}
\tkzInterLC(X,Z)(O,W) \tkzGetPoints{C}{D}

\tkzMarkAngle[color=blue,size=0.5](A,D,X)
\tkzMarkAngle[color=blue,size=0.5](X,B,C)
\tkzMarkAngle[color=red,size=0.5](X,A,D)
\tkzMarkAngle[color=red,size=0.5](B,C,X)
\tkzDrawSegment(X,B)
\tkzDrawSegment(X,C)
\tkzDrawSegment(C,B)
\tkzDrawSegment(A,D)
\tkzDrawCircle(O,W)
\tkzDrawPoints[fill=white](A,B,C,D,X)

\tkzLabelPoint[above left](A){$A$}
\tkzLabelPoint[above right](B){$B$}
\tkzLabelPoint(C){$C$}
\tkzLabelPoint[below left](D){$D$}
\tkzLabelPoint[left](X){$X$}

\end{tikzpicture}
\end{center}

Traitons le cas où le quadrilatère $ ABCD$ est convexe (le cas où il est croisé se fait de même manière).

Dans ce cas, le point $X$ se trouve à l'extérieur de $ABCD$ et puisque $ABCD$ est cyclique, on a

\[\widehat{XDA}=180^\circ-\widehat{ADC}= \widehat{ABC}=\widehat{XBC}\]

d'après le théorème de l'angle inscrit. De manière similaire, $\widehat{XAD}$ = $\widehat{XCB}$. Les triangles $XBC$ et $XDA$ partagent donc deux angles égaux, ils sont bien semblables.
\end{sol}

\begin{exo}
Soit $ABC$ un triangle rectangle en $A$ et $H$ le pied de la hauteur issue de $A$. Montrer que $AH^2 = HB \cdot HC$, $AB^2=BH\times BC$ et $AC^2=CH\times BC$.
\end{exo}

\begin{sol}
\begin{center}
\begin{tikzpicture}
[scale=1]
\tkzInit[ymin=-1,ymax=4,xmin=-1,xmax=6]
\tkzClip

\tkzDefPoint(0,0){A}
\tkzDefPoint(5,0){B}
\tkzDefPoint(0,3){C}
\tkzDefLine[perpendicular=through A](B,C) \tkzGetPoint{h}
\tkzInterLL(B,C)(A,h) \tkzGetPoint{H}

\tkzMarkRightAngle[color=red](B,A,C)
\tkzMarkRightAngle[color=red](A,H,B)
\tkzMarkAngle[color=blue](A,C,B)
\tkzMarkAngle[color=blue](B,A,H)
\tkzDrawSegment(A,B)
\tkzDrawSegment(B,C)
\tkzDrawSegment(C,A)
\tkzDrawSegment(A,H)
\tkzDrawPoints[fill=white](A,B,C,H)

\tkzLabelPoint[below left](A){$A$}
\tkzLabelPoint[right](B){$B$}
\tkzLabelPoint[above](C){$C$}
\tkzLabelPoint[above right](H){$H$}
\end{tikzpicture}
\end{center}

Nous avons

\[ \widehat{HAB} = 90^{\circ} - \widehat{HBA}=\widehat{ACH}\]

De même,

\[ \widehat{HAC} = 90^{\circ}-\widehat{HAB}=\widehat{ABH}\]
Les triangles $HAB$ et $HCA$ ont deux angles en commun et sont donc semblables. On en déduit

\[\frac{HA}{HB} = \frac{HC}{HA}\]

ce qui se réécrit $HA^2 = HB \cdot HC$.

Notons ensuite que puisque $\widehat{ACH}=\widehat{ACB}$, les triangles $ACH$ et $BCA$ sont semblables. On a donc $\dfrac{AC}{CB}=\dfrac{CH}{AC}$ qui se réécrit $AC^2=CB\cdot CH$. La relation $AB^2=CB\cdot HB$ se démontre de la même façon.

\end{sol}

\begin{exo}
Soit $ABC$ un triangle et $D,E$ les pieds des hauteurs issues de $A$ et $B$ respectivement. Montrer que les triangles $CDE$ et $CAB$ sont semblables.
\end{exo}

\begin{sol}
\begin{center}
\begin{tikzpicture}
[scale=1]
\tkzInit[ymin=-1,ymax=7,xmin=-5,xmax=5]
\tkzClip

\tkzDefPoint(1,6){C}
\tkzDefPoint(3,0){B}
\tkzDefPoint(-3,0){A}
\tkzDefLine[perpendicular=through A](B,C) \tkzGetPoint{d}
\tkzInterLL(A,d)(B,C) \tkzGetPoint{D}
\tkzDefLine[perpendicular=through B](A,C) \tkzGetPoint{e}
\tkzInterLL(B,e)(A,C) \tkzGetPoint{E}
\tkzDefMidPoint(A,B) \tkzGetPoint{M}

\tkzMarkRightAngle[color=red](B,D,A)
\tkzMarkRightAngle[color=red](A,E,B)
\tkzMarkAngle[color=blue](C,D,E)
\tkzMarkAngle[color=blue](B,A,C)
\tkzDrawSegment(A,B)
\tkzDrawSegment(B,C)
\tkzDrawSegment(C,A)
\tkzDrawSegment(A,D)
\tkzDrawSegment(B,E)
\tkzDrawSegment(E,D)
\tkzDrawCircle[dashed](M,B)
\tkzDrawPoints[fill=white](A,B,C,D,E)

\tkzLabelPoint[left](A){$A$}
\tkzLabelPoint[right](B){$B$}
\tkzLabelPoint[above](C){$C$}
\tkzLabelPoint[above right](D){$D$}
\tkzLabelPoint[above left](E){$E$}
\end{tikzpicture}
\end{center}

Puisqu'on a

\[\widehat{ADB}=90^\circ=\widehat{AEB}\]

les points $E,D,B$ et $A$ sont cocycliques. On a alors, de même que pour le premier exercice, d'après le théorème de l'angle inscrit :

\[\widehat{EDC}=180^\circ-\widehat{EDB}=\widehat{BAE}=\widehat{BAC}\]

et donc les triangles $CDE$ et $CAB$ partagent deux angles égaux et sont bien semblables.
\end{sol}

\begin{exo}
Soit $ABCD$ un parallèlogramme, $M$ un point du segment $[AC]$. Soit $E$ le projeté orhtogonal de $M$ sur le segment $[AB]$ et $F$ le projeté orhtogonal sur le segment $[AD]$. Montrer que $\frac{ME}{MF}=\frac{AD}{AB}$.
\end{exo}

\begin{sol}
\begin{center}
\begin{tikzpicture}
[scale=1.25]
\tkzInit[ymin=-1,ymax=4,xmin=-2.5,xmax=4]
\tkzClip

\tkzDefPoint(-1.5,3.5){A}
\tkzDefPoint(3,3.5){B}
\tkzDefPoint(2.5,0){C}
\tkzDefPoint(-2,0){D}
\tkzDefPoint(0,0){O}
\tkzDefLine[perpendicular=through O](D,C) \tkzGetPoint{m}
\tkzInterLL(O,m)(A,C) \tkzGetPoint{M}
\tkzDefLine[perpendicular=through M](A,D) \tkzGetPoint{f}
\tkzInterLL(A,D)(M,f) \tkzGetPoint{F}
\tkzDefLine[perpendicular=through M](A,B) \tkzGetPoint{e}
\tkzInterLL(A,B)(M,e) \tkzGetPoint{E}
\tkzDefCircle[circum](A,E,F) \tkzGetPoint{o}

\tkzMarkRightAngle[color=red](A,E,M)
\tkzMarkRightAngle[color=red](M,F,A)
\tkzMarkAngle[color=blue,size=0.5](M,A,E)
\tkzMarkAngle[color=blue,size=0.5](M,F,E)
\tkzMarkAngle[color=blue,size=0.5](M,C,D)
\tkzDrawSegment(A,B)
\tkzDrawSegment(B,C)
\tkzDrawSegment(C,A)
\tkzDrawSegment(C,D)
\tkzDrawSegment(D,A)
\tkzDrawSegment(M,E)
\tkzDrawSegment(M,F)
\tkzDrawSegment(E,F)
\tkzDrawCircle[dashed](o,A)
\tkzDrawPoints[fill=white](A,B,C,D,E,F,M)

\tkzLabelPoint[above left](A){$A$}
\tkzLabelPoint[above right](B){$B$}
\tkzLabelPoint[below right](C){$C$}
\tkzLabelPoint[below left](D){$D$}
\tkzLabelPoint[above right](E){$E$}
\tkzLabelPoint[left](F){$F$}
\tkzLabelPoint[right](M){$M$}
\end{tikzpicture}
\end{center}


Puisque $\widehat{AEM}=90^\circ=180-\widehat{AFM}$, les points $A,E,M$ et $F$ sont cocycliques. D'après le théorème de l'angle inscrit, on a donc que $\widehat{MFE}=\widehat{MAE}=\widehat{CAB}$ et $\widehat{MEF}=\widehat{MAF}=\widehat{CAD}=\widehat{ACB}$. Les triangles $MFE$ et $BAC$ sont donc semblables.

Dans le parallélogramme $ABCD$, $CD=AB$ donc on déduit l'égalité de rapport
\[\frac{ME}{MF}=\frac{BC}{AB}=\frac{AD}{AB}\]
\end{sol}

\begin{exo}
Soit $ABCD$ un parallélogramme. Soit $M$ le milieu du segment $[BC]$ et $N$ le milieu du segment $[CD]$. Les droites $(AN)$ et $(BD)$ se coupent en $Q$ et les droites $(AM)$ et $(BD)$ se coupent en $P$. Montrer que $BP=PQ=QD$.
\end{exo}

\begin{sol}
\begin{center}
\begin{tikzpicture}
[scale=1]
\tkzInit[ymin=-1,ymax=5,xmin=-3,xmax=6]
\tkzClip

\tkzDefPoint(5,3){A}
\tkzDefPoint(3,0){B}
\tkzDefPoint(-2,0){C}
\tkzDefPoint(0,3){D}
\tkzDefMidPoint(B,C) \tkzGetPoint{M}
\tkzDefMidPoint(C,D) \tkzGetPoint{N}
\tkzInterLL(A,N)(B,D) \tkzGetPoint{Q}
\tkzInterLL(A,M)(B,D) \tkzGetPoint{P}

\tkzDrawSegment(A,B)
\tkzDrawSegment(B,C)
\tkzDrawSegment(C,D)
\tkzDrawSegment(D,A)
\tkzDrawSegment(B,D)
\tkzDrawSegment(A,M)
\tkzDrawSegment(A,N)
\tkzDrawPoints[fill=white](A,B,C,D,M,N,P,Q)

\tkzLabelPoint[right](A){$A$}
\tkzLabelPoint[below right](B){$B$}
\tkzLabelPoint[below left](C){$C$}
\tkzLabelPoint[above left](D){$D$}
\tkzLabelPoint[below](M){$M$}
\tkzLabelPoint[above left](N){$N$}
\tkzLabelPoint[below](P){$P$}
\tkzLabelPoint[below](Q){$Q$}
\end{tikzpicture}
\end{center}

D'après le théorème de Thalès dans le papillon $DNQAB$, $\frac{DQ}{QB}=\frac{DN}{AB}=\frac1{2}$. Donc $QB=2DQ$ et $DQ=\frac1{3}DB$.

D'après le théorème de Thalès dans le papillon $BMPAD$, $\frac{PB}{PD}=\frac{MB}{AD}=\frac1{2}$ et donc $PB=\frac1{3}DB$.

En conséquence, on a aussi $QP=\frac1{3}DB$ donc on a les égalités de longueurs voulues.

\end{sol}

\begin{exo}
Soit $ABCD$ un losange. Soit $F$ un point du segment $[AD]$ et $E$ un point du segment $[AB]$. Les droites $(FC)$ et $(BD)$ se coupent en $L$, les droites $(EC)$ et $(BD)$ se coupent en $K$. Les droites $(FK)$ et $(BC)$ se coupent en $Q$ et les droites $(EL)$ et $(DC)$ se coupent en $P$. Montrer que $CP=CQ$.
\end{exo}

\begin{sol}
\begin{center}
\begin{tikzpicture}
[scale=1]
\tkzInit[ymin=-7,ymax=7,xmin=-5,xmax=5]
\tkzClip

\tkzDefPoint(0,6){A}
\tkzDefPoint(3,0){B}
\tkzDefPoint(-3,0){D}
\tkzDefPoint(0,-6){C}
\tkzDefPointBy[homothety=center A ratio 0.4](D) \tkzGetPoint{F}
\tkzDefPointBy[homothety=center A ratio 0.7](B) \tkzGetPoint{E}
\tkzInterLL(F,C)(B,D) \tkzGetPoint{L}
\tkzInterLL(E,C)(B,D) \tkzGetPoint{K}
\tkzInterLL(F,K)(B,C) \tkzGetPoint{Q}
\tkzInterLL(E,L)(D,C) \tkzGetPoint{P}

\tkzDrawSegment(A,B)
\tkzDrawSegment(B,C)
\tkzDrawSegment(C,D)
\tkzDrawSegment(D,A)
\tkzDrawSegment(B,D)
\tkzDrawSegment(E,C)
\tkzDrawSegment(F,C)
\tkzDrawSegment(E,P)
\tkzDrawSegment(F,Q)
\tkzDrawPoints[fill=white](A,B,C,D,E,F,L,K,P,Q)

\tkzLabelPoint[above](A){$A$}
\tkzLabelPoint[right](B){$B$}
\tkzLabelPoint[below](C){$C$}
\tkzLabelPoint[left](D){$D$}
\tkzLabelPoint[above right](E){$E$}
\tkzLabelPoint[above left](F){$F$}
\tkzLabelPoint[above](K){$K$}
\tkzLabelPoint[below right](L){$L$}
\tkzLabelPoint[below left](P){$P$}
\tkzLabelPoint(Q){$Q$}

\end{tikzpicture}
\end{center}

On dispose de plusieurs papillons, on va donc les examiner chacun et tirer les informations utiles.

D'après le théorème de Thalès dans le papillon $QBKDF$, $\frac{QB}{DF}=\frac{BK}{DK}$.

D'après le théorème de Thalès dans le papillon $EBKDC$, $\frac{EB}{DC}=\frac{BK}{DK}$ et ainsi, en combinant les deux égalités $QB=\frac{DF \cdot DF}{DC}$.

D'après le théorème de Thalès dans le papillon $DPLEB$, $\frac{DP}{EB}=\frac{DL}{LB}$.

D'après le théorème de thalès dans le papillon $DFLCB$, $\frac{DF}{CB}=\frac{DL}{LB}$. Ainsi, en combinant les deux égalités, $DP=\frac{DF\cdot EB}{BC}=\frac{DF\cdot EC}{CD}=QB$.

On a donc bien $CP=CQ$.

\end{sol}

\begin{exo}
%concours 15 MT P4
Soit $ABC$ un triangle et soit $D,E$ et $F$ les milieux respectifs des segments $[BC], [CA]$ et $[AB]$. Soit $X$ un point à l'intérieur du triangle $ABC$. Soient $A',B'$ et $C'$ les symétriques du point $X$ par rapport respectivement aux points $D,E$ et $F$. Montrer que les droites $(AA'),(BB'),(CC')$ sont concourrantes.
\end{exo}

\begin{sol}
\begin{center}
\begin{tikzpicture}
[scale=0.8]
\tkzInit[ymin=-3,ymax=7,xmin=-7,xmax=7]
\tkzClip

\tkzDefPoint(1,6){A}
\tkzDefPoint(3,0){B}
\tkzDefPoint(-3,0){C}
\tkzDefPoint(-1,2){X}
\tkzDefMidPoint(B,C) \tkzGetPoint{D}
\tkzDefMidPoint(C,A) \tkzGetPoint{E}
\tkzDefMidPoint(A,B) \tkzGetPoint{F}

\tkzDefPointBy[symmetry=center D](X) \tkzGetPoint{A'}
\tkzDefPointBy[symmetry=center E](X) \tkzGetPoint{B'}
\tkzDefPointBy[symmetry=center F](X) \tkzGetPoint{C'}

\tkzInterLL(A,A')(B,B') \tkzGetPoint{Y}

\tkzMarkSegment[color=blue,mark=s||](A',Y)
\tkzMarkSegment[color=blue,mark=s||](A,Y)
\tkzMarkSegment[color=red,mark=|](B',Y)
\tkzMarkSegment[color=red,mark=|](B,Y)
\tkzDrawSegment(A,B)
\tkzDrawSegment(B,C)
\tkzDrawSegment(C,A)
\tkzDrawSegment(A,A')
\tkzDrawSegment(B,B')
\tkzDrawSegment[dashed](C,C')
\tkzDrawSegment(X,A')
\tkzDrawSegment(X,B')
\tkzDrawSegment[dashed](A',B')
\tkzDrawSegment(D,E)

\tkzDrawPoints[fill=white](A,B,C,D,E,F,X,A',B',C',Y)

\tkzLabelPoint[above](A){$A$}
\tkzLabelPoint(B){$B$}
\tkzLabelPoint[below left](C){$C$}
\tkzLabelPoint[below left](D){$D$}
\tkzLabelPoint[left](E){$E$}
\tkzLabelPoint[above right](F){$F$}
\tkzLabelPoint[left](X){$X$}
\tkzLabelPoint[below](A'){$A'$}
\tkzLabelPoint[above](B'){$B'$}
\tkzLabelPoint[right](C'){$C'$}
\tkzLabelPoint[below right](Y){$Y$}
\end{tikzpicture}
\end{center}

Par le théorème de Thalès, les points $D$ et $E$ étant les milieux des segment $[XA']$ et $[XB']$, les droites $(ED)$ et $(A'B')$ sont parallèles et $EF=\frac1{2}A'B'$. Les droites $(AB)$ et $(A'B')$ sont donc parallèles et $AB=A'B'$. Le quadrilatère $ABA'B'$ est donc un parallélogramme. Le point d'intersection des droites $(AA')$ et $(BB')$ est donc le milieu commun des segments $[AA']$ et $[BB']$. On trouve de même pour le point d'intersection des droites $(BB')$ et $(CC')$. Le point d'intersection est donc commun.
\end{sol}

\begin{exo}
%Prasolov
Soit $ABC$ un triangle rectangle isocèle en $C$, et $D,E$ deux points sur $[CA]$ et $[CB]$ tel que $CD=CE$. Puis, soit $U$ et $V$ deux points sur $[AB]$ tels que $(DU)$ et $(CV)$ sont perpendiculaires à $(AE)$. Montrer que $UV=VB$.
\end{exo}

\begin{sol}

\begin{center}
\begin{tikzpicture}
\tkzDefPoint(0,0){C}
\tkzDefPoint(5,0){B}
\tkzDefPoint(0,5){A}
\tkzDefPoint(2,0){E}
\tkzDefPoint(0,2){D}

\tkzDefLine[perpendicular=through D](A,E) \tkzGetPoint{u}
\tkzInterLL(D,u)(A,B) \tkzGetPoint{U}
\tkzDefLine[perpendicular=through C](A,E) \tkzGetPoint{v}
\tkzInterLL(C,v)(A,B) \tkzGetPoint{V}
\tkzInterLL(D,E)(C,V) \tkzGetPoint{R}
\tkzInterLL(C,V)(A,E) \tkzGetPoint{Y}

\tkzMarkRightAngle[color=red](B,C,A)
\tkzMarkRightAngle[color=red](A,Y,C)
\tkzDrawSegment(A,B)
\tkzDrawSegment(B,C)
\tkzDrawSegment(C,A)
\tkzDrawSegment(A,E)
\tkzDrawSegment(D,U)
\tkzDrawSegment(C,V)
\tkzDrawSegment(D,E)
\tkzDrawPoints[fill=white](A,B,C,D,E,U,V,R,Y)

\tkzLabelPoint[above](A){$A$}
\tkzLabelPoint[right](B){$B$}
\tkzLabelPoint[below left](C){$C$}
\tkzLabelPoint[left](D){$D$}
\tkzLabelPoint[below](E){$E$}
\tkzLabelPoint[above right](U){$U$}
\tkzLabelPoint[above right](V){$V$}
\tkzLabelPoint[below](R){$R$}
\tkzLabelPoint[above right](Y){$Y$}
\end{tikzpicture}
\end{center}

Commençons par remarquer que les droites $(DE)$ et $(AB)$ sont parallèles d'après le théorème de Thalès. Notons alors $k=\frac{CE}{CB}$.

\medskip

Soit $R$ le point d'intersection de $[DE]$ et $(CV)$ et $Y$ le point d'intersection des droites $(CV)$ et $(AE)$. On a donc $RE= VB \cdot k$ et $DR = k \cdot AV$.

Puisque le quadrilatère $DRUV$ est un parallélogramme, on a $UV=DR$. On veut donc montrer que $DR=VB$.

D'après le théorème de Thalès dans le papillon $REYAV$, $\frac{RE}{AV} = \frac{YE}{YA}$

Donc, on a

\[DR=kAV= k RE\cdot \frac{YA}{YE} = k^2 VB\cdot \frac{YA}{YE}\]

Il nous reste donc à montrer que $k^2 \cdot \frac{YA}{YE} = 1$.

Les triangles $ACE$,$CYE$ et $AYC$ sont semblables donc

\[ k = \frac{CE}{CB} = \frac{YE}{YC} = \frac{YC}{YA}\]


Donc $k^2 \cdot \frac{YA}{YE} = \frac{YE}{YC} \cdot \frac{YC}{YA} \cdot \frac{YA}{YE} = 1$, ce qui conclut.
\end{sol}

\begin{exo}
%problem.ru pb 108646

Soit $ABCD$ trapèze avec $(AD) \parallel (BC)$, $M$ l’intersection de ses diagonales et $P$ un point de $[BC]$ tel que $\widehat{APM}=\widehat{DPM}$. Montrer que la distance de $B$ à $(DP)$ est égale à la distance de $C$ à $(AP)$.
\end{exo}

\begin{sol}
\begin{center}
\begin{tikzpicture}
\tkzDefPoint(-1,3){A}
\tkzDefPoint(3,3){D}
\tkzDefPoint(0.5,0){P}

\tkzDefCircle[in](A,D,P) \tkzGetPoint{I}
\tkzDefPointBy[homothety=center P ratio 1.2](I) \tkzGetPoint{M}
\tkzDefLine[parallel=through P](A,D) \tkzGetPoint{b}
\tkzInterLL(P,b)(A,M) \tkzGetPoint{C}
\tkzInterLL(P,b)(D,M) \tkzGetPoint{B}
\tkzDefLine[perpendicular=through B](D,P) \tkzGetPoint{u}
\tkzInterLL(B,u)(D,P) \tkzGetPoint{U}
\tkzDefLine[perpendicular=through C](A,P) \tkzGetPoint{v}
\tkzInterLL(C,v)(A,P) \tkzGetPoint{V}
\tkzDefLine[perpendicular=through M](A,P) \tkzGetPoint{y}
\tkzInterLL(M,y)(A,P) \tkzGetPoint{Y}
\tkzDefLine[perpendicular=through M](D,P) \tkzGetPoint{x}
\tkzInterLL(M,x)(D,P) \tkzGetPoint{X}

\tkzMarkRightAngle[color=red](D,X,M)
\tkzMarkRightAngle[color=red](D,U,B)
\tkzMarkRightAngle[color=red](M,Y,A)
\tkzMarkRightAngle[color=red](C,V,A)
\tkzMarkAngle[color=blue,size=0.8](X,P,M)
\tkzMarkAngle[color=blue,size=0.7](M,P,Y)
\tkzDrawSegment(A,B)
\tkzDrawSegment(B,C)
\tkzDrawSegment(C,D)
\tkzDrawSegment(D,A)
\tkzDrawSegment(A,C)
\tkzDrawSegment(D,B)
\tkzDrawSegment(A,P)
\tkzDrawSegment(P,D)
\tkzDrawSegment(P,M)
\tkzDrawSegment(P,U)
\tkzDrawSegment(P,V)
\tkzDrawSegment(B,U)
\tkzDrawSegment(C,V)
\tkzDrawSegment(M,X)
\tkzDrawSegment(M,Y)
\tkzDrawPoints[fill=white](A,B,C,D,P,M,U,V,X,Y)

\tkzLabelPoint[above](A){$A$}
\tkzLabelPoint[left](B){$B$}
\tkzLabelPoint[right](C){$C$}
\tkzLabelPoint[above right](D){$D$}
\tkzLabelPoint[below](P){$P$}
\tkzLabelPoint[above](M){$M$}
\tkzLabelPoint[below](U){$U$}
\tkzLabelPoint[below](V){$V$}
\tkzLabelPoint(X){$X$}
\tkzLabelPoint[below left](Y){$Y$}
\end{tikzpicture}
\end{center}

Soit $U$ le projeté orthogonal de $B$ sur $(PD)$ et $V$ le projeté orthogonal de $C$ sur $(AP)$. Le but est de montrer que $BU=CV$. Introduisons de plus $X$ et $Y$, les projetés orthogonaux de $M$ sur $(DP)$ et $(AP)$ respectivement.

Comme

\[\widehat{PXM} = \widehat{PYM} = 90 ^ {\circ}\]

et

\[\widehat{YPM} =\widehat{XPM}\]

et que les triangles $PXM$ et $PYM$ partagent le côté $[PM]$, ces derniers sont isométriques, d'où $MX=MY$. Il en découle que nous avons gagné si on réussit à montrer que

\[\frac{MX}{BU}=\frac{MY}{CV}\]

Comme les droites $(MX)$ et $(BU)$ sont perpendiculaires à une même droite, elles sont parallélèles. Donc d'après le théorème de Thalès,

\[ \frac{DM}{DB}=\frac{MX}{BU}\]


De même,

\[\frac{AM}{AC}=\frac{MY}{CV}\]

Et on a
\[\frac{AM}{AC}=\frac{DM}{DC}\]

car $(AD) \parallel (BC)$, ce qui conclut.
\end{sol}

\begin{exo}
%Iran Round 3 2018
Soit $ABC$ un triangle et soient $D,E$ et $F$ les points de contact du cercle inscrit respectivement avec les côtés $[BC], [CA]$ et $[AB]$. Soit $P$ un point tel que $PF=FB$, les droites $(FP)$ et $(AC)$ sont parallèles et les points $P$ et $C$ sont dans le même demi-plan délimité par la droite $(AB)$. Soit $Q$ un point tel que $QE=EC$, les droites $(EQ)$ et $(AB)$ sont parallèles et les points $Q$ et $B$ sont dans le même demi-plan délimité par la droite $(AC)$. Montrer que les points $P, D$ et $Q$ sont alignés.
\end{exo}

\begin{sol}
\begin{center}
\begin{tikzpicture}
[scale=1]
\tkzInit[ymin=-2,ymax=7,xmin=-4,xmax=4]
\tkzClip

\tkzDefPoint(1,6){A}
\tkzDefPoint(3,0){B}
\tkzDefPoint(-3,0){C}
\tkzDefCircle[in](A,B,C) \tkzGetPoint{I}
\tkzDefLine[perpendicular=through I](B,C) \tkzGetPoint{d}
\tkzInterLL(d,I)(B,C) \tkzGetPoint{D}
\tkzDefLine[perpendicular=through I](C,A) \tkzGetPoint{e}
\tkzInterLL(e,I)(C,A) \tkzGetPoint{E}
\tkzDefLine[perpendicular=through I](A,B) \tkzGetPoint{f}
\tkzInterLL(f,I)(A,B) \tkzGetPoint{F}
\tkzDefLine[parallel=through F](A,C) \tkzGetPoint{p}
\tkzInterLC(p,F)(F,B) \tkzGetPoints{P}{p'}
\tkzDefLine[parallel=through E](A,B) \tkzGetPoint{q}
\tkzInterLC(q,E)(E,C) \tkzGetPoints{Q}{q'}

\tkzMarkAngle[color=red,size=0.5](C,A,B)
\tkzMarkAngle[color=red,size=0.5](P,F,B)
\tkzMarkAngle[color=red,size=0.5](C,E,Q)
\tkzMarkSegment[mark=s||](F,B)
\tkzMarkSegment[mark=s||](F,P)
\tkzMarkSegment[mark=s|](E,Q)
\tkzMarkSegment[mark=s|](E,C)

\tkzDrawSegment(A,B)
\tkzDrawSegment(B,C)
\tkzDrawSegment(C,A)
\tkzDrawCircle(I,E)
\tkzDrawSegment(F,P)
\tkzDrawSegment(E,Q)
\tkzDrawLine[densely dashed](P,Q)
\tkzDrawPoints[fill=white](A,B,C,D,E,F,P,Q)

\tkzLabelPoint(A){$A$}
\tkzLabelPoint(B){$B$}
\tkzLabelPoint(C){$C$}
\tkzLabelPoint(D){$D$}
\tkzLabelPoint(E){$E$}
\tkzLabelPoint(F){$F$}
\tkzLabelPoint(P){$P$}
\tkzLabelPoint(Q){$Q$}
\end{tikzpicture}
\end{center}

Le triangle $PFB$ est isocèle en $F$ et $\widehat{PFB}=\widehat{EAF}$ car les droites $(FP)$ et $(AC)$ sont parallèles. Donc les triangles $AEF$ et $PFB$ sont semblables. De même, on trouve que les triangles $QEC, EAF$ et $FPB$ sont semblables. On déduit que $\widehat{PBF}=90^\circ-\frac1{2}\widehat{BAC}=\widehat{QCE}$

Ainsi, par chasse aux angles, $$\widehat{PBC}=\widehat{CBA}-\widehat{PBF}=\widehat{CBA}-90^\circ+\frac1{2}\widehat{BAC}=\widehat{BCA}+90^\circ-\frac1{2}\widehat{BAC}$$
donc les droites $(PB)$ et $(CQ)$ sont parallèles. Puisque
$$\frac{PB}{CQ}=\frac{BF}{CE}=\frac{BD}{CD}$$ d'après la réciproque du théorème de Thalès, les points $P,Q$ et $D$ sont bien alignés.
\end{sol}

\begin{exo}
Soit $ABCD$ un trapèze avec $AB<CD$ et les droites $(AB)$ et $(CD)$ parallèles. Soit $P$ un point appartenant au segment $[CB]$. La parallèle à le droite $(AP)$ passant par le point $C$ coupe le segment $[AD]$ en le point $R$ et la parallèle à la droite $(DP)$ passant par le point $B$ coupe le segment $[AD]$ en le point $R'$. Montrer que $R=R'$.
\end{exo}

\begin{sol}
\begin{center}
\begin{tikzpicture}
[scale=1]
\tkzInit[ymin=-1,ymax=7,xmin=-5,xmax=5]
\tkzClip

\tkzDefPoint(-1,3){A}
\tkzDefPoint(3,3){B}
\tkzDefPoint(4,0){C}
\tkzDefPoint(-3.75,0){D}
\tkzDefPoint(3.33,2){P}

\tkzDefLine[parallel=through C](A,P) \tkzGetPoint{r}
\tkzInterLL(C,r)(A,D) \tkzGetPoint{R}
\tkzInterLL(B,C)(A,D) \tkzGetPoint{S}

\tkzDrawSegment[color=red](A,B)
\tkzDrawSegment[color=red](C,D)
\tkzDrawSegment(S,D)
\tkzDrawSegment(S,C)
\tkzDrawSegment[color=green](C,R)
\tkzDrawSegment[color=green](A,P)
\tkzDrawSegment[color=blue](R,B)
\tkzDrawSegment[color=blue](D,P)

\tkzDrawPoints[fill=white](A,B,C,D,P,S,R)

\tkzLabelPoint[left](A){$A$}
\tkzLabelPoint[right](B){$B$}
\tkzLabelPoint[right](C){$C$}
\tkzLabelPoint[left](D){$D$}
\tkzLabelPoint[right](P){$P$}
\tkzLabelPoint[above](S){$S$}
\tkzLabelPoint[left](R){$R=R'$}

\end{tikzpicture}
\end{center}

Soit $S$ le point d'intersection des droites $(CB)$ et $(AD)$. Comme les droites $(AP)$ et $(RC)$ sont parallèles, d'après le théorème de Thalès on a
$$\frac{AS}{RS}=\frac{PS}{CS}$$ soit $RS\cdot PS=AS\cdot CS$.
\\
Comme les droites $(DP)$ et $(BR')$ sont parallèles, d'après le théorème de Thalès
$$\frac{DS}{R'S}=\frac{PS}{BS}$$ soit $DS\cdot BS= PS\cdot R'S$.
\\
Enfin, les droites $(AB)$ et $(CD)$ sont parallèles donc d'après le théorème de Thalès,
$$\frac{AS}{DS}=\frac{BS}{CS}$$ soit $AS\cdot CS=BS\cdot DS$. On conclut que $PS\cdot RS=PS\cdot R'S$ donc $RS=R'S$ et comme les points $S,R$ et $R'$ sont sur la même droite, on trouve bien $R=R'$.
\end{sol}

\begin{exo}
%Pierre Haas
Soit $ABC$ un triangle et $X$ un point situé à l'intérieur du triangle $ABC$. Les droites $(AX), (BX)$ et $(CX)$ recoupent le cercle circonscrit au triangle $ABC$ en les points $P,Q$ et $R$ respectivement. Soit $U$ un point appartenant au segment $[XP]$. Les parallèles à $(AB),(AC)$ passant par $U$ coupent les droites $(XQ)$ et $(XR)$ en les points $V$ et $W$ respectivement. Montrer que les points $R,W,V,Q$ sont cocycliques.
\end{exo}

\begin{sol}
\begin{center}
\begin{tikzpicture}
[scale=1]
\tkzInit[ymin=-3,ymax=6,xmin=-4,xmax=5]
\tkzClip

\tkzDefPoint(1,5){A}
\tkzDefPoint(3,0){B}
\tkzDefPoint(-3,0){C}
\tkzDefPoint(0.5,2){X}

\tkzDefCircle[circum](A,B,C) \tkzGetPoint{O}
\tkzInterLC(A,X)(O,A) \tkzGetPoints{A}{P}
\tkzInterLC(B,X)(O,A) \tkzGetPoints{Q}{B}
\tkzInterLC(C,X)(O,A) \tkzGetPoints{C}{R}
\tkzDefMidPoint(X,P) \tkzGetPoint{u}
\tkzDefMidPoint(u,X) \tkzGetPoint{u'}
\tkzDefMidPoint(u',P) \tkzGetPoint{u''}
\tkzDefMidPoint(u'',P) \tkzGetPoint{U}
\tkzDefLine[parallel=through U](A,B) \tkzGetPoint{v}
\tkzInterLL(v,U)(X,Q) \tkzGetPoint{V}
\tkzDefLine[parallel=through U](A,C) \tkzGetPoint{w}
\tkzInterLL(w,U)(X,R) \tkzGetPoint{W}
\tkzDefCircle[circum](R,V,W) \tkzGetPoint{o}


\tkzDrawSegment[color=green](A,B)
\tkzDrawSegment(B,C)
\tkzDrawSegment[color=red](C,A)
\tkzDrawSegment(A,P)
\tkzDrawSegment(B,Q)
\tkzDrawSegment(C,R)
\tkzDrawSegment[color=green](U,V)
\tkzDrawSegment[color=red](U,W)
\tkzDrawSegment(R,W)
\tkzDrawSegment(V,Q)
\tkzDrawCircle[densely dashed](o,R)
\tkzDrawCircle(O,A)
\tkzDrawPoints[fill=white](A,B,C,X,P,Q,R,U,V,W)

\tkzLabelPoint[above](A){$A$}
\tkzLabelPoint(B){$B$}
\tkzLabelPoint[below left](C){$C$}
\tkzLabelPoint[right](X){$X$}
\tkzLabelPoint[below](P){$P$}
\tkzLabelPoint[left](Q){$Q$}
\tkzLabelPoint[above](R){$R$}
\tkzLabelPoint(U){$U$}
\tkzLabelPoint[below left](V){$V$}
\tkzLabelPoint(W){$W$}
\end{tikzpicture}
\end{center}

Nous allons traduire les diverses hypothèses en terme d'égalité de rapport.

Puisque les droites $(UV)$ et $(AB)$ sont parallèles, par le théorème de Thalès, $\frac{AX}{UX}=\frac{BX}{VX}$.
Puisque les droites $(UW)$ et $(AC)$ sont parallèles, par le théorème de Thalès, $\frac{AX}{UX}=\frac{CX}{WX}$.

On obtient $\frac{CX}{WX}=\frac{BX}{VX}$ soit $\frac{CX}{BX}=\frac{WX}{VX}$. Par puissance du point $X$ par rapport au cercle circonscrit au triangle $ABC$, on a aussi $BX\cdot QX=CX\cdot RX$ donc $\frac{CX}{BX}=\frac{QX}{RX}$.

On obtient donc $\frac{WX}{VX}=\frac{QX}{RX}$, ou encore $WX\cdot RX=VX\cdot QX$ ce qui signifie, par réciproque de la puissance d'un point, que les points $Q,V,R$ et $W$ sont cocycliques.

\end{sol}
