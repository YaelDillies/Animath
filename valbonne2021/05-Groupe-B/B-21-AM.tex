Ce cours est essentiellement tiré de l'incroyable cours de Jean-Louis Tu que vous pouvez retrouver à ce lien : \url{https://maths-olympiques.fr/wp-content/uploads/2017/09/arith_base.pdf}. On se limite aujourd'hui aux parties 2 et 4.1.\newline\newline


%Exercices


\begin{exo}
Montrer que pour tout entier $n$, $\:\dfrac{21n+4}{14n+3}$ est irréductible.
\end{exo}
\begin{exo}
Trouver les entiers relatifs $n$ tels que $n^5-2n^4-7n^2-7n+3=0$.
\end{exo}
\begin{exo}
Quels sont les nombres premiers $p$ tels que $p+2$ et $p+4$ soient également premiers ?
\end{exo}
\begin{exo}
Soient $a$ et $b$ des entiers premiers entre eux, montrer que $a+b$ et $ab$ sont premiers entre eux.
\end{exo}
\begin{exo}
Pour quels entiers strictement positifs $n$ a-t-on $5^{n-1}+3^{n-1}\mid 5^n+3^n$ ?
\end{exo}
\begin{exo}
Trouver les $n$ tels que $n^2+1\mid n^5+3$
\end{exo}
\begin{exo}
Quels sont les entiers positifs $n$ tels que $n+2009\mid n^2+2009$ et $n+2010\mid n^2+2010$ ?
\end{exo}
\begin{exo}
Existe-t-il des rationnels positifs $q$ tels que $q^3-10q^2+q-2021=0.$ \\(\emph{Indice :} $45^2=2025$)
\end{exo}
\begin{exo}
Soient $a,b$ des entiers premiers entre eux.
Montrer que $$\mathrm{pgcd}(a^n-b^n,a^m-b^m)=a^{\mathrm{pgcd}(m,n)}-b^{\mathrm{pgcd}(m,n)}.$$
\end{exo}
\begin{exo}
Trouver toutes les fonctions $f:\mathbb N^*\rightarrow\mathbb N^*$ telles que $f(1)+f(2)+\cdots+f(n)\mid 1+2+\cdots n$ pour tout $n>0.$
\end{exo}
%Solutions

\begin{sol}
On a $3(14n+3)-2(21n+4)=1,$ donc par Bézout $21n+4$ et $14n+3$ sont premiers entre eux.
\end{sol}
\begin{sol}
L'équation se réécrit $n(n^4-2n^3-7n-7)=-3,$ en particulier $n\mid -3$. $n$ n'a donc plus que $4$ valeurs possibles : $-3,-1,1,3.$ On teste ces valeurs à la main, et on trouve que seuls $-1$ et $3$ sont solutions.
\end{sol}
\begin{sol}
Il est connu que parmi trois nombres consécutifs, l'un d'entre eux sera un multiple de $3$. L'idée ici est la même, et on va raisonner comme on le ferait pour des nombres consécutifs. On sait que $p$ s'écrit sous la forme $3k,\:3k+1$ ou $3k+2$. Traitons ces trois cas :\\
\begin{itemize}
    \item Si $p=3k$, $p$ est divisible par $3$ et est premier, donc $p=3.$ Réciproquement, $3,5$ et $7$ sont premiers.
    \item Si $p=3k+1,$ $p+2$ est divisible par $3$, donc $p+2=3$ et $p=1$, qui n'est pas premier.
    \item Si $p=3k+2,$ $p+4$ est divisible par $3$, donc $p+4=3$ et $p=-1$, qui n'est pas premier.
\end{itemize}
Ainsi, $p=3$ est la seule valeur qui marche.
\end{sol}
\begin{sol}
Soit $d$ un diviseur commun de $a+b$ et $ab.$ Alors $d\mid b(a+b)-ab=b^2.$ De même, $d\mid a(a+b)-ab=a^2,$ donc $d\mid \mathrm{pgcd}(a^2,b^2)=1,$ d'où $ab$ et $a+b$ sont bien premiers entre eux.
\end{sol}
\begin{sol}
Si $5^{n-1}+3^{n-1}\mid 5^n+3^n,$ alors $5^{n-1}+3^{n-1}\mid 5^n+3^n-5(5^{n-1}+3^{n-1})=-2\times 3^{n-1}.$ Ainsi, $5^{n-1}+3^{n-1}\leqslant 2\times 3^{n-1},$ d'où $5^{n-1}\leqslant 3^{n-1}$ et on en déduit que $n=1$. Réciproquement, pour $n=1$, $2$ divise bien $8$ et $n=1$ est notre seule solution.
\end{sol}
\begin{sol}
On a $n(n^2+1)(n^2-1)=n^5-n,$ donc $n^2+1\mid n^5+3-(n^5-n)=n+3.$ Or pour $n\ge 3,\:n^2+1\geqslant 3n+1\geqslant n+6+1>n+3,$ donc $n\leqslant 2.$ Réciproquement, $n=0,1,2$ sont tous trois solutions.
\end{sol}
\begin{sol}
On a $n+2009\mid n^2+2009-(n+2009)=n^2-n$ et de même $n+2010\mid n^2-n.$ or $n+2009$ et $n+2010$ sont consécutifs donc premiers entre eux, donc $(n+2009)(n+2010)\mid n^2-n.$ Or $n\geqslant 0,$ donc $(n+2009)(n+2010)> n^2-n$. La seule possibilité est donc que $n^2-n=0,$ soit $n=0$ ou $n=1$, qui sont réciproquement bien solutions.
\end{sol}
\begin{sol}
Écrivons $q=\frac ab$ avec $a,b$ premiers entre eux. On a donc $(\frac ab)^3-10(\frac ab)^2+(\frac ab)-2021=0,$ et en multipliant par $b^3,$ on trouve $a^3-10a^2b+ab^2-2021b^3=0.$ Alors $b(-10a^2+ab-2021b^2)=-a^3,$ donc $b\mid a^3$. Or $a$ et $b$ sont premiers entre eux, donc $b=1.$\\
De même, $a(a^2-10ab+b^2)=2021b^3$, donc $a\mid 2021.$ Or $2021=2025-4=45^2-2^2=(45-2)(45+2)=43\times 47,$ donc $q=a=1,43,47,2021.$ Réciproquement, aucune de ces valeurs ne fonctionne (On vérifie à la main pour $1$, et pour les autres valeurs, $q^3$ est beaucoup trop grand pour que les autres termes puissent l'annuler.
\end{sol}
\begin{sol}
On peut supposer sans perte de généralités que $m\ge n$. On écrit la division euclidienne de $m$ par $n$ : $m=nq+r$, $0\le r<n$. \\
On a :
\begin{eqnarray*}
\mathrm{pgcd}(a^m-b^m,a^n-b^n)
&=&\mathrm{pgcd}(a^n-b^n,(a^m-b^m)-(a^n-b^n)a^{m-n})\\
&=&\mathrm{pgcd}(a^n-b^n,b^n(a^{m-n}-b^{m-n}))\\
&=&\mathrm{pgcd}(a^n-b^n,a^{m-n}-b^{m-n})\quad\text{($b^n$ est premier avec $a^n-b^n$)}\\
&&\quad \vdots\\
&=&\mathrm{pgcd}(a^n -b^n,a^r -b^r) \:\:\text{en répétant les opérations précédentes $q$ fois}
\end{eqnarray*}
On peut donc suivre la même méthode que pour l'algorithme d'Euclide pour obtenir le pgcd de $m$ et $n$ dans ce cas là, et on a bien $$\mathrm{pgcd}(a^m-b^m,a^n-b^n)=\mathrm{pgcd}(a^{\mathrm{pgcd}(m,n)}-b^{\mathrm{pgcd}(m,n)},a^0-b^0)=a^{\mathrm{pgcd}(m,n)}-b^{\mathrm{pgcd}(m,n)}$$.
\end{sol}
\begin{sol}
On a $f(1)\mid 1,$ donc $f(1)=1.$ Ensuite, $1+f(2)\mid 3$, et $f(2)=2$ car $f(2)>0.$ Ensuite, $3+f(3)\mid 6,$ et encore une fois $f(3)=3$ car $f(3)>0.$ On peut donc se dire que $f(n)=n$ pour tout $n$, ce qu'on va montrer par récurrence forte.\\
Initialisation : On vient de montrer que $f(1)=1,f(2)=2,f(3)=3.$\\
Hérédité : Soit $n\geqslant 3$ tel que $f(i)=i$ pour tout $i\leqslant n$.\\ Alors $1+2+\cdots+n+f(n+1)\mid 1+2+\cdots n+1,$ donc $\frac{n(n+1)}2+f(n+1)\mid \frac{(n+1)(n+2)}2$. \\
Or pour $n\geqslant 3,\:\dfrac{\frac{n(n+1)}2}{\frac{(n+1)(n+2)}2}=\dfrac n{n+2}=1-\dfrac2{n+2}>\dfrac12,$ donc $\frac{n(n+1)}2+f(n+1)>\frac{n(n+1)}2>\frac12\frac{(n+1)(n+2)}2$. La seule valeur entière possible pour $\dfrac{\frac{(n+1)(n+2)}2}{\frac{n(n+1)}2+f(n+1)}$ est donc $1$, ou encore $f(n+1)=n+1$, et notre hérédité est finie.\\Ainsi notre récurrence tient, et $f(n)=n$ pour tout $n>0.$
\end{sol}