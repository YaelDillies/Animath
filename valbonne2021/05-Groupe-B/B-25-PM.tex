

\textbf{Attention :} Certains exercices de cette feuille peuvent être difficiles et nécessitent une habitude dans l'usage des sommes ($\sum$) et des logarithmes ($\ln$).


\subsubsection{Pour les impatientes et impatients}


On verra dans le cours que

\begin{itemize}
\item pour $p$ un nombre premier et $n$ un entier, $v_p(n)$ est l'exposant de $p$ dans la décomposition en facteurs premiers de $n$,
\item le logarithme est une fonction croissante définie sur les réels strictement positifs tels que pour tous $a$ et $b$, $\ln(ab) = \ln(a) + \ln(b)$,
\item la partie entière de d'un réel $x$ est l'unique entier vérifiant $\lfloor x \rfloor \le x < \lfloor x \rfloor +1$,
\item les symboles $\sum$ et $\prod$ permettent de se passer des "$\cdots$" et d'effectuer des calculs efficaces: $a_1 + a_2 + \cdots + a_n = \sum_{k=1}^n a_k$ et $a_1 \cdot a_2 \cdot \cdots \cdot a_n = \prod_{k=1}^n a_k$.
\item pour un réel $x$, $\pi(x)$ est le nombre de nombres premiers inférieurs à $x$.
\end{itemize}


\subsubsection{Introduction à la valuation $p$-adique}

\begin{exo}
Soient $m$ et $n$ deux entiers.

Montrer que
$$m | n$$

si et seulement si
$$\text{ pour tout } p \text{ premier, } v_p(m) \le v_p(n)$$
\end{exo}

\begin{exo}
Soit $p$ un nombre premier.

Soient $m$ et $n$ deux entiers.

Exprimer $v_p(\operatorname{pgcd}(m,n))$ et $v_p(\operatorname{ppcm}(m,n))$ en fonction de $v_p(m)$ et $v_p(n)$.

En déduire une nouvelle démonstration de
$$\operatorname{pgcd}(m,n) \operatorname{ppcm}(m,n) = m n$$
\end{exo}



\begin{exo}
Soit $d$ un entier et $r$ un entier.

Montrer que si l'équation
$$a^r = d b^r$$
admet une solution avec $(a,b) \neq (0,0)$, alors $d$ est la puissance $r$-ième d'un entier.
\end{exo}

\begin{exo}
On rappelle que pour un entier $m$ on note $\varphi(m)$ le nombre de nombres inférieurs à $m$ premiers avec $m$, et que si $a$ et $b$ sont premiers entre eux alors $\varphi(ab) = \varphi(a) \varphi(b)$.
\begin{enumerate}
\item Soit $n$ un entier admettant $p_1,\ldots,p_r$ comme facteurs premiers. Montrer que
  $$\sum_{d | n} \varphi(d) = \prod_{k=1}^r \sum_{j=0}^{v_{p_k}(n)} \varphi({p_k}^j)$$
\item En déduire que
  $$\sum_{d | n} \varphi(d) = n$$
\end{enumerate}
\end{exo}

\subsubsection{Partie entière, logarithme}

\begin{exo}
\begin{enumerate}
\item Montrer que pour tout réel $x$,
  $$x-1 < \lfloor x \rfloor \le x$$
\item Montrer que pour tout réel $x$,
  $$\lfloor 2 x \rfloor  - 2 \lfloor x \rfloor \in \{0,1\}$$
\end{enumerate}
\end{exo}

\begin{exo}
Montrer que pour tout entier $n$,
$$n \ln(2) \le \ln \left(\binom{2n} n \right) \le n \ln(4)$$
\end{exo}


\begin{exo}
\begin{enumerate}
\item Soit $n$ un entier. Montrer que pour tout nombre premier $p$
$$v_p(n) \le \frac{\ln n}{\ln p}$$
\item Soient $p_1, p_2, \ldots, p_r$ des nombres premiers distincts.

Pour un entier $n$, on note $F(n)$ le nombre d'entiers inférieurs à $n$ dont tous les diviseurs premiers sont parmi les parmi les $(p_i)_{1 \le i \le r}$.

Montrer que pour tout $n$

$$F(n) \le \left(\frac{\ln n}{\ln(2)}+1 \right)^r$$
\item On rappelle que lorsque $n \to + \infty$,
$$\frac{\ln n} n \to 0$$
En déduire une preuve combinatoire de l'existence d'une infinité de nombres premiers.
\end{enumerate}
\end{exo}

\subsubsection{Formule de Legendre}

\begin{exo}[Formule de Legendre]
Soit $p$ un nombre premier et $n$ un entier.

Montrer que
$$v_p(n!) = \sum_{k=1}^{\left\lfloor \frac{\ln n}{\ln p} \right\rfloor} \left\lfloor \frac n{p^k} \right\rfloor$$
\end{exo}


\begin{exo}
 On pose
$$N = 2021!$$
\begin{enumerate}
\item Proposer (sans la réaliser) une méthode pour calculer le nombre de chiffres de l'écriture décimale de $2021!$ à l'aide de la fonction $\ln$.

\textit{\small Le résultat est que l'écriture décimale de $2021!$ comporte $5805$ chiffres. Pour comparaison l'âge de l'univers en seconde peut s'écrire à l'aide de $18$ chiffres.}
\item Par combien de ``$0$''se termine l'écriture décimale de $N$ ?
\end{enumerate}
\end{exo}


\begin{exo}
\begin{enumerate}
\item Montrer que pour tous réels $\alpha$ et $\beta$
$$\lfloor 2 \alpha \rfloor  + \lfloor  2\beta\rfloor \ge \lfloor \alpha \rfloor + \lfloor \beta \rfloor + \lfloor \alpha + \beta \rfloor. $$
\item En déduire que pour tous entiers $n$ et $m$
$$ \frac{(2m)! (2n)!}{m! n! (m+n)!}$$
est un entier.
\end{enumerate}
\end{exo}

\subsubsection{Vers des estimées sur $\pi$}

\begin{exo}
On rappelle le point clé de la preuve d'Euclide de l'existence d'une infinité de nombres premiers
\begin{quote}
Si $p_1, p_2, \ldots, p_r$ sont des nombres premiers alors
$$\prod_{k=1}^r p_k +1$$
admet un diviseur premier qui n'est pas parmi les $(p_i)_{1 \le i \le r}$.


\end{quote}
En adaptant cet argument montrer que le $n$-ième nombre premier est inférieur à $2^{2^n}$ et en déduire qu'il existe une constante $c>0$telle que, pour tout réel $x$,

$$\pi(x) \ge c \ln(\ln(x))$$
\end{exo}


\begin{exo}
Soient $p_1, p_2, \ldots, p_r$ des nombres premiers distincts.

Pour un entier $n$, on note $F(n)$ le nombre d'entiers inférieurs à $n$ dont tous les diviseurs premiers sont parmi les parmi les $(p_i)_{1 \le i \le r}$.

Montrer que pour tout $n$,

$$F(n) \le 2^r \sqrt n$$
et en déduire qu'il existe une constante $c>0$ telle que, pour tout réel $x$,
$$\pi(x) \ge c \ln(x)$$
\end{exo}

\begin{exo}
Soient $n$ un entier naturel et $p$ un nombre premier.

Montrer à l'aide de la formule de Legendre (exercice~\ref{exo-IV-5-2-8})
$$v_p\left( \binom{2n} n\right) = \sum_{k=1}^{\left\lfloor \frac{\ln(2n)}{\ln p} \right\rfloor} \left(\left\lfloor \frac{2n}{p^k} \right\rfloor - 2 \left\lfloor \frac n{p^k} \right\rfloor  \right)$$
puis à l'aide de l'exercice~\ref{exo-IV-5-2-5} en déduire que
$$v_p\left( \binom{2n} n\right) \le \left\lfloor \frac{\ln(2n)}{\ln p} \right\rfloor$$
\end{exo}

\begin{exo}
    % \begin{enumerate}
    % \item À l'aide la formule de Legendre (exercice~\ref{exo-IV-5-2-8}) montrer que pour tout entier $n$
    % $$\ln \left( \binom{2n} n\right) = \sum_{\substack{p \text{ premier} \\ p \le 2n }} \sum_{k=1}^{\left\lfloor \frac{\ln(2n)}{\ln p} \right\rfloor} \left(\left\lfloor \frac{2n}{p^k} \right\rfloor - 2 \left\lfloor \frac n{p^k} \right\rfloor  \right) \ln p$$
    % \item En utilisant l'exercice~\ref{exo-IV-5-2-5} montrer que pour tout entier $n$
    % $$\sum_{\substack{p \text{ premier} \\ p \le 2n }} \sum_{k=1}^{\left\lfloor \frac{\ln(2n)}{\ln p} \right\rfloor} \left(\left\lfloor \frac{2n}{p^k} \right\rfloor - 2 \left\lfloor \frac n{p^k} \right\rfloor  \right) \ln p \le \ln(2n) \pi(2n)$$
    En utilisant l'exercice~\ref{exo-IV-5-2-13} montrer que
    $$\ln \left( \binom{2n} n\right) \le \sum_{\substack{p \text{ premier} \\ p \le 2n }} \left\lfloor \frac{\ln(2n)}{\ln p} \right\rfloor \ln p$$ puis en déduire à l'aide de l'exercice~\ref{exo-IV-5-2-6} qu'il existe une constante $c>0$telle que pour tout réel $x$,
    $$\pi(x) \ge c \frac{x}{\ln(x)}$$
    % \end{enumerate}
\end{exo}


\begin{exo}
Il a été montré que (théorème des nombres premiers)
$$\lim_{x \to + \infty} \frac{\pi(x)}{\left(\frac{x}{\ln(x)}\right)} = 1$$

Comparer les résultats des exercices~\ref{exo-IV-5-2-11},~\ref{exo-IV-5-2-12} et~\ref{exo-IV-5-2-13} entre eux, et les comparer le théorème des nombres premiers.
\end{exo}

%%%%%%%%
%% Les exercices suivant n'ont pas été donnés en TD pendant le stage (la feuille était déjà bien trop longue).
%% Il faudrait que je trouve le courage de rédiger une correction.
%% Une référence est le Petit compagnon des nombres de Boyer entre les pages 172 et 193
%%%%%%%%



% \subsubsection{Valuation $p$-adique de $\binom{2n} n$: exercices supplémentaires}

% On rappelle le résultat de l'exercice~\ref{exo-IV-5-2-13 :}

% \begin{quote}
%     Si $p$ est un nombre premier et $n$ un entier alors
%     $$v_p\left( \binom{2n} n\right) \le \left\lfloor \frac{\ln(2n)}{\ln p} \right\rfloor$$
% \end{quote}

% \begin{exo}
%     Montrer que pour tout entier $n$,

%     $$\operatorname{ppcm} \left(1,2,\cdots,n \right) \ge 2^n$$
% \end{exo}

% \begin{exo}
%     Montrer que pour tout réel positif $x$,
%     $$\sum_{\substack{p \text{ premier } \\ p \le x}} \ln p \le \ln(4) x$$
% \end{exo}

% \begin{exo}
%     L'objectif de cet exercice est de démontrer le théorème de Bertrand:
%     \begin{quote}
%         Pour tout entier $n$ il existe un nombre premier compris entre $n$ et $2n$.
%     \end{quote}

%     Dans la suite $n \ge 5$ est un entier fixé, et on note
%     $$\alpha_n = \# \left\lbrace p \text{ premier }, n \le p \le 2n \right\rbrace$$
%     \begin{enumerate}
%         \item Démontrer que pour tout nombre premier $p$,
%               $$\text{ si }\frac{2n}{3} < p < n ~~ \text{ alors }\ v_p\left( \binom{2n} n\right)=0$$
%         \item En déduire que
%               $$\binom{2n} n \le (2n)^{\sqrt{2n}} \left(\prod_{\substack{p \text{ premier } \\p \le \frac{2n}{3}}} p \right) (2n)^{\alpha_n}$$
%         \item Expliquer comment déduire de la question précédente une preuve du théorème de Bertrand.
%     \end{enumerate}
% \end{exo}



%%%%%%%
%% Solutions
%%%%%%%

\subsubsection{Solutions}


\begin{sol}
\framebox{$\Rightarrow$} Supposons que $m | n$, c'est à dire qu'il existe un entier $q$ tel que
$$n = m q$$
Soit $p$ un nombre premier. En considérant la valuation $p$-adique de l'égalité précédente on obtient
$$v_p(n) = v_p(m q)$$
c'est à dire
$$v_p(n) = v_p(m) + v_p( q)$$

Comme $v_p(q) \ge 0$ on en déduit qu'en particulier
$$v_p(n) \ge v_p(m)$$

Et cela est vrai quelquesoit le nombre premier $p$.

\framebox{$\Leftarrow$} Supposons que pour tout nombre premier $p$,
$v_p(m) \le v_p(n)$. Notons $p_1, p_2, \ldots, p_r$ les facteurs premiers de $n$.

L'idée est de s'inspirer du sens direct et de ``compléter l'écart'' entre les valuations $p$-adiques de $m$ et $n$.

Posons

$$q = \prod_{k=1}^r p_k^{v_{p_k}(n)-v_{p_k}(m)}$$

qui est bien défini car pour tout $k$, $v_{p_k}(n)-v_{p_k}(m) \ge 0$.

Alors on a

$$\begin{aligned}
        q m & = \prod_{k=1}^r p_k^{v_{p_k}(n)-v_{p_k}(m)} ~ \cdot ~\prod_{k=1}^r p_k^{v_{p_k}(m)} \\
            & = \prod_{k=1}^r p_k^{v_{p_k}(n)-v_p(m)+v_p(m)} = n                                   \\
            & = \prod_{k=1}^r p_k^{v_{p_k}(n)}                                                     \\
            & = n
    \end{aligned}$$
et donc
$$m | n$$
\end{sol}


\begin{sol}
On peut utiliser l'exercice précédent, en effet le pgcd de $m$ et $n$ est le plus grand entier $d$ tel $d |m$ et $d| n$. C'est à dire le plus grand entier tel que pour tout $p$ premier, $v_p(d)\le v_p(m)$ et $v_p(d) \le v_p(n)$. On reconnaît alors la définition du minimum.

Alternativement on peut utiliser la décomposition en facteurs premiers du $\operatorname{pgcd}$.

Dans tous les cas on obtient
$$\boxed{v_p\left(\operatorname{pgcd}(m,n)\right) = \min \left(v_p(m),v_p(n)\right)}$$
et de même
$$\boxed{v_p\left(\operatorname{pgcd}(m,n)\right) = \max \left(v_p(m),v_p(n)\right)}$$

Par conséquent, comme pour tous réels $x$ et $y$, $\min(x,y) + \max(x,y) = x+y$, on a pour tout nombre premier $p$,
$$\begin{aligned}
        v_p(\operatorname{pgcd}(m,n) \operatorname{ppcm}(m,n)) & = \min \left(v_p(m),v_p(n)\right) +\max \left(v_p(m),v_p(n)\right) \\
                                    & = v_p(m) + v_p(n)                                                  \\
                                    & = v_p(mn).
    \end{aligned}$$
Comme cela est vrai pour tout nombre premier $p$ on en conclut bien que
$$\boxed{\operatorname{pgcd}(m,n) \operatorname{ppcm}(m,n) = m n}$$
\end{sol}


\begin{sol}
Supposons qu'il existe $a$ et $b$ non nuls tels que
$$a^r = d b^r$$

Alors pour tout nombre premier $p$ on a
$$v_p(a^r) = v_p(d b^r)$$
et donc
$$r v_p(a) = v_p(d)+r v_p(b)$$
On en déduit que $r | v_p(d)$ et c'est à dire que
$$\frac{v_p(d)}{r} \in \N$$
Or
$$ d = \prod_{ p \text{ premier }} p^{v_p(d)}$$
et donc en définissant l'entier $\delta$ par
$$ \delta= \prod_{ p \text{ premier }} p^{\frac{v_p(d)}{r}}$$
on obtient
$$ \boxed{d = \delta^r}$$
ce qui est bien le résultat recherché.

\underline{Remarque :}
On a donc montré que pour tout entier $d$ et tout entier $r$, sa racine $r$-ième,  $\sqrt[r]{d}$ est soit entière, soit irrationnelle.
\end{sol}


\begin{sol}
\begin{enumerate}
\item  En développant le produit de gauche et en utilisant la multiplicativité de $\varphi$ on obtient
$$\begin{aligned}
  \prod_{k=1}^r \sum_{j=0}^{v_{p_k}(n)} \varphi({p_k}^j) & = \sum_{\substack{ 0 \le j_1 \le v_{p_1}(n) \\ 0 \le j_2 \le v_{p_2}(n) \\ \vdots\\  0 \le j_r \le v_{p_r}(n)
      }}\prod_{k=1}^r \varphi({p_k}^{j_k})                                                               \\
                                                          & = \sum_{\substack{ 0 \le j_1 \le v_{p_1}(n) \\ 0 \le j_2 \le v_{p_2}(n) \\ \vdots\\  0 \le j_r \le v_{p_r}(n)
      }}\varphi\left(\prod_{k=1}^r {p_k}^{j_k}\right)
\end{aligned} $$
Or l'ensemble des diviseurs de $n$ est l'ensemble des $\prod_{k=1}^r {p_k}^{j_k}$ tels que pour tout $k$, $0 \le j_k \le v_{p_k}(n)$.

On a donc bien
$$\boxed{\prod_{k=1}^r \sum_{j=0}^{v_{p_k}(n)} \varphi({p_k}^j) = \sum_{d | n} \varphi(d)}$$
\item On a de plus par un comptage que pour tout nombre premier $p$ et toute puissance $j\ge 1$,
$$\varphi(p^j) = p^j-p^{j-1}$$

Par conséquent pour tout entier $k$, (il s'agit d'une \emph{somme télescopique})
$$\begin{aligned}
  \sum_{j=1}^{v_{p_k}(n)} \varphi({p_k}^j) & = \sum_{j=1}^{v_{p_k}(n)} \left( {p_k}^j - {p_k}^{j-1}\right)                                                                                                         \\
                                          & = \left(p^{v_{p_k}(n)} - p^{v_{p_k}(n)-1}\right) + \left(p^{v_{p_k}(n)-1} - p^{v_{p_k}(n)-2}\right) + \cdots + \left(p^{2} - p^{1}\right)+ \left(p^{1} - p^{0}\right) \\
                                          & = p^{v_{p_k}(n)}-1
\end{aligned} $$
et donc
$$\begin{aligned}
  \sum_{j=0}^{v_{p_k}(n)} \varphi({p_k}^j) = \sum_{j=1}^{v_{p_k}(n)} \varphi({p_k}^j) +1  = p^{v_{p_k}(n)}.
\end{aligned}$$

On obtient finalement
$$ \sum_{d | n} \varphi(d) = \prod_{k=1}^r p^{v_{p_k}(n)}$$
c'est à dire
$$\boxed{ \sum_{d | n} \varphi(d) = n}$$
\end{enumerate}
\end{sol}

\begin{sol}
\begin{enumerate}
\item Par définition on a $\lfloor x \rfloor \le x$ et $x <\lfloor x \rfloor +1$, i.e $x -1 <\lfloor x \rfloor $. D'où le résultat.
\item Pour tout réel $x$ de par la question précédente on a
$$2x-1 < \lfloor 2x  \rfloor \le 2x$$
et de plus, comme $x-1 < \lfloor x  \rfloor \le x$,
on a aussi, en multipliant par $-2 < 0$
$$-2x \le 2 \lfloor x  \rfloor < -2x-2$$

En sommant les deux encadrement (et en remarquant qu'une des deux inégalités est stricte) on obtient
$-1 <\lfloor 2x  \rfloor - 2 \lfloor x  \rfloor< 2$
c'est à dire, comme $\lfloor 2x  \rfloor - 2 \lfloor x  \rfloor$ est un entier,
$$\boxed{\lfloor 2 x \rfloor  - 2 \lfloor x \rfloor \in \{0,1\}}$$

\underline{Remarque :} on aurait aussi pu procéder par disjonction de cas, selon que $x - \lfloor x \rfloor$ est dans $\left[0,\frac{1}{2}\right[$ ou dans $\left[\frac{1}{2},1\right[$.
\end{enumerate}
\end{sol}

\begin{sol}
On peut utiliser des arguments combinatoires, ou alternativement effectuer un simple calcul.

Tout d'abord, $\binom{2n} n \le 2^{2n} = 4^n$ donc 
$$\boxed{\ln \left(\binom{2n} n \right) \le n \ln(4)}$$
Pour la minoration on peut remarquer que
$$\begin{aligned}\binom{2n} n &= \frac{(2n) (2n-1) \cdots (n+1) }{n (n-1) \cdots 1}\\
&= \frac{\prod_{k=1}^ n (n+k)}{\prod_{k=1}^n k} \\
&= \prod_{k=1}^n \left(\frac n{k}+1 \right) \end{aligned}$$
or pour tout $k$ entre $1$ et $n$, $\frac n{k}+1 \ge 2 >0$. Donc par produit d'inégalités positives
$$\binom{2n} n \ge \prod_{k=1}^n 2 = 2^n$$
ce qui donne bien 
$$\boxed{\ln \left(\binom{2n} n \right) \ge n \ln(2)}$$
\end{sol}

\begin{sol}
\begin{enumerate}
\item Soit $p$ un nombre premier.

Comme $p^{v_p(n)}$ est un facteur de la décomposition de $n$ en produit de nombres premiers et que les autres facteurs sont plus grands que $1$, on a en particulier
$$p^{v_p(n)} \le n$$
en passant au $\ln$ on déduit que
$$\boxed{v_p(n) \le \frac{\ln n}{\ln p}}$$

\item Les entiers dont tous les diviseurs sont parmi $p_1,\ldots,p_r$ et plus petits que $n$ sont

TODO

$$\boxed{F(n) \le \left(\frac{\ln n}{\ln(2)}+1 \right)^r}$$
\item Supposons par l'absurde qu'il n'existe qu'un nombre fini de nombres premiers et soient $p_1,\ldots,p_r$ tous les nombres premiers.

Alors comme tout entier est produit de facteurs premiers on en déduit que pour tout $n$
$$F(n) = n$$

On a donc pour tout entier $n$
$$n \le \left(\frac{\ln n}{\ln(2)}+1 \right)^r$$
ce qui est une contradiction quand $n \to + \infty$.

Plus précisément cela impliquerait que
$$n^\frac{1}{r} \le 2\frac{r \ln\left(n^\frac{1}{r} \right)}{\ln(2)}$$
et donc que
$$0 <\frac{\ln(2)}{2r} \ge \frac{\ln(n^\frac{1}{r})}{n^\frac{1}{r}}$$
ce qui contredit
$$\frac{\ln(n^\frac{1}{r})}{n^\frac{1}{r}} \to 0$$

Par l'absurde on a donc \framebox{il existe une infinité de nombres premiers}.

\underline{Remarque :} on peut remarquer que cette preuve utilise peu d'arguments arithmétiques et est essentiellement combinatoire. De plus on voit que travailler l'inégalité précédente permet d'obtenir une minoration de $r$. Par conséquent en quantifiant cette preuve on peut en fait obtenir une \emph{minoration du nombre de nombre premier plus petits que $n$} (pour des méthodes similaires voir les exercices~\ref{exo-IV-5-2-11},~\ref{exo-IV-5-2-12},~\ref{exo-IV-5-2-13} et~\ref{exo-IV-5-2-14}).
\end{enumerate}
\end{sol}


\begin{sol}
En remarquant que
$$v_p(n!) = \sum_{k=1}^n v_p(k)$$
on se ramène à un exercice de combinatoire.

Une façon de procéder est d'utiliser le fait que pour toute fonction $f : \N \to \N$ on a (faire un schéma)
$$\sum_{k=1}^n f(k) = \sum_j \# \left\lbrace k \in \{1,\ldots,n\} \mid f(k) \ge j\right\rbrace$$

Or pour tout $j$  on a $\# \left\lbrace k \in \{1,\ldots,n\} \mid v_p(k) \ge j\right\rbrace$ est le nombre d'entiers entre $1$ et $n$ divisible par $p^j$ c'est à dire
$$\left\lfloor \frac n{p^j} \right\rfloor$$
Ce terme valant en particulier  $0$ dès que $p^j > n$ i.e $j > \left\lfloor \frac{\ln n}{\ln p} \right\rfloor$ on en déduit donc

$$\boxed{v_p(n!) = \sum_{j=1}^{\left\lfloor \frac{\ln n}{\ln p} \right\rfloor} \left\lfloor \frac n{p^j} \right\rfloor}$$

\underline{Remarque :} Étant donné que $\left\lfloor \frac n{p^j} \right\rfloor$ est nul dès que $j > \left\lfloor \frac{\ln n}{\ln p} \right\rfloor$ on écrit parfois le résultat sous la forme
$${v_p(n!) = \sum_{j=1}^{+\infty}  \left\lfloor \frac n{p^j} \right\rfloor}$$
ici la somme jusqu'à l'infini ayant un sens car \textbf{seuls un nombre fini de termes sont non nuls} (et il s'agit donc d'une somme finie).
\end{sol}


\begin{sol}
\begin{enumerate}
\item Plutôt que de calculer explicitement $N$ on a que le nombre $d$ de chiffres de l'écriture décimale de $N$ est l'entier vérifiant
$$10^{d-1} \le N < 10^{d}$$
ce qui donne en passant au $\ln$
$$d \le \frac{\ln(N)}{\ln(10)}+1 < d+1$$
ce qui est la définition de $\left\lfloor \frac{\ln(N)}{\ln(10)} \right\rfloor+1$. Or comme
$$\ln(2021!) = \sum_{k=1}^{2021} \ln(k)$$
on en déduit qu'il suffit de calculer (en faisant attention aux erreurs d'arrondi)
$$ \boxed{d = \left\lfloor \frac{1}{\ln(10)} \sum_{k=1}^{2021} \ln(k) \right\rfloor +1}$$
\item Le nombre de ``$0$'' terminant l'écriture décimale de $N$ est la plus grande puissance $p$ telle que $10^p | N$.

Comme $10 = 2 \cdot 5$ on en déduit qu'il s'agit de
$$\min(v_2(N),v_5(N))$$

Or par la formule de Legendre on sait que pour tout entier $p$
$$v_p(2021!) = \sum_{j=1}^{\left\lfloor \frac{\ln(2021)}{\ln p} \right\rfloor} \left\lfloor \frac{2021}{p^j} \right\rfloor$$
on obtient donc que $v_2(N) \ge v_5(N)$ et que
$$\begin{aligned}
v_5(N)
& =  \left\lfloor \frac{2021}{5} \right\rfloor + \left\lfloor \frac{2021}{25} \right\rfloor +\left\lfloor \frac{2021}{125} \right\rfloor + \left\lfloor \frac{2021}{625} \right\rfloor \\
& = 404 + 80 + 16 +3 \\
& = \boxed{513}
\end{aligned}$$
\end{enumerate}
\end{sol}


\begin{sol}
\begin{enumerate}
\item On remarque que l'inégalité est invariante par addition d'entiers à $\alpha$ et $\beta$.

Il suffit donc de traiter le car $(\alpha,\beta) \in [0,1[ \cdot [0,1[$.

On vérifie alors que
$$\boxed{\lfloor 2 \alpha \rfloor  + \lfloor  2\beta\rfloor \ge \lfloor \alpha \rfloor + \lfloor \beta \rfloor + \lfloor \alpha + \beta \rfloor}. $$
\item La question revient à prouver que pour tout nombre premier $p$
$$v_p \left((2m)! (2n)! \right) \ge v_p\left( m! n! (m+n)!\right)$$
Or, d'après la formule de Legendre le membre de gauche est
$$\begin{aligned}
  v_p \left((2m)! (2n)! \right) & = \sum_{k=1}^{+ \infty} \left\lfloor \frac{2m}{p^k} \right\rfloor + \sum_{k=1}^{+ \infty} \left\lfloor \frac{2n}{p^k} \right\rfloor\\
  & = \sum_{k=1}^{+\infty} \left(\left\lfloor 2 \frac m{p^k} \right\rfloor + \left\lfloor 2 \frac n{p^k} \right\rfloor\right)
\end{aligned}$$
tandis que celui de droite est
$$\begin{aligned}
  v_p \left((m)! (n)!(m+n)! \right) & = \sum_{k=1}^{+ \infty} \left\lfloor \frac m{p^k} \right\rfloor + \sum_{k=1}^{+ \infty} \left\lfloor \frac n{p^k} \right\rfloor+ \sum_{k=1}^{+ \infty} \left\lfloor \frac{m+n}{p^k} \right\rfloor\\
  & = \sum_{k=1}^{+\infty} \left(\left\lfloor  \frac m{p^k} \right\rfloor + \left\lfloor \frac n{p^k} \right\rfloor+ \left\lfloor \frac m{p^k} +\frac n{p^k} \right\rfloor\right)
\end{aligned}$$
La question précédente permet alors de conclure que pour tout nombre premier $p$,
$$\boxed{v_p \left((2m)! (2n)! \right) \ge v_p\left( m! n! (m+n)!\right)}$$

\underline{Remarque :} À priori il faut faire très attention aux calculs avec des sommes infinies. Mais ici seuls {un nombre fini de termes sont non nuls} et il n'y a donc pas de problème particulier.
\end{enumerate}
\end{sol}

\begin{sol}
  Notons $p_1 < p_2 < \ldots$ les nombres premiers rangés par ordre croissant. Ainsi pour tout $n$, $p_n$ est le $n$-ième nombre premier.
  
  Montrons par récurrence (forte) sur $n$ que pour tout entier $n$:

  $$\mathcal{H}_n: \quad p_n \le 2^{2^n}$$
  
  \textbf{Initialisation :} Pour $n=1$, $p_1 = 2 \le 2^{2^1}$.


  \textbf{Hérédité :} Soit $n$ un entier et supposons que pour tout $k \le n$, l'hypothèse $\mathcal{H}_k$ est vérifiée i.e que
  $$\forall k \le n, \quad p_k \le 2^{2^k}$$

  On a (argument d'Euclide)
  $$\prod_{k=1}^n p_k +1$$
  admet un diviseur premier qui n'est pas un des $(p_k)_{1 \le k \le n}$.

  C'est donc un $p_j$ avec $$j > n$$ et en particulier
  $$p_j \le \prod_{k=1}^n p_k +1$$
  et comme les $p_j$ rangés par ordre croissant on en déduit que
  $$p_{n+1} \le \prod_{k=1}^n p_k +1$$


  Par l'hypothèse de récurrence on a alors
  $$p_{n+1} \le \prod_{k=1}^n 2^{2^k} +1$$
  et étant donné que membre de droite se calcule
  $$\begin{aligned}
      [TODO]
  \end{aligned}$$
  on en déduit que
  $$p_{n+1} \le 2^{2^{n+1}}$$
  ce qui est l'hypothèse de récurrence au rang $n+1$.


  \textbf{Conclusion} On a donc montré $\mathcal{H}_1$ et $\left(\forall k \le n, \mathcal{H}_k \right) \Rightarrow \mathcal{H}_{n+1}$, par récurrence forte on a donc pour tout entier $n$
  $$\boxed{p_n \le 2^{2^n}}$$

  Comme le $n$-ième premier est inférieur à $2^{2^n}$ on a donc au moins $n$ nombres premiers inférieurs à $2^{2^n}$ et donc 
  $$\pi(2^{2^n}) \ge n$$

  Soit $y$ un réel. Alors par croissance de $\pi$ on a
  $$\pi(2^{2^{y}}) \ge \pi(2^{2^{\lfloor y \rfloor}})  \ge \lfloor y \rfloor \ge y-1$$

  Pour tout réel $x$, en posant alors
  $$ y = \frac{1}{\ln(2)}\ln \left( \frac{\ln(x)}{\ln(2)} \right) $$

  on en déduit
  \begin{equation*}
      \begin{aligned}
          \pi(x) =\pi \left( 2^{2^y} \right) \ge y -1 &= \frac{1}{\ln(2)}\ln \left( \frac{\ln(x)}{\ln(2)} \right) -1\\
          &=\frac{\ln(\ln(x))}{\ln(2)}- \frac{\ln(\ln(2))}{\ln(2)} -1.
      \end{aligned}
  \end{equation*}
  et par conséquent il existe $c > 0$ tel que pour tout $x$ réel
  $$\boxed{\pi(x) \ge \ln(\ln(x))}$$

  \underline{Remarque :} ces encadrements sont de très mauvaises qualités, en effet on a $p_n \sim n \ln n$ et $\pi(x) \sim \frac{x}{\ln(x)}$.
\end{sol}

\begin{sol}
Soit $m$ un entier inférieur à $n$ dont les facteurs premiers sont parmi $p_1, \ldots p_r$.

On peut $m$ peut s'écrire
$$ m = c^2 s$$
avec $s$ tel que pour tout $k$, $v_{p_k}(s) = 0 \text{ ou } 1$ (on pose pour tout $k$ la division euclidienne par $2$, $v_{p_k}(m) = 2 v_{p_k}(c) + v_{p_k}(s)$).

Comptons le nombre de tels produits:
\begin{itemize}
    \item Comme $c^2 \le m$ il y au plus $\sqrt m$ possibilités pour $c$ et comme $m \le n$ il y au plus $\sqrt n$ possibilités pour $c$.
    \item Pour tout $k$ entre $1$ et $r$ il y a deux possibilités pour $v_{p_k}(s)$ et comme les $(p_k)_{1 \le k \le r}$ sont les facteurs premiers, on a au plus $2^r$ possibilités pour $s$.
\end{itemize}

Par conséquent on a au plus $\sqrt n \cdot 2^r$ possibilités pour $m$ et donc
$$\boxed{F(n) \le 2^r \sqrt n}$$

Soit $n$ un entier et posons $p_1,\ldots,p_r$ l'ensemble de tous les nombres premiers inférieurs à $n$.

En particulier $r = \pi(n)$.

Comme tout entier inférieur à $n$ admet une décomposition en facteurs premiers inférieurs à $n$ on a donc
$$F(n) = n$$

Par conséquent l'inégalité précédente donne

$$n \le 2^{\pi(n)}\sqrt n$$

On en déduit 
$$\sqrt n \le 2^{\pi(n)}$$
puis par croissance de $\ln$ 
$$\frac{1}{2} \ln n \le \pi(n) \ln(2)$$
c'est à dire
$$\frac{1}{2 \ln(2)} \ln n \le \pi(n)$$
et par conséquent, par croissance de $\pi$, on en déduit qu'il existe $c>0$ tel que pour tout réel $x$
$$\boxed{\pi(x) \ge c \ln(x)}$$
\end{sol}


\begin{sol}
On a
$$\binom{2n} n = \frac{(2n)!}{(n!)^2}$$
donc
$$v_p\left(\binom{2n} n\right) = v_p \left( (2n)!\right)-2 v_p(n!)$$
et donc par la formule de Legendre on en déduit que, quitte à rajouter des $0$ dans les sommes,
$$\begin{aligned}
    v_p\left( \binom{2n} n\right) &= \sum_{k=1}^{\left\lfloor \frac{\ln(2n)}{\ln p} \right\rfloor} \left\lfloor \frac{2n}{p^k} \right\rfloor - 2 \sum_{k=1}^{\left\lfloor \frac{\ln n}{\ln p} \right\rfloor}\left\lfloor \frac n{p^k} \right\rfloor  \\
    &= \sum_{k=1}^{\left\lfloor \frac{\ln(2n)}{\ln p} \right\rfloor} \left\lfloor \frac{2n}{p^k} \right\rfloor - 2 \sum_{k=1}^{\left\lfloor \frac{\ln(2n)}{\ln p} \right\rfloor}\left\lfloor \frac n{p^k} \right\rfloor  
\end{aligned}$$
et donc
$$\boxed{v_p\left( \binom{2n} n\right) = \sum_{k=1}^{\left\lfloor \frac{\ln(2n)}{\ln p} \right\rfloor} \left(\left\lfloor \frac{2n}{p^k} \right\rfloor - 2 \left\lfloor \frac n{p^k} \right\rfloor  \right)}$$
De plus pour tout $k$ d'après l'exercice~\ref{exo-IV-5-2-5} on a
$$\left(\left\lfloor \frac{2n}{p^k} \right\rfloor - 2 \left\lfloor \frac n{p^k} \right\rfloor  \right) \le 1$$
et par conséquent
$$\boxed{v_p\left( \binom{2n} n\right) \le \left\lfloor \frac{\ln(2n)}{\ln p} \right\rfloor}$$

\underline{Remarque :} cette inégalité montre que $\binom{2n} n$ a non seulement ``peu de facteurs premiers'' relativement à sa taille (les facteurs premiers sont $\le 2n$) mais que ceux si ont une ``faible valuation'' (comparer à l'exercice~\ref{exo-IV-5-2-7}). Cela peut alors être utilisé pour montrer qu'il y a ``suffisamment de nombres premiers'' (voir les exercices suivants).
\end{sol}
    

\begin{sol}
Comme tous les facteurs premiers de $\binom{2n} n$ sont inférieurs à $2n$ on a
$$\binom{2n} n = \prod_{\substack{p \text{ premier} \\ p \le 2n }} p^{v_p \left( \binom{2n} n\right)}$$
et par conséquent
$$\ln \left( \binom{2n} n\right) = \sum_{\substack{p \text{ premier} \\ p \le 2n }}  v_p \left( \binom{2n} n\right)\ln p$$ 
En utilisant la majoration de l'exercice~\ref{exo-IV-5-2-13} on en déduit
$$\boxed{\ln \left( \binom{2n} n\right) \le \sum_{\substack{p \text{ premier} \\ p \le 2n }} \left\lfloor \frac{\ln(2n)}{\ln p} \right\rfloor \ln p}$$ 

De plus d'après l'exercice~\ref{exo-IV-5-2-6} on a 
$$n \ln(2) \le \ln \left( \binom{2n} n\right)$$
et comme tous les nombres premiers sont plus grand que $2$, pour tout $p$ premier on a
$$\left\lfloor \frac{\ln(2n)}{\ln p} \right\rfloor \ln p \le  \ln(2n)$$

On en déduit donc
$$n \ln(2) \le \ln(2n) \sum_{\substack{p \text{ premier} \\ p \le 2n }} 1$$
c'est à dire
$$\frac{2 n \ln(2)}{2\ln(2n)} \le \pi(2n)$$
Par croissance de $\pi$ il existe donc $c > 0$ tel que pour tout réel $x$
$$\pi(x) \ge c \frac x{\ln(x)}$$
\end{sol}

\begin{sol}
Étant donné que lorsque $x \to + \infty$ on a
$$\ln(\ln(x)) \ll \ln(x) \ll \frac{x}{\ln(x)}$$
le résultat de l'exercice~\ref{exo-IV-5-2-14} est plus précis que celui de l'exercice~\ref{exo-IV-5-2-12} qui lui même est plus précis que celui de l'exercice~\ref{exo-IV-5-2-11}.

Le théorème des nombres premiers affirmant que
$$\pi(x) \sim \frac{x}{\ln(x)}$$
l'exercice~\ref{exo-IV-5-2-14}est un résultat satisfaisant même si la constante obtenue ($\frac{\ln(2)}{2}< 1$) est non optimale et si il manque la majoration.
\end{sol}