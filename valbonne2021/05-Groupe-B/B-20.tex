\subsubsection{Énoncés}


\begin{exo}
Soit $ABC$ un triangle. Soit $D$ le pied de la bissectrice issue de $A$. La tangente au cercle circonscrit à $ABC$ en $A$ coupe la droite $(BC)$ en $P$. Montrer que le triangle $PAD$ est isocèle en $P$.
\end{exo}


\begin{exo}
Soient $a$ et $b$ deux réels strictement positifs. Montrer que
$$\left(1+\frac{a}{b}\right)^n+\left(1+\frac{b}{a}\right)^n\geq 2^{n+1} $$
Quels sont les cas d'égalité?
\end{exo}


\begin{exo}
Déterminer toutes les fonctions $f : \R\to \R$ telles que pour tous réels $x,y$, on ait
$$f(f(f(x))) + f(f(y)) = f(y) + x$$
\end{exo}


\begin{exo}
Soit $ABCD$ un quadrilatère non croisé inscrit dans un cercle $\Gamma$. Les tangentes à $\Gamma$ en $A$ et $B$ se coupent en $P$. La parallèle à la droite $(AC)$ passant par $P$ coupe $(AD)$ en $X$. La parallèle à la droite $(BD)$ passant par $P$ coupe $(BC)$ en $Y$. \\
Montrer que les points $X,Y,C,D$ sont cocycliques.
\end{exo}


\subsubsection{Solutions}


\begin{sol}
\begin{center}
\begin{tikzpicture}
[scale=0.5]
\tkzInit[ymin=-5,ymax=5,xmin=-11,xmax=6]
\tkzClip

\tkzDefPoint(-1,3){A}
\tkzDefPoint(-2,-2){B}
\tkzDefPoint(4,-2){C}

\tkzCircumCenter(A,B,C)\tkzGetPoint{O}
\tkzInCenter(A,B,C)\tkzGetPoint{I}
\tkzDefTangent[at=A](O)\tkzGetPoint{Q}

\tkzInterLL(A,Q)(B,C)\tkzGetPoint{P}
\tkzInterLL(A,I)(B,C)\tkzGetPoint{D}

\tkzDrawSegment(A,B)
\tkzDrawSegment(P,C)
\tkzDrawSegment(C,A)
\tkzDrawSegment(A,P)
\tkzDrawSegment(A,D)
\tkzDrawPoints[fill=white,color=black](A,B,C,D,P)

\tkzDrawCircle(O,B)


\tkzLabelPoint[above left](A){$A$}
\tkzLabelPoint[below left](B){$B$}
\tkzLabelPoint[right](C){$C$}
\tkzLabelPoint[left](P){$P$}
\tkzLabelPoint[below right](D){$D$}

\tkzMarkAngles[color=red, opacity=1](P,A,B A,C,B)
\tkzMarkAngles[color=blue, opacity=1](B,A,D D,A,C)


\end{tikzpicture}
\end{center}

On commence par noter $$\left\{
    \begin{array}{ll}
        \widehat{BAC}=2\alpha\\
        \widehat{ACB}=2\gamma\\
        \widehat{ABC}=2\beta
    \end{array}
\right.$$
En particulier, on remarque que
$$\widehat{DAB}=\frac{\widehat{BAC}}{2}=\frac{2\alpha}{2}=\alpha $$
On en déduit que
$$\widehat{PDA}=\widehat{BDA}=\widehat{BCA}+\widehat{CAD}=2\gamma+\alpha $$
D'autre part, on remarque que
$$\widehat{DAP}=\widehat{DAB}+\widehat{BAP}=\alpha+\widehat{BAP} $$
Que vaut $\widehat{BAP}$ ? On voit qu'il s'agit d'un angle à la tangence et on en déduit que
$$ \widehat{BAP}=\widehat{ACB}=2\gamma$$
Finalement, on en déduit que
$$\widehat{ADP}=\alpha+2\gamma=\widehat{DAP} $$
Et on a bien montré que $ADP$ était isocèle en $P$.
\end{sol}


\begin{sol}
On va utiliser à plusieurs reprises l'inégalité arithmético-géométrique sous la forme suivante :
$$\mathrm{Pour}\: \mathrm{tous}\: \mathrm{nombres}\: \mathrm{ positifs }\:a,b,\:\: \mathrm{on}\: \mathrm{a}\:\:a+b\geq 2\sqrt{ab} $$

On commence donc par écrire que
$$\left(1+\frac{a}{b}\right)\geq 2\sqrt{\frac{a}{b}} $$
Ainsi que
$$\left(1+\frac{b}{a}\right)\geq 2\sqrt{\frac{b}{a}} $$
On a donc montré que
$$\left(1+\frac{a}{b}\right)^n+\left(1+\frac{b}{a}\right)^n\geq \left(2\sqrt{\frac{a}{b}}\right)^n+\left(2\sqrt{\frac{b}{a}}\right)^n=2^n\left(\left(\sqrt{\frac{b}{a}}\right)^n+\left(\sqrt{\frac{a}{b}}\right)^n \right) $$
On utilise une dernière fois l'inégalité arithmético géométrique pour obtenir que
$$\left(\sqrt{\frac{b}{a}^n}+\sqrt{\frac{a}{b}}^n\right)^n\geq 2\left(\sqrt{\sqrt{\frac{a}{b}}\sqrt{\frac{b}{a}}}\right)^n=2\sqrt{1}=2 $$
On a donc finalement que
$$\left(1+\frac{a}{b}\right)^n+\left(1+\frac{b}{a}\right)^n\geq 2^n\left(1+\frac{a}{b}\right)^n+\left(1+\frac{b}{a}\right)^n \geq2^n\times 2=2^{n+1} $$
\end{sol}


\begin{sol}
\textbf{Analyse :} On suppose que $f$ est solution de l'équation.
\\
On remarque la présence d'un terme $"x"$ dans le membre de droite. Cela nous pousse à étudier l'injectivité et la surjectivité de $f$.
\\
\textbf{Injectivité:} Supposons qu'il existe $b,c$ des réels tels que $f(b)=f(c)$ et montrons que $b=c$. La remarque importante est que $x$ n'apparait que sous la forme $f(x)$ dans le membre de gauche, ce membre de gauche va donc être le même après la subtitution $x=b$ ou la substitution $x=c$.
\\
On pose donc $y=0$ et $x=b$ ce qui donne
$$f(f(f(b)))+f(f(0))-f(0)=b $$
Maintenant, on pose $y=0$ et $x=c$ ce qui donne
$$f(f(f(c)))+f(f(0))-f(0)=c $$ et donc on en déduit que
$$b=f(f(f(b)))+f(f(0))-f(0)=f(f(f(c)))+f(f(0))-f(0)=c $$
Ainsi, $f$ est injective.
\\
\textbf{Surjectivité:} On pose encore $y=0$ ce qui donne pour tout réel $x$ que
$$f(f(f(x)))=x+f(0)-f(f(0)) $$
Comme le membre de droite est affine, il est surjectif et on en déduit la surjectivité de $f$. Plus précisément, on se donne un réel $b$ et va chercher un antécédant $a$ de $f$ pour $b$. Pour cela, on remarque que
$$x=b-f(0)+f(f(0))\iff f(f(f(x))))=x+f(0)-f(f(0))=b $$
Et donc $f(f(x))$ est un antécédant de $f$. On a donc $f$ surjective.
\\
Cela permet de faire la subtitution $f(y)=u$ et donc on a pour tous réels $x,u$:
$$f(f(f(x))))+f(u)=u+x $$
On pose finalement $x=0$ et on a pour tout réel $u$
$$f(u)=u-f(f(f(0)))=u+C $$
\textbf{Synthèse:} On suppose que $f(u)=u+C$ avec $C$ une constante. On vérifie que
$$f(f(f(x)))+f(f(y))=x+C+C+C+y+C+C=x+y+5C $$
d'une part et on a
$$f(y)+x=x+y=C $$
On veut donc que
$$x+y+5C=x+y+C\Rightarrow C=0 $$
Et donc la seule solution est l'identité $f(x)=x$.
\end{sol}
\begin{rem}
L'injectivité n'a pas servi dans l'argument. Toutefois, cela reste un bon réflexe de la montrer dès que c'est possible.
\end{rem}


\begin{sol}
\begin{center}
\begin{tikzpicture}
[scale=1]
\tkzInit[ymin=-10,ymax=10,xmin=-5,xmax=12]
\tkzClip

\tkzDefPoint(-1,3){A}
\tkzDefPoint(2.5,2.5){B}
\tkzDefPoint(4,-2){C}
\tkzDefPoint(-2,-2){d}

\tkzCircumCenter(A,B,C)\tkzGetPoint{O}
\tkzDefTangent[at=A](O)\tkzGetPoint{u}
\tkzDefTangent[at=B](O)\tkzGetPoint{v}
\tkzInterLL(A,u)(B,v) \tkzGetPoint{P}
\tkzInterLC(O,d)(O,A) \tkzGetPoints{D'}{D}
\tkzInterLL(A,C)(B,D) \tkzGetPoint{Q}

\tkzDefLine[parallel=through P](A,C) \tkzGetPoint{x}
\tkzDefLine[parallel=through P](B,D) \tkzGetPoint{y}

\tkzInterLL(A,D)(P,x) \tkzGetPoint{X}
\tkzInterLL(B,C)(P,y) \tkzGetPoint{Y}

\tkzCircumCenter(X,Y,C)\tkzGetPoint{o}

\tkzDrawSegment(B,D)
\tkzDrawSegment(X,D)
\tkzDrawSegment(C,A)
\tkzDrawSegment(Y,C)
\tkzDrawSegment(A,P)
\tkzDrawSegment(B,P)
\tkzDrawSegment(X,P)
\tkzDrawSegment(Y,P)

\tkzDrawSegment(X,Y)
\tkzDrawSegment(A,B)
\tkzDrawSegment(C,D)

\tkzDrawPoints[fill=white,color=black](A,B,C,D,X,Y,P,Q)

\tkzDrawCircle(O,B)
\tkzDrawCircle[dashed](o,C)

\tkzLabelPoint[above left](A){$A$}
\tkzLabelPoint[above right](B){$B$}
\tkzLabelPoint[right](C){$C$}
\tkzLabelPoint[left](D){$D$}
\tkzLabelPoint[above](X){$X$}
\tkzLabelPoint[above](Y){$Y$}
\tkzLabelPoint[right](P){$P$}
\tkzLabelPoint[below](Q){$Q$}
\end{tikzpicture}
\end{center}

Voici une autre solution ne nécéssitant pas d'introduire des points supplémentaires mais seulement d'utiliser des triangles semblables. \\
On montre que les triangles $YPB$ et $BCD$ sont semblables. Par angle à la tangence on a
$$\widehat{PBY}=\widehat{ADB} $$
Puis par parallélisme on remarque que
$$\widehat{DBC}=\widehat{PYB} $$
Ce qui montre que
$$YPB\sim BCD $$
De même, on a bien que
$$YPA\sim ADC $$
On note $Q=(AC)\cap (BD)$.
Comme $(PY)//(QB)$ et $(PX)//(AQ)$, il nous suffit de montrer que
$$APY\sim AQB $$ pour obtenir $(AB)//(XY)$. On se ramène donc à montrer que
$$\frac{PX}{AQ}=\frac{PY}{PX} $$
On a d'une part
$$YP=\frac{BC\times PB}{CD} $$
D'autre part on a
$$XP=\frac{AD\times PA}{CD}$$
Comme $PA=PB$ on a finalement
$$\frac{PY}{BC}=\frac{PX}{AD} $$
Comme $BQC\sim AQD$, on peut écrire que
$$\frac{BC}{QB}=\frac{AD}{QA} $$
Et on a donc montré que
$$\frac{PY}{QB} =\frac{PX}{QA}$$
Ce qui montre que $X,Y,C$ et $D$ sont cocycliques.

\begin{center}
\begin{tikzpicture}
[scale=1]
\tkzInit[ymin=-10,ymax=10,xmin=-5,xmax=12]
\tkzClip

\tkzDefPoint(-1,3){A}
\tkzDefPoint(2.5,2.5){B}
\tkzDefPoint(4,-2){C}
\tkzDefPoint(-2,-2){d}

\tkzCircumCenter(A,B,C)\tkzGetPoint{O}
\tkzDefTangent[at=A](O)\tkzGetPoint{u}
\tkzDefTangent[at=B](O)\tkzGetPoint{v}
\tkzInterLL(A,u)(B,v) \tkzGetPoint{P}
\tkzInterLC(O,d)(O,A) \tkzGetPoints{D'}{D}

\tkzDefLine[parallel=through P](A,C) \tkzGetPoint{x}
\tkzDefLine[parallel=through P](B,D) \tkzGetPoint{y}

\tkzInterLL(A,D)(P,x) \tkzGetPoint{X}
\tkzInterLL(B,C)(P,y) \tkzGetPoint{Y}
\tkzInterLL(A,D)(P,y) \tkzGetPoint{W}
\tkzInterLL(B,C)(P,x) \tkzGetPoint{Z}

\tkzCircumCenter(X,Y,C)\tkzGetPoint{o}
\tkzCircumCenter(P,A,B)\tkzGetPoint{o'}
\tkzCircumCenter(X,Y,W)\tkzGetPoint{o''}

\tkzDrawSegment(B,D)
\tkzDrawSegment(X,D)
\tkzDrawSegment(C,A)
\tkzDrawSegment(Y,C)
\tkzDrawSegment(A,P)
\tkzDrawSegment(B,P)
\tkzDrawSegment(X,Z)
\tkzDrawSegment(Y,W)

\tkzDrawSegment(X,Y)
\tkzDrawSegment(A,B)
\tkzDrawSegment(C,D)

\tkzDrawPoints[fill=white,color=black](A,B,C,D,X,Y,P,W,Z)

\tkzDrawCircle(O,B)
\tkzDrawCircle[dashed](o,C)
\tkzDrawCircle[dashed](o',P)
\tkzDrawCircle[dashed](o'',W)

\tkzLabelPoint[above left](A){$A$}
\tkzLabelPoint[above right](B){$B$}
\tkzLabelPoint[right](C){$C$}
\tkzLabelPoint[left](D){$D$}
\tkzLabelPoint[above](X){$X$}
\tkzLabelPoint[above](Y){$Y$}
\tkzLabelPoint[left](W){$W$}
\tkzLabelPoint[right](Z){$Z$}
\tkzLabelPoint[right](P){$P$}

\tkzMarkAngles[color=red, opacity=1](P,B,A B,A,P B,D,A B,C,A P,W,A B,Z,P)

\end{tikzpicture}
\end{center}


On introduit deux nouveaux points : on note pour commencer $W=(PY)\cap (AD)$ puis on note $Z=(XP)\cap (BC)$. \\
On considère les deux droites de référence $(AD)$ et $(BC)$. On veut montrer que $(XY)$ et $(CD)$ sont antiparallèles. \\
On sait déjà que $(DC)$ et $(AB)$ sont antiparallèles. On va en fait montrer que
$$(ZW)$$
est à la fois antiparallèle à $(XY)$ et à $(AB)$. On en déduira que $(DC)//(WZ)$ et donc que $XYCD$ est cocyclique. \\
\textbf{$XYZW$ cyclique :} On note $\alpha = \widehat{PAB}$. On a alors par angle à la tangence
$$\widehat{PAB}=\widehat{ACB}=\alpha $$
Puis par parallélisme on a
$$\alpha=\widehat{ACB}=\widehat{XZY} $$
Comme $PA=PB$, on en déduit que
$$\alpha=\widehat{PAB}=\widehat{PBA} $$
On montre donc de même que
$$\alpha=\widehat{XWY}=\widehat{YZX} $$
Et donc $XYZW$ est cyclique. En d'autre termes, $(XY)$ et $(CD)$ sont antiparallèles. \\
$AWBZP$ cyclique : On va en fait montrer que l'on a d'une part $PAWB$ est cyclique et $PZBA$ est cyclique. \\
Pour cela, on remarque que, par parallélisme, on a
$$\widehat{AWP} = \widehat{ADB} = \alpha = \widehat{ABP} $$
On en déduit donc que $PAWB$ est cyclique. On a de même que $PZBA$ est cyclique. Ainsi, $A,B,Z,W$ sont tous sur le cercle circonscrit à $ABP$ et sont donc cocycliques. \\
On a donc montré que $(AB)$ et $(WZ)$ sont antiparallèles et donc on a finalment montré que $X,Y,C$ et $D$ sont cocycliques.
\end{sol}