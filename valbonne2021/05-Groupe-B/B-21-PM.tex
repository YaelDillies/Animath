\subsubsection{Idée intuitive}

Soient $n$ tiroirs et $n + 1$ chaussettes rangées dans ces tiroirs. Il existe au moins un tiroir contenant au moins deux chaussettes.

\begin{thm}[Principe des tiroirs]
Plus généralement, si $n$ éléments doivent être rangés dans $k$ tiroirs, alors il existe au moins un tiroir qui contient au moins $\lceil \frac{n}{k}\rceil$ éléments.
\end{thm}

\textit{Note : $\lceil x \rceil$ désigne la partie entière supérieure, ie l'unique entier tel que $x\le \lceil x \rceil< x +1$.}
\newline

\begin{ex}
Combien d'enfants faut-il au minimum dans une école pour que l'on soit sûr que 3 d'entre eux au moins aient leur anniversaire le même jour ?
\newline
Si on a 733 élèves, alors par le principe des tiroirs on est assuré d'avoir au moins trois élèves qui ont la même date d'anniversaire (il y a 366 dates possibles). Ce nombre est minimal car il existe une configuration à 732 élèves sans avoir trois élèves avec la même date d'anniversaire.
\end{ex}

Le principe des tiroirs est un principe très simple et intuitif mais dont les conséquences sont tout à fait impressionnantes. Ce type de raisonnement est principalement efficace pour des problèmes demandant la preuve de l'existence d'un élément. Évidemment, il faut savoir reconnaître les tiroirs et les chaussettes, ce qui peut être très difficile.

\bigskip
\bigskip

\begin{exo}
Montrer que si 3 nombres réels sont dans l'intervalle $[0, \, 1[$, alors il existe parmi eux deux nombres a et b tels que $|b - a|< \frac{1}{2}$.
\end{exo}

\begin{exo}
Montrer que, parmi les stagiaires d'Animath de cette semaine, il en existe deux qui connaissent le même nombre d'autres stagiaires.
\end{exo}

\begin{exo}
Combien d'entiers au minimum doit-on sélectionner dans l'ensemble $\{1, \, 2, \, \ldots, \, 20\}$ pour être sûr que cette sélection inclue deux entiers a et b tels que $a - b = 2$ ?
\end{exo}

\begin{exo}
Soit $x_1, \, x_2, \, \ldots, \, x_{2021}, \, x_{2022}$ des entiers. Montrer qu'il existe $i \ne j$ tels que $x_j - x_i$ soit divisible par 2021.
\end{exo}

\begin{exo}[USAMTS 2018]
Chaque point du plan est colorié soit en rouge, soit en vert, soit en bleu. Montrer qu'il existe un rectangle dont tous les sommets sont de la même couleur.
\end{exo}

\begin{exo}
Sur une table rectangulaire de dimension $2m \cdot 1m$ sont réparties 500 miettes de pain. Prouver que l'on peut trouver trois miettes qui déterminent un triangle d'aire inférieure à $50cm^2$.
\end{exo}

\begin{exo}
Soit $n$ un entier. On choisit $n+1$ nombres parmi $\{1, \, 2, \, \ldots , \, 2n\}$, montrer que l’on peut en trouver deux premiers entre eux. Montrer qu’on peut aussi en trouver deux tels que l’un divise l’autre.
\end{exo}

\begin{exo}[Bolzano-Weierstrass]
On considère une suite infinie de réel dans l'intervalle $[0, \, 1[$ $x_0, \, x_1, \, x_2, \, \ldots$.
\begin{itemize}
\item Montrer que soit l'intervalle $\left[0, \, \frac{1}{2}\right[$ soit l'intervalle $\left[\frac{1}{2}, \, 1\right[$ contient une infinité d'éléments de la suite.
\item Soit $\frac12>\varepsilon>0$. Montrer qu'il existe au moins un rationnel $\alpha\in[0,1]$ tel qu'une infinité d'éléments de la suite se trouvent dans l'intervalle $[\alpha-\varepsilon,\alpha+\varepsilon]$
\end{itemize}
\end{exo}

\begin{exo}
Montrer que le produit de cinq nombres entiers strictement positifs consécutifs ne peut pas être le carré d’un nombre entier.
\end{exo}

\begin{exo}[IMO 1997]
Une matrice carrée à $n$ lignes et $n$ colonnes, à éléments dans
l'ensemble $S = {1, 2, . . . , 2n - 1}$, est appelée une matrice d'argent si, pour tout $i =1, . . . , n$, la réunion de la i-ième ligne et de la i-ième colonne contient tous les éléments de $S$. Montrer qu'il n'existe pas de matrice d'argent pour $n = 1997$.
\end{exo}

\begin{exo}[Nordic Mathematical Contest]
Soit $n$ un nombre entier supérieur à 3 et supposons que $2n$ sommets d'un $(4n + 1)$-gone régulier soient colorés. Montrer qu'il existe trois sommets colorés formant un triangle isocèle.
\end{exo}

\bigskip



\begin{sol}
En partitionnant l'intervalle en $\left[0, \, \frac12\right[$ et $\left[\frac12, \, 1\right[$, On a par principe des tiroirs l'un des intervalles qui contient au moins 2 des 3 réels choisis, qui conviennent alors.
\end{sol}


\begin{sol}
Soit $n$ le nombre de stagiaires. Chaque stagiaire connaît entre $0$ et $n - 1$ autres stagiaires. Or il ne peut pas y avoir en même temps un stagiaire qui connaît $n - 1$ autres personnes (donc tout le monde) et un qui ne connaît personne. Il y a donc $n-1$ possibilités du nombre de stagiaire qu'une personne peut connaître et par le principe des tiroirs il y a 2 personnes qui connaît le même nombre de stagiaires.
\end{sol}


\begin{sol}
Considérons les tiroirs de la forme $\{1, \, 3\}, \ \{2, \, 4\}, \ \{5, \, 7\}, \ \{6, \, 8\}, \ \{9, \, 11\}, \ \{10, \, 12\}, \ \{13, \, 15\},$ $\{14, \, 16\}, \ \{17, \, 19\}, \ \{18, \, 20\}$. En choisissant 11 entiers, par le principe des tiroirs, il en existera deux qui seront dans le même tiroir, et donc de différence $2$. Cette quantité est bien minimale car l'ensemble $\{1, \, 2, \, 5, \, 6, \, 9, \, 10, \, 13, \, 14, \, 17, \, 18\}$ est de taille 10 et ne contient pas de tels entiers $a$ et $b$.
\end{sol}


\begin{sol}
Lorsqu'on fait la division euclidienne d'un entier par $2021$, le reste sera compris entre $0$ et $2020$, soit $2021$ possibilités. En regroupant chaque entier selon son reste dans la division euclidienne par $2021$, on obtient par principe des tiroirs deux entiers qui ont le même reste, et donc la différence des deux sera divisible par $2021$.
\end{sol}


\begin{sol}
Traçons le rectangle plein à coordonnées entières de taille $4 \cdot 82$ dont le sommet en bas à gauche est $(0, \, 0)$. Il y a $3^4 = 81$ façons de colorier une colonne de 4 points. Par principe des tiroirs, parmi les 82 colonnes, il en existe 2 identiques. Or, sur ces deux colonnes, parmi les 4 points, deux sont de la même couleur. Il suffit alors de sélectionner ces quatre sommets et de tracer le rectangle adéquat.
\end{sol}


\begin{sol}
On découpe la table en 200 petits carrés de côté 10cm. D'après le principe des tiroirs, dans au moins un de ces carrés, il y aura trois miettes, et ces miettes vont déterminer un triangle dont l'aire sera inférieure à la moitié de la surface du carré, ie $50cm^2$.
\end{sol}


\begin{sol}
\begin{itemize}
    \item En considérant les $n$ tiroirs de la forme $\{2k-1, \, 2k\}$ où $1 \le k \le n$, on voit qu'on peut trouver deux entiers consécutifs qui seront premiers entre eux.
    \item Chaque entier peut s'écrire sous la forme $2^s(2t+1)$. Or $t$ varie entre $0$ et $n-1$, il existe alors deux entiers parmi les $n + 1$ choisis qui ont la même partie impaire et qui ne diffèrent que par la puissance de 2. L'un divise donc l'autre.
\end{itemize}
\end{sol}


\begin{sol}
Pour la première question, il s'agit d'une simple application du principe des tiroirs dans le cas où l'on a une infinité de chaussettes.\\
La seconde question se résout de la même façon : commençons par choisir $n\in\N$ tel que $\frac{1}{2^n}<\varepsilon$. Prenons pour tiroirs les ensembles $\left[0,\frac1{2^n}\right]$,\dots,$\left[\frac{k}{2^n},\frac{k+1}{2^n}\right]$,\dots,$\left[\frac{2^n-1}{2^n},1\right]$. Par principe des tiroirs infinis, l'un de ces tiroirs contient une infinité d'éléments, et notons $\alpha$ son milieu. Cet intervalle est alors inclus dans $[\alpha-\varepsilon,\alpha+\varepsilon]$, ce qui achève la démonstration.
\end{sol}


\begin{sol}
Soient $a$, $b$, $c$, $d$, $e$ des entiers consécutifs tels que $abcde$ soit un carré, et $p$ un nombre premier plus grand ou égal à 5. Si l’un des nombres est multiple de $p$, alors aucun des autres nombres ne l’est, donc ce nombre est multiple de $p^2$.
\smallskip
\newline
Chacun de ces nombres appartient à l’une de ces catégories, selon le nombre de facteurs 2 et 3 :
\begin{enumerate}
\item un carré
\item deux fois un carré
\item trois fois un carré
\item six fois un carré.
\end{enumerate}
Par le principe des tiroirs, une des catégories contient deux nombres. Si ces deux nombres sont dans les catégories 2, 3 ou 4, alors leur différence vaut au moins $6$.

Si ces deux nombres sont des carrés, alors ce sont $1$ et $4$, ce qui ne laisse que $1 \cdot 2 \cdot 3 \cdot 4 \cdot 5 = 120$, qui n’est pas un carré.
\end{sol}


\begin{sol}
Appelons $i$-ième croix la réunion de la $i$-ième ligne et de la $i$-ième colonne. Soit $a$ un entier compris entre 1 et $2n-1$. Si cet entier apparaît à la position $(i, j)$, alors il apparaît à la fois dans la $i$-ième croix et dans la $j$-ième croix. Comme
maintenant $a$ doit apparaître une et une seule fois dans chaque croix, on peut faire la chose suivante : on écrit les uns à la suite des autres les nombres de 1 et 1997, et on barre au fur et à mesure les numéros des croix dans lesquelles $a$ apparaît. À la fin, toutes les croix devront être barrées.
\newline
On se rend compte que si $a$ n'apparaît pas sur la diagonale, on va barrer les nombres deux par deux. Mais cela n'est pas possible puisque 1997 est un nombre impair.
\newline
Il reste donc à prouver qu'il existe un entier a qui n'apparaît pas sur la diagonale. C'est la qu'intervient le principe des tiroirs. Il y a 3993 nombres en tout et seulement 1997 places sur la diagonale ; il y a donc au moins un nombre qui ne peut pas apparaître.
\end{sol}

\begin{sol}
Supposons par l'absurde qu'il soit possible de colorier $2n$ sommets d'un $4n+ 1$-gone de sorte que
il n'y ait pas 3 sommets colorés formant un triangle isocèle. Notons les sommets $H_{-2n}$, $H_{-2n+1}$, \dots , $H_0$, $H_1$, $H_2$, \dots, $H_{2n}$.
\newline
On considère d'abord le cas où deux sommets voisins sont colorés. Supposons sans perte de généralité que les sommets $H_0$ et $H_1$ sont colorés. Alors au plus un des sommets $H_{-i}$ et $H_i$ est coloré pour tous les $i = 1, 2, . . . , 2n$ puisqu'ils forment un triangle isocèle avec $H_0$.
\newline
De même, au plus un des sommets $H_{-i}$ et $H_{i+2}$ sont colorés pour tous les $i = 1, 2, . . . , 2n-2$, puisqu'ils forment un triangle isocèle avec $H_{1}$.
\newline
Les trois sommets $H_0$, $H_1$, $H_i$ avec $i = 2, -1, -2n$
forment aussi des triangles isocèles donc $H_{-1}$, $H_2$, $H_{-2n}$ ne sont pas colorés. Il s'ensuit qu'aucun sommets consécutifs dans les deux chaînes
$H_{-2} - H_4 - H_{-4} - H_6 - . . . - H_{2n-2} - H_{-(2n-2)} - H_{2n}$ et $H_3 - H_{-3} - H_5 - H_{-5} - . . . - H_{2n-1} - H_{-(2n-1)}$ ne sont colorés.
\newline
Comme chaque chaîne contient un nombre pair de sommets, au plus la moitié de chaque chaîne est colorée. En comptant on voit que chaque chaîne contient $2n - 2$ sommets et on en conclut qu'un sommet sur 2 est coloré dans chaque chaîne. Comme $n \ge 3$, au moins un des triangles isocèles $H_0H_{-2}H_{-4}$, $H_1H_3H_5$ ou $H_{2n-2}H_{2n}H_{-(2n-1)}$ doit être coloré. Il n'y a donc pas sommets voisins colorés.
\newline
S'il n'y a pas de sommets voisins colorés, on peut supposer sans perte de généralité que $H_i$ est coloré pour tous les $i$ impairs, mais alors $H_1H_3H_5$ est un triangle isocèle coloré. Contradiction. Il doit donc y avoir 3 sommets colorés formant un triangle isocèle.
\end{sol}


\subsubsection{Principe de l'extremum}
Le principe de l'extremum consiste à considérer le minimum ou le maximum d’une certaine quantité. Il est très efficace pour réaliser des raisonnements par l’absurde.

\begin{thm}[Principes du maximum]
\begin{itemize}
    \item Tout ensemble fini non vide de nombres réels admet un plus petit élément ainsi qu’un plus grand élément.
    \item Tout ensemble non vide d'entiers naturels admet un plus petit élément.
\end{itemize}
\end{thm}


\begin{ex}
Montrer que l'ensemble des réels appartenant à l'intervalle $]0,1]$ n'admet pas de minimum. \newline
On rappelle qu'un ensemble $A$ admet $m$ pour minimum si :
\begin{itemize}
\item $m\in A$
\item $\forall~ x\in A, ~~ x\ge m$.
\end{itemize}
\bigskip
\noindent\textit{Démonstration :} Supposons par l'absurde que $m$ soit le minimum de $]0,1]$. \newline
Notons que $1\ge m>0$ donc $1>\frac12\ge\frac{m}{2}>0$, et $\frac{m}{2}\in]0,1]$. Or $m>\frac{m}{2}$, ce qui contredit la minimalité de $m$. Donc $]0,1]$ n'admet par de minimum.
\end{ex}


\begin{exo}
À chaque point à coordonnées entières du plan, on attribue un nombre entier strictement positif, tel que chaque nombre est égal à la moyenne arithmétique de ses quatre voisins
(en haut, en bas, à gauche et à droite). Montrer que toutes les valeurs sont égales.
\end{exo}


\begin{exo}
Soit $S$ un ensemble de points tel que tout point de $S$ est le milieu d’un segment dont les extrémités sont dans $S$. Montrer que $S$ est infini.
\end{exo}


\begin{exo}
À un tournoi, chaque compétitrice rencontre chaque autre compétitrice exactement une fois. Il n’y a pas de match nul. À l’issue de la compétition, chaque joueuse fait une liste qui contient les noms des joueuses qu’elle a battues, ainsi que les noms des joueuses qui sont battues par les joueuses qu’elle a battues.
\newline
Montrer qu’il existe une liste qui contient le nom de toutes les autres joueuses.
\end{exo}

\begin{exo}
On considère un ensemble $T$ de $2n$ points du plan. Montrer qu'il est possible de relier les points par paire de sorte que deux segments ne se croisent pas.
\end{exo}

\begin{exo}
Au stage olympique, chaque élève connaît exactement trois autres élèves. Prouvons qu’on peut partager les élèves en deux groupes de sorte que chacun ne connaisse qu’une
seule personne dans son groupe.
\end{exo}

\begin{exo}
On considère un ensemble fini de points $S$ tel que toute droite passant par deux points de $S$ passe aussi par un troisième. Montrer que tous les points de $S$ sont alignés.
\end{exo}

\begin{exo}
Sept amis ont ramassé en tout cent champignons, chacun en a ramassé un nombre différent. Montrer qu’il en existe trois parmi eux qui ont ramassé à eux trois au moins 50 champignons.
\end{exo}

\begin{exo}
Lors d’une soirée dansante, aucun garçon n’a dansé avec toutes les filles, mais chaque fille a dansé avec au moins un garçon. Mq il existe deux garçons $g, g'$ et deux filles $f, f'$ tq $g$ a dansé avec $f$ mais pas avec $f'$, et $g'$ a dansé avec $f'$ mais pas avec $f$.
\end{exo}



\begin{sol}
 Il existe un minimum des entiers écrit sur l'un des points, ses quatre voisins sont supérieurs ou égaux à celui ci, ce qui n'arrive que si ils sont tous égaux à lui. Donc ses $4$ voisins sont égaux au minimum, puis de proche en proche toutes les cases du plan doivent être égales.
\end{sol}


\begin{sol}
On considère le segment de distance maximale $[AB]$, et le segment $[CD]$ de milieu $B$. On voit que soit le segment $[AC]$ soit $[AD]$ est plus long que $[AB]$, contradiction.
\end{sol}

\begin{sol}
Soit $A$ la joueuse ayant gagné le plus de matchs. S’il n’existait pas de liste contenant le nom de toutes les autres joueuses, il existerait une joueuse $B$ ayant gagné contre $A$ ainsi que contre toutes les personnes battues par $A$. Donc $B$ aurait gagné plus de matchs que $A$, absurde.
\end{sol}


\begin{sol}
Parmi toutes les configurations, on considère celle qui minimise les distances des segments. Si deux segments se croisent, on peut les décroiser et diminuer la somme des distances par inégalité triangulaire, absurde !
\end{sol}

\begin{sol}
Parmi tous les partages possibles en deux groupes, regardons le nombre total de connaissances qu’il y a cumulées en restant à l’intérieur de son groupe. Choisissons le partage qui minimise ce nombre. Prouvons que celui-ci convient.
\newline
Par l’absurde, supposons que dans le partage il existe un élève $A$ qui connaît au moins deux personnes dans son groupe. Dans l’autre groupe, il y a au plus une personne qu’il connaît : en changeant $A$ de groupe on diminuerait ainsi le nombre total de connaissances qu’il y a cumulées en restant à l’intérieur de son groupe, ce qui contredit la minimalité supposée du partage initial.
\end{sol}

\begin{sol}
 On considère $P$ le point le plus proche d'une droite reliant des points de $S$, disons $(d)$. Cette droite doit contenir trois points. Donc il en existe deux d'un côté de la projection orthogonale, disons $A$ puis $B$. On montre que $A$ est plus proche de $PB$ que $P$ de $(d)$, absurde. Donc tous les points sont confondus.
\end{sol}

\begin{sol}
On ordonne les amis par nombre de champignon différent, disons $a_1>a_2>\dots>a_7$. Si $a_4\ge 15$, on a $a_3+a_2+a_1\ge 16+17+18=51$, et c'est gagné. Donc $a_4\le 14$ et $a_4+a_5+a_6+a_7\le 14+13+12+11=50$, donc $a_1+a_2+a_3\ge 50$, et on a gagné.
\end{sol}

\begin{sol}
On considère le garçon $g$ ayant dansé avec le plus de fille. On considère une fille $f'$ avec qui il n'a pas dansé. $f'$ a dansé avec un garçon $g'$. $g'$ n'a pas pu dansé avec toutes les même filles que $g$ par hypothèse. Donc on peut trouver $f$ qui n'a pas dansé avec $g'$ mais avec $g$.
\end{sol}