
\subsubsection{Introduction}

On appelle \emph{équation diophantiennes} une équation dont on cherche des solutions \textbf{entières}.

Par exemple, la question:
\begin{quote}
  Déterminer l'ensemble des entiers \(x\) et
  \(y\) tels que \[x^2-3 y = 1\]
\end{quote}
est une question relevant des équations
diophantiennes.

Étant donné une équation diophantienne les questions se posant
naturellement sont:

\begin{itemize}
  \item  combien existe-t-il de solutions (aucune, un nombre fini, un nombre
        infini) ?
  \item  si il existe un nombre fini de solution, peut les exhiber ou donner un
        algorithme le faisant ?
  \item  si il existe une infinité de solutions, est-il possible de toutes les
        exprimer (à partir de certaines solutions ``basiques'') ?
\end{itemize}

De manière générale ces questions sont très difficiles (penser par
exemple à la célèbre équation de Fermat \(x^n+y^n=z^n\)) et les
techniques utilisées sont très différentes de celles utilisées pour des
équations réelles.

De plus (il s'agit du dixième problème de Hilbert, résolu par
Matiassevitch en 1970) il a été montré qu'il \textbf{n'existe pas de
  méthode générale} permettant de déterminer si une équation diophantienne
admet ou non des solutions.



On se contentera dans ce cours/TD d'exposer quelques techniques simples
à travers des équations classiques. Pour plus de détails sur d'autres
techniques et réflexes à acquérir on renvoie vers le chapitre 4 du
\href{https://maths-olympiques.fr/wp-content/uploads/2017/09/arith_cours.pdf}{polycopié d'arithmétique de le POFM}.


\subsubsection{Les équations linéaires}

Une équation diophantienne est dite linéaire si toutes les inconnues
sont ``sans puissance''.

Dans un outil de base est le \textbf{théorème de Bézout} ainsi que
\textbf{l'algorithme d'Euclide} permettant de construire explicitement
des solutions.

L'exemple le plus simple est:

\begin{quote}
  Déterminer l'ensemble des prix pouvant être payés avec des pièces de $2€$ et des billets de $5€$ si il est possible de rendre la monnaie.
\end{quote}

dont une formulation mathématique est la suivante:

\begin{exo}
  Déterminer l'ensemble des $n \in \mathbb{N}$ tels que l'équation
  \[5x+2y=n\]
  admette au moins une solution avec $x$ et $y$ des entiers relatifs.
\end{exo}

On peut généraliser le problème

\begin{exo}
  Soient $a$ et $b$ deux entiers.
  Déterminer l'ensemble des $n \in \mathbb{N}$ tels que l'équation
  \[ax+by=n\]
  admette au moins une solution avec $x$ et $y$ des entiers (relatifs).
\end{exo}


On peut alors raffiner la question de plusieurs façons: regarder avec plus de deux valeurs de pièces de monnaie (c'est encore Bézout et l'algorithme d'Euclide), regarder sans rendu de monnaie (c'est à dire chercher des solutions \emph{positives}), compter le nombre de solutions positives, etc.


\begin{exo}
  Déterminer l'ensemble des $n \in \mathbb{N}$ tels que l'équation
  \[5x+2y=n\]
  admette au moins une solution avec $x$ et $y$ des entiers \textbf{positifs}.
\end{exo}

\subsubsection{Méthodes ``réelles'' et majoration}

Une façon intuitive de résoudre une équation diophantienne est de résoudre l'équation dans un ensemble de nombres plus grands que $\mathbb{N}$ et d'ensuite ne garder que les solutions entières.

\begin{exo}
  Déterminer les entiers $n$ tels que
  \[7 n^2 - n^4 = 12. \]
\end{exo}


Malheureusement ce genre de méthode fonctionne rarement. De plus les équations diophantiennes sont des équations à nombreuses variables ce qui rend cette approche irréalisable.

On peut néanmoins en \textbf{effectuer des majorations}: en utilisant des calculs algébriques on établi des bornes sur les solutions éventuelles pour se ramener à traiter un nombre fini de cas.


\begin{exo}
  Déterminer tous les entiers strictement positifs $a$ et $b$ tels que
  \[\frac{1}{a} + \frac{1}{b} = 1.\]
\end{exo}


Ces majorations et disjonction de cas peuvent alors s'enchaîner.
\begin{exo}
  Déterminer le nombre de triplets $(a,b,c) \in \left(\mathbb{N}^* \right)^3$ solutions de
  \[\frac{1}{a} + \frac{1}{b} + \frac{1}{c} = 1.\]
\end{exo}

\begin{exo}
  Déterminer tous les entiers strictement positifs $a$ et $b$ tels que
  \[a^b = b^a.\]
\end{exo}

\begin{exo}[CG 90]
  Déterminer les entiers $n \in \mathbb{N}^*$ tels que l'équation
  \[ \sum_{k=1}^n \frac{1}{{x_k}^2}= 1\]
  admette une solution $(x_1,\ldots,x_n) \in (\mathbb{N}^*)^n$.
\end{exo}

Enfin il ne vaut pas oublier les factorisations et identités remarquables pour simplifier les équations. Un cas particulier très utile étant la différence de deux puissances.

\begin{exo}
  Déterminer l'ensemble des entiers naturels $n$ tels que $5^n + 4$ est un carré.
\end{exo}

\begin{exo}
  Déterminer tous les entiers positifs $n$ et $m$ tels que

  \[n^4-m^4 = 42.\]
\end{exo}

\subsubsection{Réduction modulo $n$}

Un autre outil à disposition est la réduction modulo $n$.

En effet \textbf{si} l'équation $P(x)=0$ admet une solution $x \in \mathbb{Z}$, \textbf{alors} en particulier pour tout entier $n$, l'équation $P(x) \equiv 0 \mod n$ admet une solution.

La stratégie est alors la suivante: supposer qu'il existe une solution, regarder l'équation modulo $n$, effectuer les simplifications et essayer d'en déduire des informations sur la solution.

Le cas le plus simple est celui ou l'équation n'admet pas de solution modulo $n$. Alors l'équation de départ n'admet pas de solution entière.

\begin{exo}
  Montrer que l'équation
  \[a^2 + b^2 = 2023\]
  n'admet pas de solution.
\end{exo}

La question est alors de trouver un choix judicieux de $n$, en gardant à l'esprit les heuristiques suivantes:
\begin{itemize}
  \item un $n$ faisant disparaître un terme de l'équation,
  \item pour les termes carrés $n=4$ ou $n=8$,
  \item pour les cubes $n=7$,
  \item remarquer qu'une puissance $4$ est aussi un carré,
  \item etc. (cf cours sur les modulos)
\end{itemize}

\begin{exo}
  Montrer que l'équation
  \[a^{5} - 2 b^2 = 2021\]
  n'admet pas de solutions.
\end{exo}


Souvent cette réduction modulo $n$ ne permet pas directement de conclure, mais permet de ``grappiller des informations''. Ainsi si on obtient qu'un entier $a$ vérifie $a^2 \equiv 0 \mod 4$, cela n'implique pas une contradiction, mais on peut ensuite poser $a = 2 \widetilde{a}$ et réinjecter dans l'équation pour continuer.

Pour plus de détails on renvoie encore une fois aux polycopiés d'arithmétique de la POFM et au cours sur les modules.


\subsubsection{Descente infinie}

On rappelle la propriété suivante:
\begin{quote}
  Toute partie non vide de $\mathbb{N}$ admet un plus petit élément.
\end{quote}

Cette propriété à la base du principe de récurrence a une conséquence:
\begin{quote}
  Il n'existe pas de suite strictement décroissante d'entiers naturels.
\end{quote}

Ce principe, parfois appelé \emph{descente infinie de Fermat} permet de montrer qu'une équation diophantienne n'admet pas de solution en montrant que si une solution existe, alors une autre strictement plus petite existe.

On remarquera le lien avec les méthodes de minimum: on pourrait de manière équivalente considérer la plus petite solution et obtenir une contradiction.

\begin{exo}
  Montrer que l'équation
  \[r^3 = 2\]
  n'admet pas de solution rationnelle.
\end{exo}

\begin{exo}[Bac 2003]
  Déterminer tous les entiers positifs $x$, $y$ et $z$ solutions de
  \[x^2 + y^2 = 7 z^2.\]
\end{exo}

\subsubsection{À vous de trouver la méthode}

À vous de déterminer la ou les méthodes à utiliser pour résoudre les exercices suivants:

\begin{exo}
  Déterminer l'ensemble des entiers positifs $a$ et $b$ tels que
  \[3^a = 2^b +1.\]
\end{exo}

\begin{exo}
  Déterminer l'ensemble des entiers positifs $a$ et $b$ tels que
  \[a^2 + b^2 = (ab)^2.\]
\end{exo}

\subsubsection{Solutions}

\begin{sol}
  On remarque que
  \[5 \times 1 + 2 \times (-2) = 1\]
  donc pour tout $n$ entier
  \[5 \times n + 2 \times (-2n) = n\]
  et \framebox{tous les entiers $n \in \mathbb{N}$ conviennent}.


  Il s'agit en fait tout simplement du théorème de Bézout.
\end{sol}

\begin{sol}
  Il s'agit du théorème de Bézout: il existe une solution ssi
  \[\boxed{\operatorname{pgcd}(a,b) | n}.\]
\end{sol}

\begin{sol}
  Il y a de très nombreuses façons de procéder. On peut par exemple remarquer que si $n$ convient alors $n+2$ convient aussi (il suffit de changer $y$ en $y+1$).

  On teste alors à la main les petits cas:
  \begin{itemize}
    \item $0=5 \times 0 + 2 \times 0$,
    \item si $x$ et $y$ non nuls alors $5x+2y \geq 2$ donc $1$ ne convient pas,
    \item $2=5 \times 0 + 2 \times 1$,
    \item si $x\geq 1$ ou $y\geq 2$ alors $5x+2y \geq 4$ et donc $3$ ne convient pas,
    \item $2=5 \times 0 + 2 \times 2$,
    \item $5=5 \times 1 + 2 \times 0$.
  \end{itemize}

  Et par la remarque précédente tous les entiers suivants conviennent.

  L'équation admet des solutions entières positives ssi
  \[\boxed{n \in \mathbb{N} \backslash \{1,3\}}.\]

  \underline{Remarque}: avec des $a$ et $b$ premiers entre eux on montre que le plus grand entier non atteignable ets $ab-a-b$.
\end{sol}


\begin{sol}
  Le plus simple est de reconnaître un \textbf{trinôme du second degré}.

  Avec $x = n^2$, $n$ est solution de l'équation ssi
  \[x^2-7x+12 = 0.\]

  Il s'agit d'un trinôme du second degré de discriminant
  \[\Delta = 7^2 - 4 \times 12 = 1 > 0\]
  et donc ayant deux racines réelles
  \[\frac{7 \pm 1}{2}.\]

  Donc $n$ est solution ssi $n^2 = 4$ ou $n^2 = 3$.

  Or $n^2 = 4$ admet deux solutions entières: $n=2$ et $n=-2$.

  Et $n^2 = 3$ n'admet aucune solution entière (cf la suite).


  On conclut donc que l'ensemble des solutions entières est
  \[\boxed{\{-2,2\}}.\]
\end{sol}

\begin{sol}

  \textbf{Première option: majoration}

  Soient $a$ et $b$ deux entiers strictement positifs vérifiant
  \[\frac{1}{a} + \frac{1}{b} = 1.\]
  Quitte à changer l'ordre on peut supposer que $0 < a \leq b$.

  Dans ce cas on a
  \[0 < \frac{1}{b} \leq \frac{1}{a}\]
  et donc

  \[\frac{1}{a} + \frac{1}{b} \leq \frac{2}{a}.\]
  d'où on déduit
  \[0 < a \leq 2.\]

  Il suffit alors de regarder $a = 1$ et $a=2$.

  \begin{itemize}
    \item Pour $a=1$, $\frac{1}{b} + 1 = 1$ n'admet pas de solution.
    \item Pour $a=2$, $\frac{1}{b} + \frac{1}{2} = 1$ admet $b=2$ comme solution.
  \end{itemize}

  La seule solution est donc
  \[\boxed{(a,b) = (2,2).}\]

  \textbf{Deuxième option: résolution dans $\mathbb{R}$}

  Si $a$ et $b$ sont solutions alors $a \neq 1$ et
  \[b = \frac{a}{a-1}.\]

  Donc comme $a$ et $b$ sont entiers $a-1 | a$ et donc $a-1| a-(a-1)$.

  On en déduit que $a-1 = 1$ donc $a=2$.

  Et donc $b=2$.

  La seule solution est donc
  \[\boxed{(a,b) = (2,2).}\]

\end{sol}

\begin{sol}
  Soient $a$ , $b$ et $c$ trois entiers strictement positifs vérifiant
  \[\frac{1}{a} + \frac{1}{b} + \frac{1}{c} = 1.\]
  Quitte à changer l'ordre on peut supposer que $0 < a \leq b \leq c$.

  Dans ce cas on a
  \[0 < \frac{1}{c} \leq \frac{1}{b} \leq \frac{1}{a}\]
  et donc

  \[\frac{1}{a} + \frac{1}{b}  + \frac{1}{c}\leq \frac{3}{a}.\]
  d'où on déduit
  \[0 < a \leq 3.\]

  Il suffit alors de regarder $a = 1$, $a=2$ et  $a=3$.

  \begin{itemize}
    \item Pour $a=1$, $1+\frac{1}{b} + \frac{1}{c} = 1$ n'admet pas de solution.
    \item Pour $a=2$, $\frac{1}{2}+\frac{1}{b} + \frac{1}{c} \frac{1}{b}  = 1$ revient à
          \[\frac{1}{b} + \frac{1}{c} = \frac{1}{2}.\]

          Comme $0 < b \leq c$, par le même raisonnement que précédemment on a
          \[\frac{1}{2} \leq \frac{2}{b}\]
          c'est à dire $b \leq 4$. De plus $b \geq a = 2$. Il suffit donc de regarder $b=2$, $b=3$ et $b=4$.
          \begin{itemize}
            \item Pour $b=2$, \[\frac{1}{2} + \frac{1}{c} = \frac{1}{2}\] n'admet pas de solution.
            \item Pour $b=3$, \[\frac{1}{3} + \frac{1}{c} = \frac{1}{2}\] admet $c=6$ comme solution.
            \item Pour $b=4$, \[\frac{1}{4} + \frac{1}{c} = \frac{1}{2}\] admet $c=4$ comme solution.
          \end{itemize}
    \item Pour $a=2$, $\frac{1}{3}+\frac{1}{b} + \frac{1}{c} \frac{1}{b}  = 1$ revient à
          \[\frac{1}{b} + \frac{1}{c} = \frac{2}{3}.\]

          Comme $0 < b \leq c$, par le même raisonnement que précédemment on a
          \[\frac{2}{3} \leq \frac{2}{b}\]
          c'est à dire $b \leq 3$. De plus $b \geq a = 3$. Il suffit donc de regarder $b=3$.
          \begin{itemize}
            \item Pour $b=3$, \[\frac{1}{3} + \frac{1}{c} = \frac{2}{3}\] admet $c=3$ comme solution.
          \end{itemize}
  \end{itemize}


  Les solutions vérifiant $a \leq b \leq c$ sont donc $(2,3,6)$, $(2,4,4)$ et $(3,3,3)$.

  En comptant les permutations de ces solutions, on obtient donc $6+3+1$ c'est à dire
  \[\boxed{10 \text{ triplets solutions}}.\]
\end{sol}

\begin{sol}
  On observe que $a=b$ est toujours solution.



  Fixons $a$ et résolvons
  \[a^x = x^a\]
  d'inconnue $x \in \mathbb{R}$.

  En passant au $\ln$ l'équation se réécrit

  \[\frac{\ln(x)}{x} = \frac{\ln(a)}{a}.\]

  Posons
  \[f : \begin{aligned}
      ]0,+\infty[ & \to \mathbb{R}                              \\
      x           & \mapsto \frac{\ln(x)}{x} - \frac{\ln(a)}{a}
    \end{aligned}\]
  et étudions la fonction $f$.

  La fonction $f$ est dérivable sur son ensemble de définition, et sa dérivée est, pour tout $x > 0$
  \[f'(x) = \frac{\frac{1}{x} \times x - \ln(x) \times 1}{x^2} = \frac{1-\ln(x)}{x^2}.\]

  La fonction $f$ est donc
  \begin{itemize}
    \item \textbf{croissante} sur $]0,e^1[$
    \item puis \textbf{décroissante} sur $]e^1,+\infty[$.
  \end{itemize}

  Comme de plus $f(a)=0$,

  \begin{itemize}
    \item Si $a > e^1$ on a le tableau de variations


          et l'équation $f(x) = 0$ admet une unique solution dans $]0,a[$ n'admet aucune solution dans $]a,+\infty[$.
    \item Si $a < e^1$ on a le tableau de variations


          et l'équation $f(x) = 0$ admet n'admet aucune dans $]0,a[$ et admet une unique solution $]a,+\infty[$.
  \end{itemize}

  Retournons au problème initial.

  Quitte à réordonner on peut supposer $a < b$.

  Si $a>e^1$ il n'y a donc pas de solution.

  Comme $a$ est un entier strictement positif et que $e^1 < 3$ il suffit donc de regarder $a=1$ et $a=2$.

  \begin{itemize}
    \item Pour $a=1$, $1^b = b^1$ n'admet pas de solution $b > 1$.
    \item Pour $a=2$, $2^b = b^2$ admet $b=4$ comme solution, et par l'étude précédente c'est la seule.
  \end{itemize}

  On conclut donc que la seule solution avec $a<b$  est $a=2$ et $b=4$.


  Par conséquent l'ensemble des solutions est
  \[\boxed{\left\lbrace (n,n), n \in \mathbb{N}^*\right\rbrace \cup \{(2,4),(4,2)\}}.\]

  \underline{Remarque:} il aurait été bien plus simple de, dès le début
  \begin{itemize}
    \item supposer $a < b$
    \item transformer le problème en
          \[\frac{\ln(a)}{a} = \frac{\ln(b)}{b}\]
          et étudier $\frac{\ln(x)}{x}$ sur $]0,+\infty[$. On en déduit directement que
          \[a < e^1 < b.\]
  \end{itemize}
\end{sol}
\begin{sol}
  On remarque tout d'abord que si $n$ et $m$ conviennent alors avec
  \[\frac{1}{{x_1}^2} + \cdots + \frac{1}{{x_n}^2} =1 \]
  et
  \[\frac{1}{{y_1}^2} + \cdots + \frac{1}{{y_m}^2} =1 \]
  on a
  \[\begin{aligned}
      \frac{1}{{x_1 y_1}^2} + \cdots + \frac{1}{{x_1 y_m}^2} + \frac{1}{{x_2}^2} + \cdots + \frac{1}{{x_n}^2} & = \left(\frac{1}{{y_1}^2} + \cdots + \frac{1}{{y_m}^2} \right) \frac{1}{{x_1}^2}+
      \\ = \frac{1}{{x_2}^2} + \cdots + \frac{1}{{x_n}^2} =1
    \end{aligned}\]
  et donc $n+m-1$ convient aussi.


  On cherche donc ce qui se passe pour les petits cas, en espérant ensuite reconstituer tous les entiers assez grands à partir de ces derniers.


  \begin{itemize}
    \item Pour $n=2$, si
          \[\frac{1}{{x_1}^2} + \frac{1}{{x_2}^2} = 1\]
          alors
    \item Pour $n=3$, on peut effectuer des majorations (cf exercice précédent) et on obtient qu'il n'y a pas de solution.
    \item Pour $n=4$ on peut remarquer directement que
          \[\frac{1}{2^2} + \frac{1}{2^2} + \frac{1}{2^2} + \frac{1}{2^2} = 1.\]
          Donc $n=4$ convient.
    \item Pour $n=5$, on majore comme dans les exercices précédents. Plus précisément si
          \[\frac{1}{{x_1}^2} + \frac{1}{{x_2}^2} + \frac{1}{{x_3}^2} + \frac{1}{{x_4}^2} + \frac{1}{{x_5}^2} = 1\]
          , en se ramenant à $1 < x_1 \leq x_2 \leq x_3 \leq x_4 \leq x_5$ on obtient
          \[1 \leq \frac{5}{{x_1}^2}\]
          donc $1 < x_1 \leq \sqrt{5}$. Il suffit de regarder $x_1 = 2$.

          Dans ce cas on obtient
          \[\frac{1}{{x_2}^2} + \frac{1}{{x_3}^2} + \frac{1}{{x_4}^2} + \frac{1}{{x_5}^2} = \frac{3}{4}\]
          et donc de même
          \[\frac{3}{4} \leq \frac{4}{{x_2}^2}\]
          d'où $x_1 =2 \leq x_2 \leq \sqrt{\frac{16}{3}} < 3$. Il suffit donc de regarder $x_2 = 2$.

          Dans ce cas on obtient
          \[\frac{1}{{x_3}^2} + \frac{1}{{x_4}^2} + \frac{1}{{x_5}^2} = \frac{1}{2}\]
          et donc de même
          \[\frac{1}{2} \leq \frac{3}{{x_3}^2}\]
          d'où $x_2 =2 \leq x_2 \leq \sqrt{6} < 3$. Il suffit donc de regarder $x_3 = 2$.

          Dans ce cas on obtient
          \[\frac{1}{{x_4}^2} + \frac{1}{{x_5}^2} = \frac{1}{4}\]
          et donc de même
          \[\frac{1}{4} \leq \frac{2}{{x_4}^2}\]
          d'où $x_3 =2 \leq x_2 \leq \sqrt{8} < 3$. Il suffit donc de regarder $x_4 = 2$.

          Dans ce cas on obtient
          \[\frac{1}{{x_5}^2} = 0\]
          qui n'admet pas de solution.

    \item Pour $n=6$ on peut de même que précédemment effectuer des majorations en chaîne pour se ramener à un nombre fini de cas à traiter.

          Plus précisément, en ordonnant de même que précédemment $1 < x_1 \leq x_2 \leq x_3 \leq x_4 \leq x_5 \leq x_6$ on obtient successivement

          \[1 \leq \frac{6}{{x_1}^2}\]
          c'est à dire $1 < x_1 \leq \sqrt{6} < 3$ et il suffit donc de regarder $x_1 = 2$. On obtient alors
          \[\frac{3}{4} \leq \frac{5}{{x_2}^2}\]
          c'est à dire $x_2 = 2 \leq x_1 \leq \sqrt{\frac{20}{3}} < 3$ et il suffit donc de regarder $x_2 = 2$. On obtient alors
          \[\frac{1}{2} \leq \frac{4}{{x_3}^2}\]
          c'est à dire $x_2 = 2 \leq x_3 \leq \sqrt{8} < 3$ et il suffit donc de regarder $x_3 = 2$. On obtient alors
          \[\frac{1}{4} \leq \frac{3}{{x_4}^2}\]
          c'est à dire $x_4 = 2 \leq x_1 \leq \sqrt{12} < 4$ et il suffit donc de regarder $x_4 = 2$ ou $x_4=3$. Pour $x_4 = 2$ on obtiendrait que la somme restante est nulle, ce qui est impossible. Il suffit donc de regarder $x_4 =3$. On obtient alors
          \[\frac{1}{4}-\frac{1}{9} \leq \frac{2}{{x_5}^2}\]
          c'est à dire $x_4 = 3 \leq x_5 \leq \sqrt{\frac{72}{5}} < 4$ il suffit donc de regarder $x_5 = 3$.

          On obtient alors

          \[\frac{1}{{x_6}^2} = \frac{5}{36} - \frac{1}{9}\]
          et $x_6 = 6$ convient.


          Par conséquent on obtient une solution qui est $(2,2,2,3,3,6)$.
    \item Le cas $n=7$ se construit à partir des précédents car $7 = 4 + 4 -1$.
    \item Pour $n=8$ après avoir effectué des majorations et regardé un nombre fini de cas on obtient que $(2,2,2,3,3,6,9,9,18)$ est solution.
    \item Tous les entiers $\geq 9$ conviennent. En effet par le cas $n=4$, si $m$ convient alors $m+3$ convient. Et les cas $6,7,8$ conviennent.
  \end{itemize}

  En conclusion les entiers $n$ qui conviennent sont \framebox{tous les entiers sauf $2$, $3$, $5$}.

\end{sol}
\begin{sol}
  On cherche donc les $n$ tels que l'équation
  \[5^n + 4 = a^2.\]
  Ce qui peut se réécrire
  \[5^n = a^2 - 4.\]
  En reconnaissant une identité remarquable, cela revient à
  \[(a-2)(a+2) = 5^n.\]

  Par conséquent, comme $5$ est premier, on en déduit que $a-2$ et $a+2$ sont des puissances de $5$, i.e il existe $r$ et $s$ entiers tels que
  \[\begin{aligned}
      a + 2 & = 5^r \\
      a - 2 & = 5^s
    \end{aligned}.\]

  Leur différence est donc, en supposant quitte à changer l'ordre que $r \geq s$,
  \[\begin{aligned}
      (a+2)-(a-2) = 4 & = 5^r-5^s          \\
                      & = 5^s (5^{r-s}-1).
    \end{aligned}\]

  La seule possibilité est donc $s=0$ et $r-s=1$.

  On en déduit que si $n$ est solution alors $5^n + 4 = 9$.

  On vérifie que $5^2 + 4 = 9$.

  Par conséquent il existe une unique solution qui est
  \[\boxed{n=2}\]


\end{sol}

\begin{sol}
  On a l'identité remarquable
  \[n^4 - m^4 = (n-m)(n^3+n^2 m + n m^2 + m^3).\]
  Comme $n > m$ on a donc
  \[42 \geq 4m^3\]
  par conséquent il faut que
  \[m \leq \sqrt[3]{\frac{42}{4}} < 3.\]

  Il suffit donc de regarder $m = 0,1,2$.

  \begin{itemize}
    \item Pour $m=0$, $n^4 = 42$ n'a pas de solutions entières.
    \item Pour $m=1$, $n^4 = 43$ n'a pas de solutions entières.
    \item Pour $m=2$, $n^4 = 50$ n'a pas de solutions entières.
  \end{itemize}

  Donc l'équation n'a \framebox{pas de solution}.
\end{sol}

\begin{sol}
  On regarde modulo $4$.

  Comme un carré est congru à $0$ ou $1$ modulo $4$ on a

  \[a^2 + b^2 \equiv 0,~1 \text{ ou } 2 \mod 4.\]

  Mais \[2023 \equiv 3 \mod 4.\]

  Donc il n'y a pas de solution.
\end{sol}

\begin{sol}
  On regarde modulo $11$.

  On a $2021 \equiv 8 \mod 11$.

  La table des carrés et des puissances $5$èmes modulo $11$ est

  \begin{center}
    \begin{tabular}{|l|c|c|c|c|c|c|c|c|c|c|c|c|c|c|}
      \hline
      $x \mod 11$   & $0$ & $1$ & $2$  & $3$ & $4$ & $5$ & $6$  & $7$  & $8$  & $9$ & $10$ \\
      \hline
      $x^2 \mod 11$ & $0$ & $1$ & $4$  & $9$ & $5$ & $3$ & $3$  & $5$  & $9$  & $4$ & $1$  \\
      $x^5 \mod 11$ & $0$ & $1$ & $10$ & $1$ & $1$ & $1$ & $10$ & $10$ & $10$ & $1$ & $10$ \\
      \hline
    \end{tabular}
  \end{center}

  Et il est donc impossible d'avoir
  \[a^{5} - 2 b^2 \equiv 8 \mod 11.\]


  L'équation n'admet donc \framebox{pas de solution}.

  \underline{Remarque:} Pourquoi modulo $11$ ? On cherche un entier $n$ pour lequel les puissances $2$ et $5$ sont simples modulo $n$. Comme le ppcm de $2$ et $5$ est $10$ on cherche par exemple $n$ tel que $\varphi(n)=22$ ce qui permet de simplifier toutes les puissances $10$ème. Ici $11$ convient car $11$ est premier et $11-1 = 10$.
\end{sol}
\begin{sol}
  Avec $r = \frac{p}{q}$ où $p$ et $q$ sont des entiers (positifs), $q \neq 0$, cela revient à montrer que l'équation
  \[p^3 = 2 q^3\]
  n'admet pas de solutions entières.

  Supposons qu'il existe $(p,q)$ solutions.

  Alors $2 | p^3$ et donc, comme $2$ est premier on a $2 | p$.

  Par conséquent on peut écrire $p = 2 \tilde{p}$ avec $\tilde{p}$ entier.

  Alors
  \[8 \tilde{p}^3 = 2 q^3\]
  i.e
  \[4 \tilde{p}^3 =  q^3.\]

  De même, comme $2$ est premier, $2 | q^3$ donc $2 | q$, et il existe un entier $\tilde{q}$ tel que $q = 2 \tilde{q}$.

  On a alors

  \[\tilde{p}^3 = 2 \tilde{q}.\]


  On a donc montrer que si $(p,q)$ est une solution alors on peut construire une solution $(\tilde{p},\tilde{q})$ avec $\tilde{p} < p$.

  Par principe de descente infinie, il n'existe donc \framebox{pas de solution}.

  \underline{Remarque}: De même que pour la preuve de l'irrationalité de $\sqrt{2}$, on peut rédiger différent, par exemple en prenant une solution vérifiant $\operatorname{pgcd}(p,q) = 1$, en prenant une solution minimale, on en considérant la valuation $2$-adique.
\end{sol}
\begin{sol}
  On observe que $x=y=z=0$ est solution.

  Montrons que c'est la seule.


  Soit $(x,y,z) \neq (0,0,0)$ une solution. Sans perte de généralité on peut supposer $x \neq 0$.




  On peut écrire modulo $7$
  \[x^2 = - y^2 \equiv 0 \mod 7\]


  Supposons que $7 \not| x$.
  Comme $7 \not| x$, comme $7$ est premier, $x$ est inversible modulo $7$.

  En notant $\xi$ un inverse de $x$ modulo $7$, i.e un entier tel que
  \[x \xi \equiv 1 \mod 7\]
  et en multipliant par $\xi$ on en déduit

  \[(\xi y)^2 \equiv -1 \mod 7.\]


  Or la table des carrés modulo $7$ est:


  \begin{center}
    \begin{tabular}{|l|c|c|c|c|c|c|c|}
      \hline
      $x \mod 7$   & $0$ & $1$ & $2$ & $3$ & $4$ & $5$ & $6$ \\
      \hline
      $x^2 \mod 7$ & $0$ & $1$ & $4$ & $2$ & $2$ & $4$ & $1$ \\
      \hline
    \end{tabular}
  \end{center}
  et on aboutit donc à une contradiction.


  Donc pour toute solution non nulle $7 | x$.

  Si $7 | x$ alors $7 | y$ et donc en posant $x = 7 \tilde{x}$ et $y= 7 \tilde{y}$ on obtient
  \[7 \tilde{x} + 7 \tilde{y} =  z\]
  donc on peut écrire $z = 7 \tilde{z}$. Et obtient alors
  \[\tilde{x}^2 + \tilde{y}^2 = 7 \tilde{z}^2.\]

  On obtient donc une solution $(\tilde{x},\tilde{y},\tilde{z})$ avec $0 < \tilde{x} < x$.

  Par principe de descente infinie il n'existe pas de solution non nulle.

  \underline{Remarque:} ici le point clé est que $-1$ n'est pas carré modulo le nombre premier $7$. Un résultat très classique est que $-1$ est un carré modulo le nombre premier impair $p$ ssi $p \equiv 1 \mod 4$.
\end{sol}