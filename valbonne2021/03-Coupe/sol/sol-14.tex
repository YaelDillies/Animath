\begin{center}
\begin{tikzpicture}[scale=1]
\tkzDefPoint(1,5){X}
\tkzDefPoint(3,0){Y}
\tkzDefPoint(-3,0){Z}
\tkzDefPoint(0.5,3){M}
\tkzInterLL(Y,M)(X,Z) \tkzGetPoint{N}
\tkzInterLL(Z,M)(X,Y) \tkzGetPoint{L}

\tkzDrawSegment(X,Y)
\tkzDrawSegment(Y,Z)
\tkzDrawSegment(Z,X)
\tkzDrawSegment(Y,N)
\tkzDrawSegment(L,Z)

\tkzDrawPoints[fill=white,color=black](X,Y,Z,M,N,L)

\tkzLabelPoint[above](X){$X$}
\tkzLabelPoint(Y){$Y$}
\tkzLabelPoint[below left](Z){$Z$}
\tkzLabelPoint[below](M){$M$}
\tkzLabelPoint[above left](N){$N$}
\tkzLabelPoint[above right](L){$L$}

\begin{scope}[shift={(10,0)}]
\tkzDefPoint(-2,6){A}
\tkzDefPoint(-3,0){B}
\tkzDefPoint(3,0){C}
\tkzDefPoint(2,6){D}
\tkzDefPoint(-0.25,2.5){P}

\tkzMarkSegment[color=blue,mark=s||](A,B)
\tkzMarkSegment[color=blue,mark=s||](C,D)
\tkzDrawSegment(A,B)
\tkzDrawSegment(B,C)
\tkzDrawSegment(C,D)
\tkzDrawSegment(D,A)
\tkzDrawSegment(A,C)
\tkzDrawSegment(B,D)
\tkzDrawSegment(A,P)
\tkzDrawSegment(B,P)
\tkzDrawSegment(C,P)
\tkzDrawSegment(D,P)
\tkzDrawPoints[fill=white,color=black](A,B,C,D,P)

\tkzLabelPoint[above left](A){$A$}
\tkzLabelPoint[below left](B){$B$}
\tkzLabelPoint[below right](C){$C$}
\tkzLabelPoint[above right](D){$D$}
\tkzLabelPoint[below](P){$P$}
\end{scope}
\end{tikzpicture}
\end{center}

\begin{enumerate}
\item
Il suffit de vérifier directement que
\begin{align*}
MY + MZ
& \le MY + MN + NZ && \text{par inégalité triangulaire} \\
& \le NY + NZ && \text{car $M$, $N$ et $Y$ sont alignés} \\
& \le NX + XY + NZ && \text{par inégalité triangulaire} \\
& \le XZ + XY && \text{car $X$, $N$ et $Z$ sont alignés.}
\end{align*}

\item
Nous allons utiliser la question précédente. On commence par l'inégalité de gauche. Les diagonales du quadrilatère $ABCD$ partagent celui-ci en quatre triangles de sorte que le point $P$ appartient à au moins deux des triangles $ABC$, $BCD$, $ABD$ et $ACD$.

Si $P$ appartient au triangle $ABC$, la question 1.
indique que $PA + PC \le BA + BC < 2 AB$.
Sinon, $P$ appartient au triangle $ACD$, et la question
1. indique que $PA + PC \le DA + DC < 2AB$.
Dans les deux cas, on a bien $PA + PC < 2AB$.

On montre de même que $PB + PD < 2AB$, de telle sorte que
$$PA+PB+PC+PD < 4AB.$$

On montre ensuite l'inégalité de droite. D'après l'inégalité triangulaire dans le triangle $PCD$,
$$PC +PD \ge CD,$$
avec égalité si et seulement si $P$ appartient au côté
$[CD]$.
De même, dans le triangle $PAB$,
$$PA+PB \ge AB$$
avec égalité si et seulement si $P$ appartient au côté $[AB]$.
Ces deux cas d'égalité sont donc incompatibles, si bien que
$$2(PA+PB+PC+PD) > 2AB+2CD = 4AB$$

\end{enumerate}


\altsol

\begin{center}
\begin{tikzpicture}
[scale=1]
\tkzInit[ymin=-0.5,ymax=6.5,xmin=-5,xmax=5]
\tkzClip

\tkzDefPoint(-2,6){A}
\tkzDefPoint(-3,0){B}
\tkzDefPoint(3,0){C}
\tkzDefPoint(2,6){D}
\tkzDefPoint(-0.25,2.5){P}
\tkzDefLine[perpendicular=through P](A,D) \tkzGetPoint{m}
\tkzInterLL(P,m)(A,D) \tkzGetPoint{M}
\tkzDefLine[perpendicular=through P](B,C) \tkzGetPoint{n}
\tkzInterLL(P,n)(B,C) \tkzGetPoint{N}
\tkzDefLine[perpendicular=through A](B,C) \tkzGetPoint{x}
\tkzInterLL(A,x)(B,C) \tkzGetPoint{X}

\tkzMarkSegment[color=blue,mark=s||](A,B)
\tkzMarkSegment[color=blue,mark=s||](C,D)
\tkzMarkRightAngle[color=red](P,M,D)
\tkzMarkRightAngle[color=red](C,N,P)
\tkzMarkRightAngle[color=red](C,X,A)
\tkzDrawSegment(A,B)
\tkzDrawSegment(B,C)
\tkzDrawSegment(C,D)
\tkzDrawSegment(P,M)
\tkzDrawSegment(P,N)
\tkzDrawSegment(D,A)
\tkzDrawSegment(A,P)
\tkzDrawSegment(B,P)
\tkzDrawSegment(C,P)
\tkzDrawSegment(D,P)
\tkzDrawSegment[dashed](A,X)
\tkzDrawPoints[fill=white,color=black](A,B,C,D,P,M,N,X)

\tkzLabelPoint[above left](A){$A$}
\tkzLabelPoint[below left](B){$B$}
\tkzLabelPoint[below right](C){$C$}
\tkzLabelPoint[above right](D){$D$}
\tkzLabelPoint[above](M){$M$}
\tkzLabelPoint[below](N){$N$}
\tkzLabelPoint[below](X){$X$}
\tkzLabelPoint[right](P){$P$}
\end{tikzpicture}
\end{center}

On présente une deuxième façon de réaliser l'inégalité de gauche. On note $M$ et $N$ les projetés oorthogonaux du point $P$ respectivement sur les segment $[AD]$ et $[BC]$.

Alors d'après l'inégalité triangulaire, on a les inégalités suivantes :
$$\begin{array}{llllllll}
PA\le PM+MA &,& PB\le PN+NB &,& PC \le PN+NC&,&PD\le PM+MD\\
\end{array}$$

Puisque $PM+PN=MN$,$AM+MD=AD$ et $BN+NC=BC$, on obtient en sommant les inégalités que

$$ PA+PB+PC+PD \le 2MN+AD+BC$$

Ensuite, on sait que $AD< AB$ et $BC<AB$. Il est donc suffisant de montrer que $MN\le AB$. Pour cela, on introduit le projeté orthogonal du point $A$ sur la droite $[BC]$, que l'on note $X$. Le quadrilatère $AXNM$ est un rectangle puisque ses côtés sont parallèles deux à deux et qu'il possède un angle droit. On a donc $MN=AX$. Mais dans le triangle rectangle $AXB$, l'hypothénuse est plus grand que chacun des deux autres côtés, ce qui signifie que $AB>AX=MN$, comme annoncé. On a donc bien

$$PA+PB+PC+PD < 4AB$$
