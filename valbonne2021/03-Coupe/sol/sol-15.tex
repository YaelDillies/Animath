L'exercice demande de trouver le plus petit entier $n$ vérifiant une certaine propriété. Pour montrer que le plus grand entier recherché est un entier $c$, il y a donc nécessairement deux parties distinctes : l'analyse, dans laquelle on établit que tout entier $n$ vérifiant la propriété énoncée vérifie $n\ge c$, et la construction, dans laquelle on donne un exemple de $c$ réels $x_1, \ldots , x_c$ de l'intervalle $]-1,1[$ vérifiant les deux égalités mentionnées.

\bigskip

\textbf{Analyse :}
Soit $n$ le plus petit entier vérifiant la propriété de l'énoncé et soit $x_1, \ldots , x_n$ des réels de l'intervalle $]-1,1[$ pour lesquels
$$x_1+\ldots +x_n =0 \hspace{6mm} \text{ et } \hspace{6mm}
x_1^2+\ldots +x_n^2 = 2020.$$

Tout d'abord, puisque $x_i^2 < 1$ pour tout entier $i \le n$, on a $2021=x_1^2+\ldots + x_n^2 <1+ \ldots +1 =n$ donc $n\ge 2021$.

Par ailleurs, si l'un des réels, par exemple $x_n$, est nul,
alors
$$x_1+\ldots +x_{n-1} =0 \hspace{6mm} \text{ et } \hspace{6mm}
x_1^2+\ldots +x_{n-1}^2 = 2020.$$
Ainsi, $n-1$ vérifie également la propriété, en contradiction
avec la minimalité de $n$. On sait donc que
tous les réels $x_i$ sont non nuls.
Quitte à les renuméroter, on peut aussi supposer que ${-1 < x_1, \ldots , x_k <0}$ et $0 < x_{k+1}, \ldots , x_n < 1$ pour un certain entier $1\le k \le n-1$. En effet, comme la somme des réels est nulle, on dispose d'au moins un réel strictement négatif, et d'au moins un réel strictement positif.

\medskip

Tout d'abord, puisque $x_i > -1$ pour tout réel $i \le k$, on sait que
$$0 = x_1 + x_2 + \ldots + x_n > (-k) + x_{k+1} + x_{k+2} + \ldots + x_n.$$

En outre, pour tout $i \le k$, le réel $x_i$ appartient à l'intervalle $]-1,0[$, donc $x_i^2 < 1$.
De même, pour tout $i \ge k+1$, le réel $x_i$ appartient
à l'intervalle $]0,1[$, donc $x_i^2 < x_i$.
On en déduit que
$$2020 = x_1^2 + x_2^2 + \ldots + x_n^2
< k + x_{k+1} + x_{k+2} + \ldots + x_n,$$
et donc que
$$2020 - 2k < (-k) + x_{k+1} + x_{k+2} + \ldots + x_n < 0.$$

On a donc $k>1010$, soit $k\ge 1011$. Ainsi, quels que soient les réels tous non nuls vérifiant les deux égalités, il y a au moins $1011$ réels strictement négatifs. Or, si $(x_1, \ldots ,x_n)$ est un $n$-uplet vérifiant les deux égalités de l'énoncé, le $n$-uplet
$(-x_1, \ldots , -x_n)$ vérifie également les deux égalités de l'énoncé. On a donc également au moins $1011$ réels strictement positifs. On a donc en tout au moins $2022$ réels, de sorte que $n\ge 2022$.

\bigskip

\textbf{Construction :}
Réciproquement, on montre que $n=2022$ vérifie la propriété de l'énoncé. Le plus simple pour cela est de chercher un
$2022$-uplet de la forme $(x,-x,x,-x, \ldots , x, -x)$ avec $x$ bien choisi. En injectant ce $2022$-uplet
dans l'équation de droite, on trouve $2022x^2=2020$, soit
$$x = \pm\sqrt{\dfrac{2020}{2022}}.$$
Par construction, pour un tel $x$, le $2022$-uplet
$(x,-x,x,-x, \ldots , x, -x)$ vérifie bien les deux égalités.

\bigskip

En conclusion, le plus petit entier $n$ vérifiant la propriété de l'énoncé est $n=2022$.