L'exercice était dur et a peu été réussi. Nous attirons l'attention sur le fait qu'il est bien de chercher l'exercice 17 comme tout exercice de la coupe. Néanmoins, étant donné que c'est l'exercice 17, donc l'exercice à priori le plus dur de la coupe, nous ne pouvons qu'encourager les élèves à bien chercher les exercices précédents : il vaut mieux très bien réussir les 5 ou 6 premiers exercices avant de se décider à attaquer cet exercice, sur lequel l'écrasante majorité des élèves a eu au plus $1$. Plusieurs remarques : 
\begin{itemize}
\item Certains ont annoncé dès le début que telle ou telle situation était la plus optimale. Pour eux peut-être, mais cela ne constitue jamais une preuve : il y a plein d'autres possibilités.
\item Certains essaient de placer les fourmis de manière "optimale" et n'arrivent pas à en placer $8$ et concluent donc qu'on ne peut pas mieux faire. Néanmoins, cela ne constitue pas une preuve, puisqu'il aurait été possible de les placer de façon radicalement différente et potentiellement plus intelligente.
\item Certains élèves invoquent que si on met au départ chaque fourmi dans un coin, et on les déplace, cela diminue la distance minimale. Certes c'est vrai au début, mais après plusieurs mouvements de fourmis, cet argument ne fonctionne plus.
\item Beaucoup d'élèves font un simulacre de preuve, en annonçant des faits, sans rien prouver, et aboutissent sans scrupule à la fin de leur démonstration. Il est dommage de voir autant de preuves fausses : si votre preuve n'est pas correcte et que vous le savez, il ne faut pas se dire qu'aligner les phrases pour embrouiller le correcteur est une bonne idée, il vaut mieux être honnête et ne pas prétendre avoir prouver ce qu'on n'a pas prouvé, mais écrire les différentes remarques pertinentes (et valorisées) sur le problème qui, elles, rapporteront vraiment des points. L'honnêteté intellectuelle est une qualité cruciale en mathématiques : il n'est pas grave de ne pas trouver, ça arrive à tout le monde, par contre le manque d'honnêteté lui sera bien plus sévèrement sanctionné au fil de la scolarité. Pour ceux qui pensaient avoir fait une preuve juste, mais qui ne l'était manifestement pas au regard de leur note, il faut se questionner quand on rédige une preuve. Est-ce que ma preuve est rigoureuse, les faits que j'affirme sont bien démontrés et tous les cas possibles sont bien passés en revue ? Cela vous permettra d'éviter de passer à côté d'un exercice, et parfois la déception/incompréhension face à une note ne correspondant pas au ressenti sur l'exercice.
\end{itemize}