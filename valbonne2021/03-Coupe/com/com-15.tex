Le problème était très difficile et n'a été résolu que pas une poignée d'élèves. Si il était difficile de résoudre complètement l'exercice, certaines parties de la preuve étaient néanmoins abordables et pouvait donner de nombreux points partiels, comme fournir une construction pour $2022$ ou montrer que $n>2022$.  Ainsi, même si très peu d'élèves ont réussi à fournir un raisonnement correct qui démontre qu'il n'existe pas $2021$ réels satisfaisant les deux équations, la plupart des élèves ayant abordé le problème ont bien vu ces deux parties et ont récupéré de précieux points partiels. Etre capable de gratter des points partiels sans pour autant trouver la solution à un problème est un atout important pour les olympiades. La clé de l'exercice était de séparer les réels positifs des réels négatifs puis d'établir des inégalités entre les différents quantités (en utilisant des majorations simples comme $x^2\leqslant |x|$ ou même $x_1+\ldots +x_k <k$) pour finalement les combiner correctement. Il y a avait donc plusieurs occasions de gratter des points supplémentaires en établissant l'une ou l'autre des inégalités intermédiaires.
Nous détaillons ici les raisonnements rencontrés et pourquoi ils ne permettent pas de conclure.
\begin{itemize}
\item Nous rappelons que les réels appartenaient à l'intervalle $]-1;1[$, c'est-à-dire qu'ils ne pouvaient pas prendre la valeur $\pm 1$. Il est regrettable de voir que plusieurs élèves n'ont pas tenu compte de ce détail important. 
\item Plusieurs élèves affirment qu'une bonne façon de procéder est de coupler les réels avec leur opposé, et que donc $n$ est impair. Mais bien sûr, ce n'est pas parce qu'une façon pratique de trouver des réels et de choisir des réels et leur opposés qu'il n'y a pas une meilleure façon de faire et notamment qu'il n'y a pas de solution dans le cas $n=2021$. Dans le même ordre d'idée, des élèves affirment que dans le cas où $n=2021$, on a forcément trois réels de somme nulle, mais cela n'est évidemment pas vrai. 
\item Beaucoup d'élèves donnent des raisonnements très vagues dans lesquels les réels sont passés à la limite en $\pm 1$. Mais ici, on se donne un entier $n$ et  on suppose qu'il existe une solution au sytème de deux équations à $n$ inconnues, cela n'a donc pas vraiment de sens de passer les réels à la limite.
\end{itemize}