L'exercice a dans l'ensemble été très bien réussi par une large majorité d'élèves. Quelques remarques d'ensemble : \newline $\bullet$ Plusieurs copies confondent les mots \og équilatéral \fg et \og isocèle \fg. On rappelle qu'un triangle $XYZ$ est dit isocèle en $X$ si les côtés $[XY]$ et $[XZ]$ sont de même longueur, alors qu'un triangle $XYZ$ est dit équilatéral si ses trois côtés sont de même longueur. Ainsi un triangle équilatéral est toujours isocèle, mais la réciproque n'est pas vraie. \newline $\bullet$ Nous soulignons ici l'importance de bien lire l'énoncé, en effet plusieurs élèves ont placé le point $S$ à l'extérieur du carré au lieu de le placer à l'intérieur, se retrouvant ainsi pénalisés malgré un raisonnement souvent juste.
\begin{itemize}
\item En ce qui concerne la rédaction, elle est trop souvent négligée, même dans les bonnes copies : nous rappelons qu'une figure ne constitue pas une preuve, un dessin est là pour illustrer et accompagner un raisonnement, pas pour le remplacer. Attention aussi à ne pas tomber dans une sur-rédaction : il ne faut pas non plus tout détailler trop, au risque de perdre du temps qui aurait pu servir à chercher d'autres exercices ; ici, des arguments tels que la symétrie de la figure pouvaient permettre de réduire la taille de la preuve.
\item Par ailleurs, certains élèves se sont lancés dans des calculs de géométrie analytique : ça peut parfois aboutir, mais c'est une méthode qui n'est pas conseillée, car une preuve en analytique qui n'aboutit pas n'est pas valorisée, et il y a souvent des méthodes plus simples.
\item Enfin, il est important de faire une figure grande et juste, en effet plusieurs élèves se sont trompés et ne se sont pas rendus compte de leur(s) erreur(s) à cause d'une figure absente ou mal faite donc trompeuse.
\end{itemize}