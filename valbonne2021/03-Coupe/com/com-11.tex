Exercice piège. Beaucoup d'élèves ont eu la bonne idée de regarder des petites valeurs de $n$, mais ils sont nombreux aussi à s'être trompés dans les listes de diviseurs, les calculs de somme, ou à avoir oublié que $n$ n'était pas à compter dans la somme des diviseurs pairs. Beaucoup d'élèves ont raisonné plus sur le nombre de diviseurs que sur leur somme à proprement parler (et assez souvent de façon imprécise). Très peu d'élèves (moins d'un tiers des copies) ont pensé à l'astuce que pour chaque diviseur impair $d$ il y avait le diviseur pair $2d$ (qui est strictement inférieur à $n$ car $4d$ divise $n$). Beaucoup ont essayé de construire les diviseurs de $n$ à partir des diviseurs de $2020$, mais la stratégie alors est loin d'être simple car il est facile d'en oublier ou de construire plusieurs fois les mêmes, ce qui invalide la preuve. Nous invitons tous les élèves qui ont tenté cette approche à vérifier leur argument en l'appliquant par exemple à $60 \cdot 2020$.