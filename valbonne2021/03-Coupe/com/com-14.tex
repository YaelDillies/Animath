Le problème a été abordé par de nombreux élèves mais finalement peu d'élèves ont su en venir à bout. La principale difficulté (et qui n'est pas des moindres) dans ce problème était de parvenir à formaliser correctement des résultats très intuitifs. Cette difficulté s'est vue dans le fait que de nombreux élèves proposent une solution à toutes les questions mais qui obtiennent finalement pas ou peu de points, car leur tentative n'était pas assez rigoureuse. 
\newline
De ce point de vue là, la question $1)$ en particulier a été très mal résolue et seule une poignée d'élèves, que nous félicitons, sont parvenus à résoudre la question en toute rigueur, le plus souvent avec l'inégalité triangulaire, quand ce n'était pas en invoquant la théorie des ellipses. Nous détaillons ici les raisonnements les plus vus et pourquoi ceux-ci ne constituent pas une solution au problème : 
\begin{itemize}
\item Beaucoup d'élèves ont prétendu que l'on a toujours $MY\le XY$ et $MZ\le XZ$. Mais cela n'est pas vrai dans le cas général. Plus précisément, dans le cas où le côté $YZ$ est le plus grand côté du triangle, en choisissant le point $M$ suffisamment proche du point $Z$, on peut tomber dans une situation où $MY>XY$, et le raisonnement n'est donc plus valide. Il est alors clair que les inégalités du type $LZ<XZ$ et $NY<XY$ ne sont plus valides non plus. Bien sûr, on pourra alors argumenter que dans ce cas, le côté $MZ$ est si petit que l'inégalité tient forcément même dans ce cas, mais c'est précisément ce genre de raisonnement qui n'était pas accepté, car ce n'est pas un raisonnement mathématique rigoureux. La clé du problème était d'être capable de quantifier, c'est-à-dire traduire par des inégalités mathématiques, ces effets de compensation entre une éventuelle grande longueur $MY$ et une petite longueur $MZ$. Les raisonnement sous forme de prose qui racontent pourquoi on a "obligatoirement", "forcément" ou "clairement" cet effet de compensation sans aucune formule mathématique ne sont bien sûr pas convaincants d'un point de vue mathématique et sont donc à éviter.   
\item Beaucoup d'élèves se sont contentés d'invoquer que le point $M$ est situé à l'intérieur du triangle pour conclure que le périmètre du triangle $MYZ$ est forcément inférieur au périmètre du triangle $XYZ$. Mais une telle affirmation est bien sûr à justifier puisqu'elle est en fait équivalente à l'énoncé demandé. Un tel raisonnement n'a donc jamais ramené de points à ses auteurs. 
\item Beaucoup d'élèves ont affirmé, à juste titre, que l'aire du triangle $MYZ$ est inférieure à celle du triangle $XYZ$. Mais ils en déduisent alors immédiatemment que le périmètre est inférieur. Mais il n'est pas vrai qu'un triangle avec une plus grande aire qu'une autre a également un plus grand périmètre. On peut en effet construire des triangles d'aire fixée avec un périmètre abritrairement grand. 
\item Beaucoup d'élèves se sont contentés de regarder ce qu'il se passait aux extrémités du triangle. Par exemple, certains ont affirmé que l'inégalité était une égalité seulement dans le cas où $M=X$, ce qui est vrai mais reste à prouver. D'autres ont affirmé que la seule façon pour le point $M$ d'être "le plus loin possible des points $X$ et $Y$" était de se rapporcher du point $X$, mais une telle affirmation est assez vague, puisqu'il faudrait définir la notion de "distance à deux points", dont la formalisation et la preuve des diverses propriétés nécessiterait de d'abord montrer l'exercice. Enfin, certains élèves disent que la longueur $MZ$ est maximale en $M=L$ puis que la longueur $MY$ est maximale lorsque que $M=N$ puis concluent que le maximum est atteitn pour $N=L$. Mais un tel raisonnement ne fait que mettre en valeur des extremum locaux, qui ne guarantissent pas que la combinaison des deux extremum donne bien l'extremum global (cf le premier tiret). 
\end{itemize}
Pour la question 2, il fallait à nouveau invoquer l'inégalité triangulaire et invoquer l'inégalité obtenue à la question 1. Pas mal d'élèves ont réussi à réussir l'une ou l'autre de ces deux étapes. Les fréquentes erreurs sont du même type que les erreurs faites pour la question 1. Ainsi beaucoup d'élèves affirment que $PA,PB,PC,PD<AB$, alors qu'en prenant $P$ suffisamment proche de $C$, on tombait dans une situation où $PA>AB$, et là encore le jeu était d'être capable de quantifier les compensations entre les longueurs $PA$ et $PC$. Beaucoup d'élèves affirment aussi que si $PA+PB>2AB$, sans quoi $P$ serait à l'extérieur du trapèze. Si une telle affirmation est vraie, elle mérite bien sûr une justification. 