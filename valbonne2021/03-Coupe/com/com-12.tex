Problème plutôt bien réussi. Attention à bien multiplier les possibilités pour différentes portions indépendantes et non les ajouter. De nombreux élèves ont essayé de grouper les nombres. Si cela a parfois fonctionné, ce n'était nullement nécessaire. D'autres élèves ont fait des arbres de possibilités pour les premiers entiers. Si ce n'était pas nécéssaire, c'était visiblement une très bonne idée puisque ces copies se sont presque toujours révélé excellentes. Rappelons que si on colorie un nombre fini d'éléments avec un nombre fini de couleurs, on ne peut pas arriver à un nombre infini de possibilités. Certains élèves ont tenté de choisir une couleur particulière et de distinguer les coloriages possibles selon le nombre d'entiers de cette couleur. Cette méthode n'a malheureusement jamais été concluante.