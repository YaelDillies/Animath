Cet exercice a globalement été extrêmement peu traité par les candidats (seul un élève sur 4 l'a traité), et les élèves qui l’ont traité n’ont souvent pas réussi à obtenir un point...
Il faut dire qu’une des difficultés principales de l’exercice était d’avoir l’intuition que la partition existait, alors que bon nombre d’élèves ont tenté de prouver qu’elle n’existait pas. Signalons que pour éviter cet écueil, la meilleure méthode était de tester l’existence d’une partition en sous-ensembles équilibrés pour les ensembles $\{1,2,\ldots,n^2\}$ pour des petites valeurs de $n$. On pouvait trouver $\{1\}$ pour $n=1$, $\{1\}$ et $\{2,3,4\}$ pour $n=2$, $\{2,3,4\}$ et $\{1, 5, 6, 7, 8, 9\}$ pour $n=3$... puis émettre la conjecture que $\{1,2,\ldots,n^2\}$ est toujours partitionnable en deux sous-ensembles équilibrés (cf. corrigé).
Donnons deux remarques plus précises sur les copies corrigées.
\begin{itemize}
\item Comme dit plus haut, beaucoup d’élèves ont tenté d’expliquer pourquoi il ne pouvait pas y avoir de partition convenable. Certains ont réellement quantifié leurs remarques (par exemple en considérant la partie qui devait contenir $2021^2$), ce qui aurait pu être une bonne idée... A l’inverse, de nombreux élèves ont écrit une succession d’affirmations confuses, souvent sans justification. Rappelons à cet égard qu’à la Coupe Animath, sauf mention contraire, toute affirmation doit être soigneusement rédigée et justifiée, pour la rendre la plus compréhensible possible pour le correcteur

\item Une autre erreur régulièrement rencontrée est celle de considérer que l’inefficacité d’une construction proposée prouve qu’il n’existe aucune partition convenable. La plupart des élèves concernés ont d'abord montré que l'ensemble $\{1, 2, \ldots, n\}$ était partitionnable en sous-ensembles équilibrés pour les $n$ de la forme $1 + 3 + 3^2 + \ldots + 3^k$ avec $k \geqslant 0$, puis que $2021^2$ n'était pas de cette forme. Ils font enfin l'erreur fondamentale de raisonnement de dire que cette considération suffit à prouver qu'une partition convenable n'existe pas pour l'ensemble $\{1, 2, \ldots, 2021^2\}$.
\end{itemize}
