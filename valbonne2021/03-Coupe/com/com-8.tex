Beaucoup d'élèves ont trouvé l'idée du 1), à savoir que l'on aura deux fourmis sur la même arête, mais très peu ont réussi à trouver les tiroirs dans le 2) (à savoir les sommets complétés des demi-arêtes qui partent de ce sommet, ou encore de façon équivalente le sommet le plus proche d'une fourmi). Les erreurs les plus communes étaient de penser que pour avoir des situations optimales, on était obligé d'avoir des fourmis sur les sommets ou les milieux, ce qui n'est pas forcément vrai ; ou encore de penser que l'on pouvait considérer la somme des distances entre les fourmis pour le 1), mais cela ne marchait pas car on pouvait compter plusieurs fois le même bout d'arête.