\documentclass[french]{beamer}
\usetheme{Berkeley}

\usepackage[utf8]{inputenc}
\usepackage[T1]{fontenc}
\usepackage[french]{babel}

\usepackage{amsmath}
\usepackage{amssymb}
\usepackage{amsthm}
\usepackage{comment}
\usepackage{epigraph}
\usepackage{graphicx}
\usepackage{lmodern}
\usepackage{mathtools}
\usepackage{multimedia}
\usepackage{textpos}

\DeclareMathOperator{\C}{\mathcal C}
\DeclareMathOperator{\R}{\mathbb R}
\DeclarePairedDelimiter\ceil{\lceil}{\rceil}
\DeclarePairedDelimiter\floor{\lfloor}{\rfloor}
\DeclareMathOperator{\goesto}{\longrightarrow}
\DeclareMathOperator{\nimplies}{\centernot\implies}

\setbeamertemplate{theorems}[numbered]
\theoremstyle{plain}
\newtheorem{pb}{Problème}


\title{POFM $ 2020-21 $ - Cours par correspondance : \\ \textsc{Transformations du plan}}
\subtitle{Groupe C}
\author[Yaël Dillies]{Yaël Dillies}
\date{14 novembre 2020}


\begin{document}
\section{Introduction}
\begin{frame}
  \titlepage
\end{frame}


\begin{frame}{La phrase du jour}
  \epigraph{La géométrie est l'étude des propriétés invariantes sous un certain type de transformations.}{\textit{Felix Klein}}
\end{frame}


\begin{frame}{Sommaire}
  \tableofcontents
\end{frame}


\begin{frame}{Qu'est-ce qu'une transformation du plan ?}
  \begin{itemize}
    \item Fonction de $ \R^2 $ vers $ \R^2 $
    \pause
    \item Bijective
    \pause
    \item D'autres propriétés à spécifier
  \end{itemize}
\end{frame}



\section{Géométrie euclidienne}
\begin{frame}{Transformations en géométrie euclidienne}
  On veut que nos transformations préservent les angles, c'est-à-dire
  $$ \text{ pour tous points $ A, B, C $, } \widehat{f(A)f(B)f(C)} = \widehat{ABC} $$
  Quels genres de transformations obtient-on ?
\end{frame}


\subsection{Homothéties}
\begin{frame}{Définition d'une homothétie}
  L'\textit{homothétie} de centre $ O $ et de rapport $ k $ est la transformation qui associe au point $ X $ le point $ X' $ tel que $ \overline{OX'} = k\overline{OX} $ (en longueurs algébriques). \\
  En terme de vecteurs, $ \overrightarrow{OX'} = k\overrightarrow{OX} $ \\
  En terme de nombres complexes, $ X' - O = k(X - O) $. \\
  On écrit l'homothétie $ h_{O, k} $.
  \begin{figure}
    \centering
    %\includegraphics[width=0.4\textwidth]{Géométrie/Exemple d'homothétie.png}
    \caption{Un bonhomme noir et ses images bleue et rouge par les homothéties de centre $ O $ de rapport $ 2 $ et $ -\frac 12 $.}
  \end{figure}
\end{frame}


\begin{frame}{Propriétés des homothéties}
  \begin{exampleblock}{Propriétés des homothéties}
    \begin{itemize}
      \item Les homothéties conservent les droites, les cercles, les angles orientés, et les formes.
      \item L’image d’une droite est une droite parallèle.
      \item Une homothétie de rapport $ k $ multiplie toutes les longueurs par $ |k| $. En particulier, elle conserve les rapports de longueurs.
    \end{itemize}
  \end{exampleblock}
\end{frame}


\begin{frame}{Exemples d'homothéties}
  \begin{itemize}
    \item Les symétries centrales sont les homothéties de rapport $ -1 $.
    \item Si $ AB /\!/ A'B' $, alors il existe exactement une homothétie envoyant $A$ sur $A'$ et $ B $ sur $ B' $.
    \item Si $ \C, \C' $ sont deux cercles, alors il existe exactement deux homothéties envoyant $ \C $ sur $ \C' $, une de rapport positif et une de rapport négatif.
  \begin{figure}
    \centering
    %\includegraphics[width=0.5\textwidth]{Géométrie/Homothéties entre cercles.png}
    \caption{$ O_+ $ est le centre de l'homothétie de rapport positif envoyant $ \C $ sur $ \C' $ et $ O_- $ le centre de celle de rapport négatif.}
  \end{figure}
  Remarque : Les deux homothéties existent toujours, même si $ \C $ et $ \C' $ s'intersectent ou si l'un contient l'autre.
  \end{itemize}
\end{frame}


\begin{frame}{Concourance des médianes}
  \begin{figure}
    \centering
    %\includegraphics[width=0.5\textwidth]{Géométrie/Intersection des médianes.png}
    \caption{Grâce à une homothétie bien choisie, montrer que $ \overline{B'G} = -\frac 12\overline{BG}, \overline{C'G} = -\frac 12\overline{CG} $. En déduire que les médianes sont concourantes.}
  \end{figure}
\end{frame}


\begin{frame}{Composition d'homothéties}
  \begin{exampleblock}{Lemme de composition d'homothéties}
    Si $ h_1, h_2 $ sont deux homothéties de rapport $ k_1, k_2 $ et de centre $ O_1, O_2 $, alors $ h_2 \circ h_1 $ est une homothétie de rapport $ k_1k_2 $ et de centre un point de la droite
  \end{exampleblock}
  \begin{figure}
    \centering
    %\includegraphics[width=0.6\textwidth]{Géométrie/Composition d'homothéties.png}
    \caption{Remarque : quand $ k_1k_2 = 1 $, le centre de $ h_2 \circ h_1 $ va à l'infini et $ h_2 \circ h_1 $ devient une translation.}
  \end{figure}
\end{frame}


\begin{frame}{Théorème de Monge}
  \begin{figure}
    \centering
    %\includegraphics[width=0.8\textwidth]{Géométrie/Théorème de Monge.png}
    \caption{Les homothéties envoyant $ \C_1 $ sur $ \C_2 $, $ \C_2 $ sur $ \C_3 $, $ \C_1 $ sur $ \C_3 $ sont de centres $ X, Y, Z $.}
  \end{figure}
\end{frame}


\begin{frame}{Cercle d'Euler}
  \begin{figure}
    \centering
    %\includegraphics[width=0.5\textwidth]{Géométrie/Cercle d'Euler complet.png}
    \caption{Tout ce beau monde est cocyclique.}
  \end{figure}
\end{frame}


\begin{frame}{Homothéties entre le cercle circonscrit et le cercle d'Euler}
  \begin{figure}
    \centering
    %\includegraphics[width=0.5\textwidth]{Géométrie/Cercle d'Euler centres.png}
    \caption{Quels sont donc leurs centres ?}
  \end{figure}
\end{frame}


\subsection{Rotations}
\begin{frame}{Définition d'une rotation}
  La \textit{rotation} de centre $ O $ et d'angle $ \theta $ est la transformation qui associe au point $ X $ le point $ X' $ tel que $ OX' = OX $ et $ \widehat{X'OX} = \theta $ (en angles orientés). \\
  En terme de nombres complexes, $ X' - O = e^{i\theta}(X - O) $. \\
  On écrit la rotation $ r_{O, \theta} $.
  \begin{figure}
    \centering
    %\includegraphics[width=0.3\textwidth]{Géométrie/Exemple de rotation.png}
    \caption{Un bonhomme noir et son image bleue par la rotation de centre $ O $ d'angle $ 75^\circ $.}
  \end{figure}
\end{frame}


\begin{frame}{Propriétés des rotations}
  \begin{exampleblock}{Propriétés des rotations}
    \begin{itemize}
      \item Les rotations conservent les longueurs. Ce sont des \textit{isométries}.
      \item Les rotations conservent les droites, les cercles, les angles.
      \item L’image d’une droite par une rotation d’angle $ \theta $ fait un angle de $ \theta $ avec celle-ci.
    \end{itemize}
  \end{exampleblock}
\end{frame}


\begin{frame}{Exemples de rotations}
  \begin{itemize}
    \item Les symétries centrales sont les rotations d'angle $ \pi $.
    \item Si $ AB = A'B' $, alors il existe exactement une rotation (dégénérée en translation si $ AB /\!/ A'B' $) envoyant $A$ sur $ A'$ et $ B $ sur $ B' $. (essayez de la construire !)
  \end{itemize}
\end{frame}


\begin{frame}{Composition de rotations}
  \begin{exampleblock}{Lemme de composition de rotations}
    Si $ r_1, r_2 $ sont deux rotations d'angles $ \theta_1, \theta_2 $, alors $ r_2 \circ r_1 $ est une rotation d'angle $ \theta_1 + \theta_2 $.
  \end{exampleblock}
  Remarque : quand $ \theta_1 + \theta_2 \equiv 0 \mod 360^\circ $, le centre de $ r_2 \circ r_1 $ va à l'infini et $ r_2 \circ r_1 $ devient une translation.
\end{frame}


\subsection{Symétries axiales}
\begin{frame}{Définition d'une symétrie axiale}
  La \textit{symétrie axiale} d'axe la droite $ d $ est la transformation qui au point $ X $ associe $ X' $ tel que $ d $ soit la médiatrice de $ [XX'] $.
\end{frame}


\begin{frame}{Propriétés des symétries axiales}
  \begin{exampleblock}{Propriétés des symétries axiales}
    \begin{itemize}
      \item Les symétries axiales conservent les longueurs. Ce sont des isométries.
      \item Les symétries axiales conservent les droites, les cercles, mais changent le signe des angles orientés.
      \item Pour tous points $ A, A'$, il existe exactement une symétrie axiale envoyant $A$ sur $A'$.
    \end{itemize}
  \end{exampleblock}
\end{frame}


\begin{frame}{Composition de symétries axiales}
  \begin{exampleblock}{Lemme de composition des symétries axiales}
    Si $ s_1, s_2 $ sont deux symétries axiales dont les axes forment un angle de mesure $ \alpha $, alors $ s_2 \circ s_1 $ est la rotation d'angle $ 2\alpha $ de centre l'intersection des deux axes.
  \end{exampleblock}
  \begin{figure}
    \centering
    %\includegraphics[width=0.3\textwidth]{Géométrie/Composition de symétries axiales.png}
    \caption{Remarque : si les deux axes sont parallèles, alors la rotation est dégénérée en translation.}
  \end{figure}
\end{frame}


\subsection{Similitudes}
\begin{frame}{Définition d'une similitude}
  Une \textit{similitude directe} est toute transformation préservant les angles orientés. \\
  Une \textit{similitude indirecte} est la composée d'une similitude directe avec une symétrie axiale. \\
  Les homothéties, rotations, translations, symétries axiales sont toutes des similitudes. C'est en un sens \textbf{la forme la plus générale} de transformation en géométrie euclidienne. \\
  S'il existe une similitude directe envoyant $A$ sur $ A' $, $ B $ sur $ B' $, $ C $ sur $ C' $, on écrit $ ABC \sim A'B'C' $. \\
  S'il existe une similitude indirecte envoyant $A$ sur $ A' $, $ B $ sur $ B' $, $ C $ sur $ C' $, on écrit $ ABC \overline\sim A'B'C' $.
\end{frame}


\begin{frame}{Propriétés des similitudes}
  \begin{exampleblock}{Propriétés des similitudes}
    \begin{itemize}
      \item Les similitudes conservent les droites, les cercles, les angles (non orientés).
      \item Les similitudes conservent les longueurs. Ce sont des isométries. Et en fait toute isométrie est une similitude.
      \item Toute similitude directe peut être écrite comme la composition d'une homothétie (potentiellement dégénérée en translation) et d'une rotation de même centre.
    \end{itemize}
  \end{exampleblock}
\end{frame}


\begin{frame}{Théorème de Ptolémée}
  \begin{exampleblock}{Théorème de Ptolémée}
    Pour tous points $ A, B, C, D $,
    $$ AC \cdot BD \le AB \cdot CD + AD \cdot BC $$
    avec égalité ssi $ ABCD $ sont cocycliques dans cet ordre.
  \end{exampleblock}
  Indice : Introduire $ E $ tel que $ ABE \sim ADC $.
\end{frame}


\begin{frame}{Deux similitudes pour le prix d'une}
  \begin{exampleblock}{Similitudes internes}
    Les similitudes envoyant $ \begin{cases}A \mapsto A' \\ B \mapsto B'\end{cases} $ et $ \begin{cases}A \mapsto B \\ A' \mapsto B'\end{cases} $ sont de même centre.
  \end{exampleblock}
\end{frame}


\begin{frame}{Théorème de Miquel}
  \begin{exampleblock}{Théorème de Miquel}
    Le point de Miquel $ M $ est le centre de plein de similitudes.
  \end{exampleblock}
  \begin{figure}
    \centering
    %\includegraphics[width=0.5\textwidth]{Géométrie/Théorème de Miquel.png}
    \caption{Trouvez-les toutes !}
  \end{figure}
\end{frame}


\begin{frame}{Puissance d'un point}
  \begin{figure}
    \centering
    %\includegraphics[width=0.8\textwidth]{Géométrie/Puissance d'un point.png}
  \end{figure}
\end{frame}
\end{document}